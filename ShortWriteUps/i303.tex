\Version {RDWORD}                    \Routid{I303}
\Keywords{READ FORMAT FREE NUMBER}
\Author{G. Kozlovsky}             \Library{KERNLIB}
\Submitter{}                \Submitted{07.09.1986}
\Language{Fortran}       %\Revised{}
\Cernhead {Read a Format-Free Number}
{\tt RDWORD} is a lexical analyser which reads numbers, alphabetic names
and special characters from its internal buffer in format free form.
The definitions of the three above mentioned lexical entities are as
follows:
\par
A number has the form {\tt sd.dEsdx}.
\begin{DLtt}{1234}
\item[s:] {\tt '+'} or {\tt '-'} or null;
\item[d:] Zero or more decimal digits;
\item[.:] Decimal point or may be omitted;
\item[E:] {\tt 'e'} or {\tt 'E'} or null;
\item[x:] Terminating character which may be a blank, a letter,
a special character, a second appearance of {\tt '.'} or {\tt 'E'},
a {\tt '+'} or a {\tt '-'} followed by a digit.
\end{DLtt}
The sequence {\tt d.d} must contain at least one digit ({\tt 0.} or
{\tt .0} are valid numbers, but not a decimal point alone).
If {\tt E} is omitted, the following {\tt sd} must be omitted.
If {\tt E} is  present, it must be followed by a {\tt d}
containing at least one digit.
\par
An alphabetic name has the form {\tt ax} and an alphanumeric name the
form {\tt ady}.
\begin{DLtt}{1234}
\item[a:] One or more alphabetic characters ({\tt a,...,z,A,...,Z});
\item[d:] Zero or more decimal digits;
\item[x:] Terminating character, which can be any non-alphabetic
character;
\item[y:] Terminating character, which can be any character except
an alphabetic character or a decimal digit.
\end{DLtt}
A special character is any character except an alphabetic character,
decimal digit or blank.
\par
Lexical entities may be separated by any number of blanks. The internal
buffer has a length of 80 characters. A blank is added logically as an
81st character.
\Structure
{\tt SUBROUTINE} subprograms\\
User Entry Names:
\Rdef{RDWORD}, \Rdef{RDLOAD}, \Rdef{RDSKIP}, \Rdef{RDENDB} \\
Internal Entry Names:
\begin{tabular}[t]{l*{7}{@{\hspace{4pt}}l}}
{\tt RALPH},  & {\tt RDIGI},  & {\tt RDINT},  & {\tt RDMANT}, &
{\tt RDNAME}, & {\tt RDNEXT}, & {\tt RDSKPB}, & {\tt RDUPCH}, \\
{\tt RDUPST}
\end{tabular} \\
\Usage
\begin{verbatim}
    CALL RDLOAD(STRING)
\end{verbatim}
Puts {\tt STRING} (of type {\tt CHARACTER*(*)})
into the internal buffer of {\tt RDWORD} and sets the
internal pointer to the first position of the buffer. If {\tt STRING} is
shorter than the buffer length the rest of the buffer is filled with
blanks. If {\tt STRING} is longer than the buffer (80 characters), the
remainder of it will be truncated.
\newpage
\begin{verbatim}
    CALL RDWORD(INUM,FPNUM,NAME,KTYPE)
\end{verbatim}
Causes the next object (if any) to be read from the internal
buffer starting from the current position of the internal pointer to
the buffer while skipping leading blanks until the first non-blank
character or the end of the buffer is encountered, and positions the
internal pointer to the terminating character of the object.
\par
On exit, the arguments have the following values: \\
{\bf When an integer or floating point number was read:}
\begin{DLtt}{12345678}
\item[INUM] ({\tt INTEGER}) The integer or the integer part of the
floating point number read.
\item[FPNUM] ({\tt REAL}) The number read.
\item[NAME] ({\tt CHARACTER*(*)}) Filled with blanks.
\item[KTYPE] ({\tt INTEGER}) Type of object read: \\
$\mathtt{= 1:}$ Floating point number; \\
$\mathtt{= 2:}$ Integer number.
\end{DLtt}
Reading and conversion by {\tt RDWORD} takes place from an internal
file. {\tt I}-format is used to read integers and {\tt F}-format is used
to read floating point numbers (this format can handle an exponent on
input). Consequently, in the case of wrong input format, an error
message may be issued by a Fortran system input routine.
\par
{\bf When a name or special character was read:}
\begin{DLtt}{12345678}
\item[INUM] ({\tt INTEGER}) Zero.
\item[FPNUM] ({\tt REAL}) Zero.
\item[NAME] ({\tt CHARACTER*(*)}) The name in UPPERCASE (for
alphabetic) characters or the special character, left justified and
filled with blanks.
\item[KTYPE] ({\tt INTEGER}) Type of object read: \\
$\mathtt{= \ 0:}$ Special character; \\
$\mathtt{= -1:}$ Name.
\end{DLtt}
If the name read is longer than allowed by {\tt NAME}, it is truncated
and an error message
\begin{verbatim}
    ERROR IN RDWORD: NAME READ IS TOO LONG. IT IS TRUNCATED TO: NTRUNC
\end{verbatim}
is printed. {\tt NTRUNC} stands for the truncated name.
\par
{\bf When an "end of buffer" is reached:}
\begin{DLtt}{12345678}
\item[INUM] ({\tt INTEGER}) Zero.
\item[FPNUM] ({\tt REAL}) Zero.
\item[NAME] ({\tt CHARACTER*(*)}) Filled with blanks.
\item[KTYPE] ({\tt INTEGER}) $\mathtt{= -100}$.
\end{DLtt}
\begin{verbatim}
    CALL RDSKIP
\end{verbatim}
Sets the internal pointer to 81 which means "end of buffer".
\begin{verbatim}
    CALL RDMODE(I)
\end{verbatim}
A call with $\mathtt{I=2}$ will force {\tt RDWORD} to read, as variables of
type {\tt CHARACTER}, sets of alphanumeric characters starting with a
letter, such as {\tt ALPHA99}.
\par
The default mode (purely alphabetic names that will be terminated by a
digit will be restored by a call with $\mathtt{I=1}$. \\
In any expression of type {\tt LOGICAL}
\begin{verbatim}
    RDENDB()
\end{verbatim}
has the value {\tt .TRUE.} if the end of the buffer was reached, and
{\tt .FALSE.} otherwise. {\tt RDENDB} is of type {\tt LOGICAL}.
\\ $\bullet$
