\Version {GETBYT}                      \Routid{M439}
\Keywords{BIT RETRIEVE SET STRING}
\Author{T. Lindel\"of, R. Matthews, A. Shevel} \Library{KERNLIB}
\Submitter{T. Lindel\"of}                  \Submitted{01.07.1979}
\Language{Assembler}                   %  %\Revised{}
\Cernhead {Set or Retrieve a Bit String}
{\tt GETBYT} extracts and right-adjusts a group of bits of any length up
to a full word from a bit string which may extend across word
boundaries. {\tt SETBYT} is the inverse of {\tt GETBYT}.
\Structure
{\tt SUBROUTINE} subprogram \\
User Entry Names: \Rdef{GETBYT}, \Rdef{SETBYT}\\
Internal Entry Names: {\tt SHRERR}\\
Files Referenced: Printer
\Usage
\begin{verbatim}
    CALL GETBYT(ADDR,IBEG,ILEN,IRES)
\end{verbatim}
\begin{DLtt}{123456}
\item [ADDR] Name of an array containing the desired group of bits.
\item [IBEG] The bit position within {\tt ADDR} of the
left-most bit of the group (bits are numbered starting at 1 with the
left-most or most significant bit in {\tt ADDR(1)}).
\item [ILEN] Length of the group in bits (at most one word).
\item [IRES] Will contain the desired group, right-justified and
zero-filled.
\end{DLtt}
\begin{verbatim}
    CALL SETBYT(ADDR,IBEG,ILEN,IBYT)
\end{verbatim}
causes the {\tt ILEN} right-most bits of {\tt IBYT} to replace the
group of bits of length {\tt ILEN} starting at the {\tt IBEG}-th bit
in the array {\tt ADDR} (bits are numbered starting at 1 with the
left-most or most significant bit in {\tt ADDR(1)}). Replacement goes
across word boundaries, i.e. the most significant (left-most) bit of
{\tt ADDR(N+1)} is adjacent to the least significant (right-most) bit
of {\tt ADDR(N)}.
\Errorh
Calling either {\tt GETBYT} or {\tt SETBYT} with $\mathtt{IBEG < 1}$
or $\mathtt{ILEN > }$ the number of bits in one word (errors) will result
in a diagnostic message. After more than 20 such errors the job will
come to a {\tt STOP}.
\Examples
IBM: \\
If {\tt ADDR(1)} and {\tt ADDR(2)} contain the
32-bit configurations {\tt '0...001110001'} and {\tt '110100...0'}
respectively, then
\begin{verbatim}
    CALL GETBYT(ADDR,27,10,IRES)
\end{verbatim}
will set {\tt IRES} to {\tt '0...001100011101'} or decimal {\tt 797}.
\par
If {\tt IBYT} contains the integer value 3 (binary {\tt '11'})
and $\mathtt{ADDR(1)=ADDR(2)=0}$, then
\begin{verbatim}
    CALL SETBYT(ADDR,32,2,IBYT)
\end{verbatim}
will set {\tt ADDR(1)} to {\tt 0...001'} and {\tt ADDR(2)} to
{\tt '100...0'}.
\\ $\bullet$
