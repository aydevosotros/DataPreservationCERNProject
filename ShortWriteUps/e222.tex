% 05.10.94 ksk
\Version{RCHEBN}                           \Routid{E222}
\Keywords{LINEAR SYSTEM CHEBYCHEV NORM OVERDETERMINED}
\Authors{K.S. K\"olbig}    \Library{MATHLIB}
\Submitter{}              \Submitted{01.12.1994}
\Language{Fortran}                    %\Revised{}
\Cernhead {Solution of Overdetermined Linear System in the Chebychev
Norm}
Subroutine subprograms {\tt RCHEBN} and {\tt DCHEBN} find the
Chebyshev or minimax solution to a set of overdetermined linear equations
$\mathbf{Ax=b}$, i.e. the vector {\bf x} which minimizes
$$ c \ = \ \max_{1\leq i\leq m} c_i \ = \ \max_{1\leq i\leq m}
\left| b_i - \sum_{j=1}^n a_{ij}x_j \right|.$$
\par
On computers other than CDC or Cray, only the double-precision
version {\tt DCHEBN} is available. On CDC and Cray computers, only
the single-precision version {\tt RCHEBN} is available.
\Structure
{\tt SUBROUTINE} subprograms \\
User Entry Names: \Rdef{RCHEBN}, \Rdef{DCHEBN} \\
External References:
\begin{htmlonly}
\begin{tabular}{lllll}
\end{htmlonly}
\begin{latexonly}
\begin{tabular}[t]{l*{5}{@{\hspace{4pt}}l}}
\end{latexonly}
\Rind{RVSCA}{F002},  & \Rind{RVSCL}{F002},  & \Rind{RVSCS}{F002},  
& \Rind{RVSET}{F002},  & \Rind{RVXCH}{F002}, \\
\Rind{DVSCA}{F002},  & \Rind{DVSCL}{F002},  & \Rind{DVSCS}{F002},  
& \Rind{DVSET}{F002},  & \Rind{DVXCH}{F002} \\
\end{tabular}
\Usage
For $\mathtt{t=R}$ (type {\tt REAL}), $\mathtt{t=D}$ (type
{\tt DOUBLE PRECISION}),
\begin{verbatim}
    CALL tCHEBN(M,N,A,MDIM,B,TOL,RELERR,X,RESMAX,IRANK,ITER,ICODE)
\end{verbatim}
\begin{DLtt}{12345678}
\item[M] ({\tt INTEGER}) Number $m$ of equations.
\item[N] ({\tt INTEGER}) Number $n\,(\le m)$ of unknowns.
\item[A] (type according to {\tt t})
Two-dimensional array of dimension {\tt (MDIM,d)},
where $\mathtt{d} \ge n+3$. On entry, {\tt A(I,J)} must contain the
coefficients $a_{ij}\,(i=1,\ldots,m;\,j=1,\ldots,n)$ of matrix {\bf A}.
The contents of {\tt A} is destroyed during execution.
\item[MDIM] ({\tt INTEGER}) Declared first dimension of array {\tt A},
where $\mathtt{MDIM} \ge m+1$.
\item[B] (type according to {\tt t})
One-dimensional array of length $\ge m+1$.
On entry, the first $m$ elements of {\tt B} must contain the vector
{\bf b}. On exit, these elements contain the residuals $c_i$.
\item[TOL] Tolerance parameter which should be set to a
value somewhat greater than the machine precision.
\item[RELERR] (type according to {\tt t}) On entry, {\tt RELERR}
should be set to zero if the true minimax solution is required.
(For {\tt RELERR} non-zero see {\bf Notes}).
\item[X] (type according to {\tt t}) One-dimensional array of length
$\ge n+3$. On exit, the first $n$ elements of {\tt X} contain the
solution vector {\bf x}.
\item [RESMAX] (type according to {\tt t}) On exit, {\tt RESMAX}
contains the value $c$ of the maximum residual.
\item [IRANK]({\tt INTEGER}) On exit, {\tt IRANK} contains an estimate of
the rank of the matrix {\tt A}. (This estimate may depend on {\tt TOL)}.
\item [ITER]({\tt INTEGER}) On exit, {\tt ITER} contains the number of
simplex iterations performed.
\item [ICODE]({\tt INTEGER}) On exit, {\tt ICODE} contains one of the
following: \\
$\mathtt{= 0:}$ Solution {\bf x} is not unique, \\
$\mathtt{= 1:}$ Solution {\bf x} is unique,\\
$\mathtt{= 2:}$ Calculation terminated prematurely because of rounding
error.
\end{DLtt}
\newpage
\Method
Modified simplex method of linear programming applied to the dual of
the stated minimax problem.
\Notes
\begin{enumerate}
\item  If {\tt RELERR} on entry contains a non-zero positive value $r$,
{\tt RELERR} on exit contains a value $r'<r$,
and the computed solution {\bf x}$'$ in {\tt X} and the maximum residual
$c'$ in {\tt RESMAX} are such that $(c'-c)/c < r'$,
where $c$ is the maximum residual corresponding to the true minimax
solution {\bf x}.  By setting {\tt RELERR} non-zero (e.g.
{\tt RELERR = 0.1}), the number of simplex iterations is usually reduced.
\item If {\tt RESMAX} is within one or two orders of magnitude of
{\tt TOL}, the computed residuals in {\tt B}
on exit may contain few significant digits, and may have been set to
zero if $\mathtt{RESMAX < TOL}$.
\end{enumerate}
\Source
The subprograms are based on a Fortran algorithm given in Ref. 1.
\Refer
\begin{enumerate}
\item I. Barrodale and C. Phillips, Algorithm 495:
Solution of an overdetermined system of linear equations in the
Chebyshev norm, ACM Trans. Math. Software {\bf 1} (1975) 264--270.
\end{enumerate}
$\bullet$
