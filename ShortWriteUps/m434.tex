% 27 May 1992 mg
\Version{VXINV}                          \Routid{M434}
\Keywords{BYTE FAST INVERSION VAX}
\Author{F. Carminati, M. Jonker, J. Zoll} \Library{KERNLIB, VAX and DECSTATION only}
\Submitter{}                                  \Submitted{05.10.1987}
\Language{Fortran or Assembler}      %\Revised{}
\Cernhead {Fast VAX Byte Inversion}
These routines do VAX byte inversions {\tt 1-2-3-4} to {\tt 4-3-2-1}
in each word of an array, either in-place or copied.
\Structure
{\tt SUBROUTINE} subprogram \\
User Entry Names: \Rdef{VXINVB}, \Rdef{VXINVC}
\Usage
\begin{verbatim}
    CALL VXINVB(IXV,N)
\end{verbatim}
inverts four bytes in each of the {\tt N words}
at array {\tt IXV}, in-place.
\begin{verbatim}
    CALL VXINVC(IV,IXV,N)
\end{verbatim}
copies the {\tt N} words at array {\tt IV} to array {\tt IXV}, with
the bytes inverted in each word.
\par
On DEC machines  bytes read from a disk file are
loaded in memory in reverse order. One of the above routines,
applied to the result of a binary read from a disk file, causes
the bytes to be stored in each 32 bits word in the same order than
in the disk file. This is useful when reading a binary file
transferred through a network from a foreign system, in order to
preserve the order of the bytes in each 32 bits word.
Please note that several network utilities include the possibility
to perform a bytes inversion in the network protocol. Note also
that when reading or writing from a magnetic tape, the bytes may be
swapped in pairs and not in groups of 4.
\\ $\bullet$
