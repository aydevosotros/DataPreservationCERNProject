\Version{RTHETA}                     \Routid{C349}
\Keywords{JACOBIAN THETA FUNCTION}
\Author{G.A. Erskine}                  \Library{MATHLIB}
\Submitter{K.S. K\"olbig}                \Submitted{07.06.1992}
\Language{Fortran}                      \Revised{}
\Cernhead{Jacobian Theta Functions}
Function subprograms {\tt RTHETA} and {\tt DTHETA} calculate the
Jacobian theta functions
$$ \begin{array}{rcl}
\vartheta_0(x,q) & = & 1 + 2 \displaystyle \sum_{n=1}^\infty
(-1)^n q^{n^2} \cos 2n \pi x, \\[6mm]
\vartheta_1(x,q) & = & 2 \displaystyle \sum_{n=0}^\infty
(-1)^n q^{{\left( n+\frac{1}{2} \right)}^2} \sin (2n+1) \pi x, \\[6mm]
\vartheta_2(x,q) & = & 2 \displaystyle \sum_{n=0}^\infty
q^{{\left( n+\frac{1}{2} \right)}^2} \cos (2n+1) \pi x, \\[6mm]
\vartheta_3(x,q) & = & 1 + 2 \displaystyle \sum_{n=1}^\infty
q^{n^2} \cos 2n \pi x, \\[6mm]
\vartheta_4(x,q) & = & \vartheta_0(x,q),
\end{array} $$
for real arguments $x$ and $0 \le q<1$. $\vartheta_1(x+\frac{1}{2},1)$
and $\vartheta_2(x,1)$ are undefined if $x$ is an integer; otherwise
$\vartheta_k(x,1)=0,\, k=1,2,3,4$.
\par
Note that several conflicting definitions of these functions occur in
the literature. In particular, the argument in the trigonometric terms
is often defined to be $x$ instead of $\pi x$.
\par
On CDC and Cray computers, the double-precision version {\tt DTHETA}
is not available.
\Structure
{\tt FUNCTION} subprogram \\
User Entry Names: \Rdef{RTHETA}, \Rdef{DTHETA} \\
Files Referenced: {\tt Unit 6} \\
External References: \Rind{MTLMTR}{N002}, \Rind{ABEND}{Z035}
\Usage
In any arithmetic expression,
\begin{center}
{\tt RTHETA(K,X,Q)} \quad or \quad {\tt DTHETA(K,X,Q)} \quad
has the value  \quad $\vartheta_{\mathtt{K}}(\mathtt{X,Q})$,
\end{center}
where {\tt RTHETA} is of type {\tt REAL}, {\tt DTHETA} is of type
{\tt DOUBLE PRECISION}, {\tt X} and {\tt Q} are of the same
type as the function name, and {\tt K} is of type {\tt INTEGER}.
\Method
If $t\,(0 \le t \le \frac{1}{2})$ differs from $x$ or $-x$ by an
integer, it follows from the periodicity and symmetry properties of the
functions that $\vartheta_1(x,q)=\pm \vartheta_1(t,q)$ and
$\vartheta_3(x,q)=\vartheta_3(t,q)$. In a region for which the
approximation is sufficiently accurate, $\vartheta_1$ is set equal to
the first $(n=0)$ term of the transformed series
$$ \vartheta_1(t,q) \ = \ 2(\lambda/\pi)^{1/2} e^{-\lambda t^2}
\displaystyle \sum_{n=0}^\infty (-1)^n e^{-\lambda(n+\frac{1}{2})^2}
\sinh (2n+1)\lambda t, $$
and $\vartheta_3$ is set equal to the first two (i.e. $n \le 1$)
terms of
$$ \vartheta_3(t,q) \ = \ (\lambda/\pi)^{1/2} e^{-\lambda t^2}
\displaystyle \left( 1+2 \sum_{n=1}^\infty
e^{-\lambda n^2} \cosh 2n\lambda t \right), $$
where $\lambda = \pi^2/|\ln q|$. Otherwise the trigonometric series
for $\vartheta_1(t,q)$ and $\vartheta_3(t,q)$ are used.
\par
For all $x$, $\vartheta_0$ and $\vartheta_2$ are computed from
$\vartheta_0(x,q)=\vartheta_3(\frac{1}{2}-|x|,q)$,
$\vartheta_2(x,q)=\vartheta_1(\frac{1}{2}-|x|,q)$.
\Restrict
1. \quad $\mathtt{0 \le Q \le 1}$. \\
2. \quad $\mathtt{K = 0,1,2,3,4}$. \\
3. \quad \parbox[t]{90mm}{If $\mathtt{Q = 1}$ and $\mathtt{K = 1}$,
$\mathtt{X-}\frac{1}{2}$ must not be an integer. \\
If $\mathtt{Q = 1}$ and $\mathtt{K = 2}$,
$\mathtt{X}$ must not be an integer.}
\Errorh
Error {\tt C349.1:} Restriction 1 is not satisfied. \\
Error {\tt C349.2:} Restriction 2 is not satisfied. \\
Error {\tt C349.3:} Restriction 3 is not satisfied. \\
In all cases, the function value is set equal to zero, and a message
is written on {\tt Unit 6}, unless subroutine {\tt MTLSET} (N002)
has been called.
\Accuracy
For {\tt DTHETA} (and for {\tt RTHETA} on CDC and Cray computers),
the error when {\tt Q} is less than approximately {\tt 0.9} does not
exceed two decimal digits in the last place. For larger values of
{\tt Q} (provided the computed result is non-zero), the error is at
worst comparable in magnitude to the mathematical error which would be
caused by one-bit rounding errors in the arguments {\tt X} and {\tt Q}.
\par
On computers other than CDC and Cray, non-zero values of {\tt RTHETA}
have full machine accuracy.
\Notes
Successive references using the same value of {\tt Q} are executed
faster than those in which {\tt Q} changes.
\par
Many functional relations, including relations between the theta
functions and the Jacobian elliptic functions, are given in Refs. 1--4.
\Refer
\begin{enumerate}
\item W. Magnus, F. Oberhettinger and R.P. Soni, Formulas and theorems
for the special functions of mathematical physics, Springer-Verlag
Berlin (1966) 371--377.
\item F. T\"olke, Praktische Funktionenlehre, Bd. II, Springer-Verlag
Berlin (1966) 1--38.
\item P.F. Byrd and M.D. Friedman, Handbook of elliptic integrals for
engineers and scientists, 2nd Edition, Springer-Verlag Berlin (1971)
315--320.
\item E.T. Whittaker and G.N. Watson, A course of modern analysis,
4th Edition, Cambridge University Press, Cambridge (1946) Chapter 21.
\end{enumerate}
$\bullet$
