\Version{RRIZET}                       \Routid{C315}
\Keywords{RIEMANN ZETA FUNCTION}
\Author{K.S. K\"olbig}                 \Library{MATHLIB}
\Submitter{}                           \Submitted{07.06.1992}
\Language{Fortran}        %         \Revised{ }
\Cernhead{Riemann Zeta Function}
Function subprograms {\tt RRIZET} and {\tt DRIZET}
calculate the Riemann zeta function
$$ \zeta(x) \ = \ \displaystyle \sum_{k=1}^\infty \, k^{-x} \ = \
\displaystyle \frac{1}{\Gamma(x)} \int_0^\infty
\frac{t^{x-1}}{e^{t}-1} dt \qquad (x > 1) $$
for real arguments $x \neq 1$, where $\zeta(x)$ is defined by analytic
continuation for $x < 1$. For $x = 1$, $\zeta(x)$ has a pole of order
one.
\par
On CDC and Cray computers, the double-precision version
{\tt DRIZET} is not available.
\Structure
{\tt FUNCTION} subprograms\\
User Entry Names: \Rdef{RRIZET}, \Rdef{DRIZET}\\
Files Referenced: {\tt Unit 6} \\
External References: \Rind{GAMMA}{C302}, \Rind{DGAMMA}{C302},
\Rind{MTLMTR}{N002}, \Rind{ABEND}{Z035}
\Usage
In any arithmetic expression,
\begin{center}
{\tt RRIZET(X)} \quad or \quad {\tt DRIZET(X)} \quad
\end{center}
has the value $\zeta(\mathtt{X})$ if $\mathtt{X < 1}$, and
$\zeta(\mathtt{X})-1$ if $\mathtt{X > 1}$,
where {\tt RRIZET} is of type {\tt REAL}, {\tt DRIZET} is of type
{\tt DOUBLE PRECISION}, and where {\tt X} has the same type as the
function name.
\Method
Rational Chebyshev approximation. For $x < \frac{1}{2}$ the
reflection formula for $\zeta(x)$ is used.
\Accuracy
{\tt RRIZET} (except on CDC and Cray computers)
has full single-precision accuracy.
For most values of the argument {\tt X}, {\tt DRIZET}
(and {\tt RRIZET} on CDC and Cray computers) has an accuracy of
approximately one significant digit less than the machine precision.
\Errorh
Error {\tt C315.1:} $\mathtt{X = 1}$. The function value is set
to zero, and a message is written on {\tt Unit 6},
unless subroutine {\tt MTLSET} (N002) has been called.
\Refer
\begin{enumerate}
\item W.J. Cody, K.E. Hillstrom, and H.C. Thather, Jr., Chebyshev
approximations for the \\
Riemann zeta function, Math. Comp. {\bf 25} (1971) 537--547.
\end{enumerate}
$\bullet$
