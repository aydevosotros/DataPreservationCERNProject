 \Version{TRAPER}                              \Routid{D108}
\Keywords{NUMERICAL INTEGRATION TRAPEZOIDAL RULE ESTIMATION ERROR}
\Author{K.S. K\"olbig}                        \Library{MATHLIB}
\Submitter{}                                  \Submitted{01.03.1968}
\Language{Fortran}                          %\Revised{}
\Cernhead{Trapezoidal Rule Integration with an Estimated Error}
Let a function $f(x)$ be given by its values at certain discrete points
$x_\nu (\nu =1,2,\ldots,n)$.
Let the function values $y_\nu$ be accompanied by an estimated
standard deviation $\varepsilon_\nu$ (square root of the variance).
Subroutine subprogram {\tt TRAPER} then approximates the integral
$$ I \ = \ \displaystyle \int_A^B f(x)dx \ \simeq \
\sum_\nu w_\nu y_\nu $$
by a linear combination of the $y_\nu$ using the trapezoidal rule.
It calculates the standard deviation $\sigma$ of $I$ by
$$ \sigma \ = \ \sqrt{\sum_\nu w_\nu^2\varepsilon_\nu^2}.$$
The function values  $f(A)$ and  $f(B)$ are
calculated by linear interpolation.
\Structure
{\tt SUBROUTINE} subprogram  \\
User Entry Names: \Rdef{TRAPER}
\Usage
\begin{verbatim}
    CALL TRAPER(X,Y,E,N,A,B,RE,SD)
\end{verbatim}
\begin{DLtt}{123456}
\item[X,Y,E] ({\tt REAL}) Arrays of length $\geq n $ containing
$ x_\nu,y_\nu,\varepsilon_\nu $, respectively.
\item[N]({\tt INTEGER}) Number of function values
\item[A,B]({\tt REAL})  Limits of integration.
\item[RE]({\tt REAL}) On exit, {\tt RE} contains an approximate value
of the integral $I$.
\item[SD]({\tt REAL}) On exit, {\tt SD} contains an approximate value of
the standard deviation $\sigma$.
\end{DLtt}
If no $\varepsilon_\nu$ are given, the array {\tt E} should be
filled with zeros.
\Restrict
Although there are no restrictions on {\tt A} and {\tt B}
({\tt B} may be less than {\tt A}), care must be taken if one or
both of them is either smaller than {\tt X(1)} or bigger than {\tt X(N)}.
In these cases $f(\mathtt{A})$ or $f(\mathtt{B})$ are extrapolated
linearly
from {\tt Y(1)} and {\tt Y(2)} or {\tt Y(N-1)} and {\tt Y(N)}
respectively, which may lead to unreasonable results.
If $\mathtt{A = B}$ or $\mathtt{N < 2}$, {\tt RE} and {\tt SD}
will be set to zero. It is assumed that all the $x_\nu$ are distinct.
No test is made for this.
\Notes
This program should only be used for the problem
described above. For general-purpose numerical integration to a
preassigned accuracy use {\tt GAUSS} (D103).
\\ $\bullet$
