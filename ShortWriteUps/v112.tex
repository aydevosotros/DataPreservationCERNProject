\Version {MUNOMI}                       \Routid{V112}
\Keywords{DISTRIBUTION MULTINOMIAL NUMBER RANDOM}
\Author{D. Drijard}                     \Library{MATHLIB}
\Submitter{}                        \Submitted{15.09.1978}
\Language{Fortran}                      %\Revised{}
\Cernhead{Multinomial Random Numbers}
{\tt MUNOMI} generates a vector of random integers
$n_i \, (i=1,2,\ldots,N^*)$ with probabilities $p_i$
according to a multinomial law:
$$ Prob(n_1,n_2,\ldots,n_{N^*}) \ = \
\left( \sum_{i=1}^{N^*} n_i! \right)  \prod_{i=1}^{N^*}
\left( \frac {p_i^{n_{i}}} {n_i!} \right).$$
\Structure
{\tt SUBROUTINE} subprogram \\
User Entry Names: \Rdef{MUNOMI}\\
External References: \Rind {RNDM}(V104), \Rind{UBLANK}(V300)
\Usage
\begin{verbatim}
    CALL MUNOMI(NCH,NTOT,P,N,IERR)
\end{verbatim}
\begin{DLtt}{123456}
\item[NCH] ({\tt INTEGER}) Number $N^*$ of random integers $n_i$
requested.
\item[NTOT] ({\tt INTEGER}) Equals $\sum_{i=1}^{N^*} n_i$,
specified by the user.
\item[P] ({\tt REAL}) One-dimensional array of length {\tt NCH} at least.
On entry, it contains in {\tt P(i)} the probability of channel {\tt i}.
On return, it contains the cumulative channel probabilities so that
$\mathtt{P(NCH)=1}$. If $\mathtt{P(NCH)=1}$ on entry, it is assumed that
{\tt P(i)} contains the cumulative probabilities rather than the
individual probabilites, which saves some time.
\item[N] ({\tt INTEGER}) One-dimensional array of length {\tt NCH} at
least. On return, {\tt N(i)} contains the generated random integers.
\item[IERR] Error flag. \\
$\mathtt{= 0:}$ Normal case, \\
$\mathtt{= 1:}$ At least one $\mathtt{P(i) < 0}$, \\
$\mathtt{= 2:}$ $\sum \mathtt{P(i) > 1}$.
\end{DLtt}
\Notes
{\tt MUNOMI} is very slow for large values of {\tt NCH} or {\tt NTOT}.
For $\mathtt{NCH=2}$, use {\tt BINOMI} (V111).
\Source
Los Alamos report LA-5061-MS.
\\ $\bullet$
