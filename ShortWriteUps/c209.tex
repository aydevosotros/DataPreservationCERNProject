% 19 oct 94  ksk
\Version{CPOLYZ}                 \Routid{C209}
\Keywords{ZERO COMPLEX POLYNOMIAL SOLUTION ROOT}
\Author{T. Pomentale}             \Library{MATHLIB}
\Submitter{K.S. K\"olbig}        \Submitted{07.06.1992}
\Language{Fortran}          %    \Revised{ }
\Cernhead{Zeros of a Complex Polynomial}
Subroutine subprograms {\tt CPOLYZ} and {\tt WPOLYZ}
compute the zeros of the polynominal
$$ P(z)=c_0z^n+c_1z^{n-1}+\cdots+c_{n-1}z+c_n$$
of degree $n$ with complex coefficients $c_k$ and $c_0 \ne 0$.
\par
On computers other than CDC or Cray, only
the double-precision version {\tt WPOLYZ} is available.
On CDC and  Cray computers, only the single-precision version
{\tt CPOLYZ} is available.
\Structure
{\tt SUBROUTINE} subprograms \\
User  Entry  Names: \Rdef{CPOLYZ}, \Rdef{WPOLYZ} \\
Files  Referenced:  {\tt Unit 6} \\
External References: \Rind{MTLMTR}{N002}, \Rind{ABEND}{Z035}
\Usage
For $\mathtt{t=C}$ (type {\tt COMPLEX}), $\mathtt{t=W}$ (type
{\tt COMPLEX*16}),
\begin{verbatim}
    CALL tPOLYZ(C,N,MAXIT,Z,R)
\end{verbatim}
\begin{DLtt}{12345}
\item[C] (type according to {\tt t}) One-dimensional array of
dimension {\tt (0:d)}, where $\mathtt{d \ge N}$, containing the
coefficients $c_k, (k = 0,1,\ldots,n)$.
\item[N] ({\tt INTEGER}) The degree $n$.
\item[MAXIT] ({\tt INTEGER}) The maximum number of iterations permitted.
\item[Z] (type according to {\tt t}) One-dimensional array of length
$\mathtt{\ge N}$. On entry, $\mathtt{Z(1),\ldots,Z(N)}$ must contain
starting approximations for the zeros $z_i$. If no starting
approximations are available, the $\mathtt{Z(i)}$ should be set
to zero. On exit, $\mathtt{Z(i)}$ contains an approximation to the zero
$z_i$.
\item[R] ({\tt REAL} for $\mathtt{t=C}$, {\tt DOUBLE PRECISION} for
$\mathtt{t=W}$) One-dimensional array of dimension $\ge \mathtt{N}$.
On exit, $\mathtt{R(1),\ldots,R(N)}$ contain an estimated radius $r_i$
of a circle centered at $\mathtt{Z(i)}$ within which the true zero
$z_i$ is expected to lie.
\end{DLtt}
\Notes
Note that, because of accumulation of rounding errors, unreliable
results can be obtained for large $n$ even for well-conditioned
polynomials.
\Errorh
Error {\tt C209.1:} $c_0=0$. \\
Error {\tt C209.2:} The number of iterations exceeds {\tt MAXIT}. \\
Error {\tt C209.3:} An estimated radius $r_i$ cannot be computed for a
certain value of $i$. \\
In all cases, a message is written on {\tt Unit 6},
unless subroutine {\tt MTLSET} (N002) has been called.
\Refer
\begin{enumerate}
\item T. Pomentale, Homotopy iterative methods for
polynomial equations, J. Inst. Maths. Applics. {\bf 13} (1974) 201--213.
\end{enumerate}
$\bullet$
