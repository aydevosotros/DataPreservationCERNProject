% 12 may 1995 ksk
\Version {RNBNML}                             \Routid{V137}
\Keywords{DISTRIBUTION BINOMIAL NUMBER RANDOM}
\Author{D. Drijard, K.S. K\"olbig}                \Library{MATHLIB}
\Submitter{}                               \Submitted{15.10.1994}
\Language{Fortran}                   %\Revised{}
\Cernhead {Binomial Random Numbers}
Subroutine subprogram
{\tt RNBNML} generates a random integer $N>0$ according to the
binomial distribution
$$ Prob(N=n) \ = \ \binom{M}{n} P^n\,(1-P)^{M-n} $$
where the 'sample size' $M>0$ and the probability
$P$ ($0 \le P \le 1$) are specified by the user.
\Structure
{\tt SUBROUTINE} subprogram \\
User Entry Names: \Rdef{RNBNML}\\
External References: \Rind{RANLUX}{V115}
\Usage
\begin{verbatim}
    CALL RNBNML(M,P,N,IERR)
\end{verbatim}
\begin{DLtt}{123456}
\item [M] ({\tt INTEGER}) Sample size $M$.
\item [P] ({\tt REAL}) Probability $P$.
\item [N]({\tt INTEGER}) The generated random number $N$, binomially
distributed in the interval $0 \le N \le M$ with mean
$P \times M$.
\item [IERR]({\tt INTEGER}) Error flag. \\
$\mathtt{= 0:}$ Normal case, \\
$\mathtt{= 1:}$ $\mathtt{P \le 0}$ or $\mathtt{P \ge 1}$.
\end{DLtt}
\Notes
{\tt RNBNML} should not be used when {\tt M} is 'large' (say
$ >100 $). The normal approximation is then recommended instead (with
mean $\mathtt{P*M+0.5}$ and standard deviation
$\sqrt{\mathtt{M*P*(1-P)}}$).
\\ $\bullet$
