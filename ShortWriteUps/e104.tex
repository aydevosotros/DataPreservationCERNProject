\Version{FINT}                  \Routid{E104}
\Keywords{MULTIDIMENSIONAL LINEAR INTERPOLATION}
\Author{C. Letertre}            \Library{KERNLIB}
\Submitter{B. Schorr}           \Submitted{17.05.1971}
\Language{Fortran}              \Revised{27.11.1984}
\Cernhead{Multidimensional Linear Interpolation}
Function subprogram {\tt FINT} uses repeated linear interpolation to
evaluate a
function $ f(x_1,x_2,\ldots,x_n$) of $n$ variables which has been
tabulated at the nodes of an $n$-dimensional rectangular grid. It is
not necessary that the table arguments corresponding to any
coordinate $ x_i$ be equally spaced.
\Structure
{\tt FUNCTION} subprogram \\
User Entry Names: \Rdef{FINT}\\
Files Refernced: Printer\\
External References: \Rind{KERMTR}{N001}, \Rind{ABEND}{Z035}
\Usage
In any arithmetic expression,
\begin{center}
{\tt FINT(N,X,NA,A,F)}
\end{center}
has an approximate value of $f(\mathtt{X_1,X_2,\ldots,X_n})$.
\begin{DLtt}{1234}
\item [N] ({\tt INTEGER}) Number of coordinates $n$ required
to define the function $f$.
\item [X]({\tt REAL}) One-dimensional array.
$\mathtt{X(j)},\,(j=1,2,\ldots,\mathtt{N})$, must contain the coordinates
of the point at which the interpolation is to be performed.
\item [NA] ({\tt INTEGER}) One-dimensional array.
For $j=1,2,\ldots,\mathtt{N,\,NA(j)}$ must be equal to the number
of numerical values of variable $x_j$ which are stored in array {\tt A}.
\item [A]({\tt REAL}) One-dimensional array of length not less than
the sum of $\mathtt{NA(1),\ldots,NA(N)}$. The first {\tt NA(1)}
elements of {\tt A} must
contain numerical values $a_{11}, a_{12},\ldots$ of the first variable
$x_1$ in strictly increasing order, the next {\tt NA(2)} elements of
{\tt A} must contain numerical values $a_{21}, a_{22},\ldots$ of the
second variable $x_2$ in strictly increasing order, and so on.
\item [F]({\tt REAL}) Multidimensional array with dimensions {\tt NA(1),
NA(2)}, \ldots,{\tt NA(N)}, containing values of the function $f$ at the
nodes of the rectangular grid defined by {\tt A}:\\
$\mathtt{F(i,j,\ldots,m)} = f(a_{1{\tt i}},a_{2{\tt j}},\ldots,
a_{n\mathtt{m}}),(i=1,2,\ldots,\mathtt{NA(1)},\ldots;m=1,2,\ldots,
\mathtt{NA(N))}$.
\end{DLtt}
If any coordinate $x_i$ of the given point $X$ lies
outside the range of the corresponding table arguments, the
interpolation for this coordinate is replaced by an extrapolation
based on the two nearest table arguments, with consequent loss of
accuracy.
\Method
Repeated linear interpolation with respect to variables $x_1,
x_2,\ldots$ within the grid cell which contains the given point $X$. For
$n=2$, with $(x_1,x_2)$ replaced by $(x,y)$ for clarity, the
procedure is equivalent to the following:
\par
Let $a_1,a_2,\ldots$ be the tabulated values of $x$.
Let $b_1,b_2,\ldots$ be the tabulated values of $y$. \\
Let $i$ and $j$ be the subscripts for which
$a_i\leq x<a_{i+1}, b_j\leq y <b_{j+1}$. \\
Then compute:
\begin{eqnarray*}
t & = & (x-a)/(a_{i+1}-a),\\
g_j & = & (1-t)f(a_i,b_j)+t f(a_{i+1},b_j),\\
g_{j+1} & = & (1-t)f(a_i,b_{j+1}+t f(a_{i+1},b_{j+1}),\\
u & = &  (y-b)/(b_{j+1}-b),\\
f_{appr}  & = &  (1-u) g_j + u g_{j+1}
\end{eqnarray*}
\Restrict
\begin{enumerate}
\item  $\mathtt{1\leq N \leq 5}$. {\tt FINT} is set equal to zero if
{\tt N} is not in this range.
\item  $\mathtt{NA(j) \geq 1,\,(j=1,2,\ldots,N)}.$
\item The table arguments for each variable must be in strictly
increasing order.
\end{enumerate}
There is no test for conditions 2 or 3.
\Errorh
{\tt E104.1:}  $\mathtt{N<1}$ or $\mathtt{N>5}$.
{\tt FINT} is set equal to zero, and a message is printed unless
subroutine {\tt KERSET (N001)} has been called.
\Examples
Given a function of two variables $g(x,y)$ defined by a {\tt FUNCTION}
subprogram {\tt G}, to construct a table of values of
$f_{km} = g(\sqrt{k},\log m)$ for $k=1,2,\ldots,10, m=1,2,\ldots,15$,
and to interpolate in this table to set {\tt GINT} equal to an
approximate value of $g(1.7,2.9)$. The program is written in a form which
allows generalization to functions of more than two variables.
\begin{verbatim}
      PARAMETER (NA1=10,NA2=15)
      DIMENSION X(2),NA(2),A(NA1+NA2),F(NA1,NA2)
      DATA NA/NA1,NA2/
C     STORE ARGUMENT ARRAY
      K1=0
      K2=K1+NA1
      DO 1 J = 1,MAX(NA1,NA2)
        IF (J .LE. NA1) A(J+K1)=SQRT(FLOAT(J))
        IF (J .LE. NA2) A(J+K2)=LOG(FLOAT(J))
    1 CONTINUE
C     STORE FUNCTION ARRAY
      DO 3 J1 = 1,NA1
        DO 2 J2 = 1,NA2
          F(J1,J2)=G(A(J1+K1),A(J2+K2))
    2   CONTINUE
    3 CONTINUE
C     INTERPOLATE IN TABLE
      X(1)=1.7
      X(2)=2.9
      GINT=FINT(2,X,NA,A,F)
      ...
\end{verbatim}
$\bullet$
