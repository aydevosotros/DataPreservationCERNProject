%  07 nov 94  ksk
\Version{RRTEQ3}                   \Routid{C207}
\Keywords{CUBIC EQUATION ROOT}
\Author{K.S. K\"olbig}            \Library{MATHLIB}
\Submitter{  }                    \Submitted{15.01.1988}
\Language{Fortran}                \Revised{01.12.1994}
\Cernhead{Roots of a Cubic Equation}
Subroutine subprograms {\tt RRTEQ3} and {\tt DRTEQ3}
compute the three roots of
$$ x^3 + rx^2 + sx + t = 0  \qquad     (*) $$
for real coefficients $r, s, t$.
\par
On computers other than CDC or Cray, only
the double-precision version {\tt DRTEQ3} is available.
On CDC and Cray computers, only the single-precision version
{\tt RRTEQ3} is available.
\Structure
{\tt SUBROUTINE} subprograms   \\
User Entry  Names: \Rdef{RRTEQ3}, \Rdef{DRTEQ3} \\
Obsolete User Entry Names: \Rdef{RTEQ3} $\equiv$ \Rdef{RRTEQ3} \\
\Usage
For $\mathtt{t=R}$ (type {\tt REAL}), $\mathtt{t=D}$ (type
{\tt DOUBLE PRECISION}),
\begin{verbatim}
    CALL tRTEQ3(R,S,T,X,D)
\end{verbatim}
\begin{DLtt}{123456}
\item[R,S,T] (type according to {\tt t}) Coefficients $r,s,t$ in $(*)$.
\item[X] (type according to {\tt t})
One-dimensional array of length $\ge 3$.
On exit, {\tt X} is set as described below.
\item[D] (type according to {\tt t})
On exit, {\tt D} is set to the value of the discriminant of $(*)$: \\
$\mathtt{> 0:}$ One real root {\tt X(1)} and two complex conjugate roots
$\mathtt{X(2)}+i\mathtt{X(3)}$, $\mathtt{X(2)}-i\mathtt{X(3)}$; \\
$\mathtt{= 0:}$ Three real roots {\tt X(1)}, {\tt X(2)}, {\tt X(3)},
 where at least $\mathtt{X(2)} = \mathtt{X(3)}$; \\
$\mathtt{< 0:}$ Three distinct real roots {\tt X(1)}, {\tt X(2)},
{\tt X(3)}.
\end{DLtt}
\Method
The classical method of Tartaglia-Vieta is used. In
certain cases, the solutions are improved by Newton iteration.
\Accuracy
Depends on the coefficients $r,s,t$. The values of
{\tt X(1)}, {\tt X(2)}, {\tt X(3)} and of {\tt D} may be inaccurate
if {\tt |D|} is very small, even in the case of ``exact'' coefficients.
\\ $\bullet$
