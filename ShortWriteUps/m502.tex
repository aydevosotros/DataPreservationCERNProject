% 28.09.94 ksk
\Version {UOPTC}                      \Routid{M502}
\Keywords{DECODE OPTION CHARACTER STRING SELECTION}
\Author{J. Zoll, P. Rastl}            \Library{KERNLIB}
\Submitter{}                          \Submitted{21.09.1971}
\Language{Fortran or Assembler} \Revised{16.09.1991}
\Cernhead {Decoding Options Characters}
{\tt UOPTC} and {\tt UOPT}
compare a string of {\it actual} option-characters against
a similar string of {\it possible} option-characters
filling an {\tt INTEGER} vector with {\tt 1}'s and {\tt 0}'s,
indicating for each possible option whether or not it was taken.
\Structure
{\tt SUBROUTINE} subprogram \\
User Entry Names: \Rdef{UOPTC}, \Rdef{UOPT}\\
\Usage
\begin{verbatim}
    CALL UOPTC(CHACT,CHPOSS,IOPT)
\end{verbatim}
\begin{DLtt}{123456}
\item [CHACT] ({\tt CHARACTER}) String of actual option-characters.
\item [CHPOSS] ({\tt CHARACTER}) String of possible option-characters.
\item [IOPT] ({\tt INTEGER}) Vector of at least {\tt LEN(CHPOSS)} words,
the $j$-th word of which is set to {\tt 1} or {\tt 0},
depending on whether the $j$-th possible character does or does not
occur in {\tt CHACT}.
\end{DLtt}
\begin{verbatim}
    CALL UOPT(IACT,IPOSS,IOPT,N)
\end{verbatim}
\begin{DLtt}{123456}
\item [IACT] Hollerith string of actual option-characters.
It is terminated by the first character not occuring
in the string of possibilities.
\item [IPOSS] Hollerith string of {\tt N} possible option-characters
($\mathtt{N \leq 30}$).
\item [IOPT] A vector of at least {\tt N} words,
the $j$-th word of which is set to {\tt 1} or {\tt 0},
depending on whether the $j$-th possible character
does or does not occur in the {\tt IACT} string.
\end{DLtt}
\Examples
\begin{verbatim}
    CALL UOPTC('+AB','ABC+/Y',IOPT)
    CALL UOPT (4H+AB.,6HABC+/Y,IOPT,6)
\end{verbatim}
will set the first 6 elements of {\tt IOPT} to {\tt 1,1,0,1,0,0}.
\Notes
{\tt UOPT} was written for Fortran 4 and should no longer
be used for new programs.
\\ $\bullet$
