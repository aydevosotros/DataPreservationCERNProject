%  03 nov 94  ksk
\Version{RRKSTP}                          \Routid{D200}
\Keywords{FIRST-ORDER DIFFERENTIAL EQUATION RUNGE KUTTA}
\Authors {G.A. Erskine}                 \Library{MATHLIB}
\Submitter{}                             \Submitted{01.09.1983}
\Language{Fortran}               \Revised{01.03.1994}
\Cernhead{First-order Differential Equations (Runge--Kutta)}
Subroutine subprograms {\tt RRKSTP}  and {\tt DRKSTP} advance the
solution of the system of $n \ge 1$ simultaneous first-order
differential equations
$$ \frac{dy_i}{dx} \ = \
f_i(x,y_1,\ldots,y_n), \qquad (i = 1,2,\ldots,n)$$
by a single step of length $h$ in the independent variable $x$.
\par
On CDC and Cray computers, the double-precision version {\tt DRKSTP}
is not available.
\Structure
{\tt SUBROUTINE} subprograms \\
User Entry   Names : \Rdef{RKSTP}, \Rdef{DRKSTP}\\
Obsolete User Entry Names : \Rdef{RKSTP} $\equiv$ {\tt RRKSTP} \\
Files  Referenced : {\tt Unit 6} \\
External  References: user-supplied {\tt SUBROUTINE} subprogram
\Usage
For $\mathtt{t=R}$ (type {\tt REAL}), $\mathtt{t=D}$ (type
{\tt DOUBLE PRECISION}),
\begin{verbatim}
    CALL tRKSTP(N,H,X,Y,SUB,W)
\end{verbatim}
\begin{DLtt}{123456}
\item[N] ({\tt INTEGER}) Number $n$ of equations.
\item[H] (type according to {\tt t}) The  step-length $h$.
\item[X] (type according to {\tt t}) On entry, {\tt X} must be equal
to the initial value of the independent variable $x$. On exit,
{\tt X} is equal to $x+h$.
\item[Y] (type according to {\tt t})
One-dimensional array of length $\ge $ {\tt N}.
On entry, $\mathtt{Y(i),(i=1,\ldots,N)}$, must contain $y_i(x)$.
On exit, $\mathtt{Y(i),(i=1,\ldots,N)}$, contains approximate values
 $y_i(x+h)$.
\item[SUB] Name of a user-supplied {\tt SUBROUTINE} subprogram,
declared {\tt EXTERNAL} in the calling program.
\item[W] (type according to {\tt t}) Array containing at least
{\tt 3*N} elements required as working-space.
\end{DLtt}
The user-supplied subroutine {\tt SUB} should be of the form
\begin{verbatim}
    SUBROUTINE SUB(X,Y,F)
\end{verbatim}
where the variable {\tt X} and the one-dimensional arrays {\tt Y(*)}
and {\tt F(*)} are of type {\tt t}. This subroutine must set
$$\mathtt{F(I)} = f_{\mathtt{I}}(\mathtt{X,Y(1),\ldots,Y(N))} \qquad
(\mathtt{I = 1,2,\ldots,N}).$$
\Method
Using boldface quantities to denote vectors of length $n$, the
computational sequence is as follows:
\begin{eqnarray*}
\mathbf{k}_1 & = & \textstyle h \mathbf{f}(x,\mathbf{y}(x)), \\[2mm]
\mathbf{k}_2 & = & \textstyle h \mathbf{f}(x+\frac{1}{2}h,\mathbf{y}(x)+
\frac{1}{2}\mathbf{k}_1), \\[2mm]
\mathbf{k}_3 & = & \textstyle h \mathbf{f}(x+\frac{1}{2}h,\mathbf{y}(x)+
\frac{1}{2}\mathbf{k}_2), \\[2mm]
\mathbf{k}_4 & = &
\textstyle h \mathbf{f}(x+h,\mathbf{y}(x)+\mathbf{k}_3); \qquad
\mathbf{y}(x+h) \ = \ \textstyle \mathbf{y}(x)+\frac{1}{6}
(\mathbf{k}_1 + 2\mathbf{k}_2 + 2\mathbf{k}_3 + \mathbf{k}_4)
\end{eqnarray*}
The error per step is proportional to $h^5$.
\newpage
\Errorh
$\mathtt{N < 1}$ acts as do nothing.
\Refer
\begin{enumerate}
\item  F.B. Hildebrand, Introduction to numerical analysis,
(McGraw-Hill, New--York 1956) Sect. 6.16.
\end{enumerate}
$\bullet$
