%  19 apr 1996  ksk
\Version {RAN3D}                              \Routid{V130}
\Keywords{DISTRIBUTION NUMBER RANDOM VECTOR}
\Author{F. James}                              \Library{MATHLIB}
\Submitter{}                                   \Submitted{15.09.1978}
\Language{Fortran}                        %\Revised{}
\Cernhead {Random Three-Dimensional Vectors}
\begin{center}
\fbox{\parbox{120mm}{\OBSOLETE
Please note that this routine has been obsoleted in CNL 223. Users are
advised not to use it any longer and to replace it in older programs.
No maintenance for it will take place and it will eventually disappear.
\\[3mm]
Suggested replacement: {\tt RN3DIM} (V131) }}
\end{center}
{\tt RAN3D} generates random vectors, uniformly distributed over the
surface of a sphere of a given radius.
\Structure
{\tt SUBROUTINE} subprogram \\
User Entry Names: \Rdef{RAN3D} \\
External References: \Rind{NRAN}{V105}
\Usage
\begin{verbatim}
    CALL RAN3D(X,Y,Z,XLONG)
\end{verbatim}
\begin{DLtt}{12345678}
\item[X,Y,Z] ({\tt REAL}) A random 3-dimensional vector of length
{\tt XLONG}.
\item[XLONG] ({\tt REAL}) Length of the vector (to be specified on
entry).
\end{DLtt}
\Method
A random vector in the unit cube is generated using {\tt NRAN} (V105)
and is rejected if it lies outside the unit sphere. This rejection
technique uses on average about 6 random numbers per vector, where only
two are needed in principle. However, it is faster than the classical
two-number technique which requires a square root, a sine, and a cosine.
\\ $\bullet$
