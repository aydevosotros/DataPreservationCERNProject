\Version {DISVAV}                    \Routid{G111}
\Keywords{DISTRIBUTION RANDOM NUMBER INVERSE VAVILOV}
\Author{B. Schorr}                  \Library{MATHLIB}
\Submitter{}                         \Submitted{11.12. 1974}
\Language{Fortran}                  %\Revised{}
\Cernhead{Vavilov Distribution and Its Inverse}
These subprograms compute
\begin{DLtt}{12}
\item[$\bullet$] the probability density function $F(x)$,
\item[$\bullet$] the distribution function $G(x)$,
\item[$\bullet$] the inverse $Q(x)$ of $G(x)$,
\end{DLtt}
of the Vavilov distribution. $Q(x)$ can be used to generate Vavilov
distributed random numbers.
\Structure
{\tt SUBROUTINE} and {\tt FUNCTION} subprograms \\
User Entry Names: \Rdef{DISVAV}, \Rdef{DINVAV}, \Rdef{COEDIS},
\Rdef {COEDIN}\\
Internal Entry Names: {\tt VAVFCN}, {\tt VAVZRO}\\
External References: \Rind{SININT}, \Rind{COSINT} (C336),
\Rind{EXPINT} (C337) \\
{\tt COMMON} Block Names and Lengths: {\tt/VAVILI/ 4},
{\tt/VAVILO/ 311}, {\tt/VAVILA/ 202}, \\
\hspace*{60mm}{\tt/FORFCN/ 2}, {\tt/ONE/ 1}
\Usage
See {\bf Long Write-up}.
\\ $\bullet$
