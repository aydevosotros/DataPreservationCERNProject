%  18 oct 94   ksk
\Version {RBEQN}                        \Routid{F406}
\Keywords{BAND LINEAR EQUATION}
\Author{G.A. Erskine}                   \Library{KERNLIB}
\Submitter{}                             \Submitted{01.09.1983}
\Language{Fortran}                       \Revised{27.11.1984}
\Cernhead{Banded Linear Equations}
Subroutine  subprograms {\tt RBEQN} and {\tt DBEQN} solve a system of
$N$ simultaneous linear equations with $K$ right-hand sides, the
coefficient matrix being a band matrix with bandwidth $2M+1$:
$$ \sum^N_{j=1}a_{ij} x_{jk} = b_{ik},\qquad (i=1,2,\ldots,N, \
k=1,2,\ldots,K); \qquad
(a_{ij} = 0 \mbox{\ for\ } |i-j| > M).$$
Only those coefficients $a_{ij}$ for which $|i-j|\le M$
need be supplied on entry (see {\bf Usage}).
\Structure
{\tt SUBROUTINE}  subprograms \\
User Entry Names: \Rdef{RBEQN}, \Rdef{DBEQN}\\
Files Referenced: Printer\\
External References: \Rind{KERMTR}{N001}, \Rind{ABEND}{Z035}
\Usage
For $\mathtt{t=R}$ (type {\tt REAL}), $\mathtt{t=D}$ (type
{\tt DOUBLE PRECISION}),
\begin{verbatim}
    CALL tBEQN(N,M,ABAND,IDIM,IFAIL,K,B)
\end{verbatim}
\begin{DLtt}{123456}
\item [N]({\tt INTEGER}) Number of equations.
\item [M]({\tt INTEGER}) Band parameter $M$.
\item [ABAND] (type according to {\tt t}) Two-dimensional array whose
first dimension has the value {\tt IDIM}.
\item [IDIM]({\tt INTEGER}) First dimension of array {\tt ABAND}
(and of array {\tt B} if $\mathtt{K>1}$).
\item [IFAIL]({\tt INTEGER}) On exit, {\tt IFAIL} will be set to
{\tt -1} if the coefficient matrix is singular, and to {\tt 0} otherwise.
\item [K]({\tt INTEGER}) Number of right-hand sides in array {\tt B}.
\item[B] (type according to {\tt t}) In general, a two-dimensional array
whose first dimension has the value {\tt IDIM}. {\tt B} may be
one-dimensional if $\mathtt{K=1}$.
\end{DLtt}
\par
On entry, {\tt ABAND} must contain the packed form of the coefficient
matrix as described below, and array {\tt B} must contain the matrix of
right-hand sides $b_{ik}$. Then, provided the coefficient matrix is
non-singular, {\tt IFAIL} will be set to 0 and the solution $x_{ik}$ will
replace $b_{ik}$ in {\tt B}. The contents of {\tt ABAND} are destroyed.
If the coefficient matrix is singular, {\tt IFAIL} will be set
to {\tt -1}. In this case the contents of {\tt ABAND} and {\tt B} are
unpredictable.
\par
The storage convention for {\tt ABAND} is that it must contain, on
entry, those coefficients $a_{ij}$ for which $|i - j|\le \mathtt{M}$,
stored "left-justified" as an array of {\tt N} rows and at most
$\mathtt{2M+1}$ columns. For example, if $\mathtt{N=4}$ and
$\mathtt{M=1}$, the coefficient matrix
$$\left(\begin{array}{llll}
a_{11}& a_{12} & 0 &  0  \\
a_{21}& a_{22} & a_{23}& 0 \\
0     &a_{32}  & a_{33}&a_{34}\\
0     &0       & a_{43}&a_{44} \\
\end{array} \right)
\qquad \mbox{is stored as} \qquad
\left(\begin{array}{lll}
a_{11}& a_{12}& X\\
a_{21}& a_{22}& a_{23}\\
a_{32}& a_{33}& a_{34}\\
a_{43}& a_{44}& X
\end{array} \right) $$
where $X$ denotes elements whose value need not to be set.
\newpage
\par
If {\tt ALPHA(I,J)} is a function subprogram or statement function
which computes $a_{ij}$, the following Fortran statements will
set {\tt ABAND} correctly:
\begin{verbatim}
      DO 2 I =1,N
      L = 1
      DO 1 J = MAX(I-M,1),MIN(I+M,N)
      ABAND(I,L) = ALPHA(I,J)
      L = L+1
    1 CONTINUE
    2 CONTINUE
\end{verbatim}
\Method
Gaussian elimination with row interchanges. The storage organization is
as described in the reference.
\Errorh
If the integer arguments do not satisfy the conditions
$\mathtt{1 \leq M+1 \le  N \leq IDIM,\,K \le 0}$,
a message is printed and program execution is terminated by
calling {\tt ABEND} (Z035).
\Refer
\begin{enumerate}
\item J.H. Wilkinson and C. Reinsch (eds.), Handbook
for automatic computation, Vol.2: Linear algebra (Springer-Verlag,
New York 1971) 54.
\end{enumerate}
$\bullet$
