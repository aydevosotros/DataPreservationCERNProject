\Version {IUSAME}                     \Routid{M501}
\Keywords{LOCATING STRING SAME IDENTICAL WORD}
\Author{C. Letertre}                   \Library{KERNLIB}
\Submitter{}                           \Submitted{21.08.1971}
\Language{Fortran or Assembler}         \Revised{15.09.1978}
\Cernhead {Locating a String of Same Words}
{\tt IUSAME} locates the first of a continuous sequence of identical
words occuring at least a given number of times. It returns the
number of contiguous identical words in the sequence.
\Structure
{\tt FUNCTION} subprogram \\
User Entry Names: \Rdef{IUSAME}
\Usage
\begin{verbatim}
    NSAME = IUSAME(VECT,JL,JR,MIN,JSAME)
\end{verbatim}
\begin{DLtt}{1234567890}
\item [VECT(JL)] Start of the portion of the vector to be analysed.
\item [VECT(JR)] End of the portion of the vector to be analysed.
\item [MIN] Minimum length of a string to be considered a string.
\end{DLtt}
The function returns the length of the string as function
value, and also the position of the first element of the string:
{\tt VECT(JSAME)}.
\par
If no string of at least {\tt MIN} elements has been found starting
at or after {\tt VECT(JL)}, the function returns $\mathtt{NSAME=0}$ and
$\mathtt{JSAME=JR+1}$.
\\ $\bullet$
