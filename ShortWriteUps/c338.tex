\Version{CEXPIN}              \Routid{C338}
\Keywords{EXPONENTIAL INTEGRAL COMPLEX}
\Author{K.S. K\"olbig}          \Library{MATHLIB}
\Submitter{}                \Submitted{01.05.1990}
\Language{Fortran}              \Revised{15.03.1993}
\Cernhead{Exponential Integral for Complex Argument}
Function subprograms {\tt CEXPIN} and {\tt WEXPIN} calculate
the the exponential integral
$$ E(z) \ = \ \displaystyle \int^{z}_{0}\,t^{-1}\,(1 - e^{-t})\,dt $$
for complex arguments $z$.
\par
The double-precision version {\tt WEXPIN} is available only on computers
which support a {\tt COMPLEX*16} Fortran data type.
\Structure
{\tt FUNCTION} subprograms \\
Use Entry Names : \Rdef{CEXPIN}, \Rdef{WEXPIN} \\
Files referenced : {\tt Unit 6} \\
External References: \Rind{MTLMTR}{N002}, \Rind{ABEND}{Z035}
\Usage
In any arithmetic expression,
\begin{center}
{\tt CEXPIN(Z)} \quad or \quad {\tt WEXPIN(Z)} \quad has the value \quad
$E(\mathtt{Z})$,
\end{center}
where {\tt CEXPIN} is of type {\tt COMPLEX}, {\tt WEXPIN} is of type
{\tt COMPLEX*16}, and {\tt Z} has the same type as the function name.
\Method
Pad\'e approximants are used to compute $E(z) = E(x + iy)$ in the
following (partly overlapping) regions of the $z$-plane:
\begin{center}\begin{tabular}{lrcl}
(i)  & $ \textstyle (\frac{1}{7}(x - 1))^2 + (\frac{1}{5}y)^2$ &
$\le 1$ & $(x \ge - 5)$,\\[3mm]
(ii)& $ \textstyle (\frac{1}{15}(x + 12))^2 + (\frac{1}{12}y)^2$ &
$\ge 1$ & $(x \ge - 12)$,\\[3mm]
(iii)& $\textstyle(\frac{1}{12}y)^2$ & $\ge 1$ & $(x < - 12)$.
\end{tabular}\end{center}
In the remaining region, consisting mainly of a strip along the negative
real axis, $E(z)$ is computed by numerical integration (which is very
much slower than the evaluation of the Pad\'e approximations).
\Accuracy
{\tt CEXPIN} (except on CDC and Cray computers)
has full single-precision accuracy.
For most values of the argument {\tt Z}, {\tt WEXPIN}
(and {\tt CEXPIN} on CDC and Cray computers) has an accuracy of
approximately two significant digits less than the machine precision.
\Errorh
Error {\tt C338.1}: Numerical integration not successful (unlikely).
The function value is set equal to zero, and a message is written on
{\tt Unit 6}, unless subroutine {\tt MTLSET} (N002) has been called.
\Refer
\begin{enumerate}
\item Y.L. Luke, the special functions and their approximations, v. II,
(Academic Press, New York 1969) 198--199, 402--416.
\end{enumerate}
$\bullet$
