\Version {NOARG}                            \Routid{Z029}
\Keywords{ARGUMENT NOARG NUMBER STATEMENT SUPPLIED}
\Author{A. Yule}                             \Library{KERNLIB}
\Submitter{T. Lindel\"of}                      \Submitted{15.01.1969}
\Language{Assembler}                           \Revised{27.11.1984}
\Cernhead {Number of Arguments Supplied in a Call Statement}
\begin{center}
\fbox{\parbox{120mm}{
\begin{center}
{\bf OBSOLETE}
\end{center}
Please note that this routine has been obsoleted in CNL 204. Users are
advised not to use it any longer and to replace it in older programs.
No maintenance for it will take place and it will eventually disappear.
}}
\end{center}
{\tt NOARG} finds the number of actual arguments supplied in a
Fortran {\tt CALL} statement. {\tt NOARG} is maintained only for
backwards compatibility with Fortran 4 programs. Its use in
Fortran 77 is unreliable and not recommended. On Apollo it returns
the value of the parameter unchanged.
\Structure
{\tt SUBROUTINE} subprogram \\
User Entry Names: \Rdef{NOARG}
\Usage
\begin{verbatim}
    CALL NOARG(I)
\end{verbatim}
in a {\tt SUBROUTINE} subprogram will set {\tt I} equal to the number of
actual arguments supplied in the call which transferred control to the
subprogram.
\Notes
Referring in a subprogram to any parameter beyond the number of
arguments supplied, as returned by {\tt NOARG}, will lead to
unpredictable results.
\par
On CDC, using the FTN (Fortran 4) compiler, the calling program
must be compiled with the FTN option {\tt Z}, otherwise {\tt NOARG}
will not work if the number of arguments is zero.
\par
On CDC, using the FTN5 (Fortran 77) compiler, $\mathtt{OPT=2}$ must
not be used as the compiler may locally store parameters in its prolog
code at this optimization level.
\par
On IBM, using the JFORT compiler, {\tt NOARG} will return a value
one larger than the actual number of arguments if they include one
or more alternate {\tt RETURN} statements.
\Examples
\begin{verbatim}
    CALL BETA(A,B,C)
    ...
    SUBROUTINE BETA(X,Y,Z,I)
    ...
    CALL NOARG(N)
\end{verbatim}
sets $\mathtt{N=3}$, the number of actual arguments in the call to
{\tt BETA}.
\\ $\bullet$
