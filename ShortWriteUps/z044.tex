%24 mar 94 ksk
\Version {INTRAC}                        \Routid{Z044}
\Keywords{IDENTIFY INTERACTIVE INTRAC JOB}
\Author{F. Carminati, T. Lindel\"of, R. Matthews, C. Vosicki, J. Zoll}
\Library{KERNLIB}
\Submitter{}                              \Submitted{01.12.1974}
\Language{Fortran or C or Assembler}           \Revised{01.06.1993}
\Cernhead {Identify Job as Interactive}
{\tt INTRAC} allows an executing module to determine whether it
is running interactively or not.
\Structure
{\tt FUNCTION} subprogram \\
User Entry Names: \Rdef{INTRAC}
\Usage
In any logical expression,
\begin{verbatim}
    INTRAC()
\end{verbatim}
has the value {\tt .TRUE.} if the module is executing interactively and
{\tt .FALSE.} otherwise.
Note that {\tt INTRAC}
must be declared {\tt LOGICAL} in the calling routine.
\Method
On {\tt UNIX} machines execution is interactive if 'standard input'
(System {\tt Unit 0}, i.e. Fortran {\tt Unit 5} normally) is
connected to a terminal. The same is true on VAX as from June 1993.
\\ $\bullet$
