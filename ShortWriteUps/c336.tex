% 31 oct 94  ksk
\Version{RSININ}                            \Routid{C336}
\Keywords{SINE COSINE INTEGRAL}
\Author{K.S. K\"olbig}                      \Library{MATHLIB}
\Submitter{}                                \Submitted{07.12.1970}
\Language{Fortran}                     \Revised{01.12.1994}
\Cernhead{Sine and Cosine Integrals}
Function subprograms {\tt RSININ}, {\tt RCOSIN} and
{\tt DSININ}, {\tt DCOSIN} calculate the sine and cosine integrals
$$ \begin{array}{r@{\quad = \quad}l}
\mbox{\rm Si}(x) & \displaystyle \int^x_0\frac{\sin t}{t}dt \\[4mm]
\mbox{\rm Ci}(x) & \displaystyle
\gamma+\ln|x|+\int^x_0 \frac{\cos t - 1}{t}dt \qquad
(x\ne 0)
\end{array} $$
for real arguments $x$,
where $\gamma = 0.57721 \ldots$ is Euler's constant.
\par
On CDC and Cray computers, the double-precision versions
{\tt DSININ} and {\tt DCOSIN} are not available.
\Structure
{\tt FUNCTION} subprograms \\
 User Entry Names: \Rdef{RSININ}, \Rdef{RCOSIN}, \Rdef{DSININ},
 \Rdef{DCOSIN}\\
Obsolete User Entry Names: \Rdef{SININT} $\equiv$ \Rdef{RSININ},
                           \Rdef{COSINT} $\equiv$ \Rdef{RCOSIN} \\
Files Referenced: {\tt Unit 6} \\
External References: \Rind{MTLMTR}{N002}, \Rind{ABEND}{Z035}
\Usage
In any arithmetic expression,
\begin{center}
{\tt RSININ(X)} \quad or \quad {\tt DSININ(X)} \quad has the value \quad
Si({\tt X}), \\
{\tt RCOSIN(X)} \quad or \quad {\tt DCOSIN(X)} \quad has the value \quad
Ci({\tt X}),
\end{center}
where {\tt RSININ} and {\tt RCOSIN} are of type {\tt REAL}, {\tt DSININ}
and {\tt DCOSIN} are of type {\tt DOUBLE PRECISION},
and {\tt X} has the same type as the function name.
\Method
Approximation by truncated Chebyshev series.
\Accuracy
{\tt RSININ} and {\tt RCOSIN} (except on CDC and Cray computers)
have full single-precision accuracy.
For most values of the argument {\tt X}, {\tt DSININ}, {\tt DCOSIN}
(and {\tt RSININ}, {\tt RCOSIN} on CDC and Cray computers) have an
accuracy of
approximately one significant digit less than the machine precision.
\Errorh
Error {\tt C336.1:}  $\mathtt{X = 0}$ for {\tt RCOSIN} or {\tt DCOSIN}.
The function value is set equal to zero, and a message is written on
{\tt Unit 6}, unless subroutine {\tt MTLSET} (N002) has been called.
\Refer
\begin{enumerate}
\item Y.L. Luke, The special functions and their approximations, v.II,
(Academic Press, New York l969) 325--326
\end{enumerate}
$\bullet$
