\Version{CGAMMA}               \Routid{C305}
\Keywords{COMPLEX GAMMA FUNCTION}
\Author{K.S. K\"olbig}           \Library{MATHLIB}
\Submitter{}                    \Submitted{02.05.1966}
\Language{Fortran}            \Revised{15.03.1993}
\Cernhead{Gamma Function for Complex Argument}
Function subprograms {\tt CGAMMA} and {\tt WGAMMA} calculate
the gamma function
$$ \Gamma(z) \ = \ \displaystyle \int_0 ^{\infty} e^{-t} t^{z-1}dt \qquad
(\mbox{\rm Re } z>0) $$
for complex arguments $z \neq -n,(n = 0,1,2,\cdots)$.
\par
The double-precision version {\tt WGAMMA} is available only on computers
which support a {\tt COMPLEX*16} Fortran data type.
\Structure
{\tt FUNCTION} subprograms \\
User Entry Names: \Rdef{CGAMMA}, \Rdef{WGAMMA}\\
Files Referenced: {\tt Unit 6} \\
External References: \Rind{MTLMTR}{N002}, \Rind{ABEND}{Z035}
\Usage
In any arithmetic expression,
\begin{center}
{\tt CGAMMA(Z)} \quad or \quad {\tt WGAMMA(Z)} \quad has the value \quad
$\Gamma(\mathtt{Z}),$
\end{center}
where {\tt CGAMMA} is of type {\tt COMPLEX}, {\tt WGAMMA} is of type
{\tt COMPLEX*16},
and {\tt Z} has the same type as the function name.
\Method
The method is described in Ref. 1.
\Accuracy
{\tt CGAMMA} (except on CDC and Cray computers)
has full single-precision accuracy.
For most values of the argument {\tt Z}, {\tt WGAMMA}
(and {\tt CGAMMA} on CDC and Cray computers) has an accuracy of
approximately one significant digit less than the machine precision.
\Errorh
Error {\tt C305.1}: $\mathtt{Z} = -n,(n = 0,1,2,\cdots).$
The function value is set equal to zero, and a message is written on
{\tt Unit 6}, unless subroutine {\tt MTLSET} (N002) has been called.
\Refer
\begin{enumerate}
\item Y.L. Luke, The special functions and their approximations,
v.II, (Academic Press, New York 1969) 304--305
\end{enumerate}
$\bullet$
