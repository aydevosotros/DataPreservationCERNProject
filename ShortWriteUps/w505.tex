\Version {FOWL}                              \Routid{W505}
\Keywords{MONTE-CARLO GENERAL PHASE SPACE}
\Author{F. James}                           \Library{POOL}  
\Submitter{}                                \Submitted{13.11.1972}
\Language{Fortran}                      \Revised{01.12.1981}
\Cernhead {General Monte-Carlo Phase-Space}
{\tt FOWL} uses the Monte-Carlo method to calculate phase space
distributions arising from particle interactions. The events are
generated according to Lorentz-invariant phase space, and after each
event the user may calculate (in a subroutine) all quantities
(effective masses, angles, moments, delta squared, etc.)
whose distribution he wants.
\par
Moreover, the user may calculate, for each quantity, a weight
(or 'matrix element', for example a Breit-Wigner) which is in general a
function of the kinematic quantities for the event. In addition, one can
investigate the effects of cutoffs, selections or biases in an actual
experiment by imposing the same selections on events in {\tt FOWL}.
The program then prints histograms and/or scatter plots of quantities
calculated by the user.
\Structure
{\tt SUBROUTINE} subprogram \\
User Entry Names: \Rdef{FOWL}\\
Files Referenced: {\tt INPUT}, {\tt OUTPUT}, {\tt PUNCH}\\
External References:
\parbox[t]{105mm}{
\Rind{RNDM}{V104}, \Rind{UBLANK}{V300},
\Rind{IUCHAN}{Y201}, \Rind{DATIME}{Z007},
user-supplied subroutine {\tt USER}.}
\Usage
See {\bf Long Write-up}.
\Source
Event generator {\tt GENEV} was adapted by K. Kajantie from a program
by G.  Lynch.
\\ $\bullet$
