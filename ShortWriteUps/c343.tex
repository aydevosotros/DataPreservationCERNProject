\Version{BSJA}                             \Routid{C343}
\Keywords{BESSEL FUNCTION NON-INTEGER ORDER}
\Author{K.S. K\"olbig}                     \Library{MATHLIB}
\Submitter{}                                \Submitted{24.01.1986}
\Language{Fortran}                         \Revised{15.03.1993}
\Cernhead{Bessel Functions J and I with Positive Argument and
Non-Integer Order}
Subroutine subprograms {\tt BSJA}, {\tt BSIA},
{\tt DBSJA}, {\tt DBSIA} and {\tt QBSJA}, {\tt QBSIA}
calculate the sequences of Bessel functions
$$ J_{a+n}(x), \ J_{a-n}(x), \ I_{a+n}(x) \
\mathrm{or} \ I_{a-n}(x),$$
for real argument $x>0$, $0\leq a<1$, and $n=0,1,2,\ldots,N$.
\par
The quadruple-precision versions {\tt QBSJA} and {\tt QBSIA}
are available only on computers which support a {\tt REAL*16}
Fortran data type.
\Structure
{\tt SUBROUTINE} subprograms\\
User Entry Names: \Rdef{BSJA}, \Rdef{BSIA}, \Rdef{DBSJA}, \Rdef{DBSIA},
\Rdef{QBSJA}, \Rdef{QBSIA} \\
Files Referenced: {\tt Unit 6} \\
External References: 
\Rind{GAMMA}{C302}, \Rind{DGAMMA}{C302}, \Rind{QGAMMA}{C302},
\Rind{MTLMTR}{N002}, \Rind{ABEND}{Z035}
\Usage
{\bf Single-precision version:} \\[2mm]
\hspace*{8mm} {\tt CALL BSJA(X,A,NL,ND,B)} \qquad or \qquad
              {\tt CALL BSIA(X,A,NL,ND,B)}
\begin{DLtt}{1234}
\item[X] ({\tt REAL}) Argument $x$.
\item[A] ({\tt REAL}) Order $a$ of the first Bessel function in the
computed sequence.
\item[NL] ({\tt INTEGER}) Specifies the order $a+\mathtt{NL}$ of the last
Bessel function in the computed sequence.
It is permissible for {\tt NL} to be negative.
\item[ND] ({\tt INTEGER}) Requested number of correct significant
decimal digits.
\item[B] ({\tt REAL}) One-dimensional array with dimension
({\tt 0:d}) where $\mathtt{d} \geq \mathtt{|NL|}$. \\
On exit, $\mathtt{B(n)}$, $\mathtt{(n = 0,1,2,\ldots,|NL|)}$,
contains $J_{a+\mathtt{n}}(\mathtt{X})$, $J_{a-\mathtt{n}}(\mathtt{X})$,
$I_{a+\mathtt{n}}(\mathtt{X})$ or $I_{a-\mathtt{n}}(\mathtt{X})$,
the plus sign of the subscript being taken if $\mathtt{NL} \geq 0 $,
the minus sign if $\mathtt{NL} < 0$.
\end{DLtt}
{\bf Double-precision version:} \\[2mm]
\hspace*{8mm} {\tt CALL DBSJA(X,A,NL,ND,B)} \qquad or \qquad
              {\tt CALL DBSIA(X,A,NL,ND,B)}  \\[2mm]
where {\tt X}, {\tt A} and {\tt B} are of type {\tt DOUBLE PRECISION}.
\\[2mm]
{\bf Quadruple-precision version:} \\[2mm]
\hspace*{8mm} {\tt CALL QBSJA(X,A,NL,ND,B)} \qquad or \qquad
              {\tt CALL QBSIA(X,A,NL,ND,B)}  \\[2mm]
where {\tt X}, {\tt A} and {\tt B} are of type {\tt REAL*16}.
\Method
For $\mathtt{NL \geq 0}$, the method of computation is a variant
of Miller's backwards recurrence algorithm (see Ref. 1). The
requested accuracy is obtained, when possible, by a
judicious choice of the initial value of the recursion index,
together with at least one repetition of the recursion with this index
increased by 5.
For $\mathtt{NL < 0}$, only the first two functions in the sequence are
computed in this way. The remaining functions are computed by the
standard Bessel function recurrence relation.
\newpage
\Restrict
$\mathtt{X > 0}$, \ $\mathtt{0 \leq A < 1}$, \ $\mathtt{|NL| \leq 100}$.
\Accuracy
If {\tt X} is close to a zero of one of the functions $J_{a+n}(x)$,
the accuracy of that particular function will be less than {\tt ND}
significant digits.
There may also be a loss of accuracy in any of the computed functions
if {\tt A} is close to 0 or 1, and in other special situations.
\Errorh
Error {\tt C343.1}: $\mathtt{X \le 0}$.\\
Error {\tt C343.2}: $\mathtt{A<0}$ or $\mathtt{A \ge 1}$. \\
Error {\tt C343.3}: $\mathtt{|NL|>100}$. \\
Error {\tt C343.4}: Difficulties of convergence. Try smaller
$\mathtt{|ND|}$.\\
In all cases, a message is written on
{\tt Unit 6}, unless subroutine {\tt MTLSET} (N002) has been called.
If Error 1 to 3 occurs, the initial contents of array {\tt B}
is left unchanged.
If the requested accuracy cannot be obtained after the initial
recursion index has been increased fifty times (Error 4),
the contents of array {\tt B} is undefined.
\Source
The subprogram is based on Algol procedures described in Ref. 1.
\Refer
\begin{enumerate}
\item W. Gautschi, Algorithm 236, Bessel functions of the first kind,
 Collected Algorithms from CACM (1972)
\end{enumerate}
$\bullet$
