\Version {ECTRAD}                      \Routid{E401}
\Keywords{POWER SERIES TELESCOPING}
\Author{D.T. Elliott}                   \Library{MATHLIB}
\Submitter{T. H{\aa}vie}                    \Submitted{01.12.1970}
\Language{Fortran}                      \Revised{15.03.1976}
\Cernhead {Telescoping of Power Series, Double-Precision}
{\tt ECTRAD} performs the following tasks:
\begin{enumerate}
\item Conversion of a Chebyshev series to a power series.
\item Conversion of a power series in $z \in [a,b]$ to a
Chebyshev series in $x \in [-1,1]$ where $x=(2z-b-a)/(b-a)$.
\item Economization of a power series in $z \in [a,b]$
to a power series in $x \in [-1,1]$ of lower degree with error
less than a specified maximum error for all $x \in [-1,1]$.
\end{enumerate}
\Structure
{\tt SUBROUTINE} subprogram\\
User Entry Names: \Rdef{ECTRAD}\\
Internal Entry Names: {\tt ECONDE}, {\tt TAYCHD}, {\tt BIN},
{\tt TRANSD}, {\tt CHEB02} \\
Files Referenced: Printer\\
{\tt COMMON} Block Names and Lengths: {\tt /COFD/ 600}
\Usage
\begin{verbatim}
    CALL ECTRAD(COFIN,N,COFOUT,M,A,B,ACC,IOP)
\end{verbatim}
The meaning of the parameters depends on the two-digit
integer {\tt IOP}, the first digit being {\tt 1, 2} or {\tt 3}
referring to the three types of applications defined above, and the
second digit taking the values {\tt 0} (the general case), {\tt 1}
(the series is symmetric, i.e. {\it even} with respect to $(a+b)/2$),
or {\tt 3} (the series is anti-symmetric, i.e. {\it odd} with respect
to $(a+b)/2$). The general case will handle any series, but the
calculation is faster for the special cases. \\[3mm]
{\bf Conversion of a Chebyshev series to a power series}
$(\mathtt{IOP=10})$:
\begin{DLtt}{12345678}
\item[COFIN] ({\tt DOUBLE PRECISION}) Coefficients of the Chebyshev
series (input).
\item[N] ({\tt INTEGER}) Number of terms in the Chebyshev series.
\item[COFOUT] ({\tt DOUBLE PRECISION}) Coefficients of the power
series (output).
\item[M $=$ N] ({\tt INTEGER}) Number of terms in the power series.
\end{DLtt}
{\bf Conversion of a power series to a Chebyshev series}
$(\mathtt{IOP = 20, 21, 22})$:
\begin{DLtt}{12345678}
\item[COFIN] ({\tt DOUBLE PRECISION}) Coefficients of the power
series (input).
\item[N] ({\tt INTEGER}) Number of terms in the power series.
\item[A,B] ({\tt DOUBLE PRECISION}) Boundary values $a$ and $b$.
\item[COFOUT] ({\tt DOUBLE PRECISION}) Coefficients of the Chebyshev
series (output).
\item[M $=$ N] ({\tt INTEGER}) Number of terms in the Chebyshev series.
\end{DLtt}
\newpage
{\bf Economization of a power series} $(\mathtt{IOP = 30, 31, 32})$:
\begin{DLtt}{12345678}
\item[COFIN] ({\tt DOUBLE PRECISION}) Coefficients of the power
series (input).
\item[N] ({\tt INTEGER}) Number of terms in the power series.
\item[A,B] ({\tt DOUBLE PRECISION}) Boundary values $a$ and $b$.
\item[ACC] ({\tt DOUBLE PRECISION}) Maximum permitted absolute error
in the economization. If this error is not achieved,
a message will be printed.
\item[COFOUT] ({\tt DOUBLE PRECISION}) Coefficients of the economized
series (output).
\item[M] ({\tt INTEGER}) Number of terms in the economized series.
\end{DLtt}
\Refer
\begin{enumerate}
\item C. Lanczos, Applied Analysis, Prentice Hall, London (1959).
\end{enumerate}
$\bullet$
