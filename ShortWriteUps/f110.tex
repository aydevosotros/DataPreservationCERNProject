%  04 jan 95   ksk
\Version {MXPACK}                          \Routid{F110}
\Keywords{TC MANIPULATION MATRIX ROW-WISE}
\Author{TC}                        \Library{KERNLIB}
\Submitter{C. Letertre}                   \Submitted{01.08.1969}
\Language{Fortran}                       \Revised{07.03.1989}
\Cernhead {TC Matrix Manipulation Package}
\begin{center}
\fbox{\parbox{120mm}{\OBSOLETE
Please note that this routine has been obsoleted in CNL 194. Users are
advised not to use it any longer and to replace it in older programs.
No maintenance for it will take place and it will eventually disappear.
\\[3mm]
Suggested replacement: {\tt RVADD} (F002), {\tt RMADD} (F003),
{\tt RMMLT} (F004) }}
\end{center}
The routines of {\tt MXPACK} compute the product of two matrices or the
product of their transposed matrices and may add or subtract to the
resultant matrix a third one, add or subtract one matrix from another,
or transfer a matrix, its negative, or a multiple of it, transpose a
given matrix, build up a unit matrix, multiply a matrix by a diagonal
(from left or from right) and may add the result to another matrix,
add to square matrix the multiple of a diagonal matrix, compute the
products $\mathbf{X=ABA'}$ ($\mathbf{A'}$ denotes the transpose of
$\mathbf{A}$) and $\mathbf{X=A'BA}$. It is assumed that matrices are
stored {\bf row-wise without gaps}, contrary to the Fortran convention.
\Structure
{\tt SUBROUTINE} subprograms  \\
User Entry Names:
\begin{htmlonly}
\begin{tabular}{llllllll}
\end{htmlonly}
\begin{latexonly}
\begin{tabular}[t]{l*{7}{@{\hspace{4pt}}l}}
\end{latexonly}
\Rdef{MXMAD},  & \Rdef{MXMAD1}, & \Rdef{MXMAD2}, & \Rdef{MXMAD3}, &
\Rdef{MXMPY},  & \Rdef{MXMPY1}, & \Rdef{MXMPY2}, & \Rdef{MXMPY3}, \\
\Rdef{MXMUB},  & \Rdef{MXMUB1}, & \Rdef{MXMUB2}, & \Rdef{MXMUB3}, &
\Rdef{MXTRP},  & \Rdef{MXUTY},  & \Rdef{MXMLRT}, & \Rdef{MXMLTR}
\end{tabular}
\Usage
{\bf Matrix Multiplication} \\[3mm]
\begin{tabular}{@{\hspace*{10mm}}l@{\qquad}l}
{\tt CALL MXMPY(A,B,C,NI,NJ,NK)} & $(A_{ij})(B_{jk}) \to (C_{ik})$ \\
{\tt CALL MXMPY1(A,Q,C,NI,NJ,NK)} & $\mathbf{AQ' \to C}$ \
($\mathbf{Q}$ is $\mathtt{NK \times NJ}$) \\
{\tt CALL MXMPY2(P,B,C,NI,NJ,NK)} & $\mathbf{P'B \to C}$ \
($\mathbf{P}$ is $\mathtt{NJ \times NI}$) \\
{\tt CALL MXMPY3(P,Q,C,NI,NJ,NK)} & $\mathbf{P'Q' \to C}$ \\
\end{tabular} \\[2mm]
If $\mathtt{NJ=0}$, $\mathbf{C}$ will be filled with zeros. \\[2mm]
{\bf Matrix Multiplication and Addition} \\[3mm]
\begin{tabular}{@{\hspace*{10mm}}l@{\qquad}l}
{\tt CALL MXMAD(A,B,C,NI,NJ,NK)} &
$(A_{ij})(B_{jk})+(C_{ik}) \to (C_{ik})$ \\
{\tt CALL MXMAD1(A,Q,C,NI,NJ,NK)} & $\mathbf{AQ'+C \to C}$ \\
{\tt CALL MXMAD2(P,B,C,NI,NJ,NK)} & $\mathbf{P'B+C \to C}$ \\
{\tt CALL MXMAD3(P,Q,C,NI,NJ,NK)} & $\mathbf{P'Q'+C \to C}$ \\
\end{tabular} \\[2mm]
If $\mathtt{NJ=0}$, $\mathbf{C}$ will not be changed. \\[2mm]
\newpage
{\bf Matrix Multiplication and Subtraction} \\[3mm]
\begin{tabular}{@{\hspace*{10mm}}l@{\qquad}l}
{\tt CALL MXMUB(A,B,C,NI,NJ,NK)} &
$(A_{ij})(B_{jk})-(C_{ik}) \to (C_{ik})$ \\
{\tt CALL MXMUB1(A,Q,C,NI,NJ,NK)} & $\mathbf{AQ'-C \to C}$ \\
{\tt CALL MXMUB2(P,B,C,NI,NJ,NK)} & $\mathbf{P'B-C \to C}$ \\
{\tt CALL MXMUB3(P,Q,C,NI,NJ,NK)} & $\mathbf{P'Q'-C \to C}$ \\
\end{tabular} \\[2mm]
If $\mathtt{NJ=0}$, $\mathbf{C}$ will be replaced by $\mathbf{-C}$.
\\[2mm]
{\bf Matrix Transposition} \\[3mm]
\begin{tabular}{@{\hspace*{10mm}}l@{\qquad}l}
{\tt CALL MXTRP(A,B,NI,NJ)} & $(A_{ij}) \to (B_{ji})$ \\
\end{tabular} \\[3mm]
{\bf Unity Matrix} \\[3mm]
\begin{tabular}{@{\hspace*{10mm}}l@{\qquad}l}
{\tt CALL MXUTY(A,NI)} & $(A_{ii})=1; \ (A_{ij})= 0,\,(i \ne j)$ \\
\end{tabular} \\[3mm]
{\bf Matrix Multiplication} \\[3mm]
\begin{tabular}{@{\hspace*{10mm}}l@{\qquad}l}
{\tt CALL MXMLRT(A,B,X,M,N)} &
$\mathbf{A}[m \times n]\,\mathbf{B}[n \times n]\,\mathbf{A'}[n \times m]
\to \mathbf{X}[m \times m]$ \\
{\tt CALL MXMLTR(A,B,X,N,M)} &
$\mathbf{A'}[n \times m]\,\mathbf{B}[m \times m]\,\mathbf{A}[m \times n]
\to \mathbf{X}[n \times n]$ \\
\end{tabular}
\Notes
In the formulae above, $(A_{ij})$ {\it etc} denotes the ensemble of
elements of the matrix $\mathbf{A}$ {\it etc}
with the row index $i$ and the column index $j$.
The Fortran variables {\tt NI}, {\tt NJ} and {\tt NK} specify the
dimensions associated with the indices $i,j$ and $k$. If
{\tt DIMENSION A(NJ,NI)} reserves space for the
matrix $\mathbf{A}$, then the element $A_{ij}$ is contained in
{\tt A(J,I)}.
\\ $\bullet$
