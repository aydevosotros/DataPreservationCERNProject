\Version{GAMMA}                        \Routid{C302}
\Keywords{GAMMA FUNCTION}
\Author{K.S. K\"olbig}    \Library{MATHLIB or Fortran Computer Library}
\Submitter{}                           \Submitted{07.06.1992}
\Language{Fortran}                     \Revised{15.03.1993}
\Cernhead{Gamma Function for Positive Argument}
Function subprograms {\tt GAMMA}, {\tt DGAMMA}  and {\tt QGAMMA}
calculate the gamma function
$$ \Gamma(x) \ = \ \displaystyle \int_0 ^{\infty} e^{-t} t^{x-1}dt
\qquad (x > 0) $$
for real argument $x > 0$.
\par
The quadruple-precision version {\tt QGAMMA} is available only on
computers which support a {\tt REAL*16} Fortran data type.
\Structure
{\tt FUNCTION} subprograms\\
User Entry Names: \Rdef{GAMMA}, \Rdef{DGAMMA}, \Rdef{QGAMMA} \\
Files Referenced: {\tt Unit 6} \\
External References: \Rind{MTLMTR}{N002}, \Rind{ABEND}{Z035}
\Usage
In any arithmetic expression,
\begin{center}
{\tt GAMMA(X)}, \quad {\tt DGAMMA(X)} \quad or \quad {\tt QGAMMA(X)}
\quad has the value \quad $\Gamma(\mathtt{X})$,
\end{center}
where {\tt GAMMA} is of type {\tt REAL}, {\tt DGAMMA} is of type
{\tt DOUBLE PRECISION}, {\tt QGAMMA} is of type {\tt REAL*16},
and  {\tt X} has the same type as the function name.
\Method
Approximation by truncated Chebyshev series and functional relations.
\Accuracy
The system-supplied version (see {\bf Notes}) has full machine
accuracy. The CERN version of {\tt GAMMA} (except on CDC
and Cray computers) has full single-precision accuracy. The
CERN version of {\tt DGAMMA}, {\tt QGAMMA}
(and of {\tt GAMMA}, {\tt DGAMMA} on CDC and
Cray computers) have an accuracy which is approximately
one digit less than machine precision.
 \Errorh
Error {\tt C302.1}: $\mathtt{X \le 0}$.
The function value is set equal to zero, and a message is written on
{\tt Unit 6} unless subroutine {\tt MTLSET} (N002) has been called.
\Notes
If the function {\tt GAMMA} or {\tt DGAMMA} is available in the
system-supplied Fortran mathematical library,
the system-supplied function will be loaded instead of the CERN version.
\Refer
\begin{enumerate}
\item Y.L. Luke, Mathematical functions and their approximations,
(Academic Press, New York 1975) 4.
\end{enumerate}
$\bullet$
