\Version{BESJ0}               \Routid{C312}
\Keywords{BESSEL FUNCTION ORDER ZERO ONE}
\Author{K.S. K\"olbig}          \Library{MATHLIB}
\Submitter{}                   \Submitted{18.10.1967}
\Language{Fortran}          \Revised{15.03.1993}
\Cernhead{Bessel Functions J and Y of Orders Zero and One }
Function subprograms
{\tt BESJ0}, {\tt BESJ1}, {\tt BESY0}, {\tt BESY1} and
{\tt DBESJ0}, {\tt DBESJ1}, {\tt DBESY0}, {\tt DBESY1}
calculate the Bessel functions
$$J_0(x), \ J_1(x), \ Y_0(x), \ Y_1(x)$$
for real arguments $x$, where $ x>0 $ for $Y_0(x)$ and $Y_1(x)$.
\par
On CDC and Cray computers, the double-precision versions
{\tt DBESJ0} etc. are not available.
\Structure
{\tt FUNCTION} subprograms\\
User Entry Names:
\Rdef{BESJ0}, \Rdef{BESJ1}, \Rdef{BESY0}, \Rdef{BESY1},
\Rdef{DBESJ0}, \Rdef{DBESJ1}, \Rdef{DBESY0}, \Rdef{DBESY1}\\
Files Referenced: {\tt Unit 6} \\
External References: \Rind{MTLMTR}{N002}, \Rind{ABEND}{Z035}
\Usage
In any arithmetic expression,
\begin{center}
\parbox{.6\textwidth}{
{\tt BESJ0(X)} \quad or \quad {\tt DBESJ0(X)} \quad has the value \quad
$J_0(\mathtt{X})$, \\
{\tt BESJ1(X)} \quad or \quad {\tt DBESJ1(X)} \quad has the value \quad
$J_1(\mathtt{X})$, \\
{\tt BESY0(X)} \quad or \quad {\tt DBESY0(X)} \quad has the value \quad
$Y_0(\mathtt{X})$, \\
{\tt BESY1(X)} \quad or \quad {\tt DBESY1(X)} \quad has the value \quad
$Y_1(\mathtt{X})$,
}\end{center}
where {\tt BESJ0} etc. are of type {\tt REAL}, {\tt DBESJ0} etc.
are of type {\tt DOUBLE PRECISION}, and {\tt X} has the same type as the
function name.
\Method
Approximation by truncated Chebyshev series.
\Accuracy
{\tt BESJ0} etc. (except on CDC and Cray computers)
have full single-precision accuracy.
For most values of the argument {\tt X}, {\tt DBESJ0} etc.
(and {\tt BESJ0} etc. on CDC and Cray computers) have an accuracy of
approximately one significant digit less than the machine precision.
\Errorh
Error {\tt C312.1}:  $\mathtt{X \leq 0}$ for $Y_0(x)$ or $Y_1(x)$.
The function value is set equal to zero, and a message is written on
{\tt Unit 6} unless subroutine \Rind{MTLSET}{N002} has been called.
\Refer
\begin{enumerate}
\item Y.L. Luke, Mathematical functions and their
approximations (Academic Press, New York 1975) 322--324.
\end{enumerate}
$\bullet$
