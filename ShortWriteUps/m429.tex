\Version {NUMBIT}           \Routid{M429}
\Keywords{BIT NUMBER WORD}
\Author{M. Metcalf}          \Library{KERNLIB}
\Submitter{}                 \Submitted{01.06.1973}
\Language{Assembler}         \Revised{15.09.1978}
\Cernhead {Number of One-Bits in a Word}
{\tt NUMBIT} counts the one-bits in a word.
\Structure
{\tt FUNCTION} subprogram \\
User Entry Names: \Rdef{NUMBIT}
\Usage
In an arithmetic expression,
\begin{center}
{\tt NUMBIT(X)}
\end{center}
has the {\tt INTEGER} value giving the number of one-bits in {\tt X}.
\Examples
\begin{verbatim}
    J=NUMBIT(5)
\end{verbatim}
sets {\tt J} to {\tt 2} as the binary representation of 5 has 2 one-bits.
\\ $\bullet$
