% 18 jan 1995  ksk
\Version {VAXAST}                    \Routid{Z037}
\Keywords{CONTROL INTERRUPT VAX VMS}
\Author{C. Mekenkamp}                  \Library{KERNLIB}
\Submitter{R. Veenhof}                \Submitted{10.03.1988}
\Language{Fortran, Vax Macro}                \Revised{}
\Cernhead {Routines to Handle Control-C Interrupts on Vax}
These routines allow you to write a program that, when interrupted
with a {\it control-}{\tt C}, resumes execution in a routine that you
specify, which is higher up in the calling tree.
\Structure
Vax Macro and Vax Fortran routines \\
User Entry Names: \Rdef{ASTINT}, \Rdef{ASTXIT}, \Rdef{ASTDCC},
\Rdef{ASTECC}, \Rdef{ASTSCS}, \Rdef{ASTECS} \\
Internal Entry Names: \Rdef{ASTCCH}
\Usage
{\tt VAXAST} should be initialised at the beginning of the program by
\begin{verbatim}
    CALL ASTINT
\end{verbatim}
The routine to which control should be returned after a
{\it control-}{\tt C}
has been typed, should have in its header
\begin{verbatim}
    EXTERNAL ASTCCH
    CALL LIB$ESTABLISH(ASTCCH)
\end{verbatim}
When a {\it control-}{\tt C} is typed on the terminal, {\tt ASTCCH} is
called. This routine is part of {\tt VAXAST}, its main job is to unwind
the stack of routine calls until the routine is found in which the
{\tt LIB\$ESTABLISH} was issued.
Your program then continues execution just after the call to the routine
that was interrupted. You may have several routines with the header
shown above. Only the last call to {\tt LIB\$ESTABLISH} has effect.
\par
When you no longer wish to make use of the {\tt VAXAST} routines:
\begin{verbatim}
    CALL ASTXIT
\end{verbatim}
You may not wish to have {\it control-}{\tt C} trapped all the time,
for instance when the program is waiting for input.
To suspend trapping for a short while, do the following:
\begin{verbatim}
       CALL ASTDCC
       ...
       CALL ASTECC
\end{verbatim}
Between {\tt ASTDCC} and {\tt ASTECC} a {\it control-}{\tt C} typed on
the terminal has the same effect as a {\it control-}{\tt Y}, i.e.
stopping the program and returning to {\tt DCL}.
Execution can, as with {\it control-}{\tt Y}, be resumed at the point it
was interrupted, via the {\tt CONTINUE} command.
\par
Not all programs survive the stack unwind {\tt ASTCCH} performs.
A classical example is the set of I/O routines in the Vax
Fortran run time library (RTL). {\tt VAXAST} replaces those routines by
variants that are stack unwind proof but perform otherwise identical
tasks. You will see 29 messages about multiply defined symbols
when you {\tt LINK} your program, you can safely ignore them.
\newpage
If there is a part in your own program where the stack should not
be unwound but during which you would like a {\it control-}{\tt C} to be
stored, do the following:
\begin{verbatim}
    CALL ASTSCS
    ...
    CALL ASTECS
\end{verbatim}
A {\it control-}{\tt C} typed between the {\tt ASTSCS} and {\tt ASTECS}
calls remains 'dormant' and takes effect only at the {\tt ASTECS} call.
\Notes
1988 C.A.J. Mekenkamp. All Rights Reserved. \\
Carlo Mekenkamp, President Krugerstraat 42, NL-1975 EH IJmuiden.
\\ $\bullet$
