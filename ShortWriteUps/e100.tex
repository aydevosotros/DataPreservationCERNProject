\Version{POLINT}                         \Routid{E100}
\Keywords{POLYNOMIAL INTERPOLATION}
\Author{F. James}                        \Library{KERNLIB}
\Submitter{}                               \Submitted{05.09.1966}
\Language{Fortran}                         \Revised{18.11.1985}
\Cernhead{Polynomial Interpolation}
Subroutine {\tt POLINT} interpolates in a table of arguments $a_j$ and
function values
$ f_j = f(a_j)$, using an interpolating polynomial of
specified degree $K-1$ which passes through $K$ successive tabular
points. The table arguments $a_j$ need not be equidistant.
Meaningful results can usually be obtained only for
small values of $K$ (typically less than 10).
\Structure
{\tt SUBROUTINE} subprogram    \\
User Entry Names: \Rdef{POLINT}\\
Files Referenced:  Printer \\
External References: \Rind{KERMTR}{N001},  \Rind{ABEND}{Z035}
\Usage
\begin{verbatim}
    CALL POLINT(F,A,K,X,R)
\end{verbatim}
\begin{DLtt}{1234}
\item [F]({\tt REAL}) One-dimensional array. {\tt F(j)} must be equal
to the value at {\tt A(j)} of the function to be interpolated,
$(j=1,2,\ldots,\mathtt{K})$.
\item [A]({\tt REAL}) One-dimensional array. {\tt A(j)} must be equal
to the table argument $a_j,(j=1,2,\ldots,\mathtt{K})$.
\item [K]({\tt INTEGER}) {\tt K-1} is the degree of the
interpolating polynomial.
\item [X]({\tt REAL}) Argument at which the interpolating polynomial
is to be evaluated.
\item [R]({\tt REAL}) On exit, {\tt R} is set equal to the value at
{\tt X} of the polynomial passing through the points
$(a_j,f_j),(j=1,2,\ldots,\mathtt{K})$.
\end{DLtt}
If {\tt X} lies outside the range of the points
$\mathtt{A(1),\ldots,A(K)}$, the interpolation becomes an extrapolation,
with consequent loss of accuracy.
\Method
Newton's divided difference formula is used.
\Restrict
$\mathtt{2 \leq K \leq 20}$. If $\mathtt{K>20}$, the interpolation
is performed as if {\tt K} had the value {\tt 20}. The original value of
{\tt K} is unchanged on exit.
\Errorh
Error {\tt E100.1:}  $\mathtt{K < 1}$. A message is printed unless
subroutine {\tt KERSET} (N001) has been called.
\Notes
{\tt POLINT} is intended for interpolation using {\it all} the
tabulated points in the array {\tt A}. To use only the tabulated points
around the value of the argument {\tt X}, use {\tt DIVDIF} (E105).
\\ $\bullet$
