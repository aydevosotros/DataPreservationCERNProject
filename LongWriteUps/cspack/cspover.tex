%.*****************************************************************>
%.* CSPACK Reference Guide - Overview                             *>
%.*  Last Mod.    4 May 1991  12.00   mg                          *>
%.*****************************************************************>
\Filename{H1cspover-GLOSSARY}
\chapter{GLOSSARY}
\par
\Filename{H2cspover-Software-packages-used-in-High-Energy-Physics}
\section{Software packages used in High Energy Physics}
\par
A short description of packages referred to in this document
are given below.
\subsection{ZEBRA - The data structure management system}
\par
\index{ZEBRA}
The data structure management package ZEBRA
was developed at CERN in order to overcome the lack of dynamic
data structure facilities in FORTRAN, the favourite computer language
in high energy physics. It implements the {\bf dynamic
creation and modification} of data structures at execution time
and their transport
to and from external media.
ZEBRA input/output is either by a sequential or direct access method.
Two data representations,
{\bf native} (no data conversion when transferred to/from the
external medium) and {\bf exchange} (a conversion to/from an
interchange format is made if necessary), allow data to be transported between
computers of the same and of different architectures.
\par
Many of the packages described below are based on Zebra.
\subsection{EPIO - A machine independant input/output package}
\par
\index{EPIO}
EPIO is an input/output package still in use by some experiments at
CERN. CSPACK provides remote file transfer and access for EPIO files.
\subsection{KUIP - The user interface package}
\index{KUIP}
\par
The purpose of KUIP
({\bf K}it for a {\bf U}ser
{\bf I}nterface {\bf P}ackage) is to handle
the dialogue between the user and the application program
It parses the commands input into the system, verifies them for
correctness and then hands over control to the relevant action
routines.
\subsection{HBOOK - The histogramming package}
\par
\index{HBOOK}
HBOOK provides a library of FORTRAN callable routines for the manipulation
of histograms, scatter plots, tables and ntuples. These may be stored
on disk files using the RZ direct access routines of the ZEBRA package.
\subsection{PAW - The Physics Analysis Workstation}
\par
The PAW system is widely used by physicists to perform
interactive data analysis and presentation.
It uses the facilities provided by packages such as HBOOK, KUIP
and of course ZEBRA.
\index{PAW}
\subsection{FATMEN - A Distributed File and Tape Management System}
\par
\index{FATMEN}
The FATMEN system provides a fully distributed file catalogue
and file access in a location, operating system and device
independent manner. The ZEBRA RZ package is used to store
the file catalogue information. The CSPACK facilities
are also used by FATMEN for catalogue update distribution,
remote file access and remote data file access.
\subsection{PATCHY - The Source Code Management System}
\par
\index{PATCHY}
PATCHY is a source code management system which has been in use
for many years. Files may be stored in a number of formats:
CARD files, compact binary PAM files or in CETA format. All
of the above formats may be transferred between different
machines by tools in the CSPACK package.
\subsection{CMZ - A Code Management system using ZEBRA}
\par
\index{CMZ}
CMZ is an advanced Code Management system, backward compatible
with PATCHY, that is based on ZEBRA. As with
HBOOK, the ZEBRA RZ package is used to store data on disk.
\Filename{H2cspover-Components-of-the-CSPACK-system}
\section{Components of the CSPACK system}
\subsection{CZ - The ZEBRA Communications Package}
\index{CZ}
\par
The CZ package is a small set of FORTRAN callable
routines used by FATMEN, PAW
and other applications. It provides a simple means of starting a remote
server and then exchanging character or binary data. The actual
communication is performed by TCPAW, running over TCP/IP, or transparent
DECnet task-to-task.
\subsection{XZ - The remote I/O package}
\index{XZ}
\par
XZ is a small package built on top of CZ which permits remote I/O,
such as OPEN, CLOSE, READ, WRITE etc. and remote file transfer.
\subsection{TCPAW - The Networking Package}
\index{TCPAW}
\index{TCP/IP}
\par
TCPAW provides the network layer for many of the tools
in the current CSPACK package is built. It consists
of FORTRAN callable C routines, and is implemented on a variety of
platforms, including VM/CMS, VAX/VMS,
and Unix systems.
\par
TCPAW uses the internet daemon (INETD) to start servers,
except on VM/CMS, where REXEC is used.
\subsection{SYSREQ - The System Service Request Facility}
\par
\index{SYSREQ}
\index{TCPREQ}
SYSREQ is a facility developed at RAL
for generalised inter-system communications. It allows
commands to be sent to, and replies received from, services running
in dedicated service machines under the VM/CMS.
For example, all communication with the HEPVM Tape Management
System (TMS), that was developed at the Rutherford Appleton
Laboratory in the UK and is now running at several of the
larger HEPVM sites, is via SYSREQ.
At CERN, a facility has
been developed to permit remote users use the facilities of SYSREQ,
by forwarding the messages and replies over TCP/IP. This system is
known as SYSREQ-TCP.
\subsection{TELNETG - A extended TELNET program}
\index{TELNET}
\index{TELNETG}
\index{GRAPHICS}
\par
TELNETG is a modified version of the standard TELNET program that
allows the input/output of a HIGZ based graphics session on a remote
system to be displayed in a graphics window on the local workstation.
TELNETG is available for Unix and VAX/VMS systems.
\subsection{TAGIBM - A 3270 terminal emulator}
\index{TAGIBM}
\index{GRAPHICS}
\par
TAGIBM is a powerful 3270 terminal emulator similar to TELNETG
but with full-screen emulation for IBM systems.
\subsection{INETD - the internet daemon}
\index{INETD}
\index{ZSERV}
\index{PAWSERV}
\par
On all systems except VM/CMS and IBM MVS, the server for ZFTP,
distributed PAW and the CZ/XZ FORTRAN routines is started
using the internet daemon (INETD), except between VAX/VMS
systems when the DECnet option is activated.
\par
The inetd daemon is normally started when your system is rebooted.
Once started, the inetd daemon listens for connections on
certain Internet sockets specified in the
/etc/inetd.conf file.  When the inetd
daemon receives a request on one of these  sockets, it determines
what  service  corresponds to that socket and then either handles
the service request itself or invokes the appropriate server,
such as ZSERV or PAWSERV.
\par
A separate process exists for each concurrent connection to a given
host.
\subsection{REXEC - the remote execution daemon}
\index{REXEC}
\index{ZSERV}
\index{PAWSERV}
\par
As INETD is not available for VM/CMS, another solution has to be used.
TCPAW uses the REXEC command to start servers on VM/CMS systems.
The REXEC daemon autologs the machine of the specified user,
having verified the username and password. This means that the
machine in question must not be in use, i.e. logged on or
disconnected. Once the machine is autologged, the ZSERV or
PAWSERV program is started.
\par
If you have problems connecting to a remote VM system,
first check that the account is not in use. If you
still have problems, ensure that your PROFILE EXEC
does not contain any statements which cause it to
run a command, e.g. EXEC MAIL, either unconditionally
or in DISCONNECTED mode.
\Filename{H1cspover-Introduction}
\chapter{Introduction}
\par
Many High Energy physics experiments use some or all of the following
packages:
\begin{UL}
\item
PATCHY or CMZ for code management
\item
Zebra FZ and RZ packages for I/O
\item
PAW, HBOOK for histogramming
\end{UL}
The transfer of the files used by these packages
is often difficult, and network access impossible.
\par
For example, PATCHY PAM files have are normally transferred
between different machines in a special interchange format,
known as CETA. Network access to PAM files between different
hardware platforms is not supported.
The transfer of Zebra files also requires the use of an
interchange format, Zebra binary (or even ASCII) exchange
format. This requires a three step process to transfer a file:
\begin{UL}
\item
Convert to exchange format
\item
Transfer
\item
Convert back to native
\end{UL}
\par
Trasnfer of such files to and from Unix machines is further
complicated by the fact that that data records, when
written by FORTRAN, contain control information which
renders the file unreadable on the remote system and
so a further step is required to add or remote these
control words.
\Filename{H2cspover-CSPACK}
\section{CSPACK}
\par
CSPACK is designed to solve the above problems,
by providing network transfer and access to the
commonly used HEP formats with transparent, on the fly
data conversion. This is performed through a file
transfer program called ZFTP.
\par
In addition, a FORTRAN callable interface allows users
to code their own applications, or call directly
routines that provide complete file transfer
of ASCII, binary, FORTRAN direct-access, Zebra FZ or
RZ and PAM files. Routines for record level access also
exist.
\par
CSPACK also includes other tools and routines for
the distributed computing environment, such as the
TELNETG program, which permits a graphics application,
such as PAW or GEANT, to be run on a remote machine
utilising a graphics window on the local workstation.
\Filename{H1cspover-Positioning}
\chapter{Positioning}
\par
Many of the tools developed in this package were first
written in the framework of the PAW~\cite{bib-PAW} project.
They were extended and enhanced for the FATMEN~\cite{bib-FATMEN}
project. The tools are based on such de-facto standards
as DECnet and TCP/IP sockets. However, new standards
are now emerging which, together with enhancements to HEP
packages, render parts of CSPACK redundant. Some of these
are described below.
\Filename{H2cspover-ZEBRA-RZ-files}
\section{ZEBRA RZ files}
\par
ZEBRA RZ files may now be written in both exchange and native
data formats. For systems that use the IEEE floating point format,
such as most Unix systems including Sun, Apollo, RS6000 etc.,
native and exchange formats are identical. It is recommended
that exchange data format be used whereever possible. Such files
may be transferred between different systems using the standard
{\tt ftp} utility and accessed at the record level using {\tt nfs}.
This obviates the need for the {\tt GETRZ} and {\tt PUTRZ}
commands in {\tt ZFTP}, for example.
\Filename{H2cspover-ZEBRA-FZ-files}
\section{ZEBRA FZ files}
\par
Exchange format has always existed for ZEBRA FZ files.
However, due to limitations of certain FORTRAN implementations,
such files have not been easily transferable to/from these systems.
(FORTRAN typically writes control words at the beginning and end
of each record in sequential files on most Unix systems. These control
words render the file unreadable from other systems across NFS,
or if the file is transferred using FTP without further conversion).
ZEBRA FZ has now been enhanced to provide I/O using the C run time
library (or FORTRAN direct-access I/O). Files written with either
of these options maybe be shared across systems using NFS or
transferred using FTP without further conversion.
\Filename{H2cspover-PATCHY-files}
\section{PATCHY files}
\par
PATCHY files may be kept in binary (PAM) or formatted (CARD) files.
Card files may be shared across systems without problems, unless
certain special characters are used. The use of card files removes
the need for the ZFTP GETP and PUTP commands.
