% file : zreffmt.tex
%
%  Set the page size  A4  =  210mm by 297 mm
%
%  width of text
%
%
%  vertical space
%
%  \setlength{\voffset}{-18mm}
   \setlength{\topmargin}{-15mm}
   \setlength{\headheight}{5mm}
   \setlength{\headsep}{18mm}
   \setlength{\textheight}{237mm}
   \setlength{\baselineskip}{13pt}
   \setlength{\footskip}{15mm}
   \setlength{\footheight}{5mm}
   \parskip 5pt
   \parindent 0pt
%
%  horizontal space
%
%  \setlength{\hoffset}{-13mm}
   \setlength{\textwidth}{140mm}
   \setlength{\oddsidemargin}{6mm}
   \setlength{\evensidemargin}{13mm}
   \setlength{\marginparsep}{5mm}
   \setlength{\marginparpush}{20mm}
   \setlength{\marginparwidth}{15mm}
%
%  macro "in" - move left margin
%
   \def\in#1{\advance\leftskip by #1}
%
%  macro "lile" - line-length, move left and right margins
%
   \def\lile#1{\advance\leftskip by #1 \advance\rightskip by #1}
%
%  macro "spsmall - small size and inter-line spacing compact
%
   \def\spsmall{\small\setlength{\baselineskip}{10pt}}
%
%  macro "lspc" - inter-line spacing compact
%
   \def\lspc{\setlength{\baselineskip}{11pt}}
%
%  macro "lspb" - inter-line spacing narrow
%
   \def\lspb{\setlength{\baselineskip}{12pt}}
%
%  macro "lspa" - inter-line spacing normal
%
   \def\lspa{\normalsize\setlength{\baselineskip}{13pt}}
%
%  Command \Func - box for Fortran FUNCTION - obsolete, use ROUTA
%
   \newcommand{\Func}[1]{\par\vspace*{2mm}
    \framebox[\textwidth]{\rule[-3mm]{0mm}{8mm}\large\tt#1}
    \par\vspace*{1mm}}
%
%  Command \Subr - box for Fortran SUBROUTINE - obsolete, use ROUTA
%
   \newcommand{\Subr}[1]{\par\vspace*{2mm}
    \framebox[\textwidth]{\rule[-3mm]{0mm}{8mm}\large\tt#1}
    \par\vspace*{1mm}}
%
%  Command \ROUTA - box for high-lighting one routine
%
   \newcommand{\ROUTA}[1]{\par\vspace*{2mm}
    \framebox[\textwidth]{\rule[-3mm]{0mm}{8mm}\large\tt#1}
    \par\vspace*{1mm}}
%
%  Commands \ROUTB S M L - boxes for high-lighting two routines
%
   \newcommand{\ROUTBS}[2]{\ROUTBX{40}{#1}{#2}}
   \newcommand{\ROUTBM}[2]{\ROUTBX{20}{#1}{#2}}
   \newcommand{\ROUTBL}[2]{\ROUTBX{10}{#1}{#2}}
%
%        Command \ROUTBX - box for high-lighting two routines
%
   \newcommand{\ROUTBX}[3]{\par%
     \setlength{\unitlength}{1mm}%
     \begin{picture}(150,19)%
       \put(0,0){\framebox(150,16){ }}%
       \put(#1,9){\large\tt #2}%
       \put(#1,3){\large\tt #3}%
     \end{picture}%
     \par\vspace*{1mm}}
%
%  Commands \ROUTC S M L - boxes for high-lighting three routines
%
   \newcommand{\ROUTCS}[3]{\ROUTCX{40}{#1}{#2}{#3}}
   \newcommand{\ROUTCM}[3]{\ROUTCX{20}{#1}{#2}{#3}}
   \newcommand{\ROUTCL}[3]{\ROUTCX{10}{#1}{#2}{#3}}
%
%        Command \ROUTCX - box for high-lighting three routines
%
   \newcommand{\ROUTCX}[4]{\par%
     \setlength{\unitlength}{1mm}%
     \begin{picture}(150,25)%
       \put(0,0){\framebox(150,22){}}%
       \put(#1,15){\large\tt #2}%
       \put(#1,9){\large\tt #3}%
       \put(#1,3){\large\tt #4}%
     \end{picture}%
     \par\vspace*{1mm}}
%
%  Commands  \bva for begin verbatim with 13pt
%
   \newcommand{\bva}{\begin{verbatim}}
%
%  Commands  \bvb for begin verbatim with 12 pt
%
   \newcommand{\bvb}{\setlength{\baselineskip}{12pt}\begin{verbatim}}
%
%  Commands  \bvc for begin verbatim with 11pt
%
   \newcommand{\bvc}{\setlength{\baselineskip}{11pt}\begin{verbatim}}
%
%  Command \ul for underline
%
   \newcommand{\ul}[1]{\underline{#1}}
%
%%
%%%%%%%%%%%%%%%%%%%%%%  verbatims  %%%%%%%%%%%%%%%%%%%%%%%%
%%   \newenvironment{asisx}{\begin{verbatim}}{\end{verbatim}}
%%   \newenvironment{asisy}{\setlength{\baselineskip}{12pt}\begin{verbatim}}{\end{verbatim}}
%%   \newenvironment{asisz}{\setlength{\baselineskip}{11pt}\begin{verbatim}}{\end{verbatim}}
%%   \newcommand{\doasisx}{\begin{asisx}}
%%   \newcommand{\doasisy}{\begin{asisy}}
%%   \newcommand{\doasisz}{\begin{asisz}}
