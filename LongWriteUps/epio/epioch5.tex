%%%%%%%%%%%%%%%%%%%%%%%%%%%%%%%%%%%%%%%%%%%%%%%%%%%%%%%%%%%%%%%%%%
%                                                                 %
%   EPIO User Guide -- LaTeX Source                               %
%                                                                 %
%   Chapter 5                                                     %
%                                                                 %
%   The following external EPS files are referenced:              %
%                                                                 %
%   Editor: Michel Goossens / CN-AS                               %
%   Last Mod.: 25 Nov 1993 23:45 mg                               %
%                                                                 %
%%%%%%%%%%%%%%%%%%%%%%%%%%%%%%%%%%%%%%%%%%%%%%%%%%%%%%%%%%%%%%%%%%%
 
\Filename{H1EPIO-technical-details}
\chapter{Technical details}
\label{sec:H1TechnicalDetails}

\Filename{H2EPIO-technical-details-status-words}
\section{Status words}
\index{status word}

One set of status words per unit will be kept in a common block (which
should reside in the root for overlayed programs). The status words
contain all information the routines have to know in order to operate.
The shorthand notations 'physical header' and 'record header' refer
to the physical and logical headers respectively of the EP format
description.
 
Words marked '*' can be defined and changed by the user through calls
to special routines (the user should not write into the status
word area directly).

r, w, and r/w under 'use' means reading or writing only, or both.

\begin{longtable}{@{}rllp{.84\linewidth}@{}}
\caption{Overview of the status words}\label{tab:statuswords}         \\
1 & * & r/w & Physical block length in 16-bit words.
              Used for writing; should always be a multiple of the 
              ``magic'' 180.
              For the UNIX and ``standard Fortran'' versions it is
              the actual block length.                                \\
  &   &     & For reading, it is the upper limit
              of the EPIO buffer \Lit{IBUF}.                          \\
  &   &     & The user has to provide sufficient space in \Lit{IBUF}. \\
  &   &     & Example: for a buffer length of 1800 16-bit words,
              \Lit{IBUF(480)} is needed on a 60-bit machine.          \\
  &   &     &  Default=1800                                           \\
2 & * &  r  & Maximum logical record length (in machine words).       \\
  &   &     & Default=999999                                          \\
3 & * & r/w & Reading : as read from physical header word 11          \\
  &   &     & Writing : Logical record word length (16 or 32-bits); 
              a word of this length will be called a ``unit'' 
              in the following.                                       \\
  &   &     & Default=16                                              \\
4 & * & r/w & P.h. word 5 (run number), read or written               \\
  &   &     & Default=10101                                           \\
5 & * & w   & If $\geq0$, logical record type identifier 
              (will be used as record header word 2).                 \\
  &   &     & If $\leq0$, for user purposes.                          \\
  &   &     & Default=0                                               \\
6 & * & w   & If $\geq0$, logical record count will be placed 
              in record header word 4                                 \\
  &   &     & If $<0$, record header word 4 will not be set 
              (i.e. for user purposes).                               \\
  &   &     & Default=0                                               \\
7 & * & r/w & Reading : as read from physical header word 2, 
              or 0 if no physical header                              \\
  &   &     & Writing : Actual physical header length. 
              Is automatically set at call to \Rind{EPADDH}.          \\
  &   &     & Default=12 (see control word 6), 24 for 32 bit p.h.     \\
8 & * & w   & Output padding flag, values \Lit{i+j}                   \\
  &   &     & \Lit{i=0} : span logical records over blocks            \\ 
  &   &     & \Lit{i=10}: do not span logical records                 \\
  &   &     & Remark: headers will never be spanned on writing        \\
  &   &     & \Lit{j=1} : pad physical block with zeros to full length
              (i.e. write fixed length blocks)                        \\
  &   &     & \Lit{j=2} : pad up to next magic multiple 
              (see control word 5)                                    \\
  &   &     & \Lit{j=3} : do not pad at all                           \\
  &   &     & Default=1                                               \\
9 & * & r/w & Physical header word 6 (physical record type)           \\
  &   &     & Default=0 (16 bit p.h.), 1 for 32 bit p.h.              \\
10&   & r/w & Logical unit number                                     \\ 
11&   & r/w & Number of blocks read or written                        \\
12&   & r/w & Number of records read or written                       \\ 
13&   & r/w & Number of read or write parities                        \\
14&   & r/w & Actual occupation of I/O buffer (16-bit words)          \\
15&   & r/w & Displacement to start of first logical record 
              (physical header)                                       \\
16&   & r/w & Status indicator: 0 at start, 1 writing, 2 reading.     \\
17&   & r   & Flag for old (=1) or new (=0) EP format. Recognized
              when reading.                                           \\
18&   & r   & Number of headerless blocks still following (from
              physical header word 9)                                 \\
19&   & r   & Normally 0, 1 if last read access resulted in E.O.F.    \\
20&   & r   & Actual number of units read, including header, in
              current logical record (logical record header word 1)   \\
21&   & r   & Logical record header length (logical record header word 3)\\
22&   & r   & Position indicator - Internal in \Rind{EPREAD}          \\
  &   &     & \Lit{=0} : pointer at start of logical record (header)  \\
  &   &     & \Lit{=1} : pointer at start of logical record (data)    \\
  &   &     & \Lit{=2} : pointer at end of physical header            \\
23&   & r   & Internal pointer \Lit{IP1} in \Rind{EPREAD}             \\
24&   & r/w & Internal unit name UNIVAC and Apollo only               \\
25& * & r/w & VAX: Channel address (magnetic tape)                    \\
  &   &     & UNIVAC: -1 if tape, otherwise sector address            \\
  &   &     & IBM: if 0 (default) \Lit{ULP} option, 
              else \Lit{NULP} (see \Rind{IOPACK})                     \\
  &   &     & \Lit{NULP} allows concatenation of files, 
              \Lit{ULP} does not                                      \\
  &   &     & Apollo: must be set before first read \Lit{Stream _$ID} \\
  &   &     & UNIX C I/O: stream address (not user settable!)         \\
26& * & r   & Maximum logical record header length, for calls
              to \Rind{EPREAD} with \Lit{MODE=20} (default 999999)    \\
27& * & r   & User wants automatic byte-swapping if 1 (default) 
              or suppression of it (=0).                              \\
  &   &     & User forces new format if 2                             \\
28&   & r   & Internal usage                                          \\
29& * & r/w & Select physical header record format                    \\
  &   &     & writing: 16 bit headers = 0 (default) or
              32 bit headers = 1                                      \\
  &   &     & reading: as deduced from header words 7 and 8           \\
30& * & w   & Vax tape append option,                                 \\
  &   &     & = 0 (default) new tape file,= 1 append.                 \\
31&   & r   & =Pointer to logical reader header, for random access    \\
32&   & r   & =Internal usage                                         \\
33& * & r/w & Select type of I/O in the UNIX version;
              see section \ref{sec:usageunix} ``Usage on UNIX''.      \\
\end{longtable}

The status words are kept in a common block

\begin{XMP}
      COMMON/EPCOMM/NMUNIT,NWUNIT,NCONT,ISTART,LASTUT,LREF,LIST($dim)
\end{XMP}

with
\begin{DLtt}{123456}
\item[NMUNIT] max. no. of units supported simultaneously, reading plus writing
              (default = 10).
\item[NWUNIT] number of status words per unit according to list above
              (default = 32).
\item[NCONT]  number of overall control words for package 
              according to list below (default = 8).
\item[ISTART] internal reference flag
\item[LASTUT] internal reference flag
\item[LREF]   internal reference flag
\item[LIST]   status word list, with \Lit{$dim = NCONT+NWUNIT*NMUNIT}, 
              this allows the support of more units than the default 10 
              (see below) default value of \Lit{$dim = 350}.
\end{DLtt}
 
\subsection*{Important note}
 
These defaults may change in a later version of the package. Users who
want to use or modify these values (e.g.: to support more than 10
simultaneous units) are urged to check the current values by printing
the start of above COMMON block after a call to \Rind{EPINIT}.

\Filename{H2EPIO-technical-details-control-words}
\section{Control words}
\index{control word}

These control words are kept in the first 8 locations of the
array \Lit{LIST} in common block \Lit{/EPCOMM/}.

\begin{DLtt}{1}
\item[1] no. of units actually in use
\item[2] max. no. of error print messages (default=100)
\item[3] actual number of error print messages
\item[4] no. bits/machine word
\item[5] no. of 16-bit words / recommended unit (default=180)
\item[6] standard physical header length (default=12) in words (16 or 32 bit)
\item[7] number of unit control words accessible to user mods
\item[8] logical print output unit
\end{DLtt}

\newpage

\Filename{H2EPIO-technical-details-remark-logical-record-headers}
\section{Important remark on logical record headers}
\label{sec:remarkslogicalrecords}
\index{logical record}\index{record!logical}

The logical record header consists of 3 words minimum, but the
standard form has four words, being:

\begin{enumerate}
\item the logical record length
\item the logical record type identifier
\item the logical record header length
\item the logical record sequence number.
\end{enumerate}
 
Of these, words 1 and 3 will always be  provided by the output
routines when writing (basic protection).
 
Words 2 and 4 of the logical record will always be assigned the values
described above when the user writes via calls to \Rind{EPOUTS}.
 
For use with \Rind{EPOUTL}, two status words, number 5 and 6, are used to
assign values in the following way:

\begin{quote}
let the status words be \Lit{S5}, \Lit{S6}, \Lit{S12}
and the header words be \Lit{H2}, \Lit{H4},
\end{quote}

then

\begin{DLtt}{12345678}
\item[if S5 $\geq0$] \Lit{H2} is set to \Lit{S5} by \Rind{EPOUTL}
\item[if S5 $< 0$]   \Lit{H2} is not set by \Rind{EPOUTL},
                     i.e. the user should have set it before
                     calling \Rind{EPOUTL}.
\item[if S6 $\geq0$] \Lit{H4} is set to \Lit{S12} (!) by \Rind{EPOUTL}
\item[if S6 $< 0$]   \Lit{H4} is not set by \Rind{EPOUTL},
                     i.e. the user should have set it before
                     calling \Rind{EPOUTL}.
\end{DLtt}

\Filename{H2EPIO-technical-details-remark-padding}
\section{Remark on padding}
\index{padding}

Full padding, partial padding, and no padding have certain
consequences on the different computers, and should therefore be
considered beforehand.
\begin{enumerate}
\item Full padding: all blocks on the output file have the same (user
      specified) length, which should be a multiple of 180 16-bit words.
      This mode will probably work on any computer for
      reading and writing.
\item Partial padding: ensures the blocks to have an integer length when
      counted in machine words (no trailing bits, or incomplete words on
      reading, no extra-bits when writing). This should normally work
      except on computers which can only read and write fixed length blocks.
\item No padding: since in the current package, full machine words are
      written on all computers, this may lead to problems when reading, on
      another machine but normally only if the physical or logical lengths
      are not known. For example, due to extra padding during the transfer,
      records of unknown length may not be processed correctly
      when received through a network.
\end{enumerate}

\newpage
\Filename{H2EPIO-technical-details-error-list}
\section{List of errors}
\index{error codes}

Table~\ref{tab:errorcodes} explains the meaning of the integer \Lit{IERR} 
returned as the last parameter of most EPIO subroutine calls,
zero meaning no error.
For each error two types of routines are quoted: those called by the user,
and in which the error condition was detected, and those in which the
error occurred.
 
Control is always returned to the user, but some of the errors
(marked by * in first column) are so serious that it becomes meaningless
to continue reading or writing on the unit concerned,
at least after a limited number of them.

\newcommand{\RA}{\Rightarrow} 
\begin{longtable}{@{}lllp{.6\linewidth}@{}}
\caption{Overview of the error codes}\label{tab:errorcodes}            \\
\textbf{Error}  & \textbf{Routine} & \textbf{User} & \textbf{Meaning}  \\
                &                  & \textbf{routine} &                \\\hline
\endhead
1   &\Rind{EPBLIN}&\Rind{EPREAD}&End of file on reading or open 
                                 failure on IBM                        \\
    &\Rind{EPBOUT}&\Rind{EPOUTL}&reading or writing                    \\
    &             &\Rind{EPOUTS}&                                      \\
2   &\Rind{EPBLIN}&\Rind{EPREAD}&r/w parity, or I/O error (IBM)        \\
    &\Rind{EPBOUT}&\Rind{EPOUTL}&                                      \\
    &             &\Rind{EPOUTS}&                                      \\
    &             &\Rind{EPRWND}&                                      \\
3   &\Rind{EPBLIN}&\Rind{EPREAD}& end of information on reading, or in
                                  some cases after an open error on IBM\\
4*  &\Rind{EPBLIN}&\Rind{EPREAD}& physical record length $\leq0$       \\
5*  &\Rind{EPBLIN}&\Rind{EPREAD}& physical record length of record just
                                  read \Lit{>} actual length of block 
                                  read or user buffer too small        \\
6   &see $\RA$    &\Rind{EPREAD}& user record chopped (\Lit{IREC} too 
                                  small, status word 2) the actual 
                                  length (including header) can be 
                                  retrieved from status word 20 
                                  using \Rind{EPGETW}                  \\
7*  &see $\RA$    &\Rind{EPREAD}& physical header error; could be a 
                                  record in the old format             \\
8   &see $\RA$    &\Rind{EPOUTS}& invalid mode specified in call       \\
    &             &\Rind{EPOUTL}&                                      \\
    &             &\Rind{EPREAD}&                                      \\
9   &see $\RA$    &\Rind{EPREAD}& call to \Rind{EPREAD} with mode 11, 
                                  12, or 13, without prior call with 
                                  \Lit{MODE=20}                        \\
    &             &             & You can also get this error reading 
                                  OLD format, with \Lit{MODE=20}
                                  followed by 11, 12 or 13             \\
10  &\Rind{EPBLIN}&\Rind{EPREAD}&end-of-run (logical record length = 0)\\
11* &\Rind{EPBOUT}&\Rind{EPOUTS}&unit not declared on JCL card,        \\
    &             &\Rind{EPOUTL}&or wrong \Lit{BLKSIZE} (only IBM)     \\
    &\Rind{EPBLIN}&\Rind{EPREAD}&                                      \\
12  &\Rind{EPBOUT}&\Rind{EPOUTS}&end of volume, or unrecovered write   \\
    &             &\Rind{EPOUTL}&parity (only IBM~\cite{bib-IOPACK})   \\
    &             &\Rind{EPCLOS}&                                      \\
    &             &\Rind{EPRWND}&                                      \\
13  &\Rind{EPUNIT}&\Rind{EPOUTS}&maximum number of units reached       \\
    &             &\Rind{EPOUTL}&                                      \\
    &             &\Rind{EPRWND}&                                      \\
    &             &\Rind{EPADDH}&                                      \\
    &             &\Rind{EPSETW}&                                      \\
    &             &\Rind{EPSETA}&                                      \\
    &             &\Rind{EPREAD}&                                      \\
    &             &\Rind{EPGETA}&                                      \\
    &             &\Rind{EPGETW}&                                      \\
14  &see $\RA$    &\Rind{EPDROP}& unit does not exist                  \\
    &             &\Rind{EPADDH}&                                      \\
    &             &\Rind{EPEND} &                                      \\
15  &see $\RA$    &\Rind{EPOUTS}& logical record header (or complete 
                                  record)                              \\
    &             &\Rind{EPOUTL}& too long to fit in physical block    \\
16  &see $\RA$    &\Rind{EPGETW}& status word address out of range     \\
    &             &\Rind{EPGETA}&                                      \\
    &             &\Rind{EPSETW}&                                      \\
    &             &\Rind{EPSETA}&                                      \\
17  &see $\RA$    &\Rind{EPADDH}& user switches from reading to writing\\
    &             &\Rind{EPREAD}& or vice versa, without rewinding unit\\
    &             &\Rind{EPOUTL}&                                      \\
    &             &\Rind{EPOUTS}&                                      \\
18  &see $\RA$    &\Rind{EPREAD}& displacement to start of logical 
                                  record inconsistent with current 
                                  logical record length, not tested 
                                  by \Rind{EPFRD}                      \\
19  &\Rind{EPOUTL}&\Rind{EPOUTL}& number of words to write $< 0$       \\
    &             &\Rind{EPOUTS}&                                      \\
20  &\Rind{EPOUTL}&\Rind{EPOUTL}& Negative record length or negative or\\
    &             &\Rind{EPOUTS}& zero header length in user call      \\
21  &\Rind{EPREAD}&\Rind{EPREAD}& Old format and \Lit{MODE\(\neq\)2}   \\
22  &\Rind{EPBLIN}&\Rind{EPREAD}& Wrong 32 bit physical header read    \\
23  &\Rind{EPREAD}&\Rind{EPREAD}& Logical record unit neither 16 nor 
                                  32 bit (status word 3).              \\
24  &\Rind{EPOUTL}&\Rind{EPOUTL}&Logical record or header length       \\
    &             &\Rind{EPOUTS}& \Lit{>} 65536 for 16 bit headers     \\
25  &\Rind{EPBLIN}&\Rind{EPREAD}&Error from \Rind{cfseek} reading
                                 in random access mode                 \\
\end{longtable}

\newpage

\Filename{H2EPIO-technical-details-format-description}
\section{Format Description}
\label{sec:EPIOformat}
\index{tape format}

\subsection{Tapes}
\index{logical record}\index{record!logical}
 
Logical records may be of fixed or variable length.
A logical record may be entirely contained in one physical record
or may overflow into one or more physical records.  Normally an
event or trigger corresponds to one logical record but it may equally
well consist of a sequence of related logical records.
 
Physical records may be of fixed or variable length.
 
A fixed record length is recommended (for ease of reading using standard
system and particularly FORTRAN, input routines). A fixed record length
is particularly important if the record has no header specifying the
length.
 
A record length which is a multiple of 360 bytes is recommended
(corresponding to an integral number of 16, 18, 24, 32, 36, 48, 60, 64-bits).
 
The following table indicates the increase in tape utilisation with record
length up to the maximum of 32K bytes for IBM 370~\cite{bib-WYLBUR}.
\begin{table}[h]
\centering
\begin{tabular}{clllll}
\textbf{Usage(\%)}            &
67   & 80     & 88     & 90     & 94      \\
\textbf{Record length (bytes)}&
3600 & 3600*2 & 3600*4 & 3600*5 & 3600*8  \\
\end{tabular}
\caption{Tape usage at 6250 bpi}
\label{tab:tapeusage}
\end{table}
 
\subsection{Files}
 
It is recommended that logical records relating to the same
experimental conditions or 'run' are grouped in one file.
 
For unlabelled tapes the end-of-file (and
end-of-run) is indicated by one EOF mark.
 
For labelled tapes the end-of-file (and end-of-run) is indicated by
an EOF mark and the associated EOF records~\cite{bib-LABTAP}.
 
End-of-data on the tape (and end-of-file) is indicated by two
consecutive EOF marks. It is NOT recommended that files (runs) span
from one tape to another.
 
\textsc{End-of-tape}. It is NOT recommended to write past the EOT reflective
marker, therefore, on detection of the EOT marker, the software should
backspace over a sufficient number of data records to allow the
required end-of-run and EOF records to be written.
 
\textsc{Record formats}.  Logical records and physical records, (i.e. physical
tape blocks) normally consist of a number of header words, followed by
data. In some cases the number of header words in a physical record
may be 0.
 
\textsc{Word lengths}.  All lengths in the physical header are expressed in words
where a word is defined as a 16-bit unsigned integer.
The logical record wordlength is defined in the physical record header.
 
\begin{DLtt}{12345}
\item[byte 0] is defined as the highest order 8-bits of the word,
\item[byte 1] is defined as the next highest order 8-bits, etc.
\item[byte 0] is written to tape before byte 1.
\end{DLtt}

\clearpage
\subsection*{Logical Record Format}
\index{logical record format}\index{record format!logical}
 
A logical record normally has a minimum of three header words followed
by data. The logical record wordlength is defined in the physical record
header. Logical record headers with 32 bit words should have their 16
bit components in ``IBM'' order.

\begin{longtable}{@{}>{\tt}lp{.86\linewidth}@{}}
\caption{Logical Record Format}\label{tab:logicalrecordformat}       \\
\multicolumn{1}{c}{\textbf{Word}} &
\multicolumn{1}{c}{\textbf{Definition}}                              \\
\multicolumn{2}{c}{\textbf{Header}}                                  \\
 1  & The logical record length (\Lit{LRL}), which  should 
      always correspond to the actual number of words in the 
      logical record, except in the following special cases:
      \begin{DLtt}{1234567890}
        \item[LRL=1] The total length of the logical record 
                     is unknown at the time of writing 
                     (the logical record header to tape).
        \item[LRL=0] End-of-data in this run.
        \item[LRL=\(2^{16}-1\)] End-of-data in this physical record
                      (16-bit word all-bits set).
      \end{DLtt}
      Note that even when the logical record wordlength 
      is 16-bits the 16-bits following the last word of the 
      current logical record will,when set, be an unambiguous 
      indication of the end-of-data in the physical record.          \\
 2  & Logical Record Type Identifier     
      It is mandatory,except in the above mentioned special cases 
      (but may simply be set to zero). No attempt is made to 
      standardise on record types as they are rather application 
      dependent.                                                     \\
 3  & Logical Record Header Length (\Lit{LHL}), whose  
      definition is strongly recommended.
      The offline package writes this word and uses it when unpacking.\\
 4  & Logical record sequence number.
      Its use is recommended. (It has been found useful online but 
      is not required by the offline unpack routines.)                \\
... &                                                                \\
\multicolumn{2}{c}{\textbf{Data}}                                    \\
LHL+1 & First Data Word                                              \\
... &                                                                \\
LRL & Last Data Word                                                 \\
\end{longtable}

 
\subsection*{Physical Record Format}
\index{physical record format}\index{record format!physical}
 
Each physical record normally has a variable length physical record
header, followed by data. However, certain records, following a record
with a suitable header may have no header.
The word length may be 16 or 32 bits but in both cases the values of
lengths and pointers are in units of 16 bits.
 
The physical header identifier fields (7 and 8) are used by
the off-line package to identify the header word length, to help in
error detection and recovery, and to decide if byte swapping is
required. For the 32 bit
header words, the least significant 16 bits are in the leftmost 16 bits
of the word, i.e. "IBM format". The off-line package achieves this by
calling \Rind{CTOIBM} before the header is output.
 
\newpage

\begin{longtable}{@{}>{\tt}lp{.86\linewidth}@{}}
\caption{Physical Record Format}\label{tab:physicalrecordformat}\\
\multicolumn{2}{c}{\textbf{Header}}                            \\
\multicolumn{1}{c}{\textbf{Word}} &
\multicolumn{1}{c}{\textbf{Definition}}                        \\
 1 & Physical Record Length        (\Lit{PRL})                 \\
 2 & Physical Record Header Length (\Lit{PHL>SHL}) (see below) \\
 3 & Physical Record Number (\Lit{NR})                         \\
   & This number identifies each record on tape.
     Each physical record counts in the numbering even if 
     there is no physical record header.
     \Lit{NR} is modulo 65536 ($2^{16}$) for 16 bit headers.   \\
 4 & Displacement (\Lit{NS} from first header word to first 
     word of first logical record in present physical record
     (or to end-of-data indicator if no logical record starts 
     in current physical record).                              \\
   & \Lit{LS=PHL}   if logical record starts 
     immediately after header                                  \\
   & \Lit{LS=0\ \ } if ALL data in this record belong to 
     current logical record                                    \\
 5 & Run Number or other Run Identifier. 
     They can be selected by Run number or other Run identifier.\\
 6 & Physical Record Type. 
     It is application dependent and is foreseen to readily
     select data from different sources which may be 
     interleaved on tape.                                      \\   
 7 & Physical Header Identifier Fields 1 =29954 (16 bit),
     522144444 (32 bit)                                        \\
 8 & Physical Header Identifier Fields 2 =31280 (16 bit),
     522144444 (32 bit)                                        \\
   & These Header Identifier fields are present to identify
     extensions to the 'EP' format~\cite{bib-MAG} and to 
     recognise the physical header during error recovery.      \\
 9 & Number of Physical Records following this one, with
     0 length headers. 0 if next record has a header.
     Physical records without headers are \textbf{not} 
     recommended, mainly to help error recovery.               \\
10 & Tape Format Version Number = 8012                         \\
11 & Logical Record word length (in-bits, 16 or 32)
     This is the word length (of header and data words) for all
     logic records which start in this physical record. 
     Initially the unpack
     routines will be written to handle 16 or 32-bit words.    \\
12 & Length of Standard Physical Record Header 
     (\Lit{SHL=12} for 16 bit or \Lit{24} for 32 bit)
     This length (\Lit{SHL}) is to be used as a displacement to 
     access the user header words (independently of any
     future addition to the standard header).                  \\
\multicolumn{2}{c}{\textbf{Data}}                              \\
SHL+1 & Start of User Header Words                             \\
...   &                                                        \\
PHL+1 & Start of data from previous logical record (if any)    \\
...   &                                                        \\
LS+1  & First word of first logical record in this physical 
        record (if any)                                        \\
...   &                                                        \\
...   & Logical record(s) data                                 \\
...   &                                                        \\
PRL   & Last word in physical record                           \\
\end{longtable}

