%%%%%%%%%%%%%%%%%%%%%%%%%%%%%%%%%%%%%%%%%%%%%%%%%%%%%%%%%%%%%%%%%%%
%                                                                 %
%   MINUIT User Guide -- LaTeX Source                             %
%                                                                 %
%   Chapter 6                                                     %
%                                                                 %
%   The following external EPS files are referenced:              %
%                                                                 %
%   Editor: Michel Goossens / CN-AS                               %
%   Last Mod.: 16 Mar. 1992 16:20 mg                              %
%                                                                 %
%%%%%%%%%%%%%%%%%%%%%%%%%%%%%%%%%%%%%%%%%%%%%%%%%%%%%%%%%%%%%%%%%%%
 
\chapter{A complete example}
 
We give here one full example of a real fit, performed first in batch
data-driven mode, then the same fit performed by Fortran calls.

\section{A data-driven fit}

The example job given here is set up for batch processing.
The \texttt{OPEN} statements assign the input and output files, and are
somewhat computer-dependent (those given here are for a Vax).
On many systems, it may be more convenient (or necessary)
to perform the file assignments in JCL rather than from the Fortran,
but whatever the user decides,
the files must be opened and the unit numbers
communicated to Minuit before the call to \Rind{MINUIT}.
 
The same job could be run interactively, in which case the input
and output files would be assigned to the terminal,
and the ``user's data'' listed below, instead of coming from a file,
would be typed in directly to the terminal.

\begin{center}\small\textbf{The User's main program}\end{center}
\begin{alltt}\footnotesize
      PROGRAM DSDQ
      EXTERNAL FCNK0
      OPEN (UNIT=5,FILE='DSDQ.DAT',STATUS='OLD')
      OPEN (UNIT=6,FILE='DSDQ.OUT',STATUS='NEW',FORM='FORMATTED')
CC      CALL MINTIO(5,6,7)   ! Not needed, default values
      CALL MINUIT(FCNK0,0)   ! User routine is called FCNK0
      STOP
      END
\end{alltt}

\begin{center}\small\textbf{The User's FCN}\end{center}
\begin{alltt}\footnotesize
      SUBROUTINE FCNK0(NPAR,GIN,F,X,IFLAG,FUTIL)
      IMPLICIT DOUBLE PRECISION (A-H,O-Z)
      REAL THPLUI, THMINI
      DIMENSION X(*),GIN(*)
C   this subroutine does not use FUTIL
      PARAMETER (MXBIN=50)
      DIMENSION THPLU(MXBIN),THMIN(MXBIN),T(MXBIN),
     +    EVTP(MXBIN),EVTM(MXBIN)
      DATA  NBINS,NEVTOT/ 30,250/
      DATA (EVTP(IGOD),IGOD=1,30)
     +         /11.,  9., 13., 13., 17.,  9.,  1.,  7.,  8.,  9.,
     +           6.,  4.,  6.,  3.,  7.,  4.,  7.,  3.,  8.,  4.,
     +           6.,  5.,  7.,  2.,  7.,  1.,  4.,  1.,  4.,  5./
      DATA (EVTM(IGOD),IGOD=1,30)
     +         / 0.,  0.,  0.,  0.,  0.,  0.,  0.,  0.,  1.,  1.,
     +           0.,  2.,  1.,  4.,  4.,  2.,  4.,  2.,  2.,  0.,
     +           2.,  3.,  7.,  2.,  3.,  6.,  2.,  4.,  1.,  5./
C
      XRE = X(1)
      XIM = X(2)
      DM = X(5)
      GAMS = 1.0/X(10)
      GAML = 1.0/X(11)
      GAMLS = 0.5*(GAML+GAMS)
      IF (IFLAG .NE. 1)  GO TO 300
C                        generate random data
      STHPLU = 0.
      STHMIN = 0.
      DO 200 I= 1, NBINS
      T(I) = 0.1*REAL(I)
      TI = T(I)
      EHALF = EXP(-TI*GAMLS)
      TH =      ((1.0-XRE)**2 + XIM**2) * EXP(-TI*GAML)
      TH = TH + ((1.0+XRE)**2 + XIM**2) * EXP(-TI*GAMS)
      TH = TH -               4.0*XIM*SIN(DM*TI) * EHALF
      STERM = 2.0*(1.0-XRE**2-XIM**2)*COS(DM*TI) * EHALF
      THPLU(I) = TH + STERM
      THMIN(I) = TH - STERM
      STHPLU = STHPLU + THPLU(I)
      STHMIN = STHMIN + THMIN(I)
  200 CONTINUE
      NEVPLU = REAL(NEVTOT)*(STHPLU/(STHPLU+STHMIN))
      NEVMIN = REAL(NEVTOT)*(STHMIN/(STHPLU+STHMIN))
      WRITE (6,'(A)') '  LEPTONIC K ZERO DECAYS'
      WRITE (6,'(A,3I10)') ' PLUS, MINUS, TOTAL=',NEVPLU,NEVMIN,NEVTOT
      WRITE (6,'(A)')
     +  '0    TIME        THEOR+      EXPTL+     THEOR-      EXPTL-'
      SEVTP = 0.
      SEVTM = 0.
      DO 250 I= 1, NBINS
      THPLU(I) = THPLU(I)*REAL(NEVPLU) / STHPLU
      THMIN(I) = THMIN(I)*REAL(NEVMIN) / STHMIN
      THPLUI = THPLU(I)
CCCCC       remove the CCC to generate random data
CCC      CALL POISSN(THPLUI,NP,IERROR)
CCC      EVTP(I) = NP
      SEVTP = SEVTP + EVTP(I)
      THMINI = THMIN(I)
CCC      CALL POISSN(THMINI,NM,IERROR)
CCC      EVTM(I) = NM
      SEVTM = SEVTM + EVTM(I)
      IF (IFLAG .NE. 4)
     + WRITE (6,'(1X,5G12.4)') T(I),THPLU(I),EVTP(I),THMIN(I),EVTM(I)
  250 CONTINUE
      WRITE (6, '(A,2F10.2)') ' DATA EVTS PLUS, MINUS=', SEVTP,SEVTM
C                      calculate chisquare
  300 CONTINUE
      CHISQ = 0.
      STHPLU = 0.
      STHMIN = 0.
      DO 400 I= 1, NBINS
      TI = T(I)
      EHALF = EXP(-TI*GAMLS)
      TH =      ((1.0-XRE)**2 + XIM**2) * EXP(-TI*GAML)
      TH = TH + ((1.0+XRE)**2 + XIM**2) * EXP(-TI*GAMS)
      TH = TH -               4.0*XIM*SIN(DM*TI) * EHALF
      STERM = 2.0*(1.0-XRE**2-XIM**2)*COS(DM*TI) * EHALF
      THPLU(I) = TH + STERM
      THMIN(I) = TH - STERM
      STHPLU = STHPLU + THPLU(I)
      STHMIN = STHMIN + THMIN(I)
  400 CONTINUE
      THP = 0.
      THM = 0.
      EVP = 0.
      EVM = 0.
      IF (IFLAG .NE. 4) WRITE (6,'(1H0,10X,A,20X,A)')
     +  'POSITIVE LEPTONS','NEGATIVE LEPTONS'
      IF (IFLAG .NE. 4) WRITE (6,'(A,3X,A)')
     +    '      TIME    THEOR    EXPTL    CHISQ',
     +    '      TIME    THEOR    EXPTL    CHISQ'
C
      DO 450 I= 1, NBINS
      THPLU(I) = THPLU(I)*SEVTP / STHPLU
      THMIN(I) = THMIN(I)*SEVTM / STHMIN
      THP = THP + THPLU(I)
      THM = THM + THMIN(I)
      EVP = EVP + EVTP(I)
      EVM = EVM + EVTM(I)
C  Sum over bins until at least four events found
      IF (EVP .GT. 3.)  THEN
         CHI1 = (EVP-THP)**2/EVP
         CHISQ = CHISQ + CHI1
         IF (IFLAG .NE. 4)
     +      WRITE (6,'(1X,4F9.3)') T(I),THP,EVP,CHI1
         THP = 0.
         EVP = 0.
      ENDIF
      IF (EVM .GT. 3)  THEN
         CHI2 = (EVM-THM)**2/EVM
         CHISQ = CHISQ + CHI2
         IF (IFLAG .NE. 4)
     +      WRITE (6,'(42X,4F9.3)') T(I),THM,EVM,CHI2
         THM = 0.
         EVM = 0.
      ENDIF
  450 CONTINUE
      F = CHISQ
      RETURN
      END
\end{alltt}

\begin{center}\small\textbf{The user's data to drive Minuit.}\end{center}

\begin{alltt}\footnotesize
set title
FIT DELTA S/ DELTA Q RULE TO LEPTONIC K ZERO DECAYS
parameters
1 'Real(X)' 0. .1
2 'Imag(X)' 0. .1
5 'Delta M'  .535 .01
10 'K Short LT' .892
11 'K Long LT'   518.3
 
fix 5
migr
print 0
set print 0
minos
restore
migrad
minos
set param 5 0.535
fix 5
contour 1 2
stop
\end{alltt}

\section{The same example in Fortran-callable mode.}

The program below takes the place of
the data in the above example.

\begin{center}\small\textbf{The User's main program and subroutine}\end{center}

\begin{alltt}\footnotesize
      PROGRAM DSDQ
C             Minuit test case.  Fortran-callable.
C             Fit randomly-generated leptonic K0 decays to the
C       time distribution expected for interfering K1 and K2,
C       with free parameters Re(X), Im(X), DeltaM, and GammaS.
      IMPLICIT DOUBLE PRECISION (A-H,O-Z)
      EXTERNAL FCNK0
CC    OPEN (UNIT=6,FILE='DSDQ.OUT',STATUS='NEW',FORM='FORMATTED')
      DIMENSION NPRM(5),VSTRT(5),STP(5),ARGLIS(10)
      CHARACTER*10 PNAM(5)
      DATA NPRM /   1   ,    2   ,     5    ,   10     ,  11    /
      DATA PNAM /'Re(X)', 'Im(X)', 'Delta M','T Kshort','T Klong'/
      DATA VSTRT/   0.  ,    0.  ,    .535  ,   .892   ,  518.3 /
      DATA STP  /   0.1 ,    0.1 ,     0.1  ,     0.   ,   0.   /
C        Initialize Minuit, define I/O unit numbers
      CALL MNINIT(5,6,7)
C        Define parameters, set initial values
      ZERO = 0.
      DO 11  I= 1, 5
       CALL MNPARM(NPRM(I),PNAM(I),VSTRT(I),STP(I),ZERO,ZERO,IERFLG)
       IF (IERFLG .NE. 0)  THEN
          WRITE (6,'(A,I)')  ' UNABLE TO DEFINE PARAMETER NO.',I
          STOP
       ENDIF
   11 CONTINUE
C
      CALL MNSETI('Time Distribution of Leptonic K0 Decays')
C       Request FCN to read in (or generate random) data (IFLAG=1)
           ARGLIS(1) = 1.
      CALL MNEXCM(FCNK0, 'CALL FCN', ARGLIS ,1,IERFLG)
C
         ARGLIS(1) = 5.
      CALL MNEXCM(FCNK0,'FIX', ARGLIS ,1,IERFLG)
         ARGLIS(1) = 0.
      CALL MNEXCM(FCNK0,'SET PRINT', ARGLIS ,1,IERFLG)
      CALL MNEXCM(FCNK0,'MIGRAD', ARGLIS ,0,IERFLG)
      CALL MNEXCM(FCNK0,'MINOS', ARGLIS ,0,IERFLG)
         CALL PRTERR
         ARGLIS(1) = 5.
      CALL MNEXCM(FCNK0,'RELEASE', ARGLIS ,1,IERFLG)
      CALL MNEXCM(FCNK0,'MIGRAD', ARGLIS ,0,IERFLG)
      CALL MNEXCM(FCNK0,'MINOS', ARGLIS ,0,IERFLG)
         ARGLIS(1) = 3.
      CALL MNEXCM(FCNK0,'CALL FCN', ARGLIS , 1,IERFLG)
         CALL PRTERR
      CALL MNEXCM(FCNK0,'STOP ', 0,0,IERFLG)
      STOP
      END
 
      SUBROUTINE PRTERR
C   a little hand-made routine to print out parameter errors
      IMPLICIT DOUBLE PRECISION (A-H,O-Z)
C  find out how many variable parameters there are
      CALL MNSTAT(FMIN,FEDM,ERRDEF,NPARI,NPARX,ISTAT)
C   and their errors
      DO 50 I= 1, NPARI
      CALL MNERRS(-I,EPLUS,EMINUS,EPARAB,GLOBCC)
      WRITE (6,45) I,EPLUS,EMINUS,EPARAB,GLOBCC
   45 FORMAT (5X,I5,4F12.6)
   50 CONTINUE
      RETURN
      END
\end{alltt}

The FCN is exactly the same in Fortran-callable mode as in
data-driven mode.
