%%%%%%%%%%%%%%%%%%%%%%%%%%%%%%%%%%%%%%%%%%%%%%%%%%%%%%%%%%%%%%%%%%%
%                                                                 %
%  GEANT manual in LaTeX form                                     %
%                                                                 %
%  Michel Goossens (for translation into LaTeX)                   %
%  Version 1.00                                                   %
%  Last Mod. Jun  8 1993  1300   MG                               %
%                                                                 %
%%%%%%%%%%%%%%%%%%%%%%%%%%%%%%%%%%%%%%%%%%%%%%%%%%%%%%%%%%%%%%%%%%%
\Origin{R.Brun, F.Bruyant, W.Gebel, M.Maire}
\Submitted{01.11.83}                   \Revised{17.12.93}
\Version{Geant 3.16}                   \Routid{HITS200}

\Makehead{Routines to store and retrieve HITS}

\Shubr{GSAHIT}{(ISET,IDET,ITRA,NUMBV,HITS,IHIT*)}
\begin{DLtt}{MMMMMMMM}
\item[ISET]  ({\tt INTEGER}) set number (see below);
\item[IDET]  ({\tt INTEGER}) detector number;
\item[ITRA]  ({\tt INTEGER}) number of the track producing this hit
\item[NUMBV] ({\tt INTEGER}) array of volume numbers corresponding 
to list {\tt NAMESV} of {\tt GSDET};
\item[HITS] ({\tt REAL}) array of values for current hit elements;
\item[IHIT] ({\tt INTEGER}) current hit number, if 0 the
hit has not been stored.
\end{DLtt}

Stores element values for current hit into the data structure {\tt
JHITS}. The values {\tt ISET}, {\tt IDET} and {\tt NUMBV} can be found
in the corresponding variables of common \FCind{/GCSETS/}. These values
are set by the routine \Rind{GFINDS} every time that a particle is in
a defined detector.

\Shubr{GSCHIT}{(ISET,IDET,ITRA,NUMBV,HITS,NHSUM,IHIT*)}
Same action as \Rind{GSAHIT}, but in case the detector identified by
{\tt ISET}, {\tt IDET} and
{\tt NUMBV} contains already a hit for the same track, the routine will
make a cummulative sum for the latest {\tt NHSUM} elements of {\tt JHITS}.
The other previous elements of {\tt JHITS} are replaced.
That facility is particularly interesting in the case of
hits generated into a calorimeter. No packing (i.e. 32 bits per
hit element) should be requested for the
last {\tt NHSUM} hits in \Rind{GSDETH} and for these hits {\tt ORIG} should
be set to 0.

\Shubr{GPHITS}{(CHSET,CHDET)}
\begin{DLtt}{MMMMMMMM}
\item[CHSET] ({\tt CHARACTER*4}) set name,
if {\tt '*'} prints all {\tt JHITS} banks of all sets;
\item[CHDET] ({\tt CHARACTER*4}) detector name,
if {\tt '*'} prints hits in all detectors of set {\tt
CHSET}.
\end{DLtt}
Prints {\tt JHITS} banks for detector {\tt CHDET} of set {\tt CHSET}.
 
\Shubr{GFHITS}{(CHSET,CHDET,NVDIM,NHDIM,NHMAX,ITRS,NUMVS,
                ITRA*,NUMBV*,HITS*,NHITS*)}
 
\begin{DLtt}{MMMMMMMM}
\item[CHSET] ({\tt CHARACTER*4}) set name;
\item[CHDET] ({\tt CHARACTER*4}) detector name;
\item[NVDIM] ({\tt INTEGER}) 1$^{st}$ dimension of arrays {\tt NUMBV} and
{\tt NUMVS}: 1$\leq${\tt NVDIM}$\leq${\tt NV} argument of \Rind{GSDET};
\item[NHDIM] ({\tt INTEGER}) 1$^{st}$ dimension of array {\tt HITS}:
1$\leq${\tt NHDIM}$\leq${\tt NH} argument of \Rind{GSDETH};
\item[NHMAX] ({\tt INTEGER}) maximum number of hits to be returned, this 
should be not larger than the second dimension of array {\tt NUMBV} and
{\tt HITS};
\item[ITRS] ({\tt INTEGER}) number of the selected track,
if {\tt ITRS=0}, all tracks are taken;
\item[NUMVS] ({\tt INTEGER}) is a 1-dimension array of {\tt NVDIM}
elements that contains
the list of volume numbers which identify the selected detector,
0 is interpreted as 'all valid numbers';
\item[ITRA] ({\tt INTEGER}) is a 1-dim array of dimension {\tt NHMAX}
that contains on output
for each hit the number of the track which has produced it;
\item[NUMBV] ({\tt INTEGER}) 2-dim array ({\tt NVDIM,NHMAX})
that containis on output for each hit the
list of volume numbers which identify the detector, all values set to 
0 means that no more volumes are stored;
\item[HITS] ({\tt REAL}) 2-dim array ({\tt NHDIM,NHMAX}) that containis 
{\tt NHITS} hits;
\item[NHITS] ({\tt INTEGER}) number of hits returned, in case the total
number of hits is greater than {\tt NHMAX, NHITS} is set to
{\tt NHMAX+1} and {\tt NHMAX} hits are returned.
\end{DLtt}

This rotine returns the hits produced by track {\tt ITRS} (or by any track) in
the detector {\tt CHDET} identified by the list {\tt NUMVS}
belonging to set {\tt CHSET}.

\begin{DLtt}{MMMMMMMMMMMM}
\item[HITS(1,I)] is element 1 of hit number {\tt I};
\item[NUMBV(1,I)] is volume number 1 of hit number {\tt I};
\item[ITRA(I)] is the track number corresponding to hit number {\tt I};
\end{DLtt}

The arrays {\tt NUMVS, NUMBV, HITS}
and {\tt ITRA} must be dimensioned to:
\begin{verbatim}
    NUMVS(NVDIM)
    NUMBV(NVDIM,NHMAX)
    HITS(NHDIM,NHMAX)
    ITRA(NHMAX)
\end{verbatim}

\Shubr{GFPATH}{(ISET,IDET,NUMBV,NLEV*,LNAM*,LNUM*)}
\begin{DLtt}{MMMMMMMM}
\item[ISET] ({\tt INTEGER}) set number;
\item[IDET] ({\tt INTEGER}) detector number;
\item[NUMBV] ({\tt INTEGER}) array of numbers which identify uniquely
detector number {\tt IDET};
\item[NLEV] ({\tt INTEGER}) number of elements filled of arrays
{\tt LNAM} and {\tt LNUM};
\item[LNAM] ({\tt INTEGER}) array of {\tt NLEV} volume names stored in 
ASCII code in integers;
\item[LNUM] ({\tt INTEGER}) array of {\tt NLEV} copy numbers.
\end{DLtt}

This routine returns the list of volume names and numbers
which identify the complete ancestry in the {\tt JVOLUM} data structure of the
volume corresponding to the detector number {\tt IDET} in set
number {\tt ISET} and which is identified by the volume numbers {\tt NUMBV}
(see \Rind{GFHITS}).
 
\Rind{GFPATH} assumes that the detectors have been declared via 
\Rind{GSDETV} and not \Rind{GSDET}. The main use of \Rind{GFPATH} is to
prepare the lists {\tt LNAM} and {\tt LNUM} required  by the routine
\Rind{GLVOLU} to fill the common \FCind{/GCVOLU/}. Once \FCind{/GCVOLU/}
is properly filled, it is possible to use the {\tt GEANT} routines to
transform from the local to the master reference system and so on.
 
