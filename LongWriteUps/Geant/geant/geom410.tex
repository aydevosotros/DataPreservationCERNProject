%%%%%%%%%%%%%%%%%%%%%%%%%%%%%%%%%%%%%%%%%%%%%%%%%%%%%%%%%%%%%%%%%%%
%                                                                 %
%  GEANT manual in LaTeX form                              %
%                                                                 %
%  Michel Goossens (for translation into LaTeX)                   %
%  Version 1.00                                                   %
%  Last Mod. Jan 24 1991  1300   MG + IB                          %
%                                                                 %
%%%%%%%%%%%%%%%%%%%%%%%%%%%%%%%%%%%%%%%%%%%%%%%%%%%%%%%%%%%%%%%%%%%
\Origin{F.Bruyant, A.C.McPherson}
\Documentation{F.Carminati, M.Lefebvre}
\Submitted{16.12.83}                \Revised{14.12.93}
\Version{Geant 3.16}                \Routid{GEOM410}

\Makehead{Ordering the contents of a volume}

In the case of a mother volume containing a large number of daughters,
tracking can be rather slow. This is due to the fact that each time
{\tt GEANT} requires to know in which daughter it is or will be tracking,
it search through the whole list of daughter volumes.
This is done by {\tt GEANT} creating for every daughter volume a list
which contain pointers to all the daughters of the same mother.

Clearly this may be avoided, because at any step of tracking, the 
coordinates and direction cosines of the current step are known. From this
and the knowledge of the geometry, a restricted list of daughter volumes 
to be searched can be built. This can be accomplished in {\tt GEANT} in
two ways, which are described in this chapter.

{\bf Note}: the user must be aware that the following routines alter the
default search list of daughters of a given mother. A user mistake can
cause wrong transport because {\tt GEANT} does not make any check on the
correctness of the list of neighbours provided by the user.

\section*{Static ordering}
\Shubr{GSNEXT}{(CHMOTH,IN,NLIST,LIST)}
\begin{DLtt}{MMMMMMMM}
\item[CHMOTH] ({\tt CHARACTER*4}) name of the mother volume to be ordered;
\item[IN]     ({\tt INTEGER}) number of the content for which a list
is established;
\item[NLIST]  ({\tt INTEGER}) number of neighbours to be considered 
during tracking;
\item[LIST]  ({\tt INTEGER}) list of neighbours to volume {\tt IN}.
\end{DLtt}

This routine stores a given ordered {\tt LIST} of {\tt NLIST} daughter
volumes to search when leaving the {\tt IN}$^{th}$ daughter
of the mother volume {\tt CHMOTH}.

If {\tt IN=0}, for each content, \Rind{GSNEXT} builds a list limited to
the contents {\tt IN+1} (if it exists), {\tt IN-1} (if it exists) and 
{\tt IN} itself.

\Shubr{GSNEAR}{(CHMOTH,IN,NLIST,LIST)}
\begin{DLtt}{MMMMMMMM}
\item[CHMOTH] ({\tt CHARACTER*4}) name of the mother volume to be ordered;
\item[IN]     ({\tt INTEGER}) number of the content for which a list
is established;
\item[NLIST]  ({\tt INTEGER}) number of neighbours to be considered 
during tracking;
\item[LIST]  ({\tt INTEGER}) list of neighbours to volume {\tt IN}.
\end{DLtt}

This routine stores a given ordered {\tt LIST} of {\tt NLIST} daughter
volumes to search when leaving the {\tt IN}$^{th}$ daughter
of the mother volume {\tt CHMOTH}.
 
If {\tt LIST(1)}=0 the particle is back into the mother when leaving the
{\tt IN}$^{th}$ daughter. This means that the {\tt IN}$^{th}$ is not 
contiguous to any other daughter or to the boundary of the mother.

If {\tt IN}=-1, the mother does not have contents contiguous
to its boundaries (status bit 4 set in mother volume bank for action in
\Rind{GGCLOS}).

If {\tt IN}=0 for each content \Rind{GSNEAR} sets {\tt LIST(1)}=0.

\Rind{GSNEAR} must be called after all contents have been position ( except
when {\tt IN}=-1)

\section*{Dynamic ordering}

The list of neighbours to search when exiting from the {\tt IN}$^{th}$
content may depend also on the direction and position of the particle.
In case where it is necessary, for performance reasons, to exploit also
this information, {\tt GEANT} offers the possibility to the user to 
build a dynamic search list.

\Shubr{GSUNEA}{(CHNAME,ISEARC)}
\begin{DLtt}{MMMMMMMM}
\item[CHNAME] ({\tt CHARACTER*4}) name of the volume where the user search
has to be activated;
\item[ISEARCH] ({\tt INTEGER}) specifies the kind of search list to
be used: a positive value must be specified with this routine to activate
user search lists.
\end{DLtt}
This routine should be called once for every volume where user volume
search is activated.

\Shubr{GUNEAR}{(ISEARC,ICALL,XC,JNEAR)}
\begin{DLtt}{MMMMMMMM}
\item[ISEARCH] ({\tt INTEGER}) number associated to the volume in which
the user search is used, it is the same number set by the user with
\Rind{GSUNEA};
\item[ICALL] ({\tt INTEGER}) type of question that the list of volumes
must answer:
\begin{DLtt}{MMMM}
\item[1] \Rind{GMEDIA}-like call, where am I?
\item[2] \Rind{GTNEXT}-like call, where can I go?
\end{DLtt}
\item[XC] ({\tt REAL}) array of 6 containing the position and the
direction cosines of the particle ($x$, $y$, $z$, $p_x/p$, $p_y/p$, $p_x/p$);
\item[JNEAR] ({\tt INTEGER}) pointer to the volume list bank which has
to be filled by the user;
\end{DLtt}

The list of volumes where {\tt GEANT} has to search to answer the question
specified by {\tt ICALL} is returned by the user starting at {\tt Q(JNEAR+1}.
{\tt GEANT} will only look at the volumes specified by the user and in
the order in which they appear in the list. Daughters are numbered from 1
to N according to the order with which they have been positioned in the
mother. The list should be filled in the following way:

\begin{tabular}{lcp{7cm}}
{\tt IQ(JNEAR+1)} & = & {\tt N}, number of volumes in the list \\
{\tt IQ(JNEAR+1+1)} & = & number of the first daughter to search \\
{\tt IQ(JNEAR+1+2)} & = & number of the second daughter to search \\
.\\
.\\
.\\
{\tt IQ(JNEAR+1+N)} & = & number of the N$^{th}$ daughter to search 
\end{tabular}

The user should be aware that this routine is called very often, almost
at every tracking step, so it should be coded with the maximum efficiency
in mind.  An example of \Rind{GUNEAR} could be the following:

\begin{verbatim}
      SUBROUTINE GUNEAR(ISEARC,ICALL,XC,JNEAR)
*---              Make sure to add GEANT main store
+SEQ, GCBANK.
      DIMENSION XC(6), MYLIST(100)
*---              Executable code
      IF(ISEARC.EQ.1) THEN
*---              Build a list using XC and ISEARC for a GMEDIA type call
*---              Put all the daughters where the particle may be in
         MYLIST(1) = ....
         .
         .
         .
         MYLIST(N) = ....
      ELSE 
*---              Build a list using XC and ISEARC for a GTNEXT type call
*---              Put all the daughters where the particle may be going
         MYLIST(1) = ....
         .
         .
         .
         MYLIST(N) = ....
      ENDIF
*---              Return the information to GEANT
      DO 10 I=1,N
         IQ(JNEAR+1+I) = MYLIST(I)
  10  CONTINUE
      IQ(JNEAR+1) = N
*---              End of GUNEAR
      END
\end{verbatim}
