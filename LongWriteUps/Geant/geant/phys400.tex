%%%%%%%%%%%%%%%%%%%%%%%%%%%%%%%%%%%%%%%%%%%%%%%%%%%%%%%%%%%%%%%%%%%
%                                                                 %
%  GEANT manual in LaTeX form                                     %
%                                                                 %
%  Version 1.00                                                   %
%                                                                 %
%  Last Mod. 12 June 1993 18:20  MG                               %
%                                                                 %
%%%%%%%%%%%%%%%%%%%%%%%%%%%%%%%%%%%%%%%%%%%%%%%%%%%%%%%%%%%%%%%%%%%
\Origin{G.N.Patrick}
\Version{Geant 3.16}\Routid{PHYS400}
\Submitted {30.03.82} \Revised{16.12.93}
\Makehead{Simulation of particle decays in flight}
\section{Subroutines}
\Shubr{GDECAY}{}
 
\Rind{GDECAY} is the control routine for the simulation of particle
decays in flight. For a given parent particle it selects from a list a
two- or three-body decay mode using the known branching ratios and calls the
routines needed to generate the vertex and secondary tracks.
It used the following input and output:
\begin{DLtt}{MMMMMMM}
\item[input:]  via common blocks \FCind{/GCTRAK/} and \FCind{/GCKINE/}
\item[output:] via common block  \FCind{/GCKING/}
\end{DLtt}
\Rind{GDECAY} is called by the tracking routines.
It calls the subroutines 
\Rind{GDECA2} for two-body decay, \Rind{GDECA3} for three-body decay,
\Rind{GLOREN} for Lorentz transformation and \Rind{GDROT} for rotation.
For the documentation of \Rind{GLOREN} and \Rind{GDROT}, see {\tt [PHYS410]}.
\Shubr{GDECA2}{(XM0,XM1,XM2,PCM)}
\begin{DLtt}{MMMMMMMM}
\item[XM0] ({\tt REAL}) mass of the parent particle
\item[XM1] ({\tt REAL}) mass of the first decay product
\item[XM2] ({\tt REAL}) mass of the second decay product
\item[PCM(3,4)] ({\tt REAL}) array containing the four-vectors of
the decay products
\end{DLtt}
{\tt GDECA2} simulates the two-body decay with isotropic angular
distribution in the center-of-mass system. It is called from
{\tt GDECAY}.
\Shubr{GDECA3}{(XM0,XM1,XM2,XM3,PCM)}
\begin{DLtt}{MMMMMMMM}
\item[XM0] ({\tt REAL}) mass of the parent particle
\item[XM1] ({\tt REAL}) mass of the first decay product
\item[XM2] ({\tt REAL}) mass of the second decay product
\item[XM3] ({\tt REAL}) mass of the third decay product
\item[PCM(3,4)] ({\tt REAL}) array containing the four-vectors of
the decay products
\end{DLtt}
{\tt GDECA3} simulates the three-body decay 
with isotropic angular distribution in the center-of-mass system.
It is called from {\tt GDECAY}.

\section{Method}
\begin{itemize}
\item
Upon entry to \Rind{GDECAY} a binary search is made in a list of
parent particles. This list is stored in the {\tt JPART} structure
and currently contains the particles defined in \Rind{GPART}
{\tt [CONS300]}. If the current particle cannot be found in the list,
control is returned {\it without} any decay generation.
Up to six decay modes and their corresponding branching ratios are then
extracted from the {\tt JPART} data banks (see {\tt [CONS310]}).
\item
A decay channel is selected according to the branching ratios.
If the sum
of the branching ratios for a particle is not equal 100\%, it is
possible that {\it no} decay is selected.
\item
Depending on whether the two- or three-body decay is selected, either
\Rind{GDECA2} or \Rind{GDECA3} is called to generate the
four-momenta of the decay products with isotropic angular
distribution in the center-of-mass system.
\item
The momentum vectors of the decay products 
are transformed into the laboratory 
system and rotated back into the {\tt GEANT} coordinate frame.
The kinematics of the products is stored in the common \FCind{/GCKING/}.
\item
When a particle decays and no branching ratio is defined, then
\Rind{GDECAY}
calls the user routine \Rind{GUDCAY}.
\end{itemize}
