%%%%%%%%%%%%%%%%%%%%%%%%%%%%%%%%%%%%%%%%%%%%%%%%%%%%%%%%%%%%%%%%%%%
%                                                                 %
%  GEANT manual in LaTeX form                              %
%                                                                 %
%  Michel Goossens (for translation into LaTeX)                   %
%  Version 1.00                                                   %
%  Last Mod. Jan 24 1991  1300   MG + IB                          %
%                                                                 %
%%%%%%%%%%%%%%%%%%%%%%%%%%%%%%%%%%%%%%%%%%%%%%%%%%%%%%%%%%%%%%%%%%%
\Origin{W.Mitaroff}
\Submitted{21.02.85}               \Revised{19.12.93}
\Version{Geant 3.16}               \Routid{HITS510}

\Makehead{Digitisation for drift chambers}

\Shubr{GCDRIF}{(RADD,ZMIN,ZMAX,DETREP,HITREP,IOUT*)}
\begin{DLtt}{MMMMMMMM}
\item[RADD] ({\tt REAL}) radius of cylindrical chamber in cm;
\item[ZMIN] ({\tt REAL}) $z$ of lower end of cylindrical chamber;
\item[ZMAX] ({\tt REAL}) $z$ of upper end of cylindrical chamber;
\item[DETREP] ({\tt REAL}) array of 8 with detector description:
\begin{DLtt}{MMMM}
\item[1] number of wires;
\item[2] wire spacing in $\phi$ (radians);
\item[3] cosine of wire angle with respect to the $z$ axis;
\item[4] sine of wire angle  with respect to the $z$ axis 
(signed like $d\phi/dz$);
\item[5] $d\phi/dz$ along wire;
\item[6] $\phi$ of point with $z=0$ on wire 1;
\item[7] drift velocity (cm nsec$^{-1}$);
\item[8] if $>0$ user routine \Rind{GUDTIM}
will be called to calculate drift time;
\end{DLtt}
\item[HITREP] ({\tt REAL}) array of 4 describing the track:
\begin{DLtt}{MMMM}
\item[1] $\phi$ coordinate of intersection;
\item[2] $z$ coordinate of intersection;
\item[3] $d\phi/dr$;
\item[4] $dz/dr$;
\end{DLtt}
\item[IOUT] ({\tt INTEGER}) array of 4 with digitisation information:
\begin{DLtt}{MMMM}
\item[1] wire number (1...{\tt NWI} with increasing phi), -1 if {\tt DETREP}
parameters are inconsistent;
\item[2] drift time in nsec, $>0$ if $\phi(hit)>\phi(wire)$;
\item[3] digitised current division information
(relative position of charge along wire, per mille);
\item[4] amount of charge deposited onto wire.
\end{DLtt}
\end{DLtt}
Digitisation routine for a cylindrical drift chamber.
 
\begin{figure}[hbt]
     \centering
%    \epsfig{file=eps/hits510-1.eps,width=14cm}
\begin{verbatim}
                        Charge                     
         .              |                        .
         |              .                        |
         =========================================  SENSE WIRE
     ...................................................> Z (cm)
         Z              Z                        Z 
          l                                       u
     ...............................................> ICD (0<ICD<1000)
         0              ICD                      1000
               ICD                (1000-ICD)
\end{verbatim}
     \caption{Coordinate system along the wire}
     \label{fg:hits510-1}
\end{figure}

Knowing the position $Z$ of the deposit of charge we can calculate
\[
\mbox{\tt ICD} = L \frac{Z-Z_l}{Z_u -Z_l} 
\]
where $L=1000$ in the
program. This is the information stored into {\tt IOUT(3)}.

\Shubr{GCDERR}{(ICD*,ERP,ERS)}
\begin{DLtt}{MMMMMMMM}
\item[ICD] ({\tt INTEGER}) digitised current division information 
($\leq\mbox{\tt ICD}\leq1000$), overwritten on output with the
modified value taking into account the errors;
\item[ERP] ({\tt REAL}) variance of Gaussian pedestal errors 
on the measured
pulse heights relative to the sum of the pulse heights;
\item[ERS] ({\tt REAL}) variance of Gaussian slope 
errors on the measured
pulse heights relative to the each pulse heights.
\end{DLtt}
Routine to calculate the error on the current division
information as obtained by \Rind{GCDRIF}.
Here we assume that {\tt ICD} has been determined by measuring the pulse heights 
$I_1, I_2$ at the two ends of the wire with the formula:
\[
\mbox{\tt ICD} = L \; \frac{I_2}{I_+} 
\mbox{\hspace{1cm}with\hspace{1cm}}  I_+ = I_1+I_2 
\]
Its error is determined by:
\[
\delta \mbox{\tt ICD} = - \frac{\mbox{\tt ICD}}{I_+} \delta I_1 + 
\frac{L-\mbox{\tt ICD}}{I_+}  \delta I_2
\mbox{\hspace{5mm}and\hspace{5mm}}
\delta I_1 = \delta_1 + \epsilon_1 \; I_1  \mbox{\hspace{3mm};\hspace{3mm}}
\delta I_2 = \delta_2 + \epsilon_2 \; I_2 
\]
 
 
$\delta_1$ and $\delta_2 $ are of dimension {\tt [I]} and represent the 
{\it pedestal} errors. $\epsilon_1$ and $\epsilon_2$ are the {\it slope} errors.
Errors are independent (no correlations), with a Gaussian distribution with 
average 0 and {\tt ERP} as relative variance for pedestals $\delta_i/I_+$ 
and {\tt ERS} as variance for slopes $\epsilon_i$.  This gives the final result
 
\[
\delta \mbox{\tt ICD}   = \underbrace{-\frac{\delta_1}{I_+}\mbox{\tt ICD} +
                 \frac{\delta_2}{I_+}(L-\mbox{\tt ICD})}_{\mbox{\it pedestals}}
               +
 \underbrace{(\epsilon_2-\epsilon_1)
\frac{\mbox{\tt ICD}(L-\mbox{\tt ICD})}{L}}_{\mbox{\it slope}}
\]

\Rind{GCDERR} sets the {\tt ICD} obtained from \Rind{GCDRIF} to
$\mbox{\tt ICD}=\mbox{\tt ICD}+\delta \mbox{\tt ICD}$ with 
$ 0 \leq \mbox{\tt ICD}\leq L$.

\Sfunc{GUDTIM}{VALUE = GUDTIM(DETREP,HITREP,IW1,DIS)}
The arguments have the same meaning than for \Rind{GCDRIF} apart from:
\begin{DLtt}{MMMMMMMM}
\item[IW1] ({\tt INTEGER}) wire number which will generate a signal;
\item[DIS] ({\tt REAL}) distance from the track to the wire;
\end{DLtt}
This function has to be written by the user to return the drift time
in nanoseconds.
