%%%%%%%%%%%%%%%%%%%%%%%%%%%%%%%%%%%%%%%%%%%%%%%%%%%%%%%%%%%%%%%%%%%
%                                                                 %
%  GEANT manual in LaTeX form                              %
%                                                                 %
%  Michel Goossens (for translation into LaTeX)                   %
%  Version 1.00                                                   %
%  Last Mod. Jan 24 1991  1300   MG + IB                          %
%                                                                 %
%%%%%%%%%%%%%%%%%%%%%%%%%%%%%%%%%%%%%%%%%%%%%%%%%%%%%%%%%%%%%%%%%%%
\Origin{L.Urb\'{a}n}
\Version{Geant 3.16}\Routid{PHYS441}
\Submitted {26.10. 84}   \Revised {11.10.93}
\Makehead{Simulation of discrete bremsstrahlung by muons}
\section{Subroutines}
\Shubr{GBREMM}{}
\Rind{GBREMM} generates a photon from bremsstrahlung of a highly 
energetic muon as a discrete process. For the 
angular distribution of the photon, is calls the function \Rind{GBTETH}.

\begin{DLtt}{MMMMMMMM}
\item[Input:] via common \Rind{/GCTRAK/};
\item[Output:] via common \Rind{/GCKING/}.
\end{DLtt}
 
\Rind{GBREMM} is called from the tracking routine 
\Rind{GTMUON} when the muon reaches its radiation 
point during the tracking stage of {\tt GEANT}.

\Sfunc{GBTETH}{VALUE = GBTETH(ENER,PARTM,EFRAC)}

\begin{DLtt}{MMMMMMMM}
\item[ENER] ({\tt REAL}) kinetic energy of the muon;
\item[PARTM] ({\tt REAL}) mass of the radiating particle ($m_{\mu}$ in
this case);
\item[EFRAC] ({\tt REAL}) ratio of the energy of the radiated photon
to the energy of the muon.
\end{DLtt}

\Rind{GBTETH} calculates the angular distribution of the \Pem\Pep pair
in photon pair production and of the photon in bremsstrahlung.
In case of bremsstrahlung it gives the scaled angle ($\eta = E\theta
m^{-1}$) of the photon.

\section{Method}
 
The differential cross-section for the emission of a photon of energy $k$ by
a muon of energy $E$ is \cite{bib-LOHM,bib-MARM}:
 
\begin{equation}
 \frac{d \sigma} {dv}= \frac{N} {v} \left (
 \frac{4} {3} -\frac{4} {3}  v+v^2  \right )  \Phi ( \delta )
\end{equation}
where
\begin{eqnarray*}
& & \begin{array}{LcL@{\hspace{1cm}}LcL}
N & = & \mbox{normalisation factor} & v  & = & \frac{k}{E}   \\
\delta   & = & \frac {m_{\mu}^2}{2E}\frac{v} {1-v} & e & = & 2.718\dots 
\end{array} \\ [0.2cm]
\Phi(\delta)  & = & \left \{
\begin{array}{LlL}
\ln\left(\frac{189 m_{\mu}}{m_{e} Z^{1/3}}\right)
        -\ln\left(\frac{189\sqrt{e}} {m_{e} Z^{1/3}}\delta +1\right)
                   & \mbox{if}& Z \leq 10    \\ [0.4cm]
\ln\left(\frac{189 m_{\mu}}{m_{e} Z^{1/3}}\right)
        -\ln\left(\frac{189\sqrt{e}} {m_{e} Z^{1/3}}\delta +1\right)
+\ln \left(\frac{2}{3}Z^{-1/3} \right)
                   & \mbox{if}& Z > 10 
\end{array} \right . \\ [0.2cm]
v_c          & = & \frac{k_c}{E}\leq v \leq
                   \left( 1- 0.75\sqrt{e}\frac{m_{\mu}}{E}Z^{1/3} \right)
                   = v_{max}
\end{eqnarray*}
Therefore, the differential cross-section can be written as
\begin{equation}
 \frac{d \sigma} {dv} = f(v) g(v)
\end{equation}
with
\begin{eqnarray*}
f(v) & = & \left[v \ln \left( \frac{v_{max}}{v_c}\right)\right]^{-1} \\
g(v) & = & \frac{1}{\Phi(0)} \left(1-v +\frac{3}{4}v^2\right)\Phi(\delta)
\end{eqnarray*}
We can sample the photon energy in the following way
($r_1$, $r_2$ uniformly distributed random numbers in $]0,1[$):
 
\begin{enumerate}
\item sample $v$:
\[
v= v_c \left(\frac{v_{max}}{v_c}\right)^{r_1}
\]

\item compute the rejection function $g(v)$ and:
\begin{enumerate}
\item if $ r_2 > g(v)$ go back to step 1
\item if $ r_2 \leq g(v)$, accept $v$ and $k = v E$.
\end{enumerate}
\end{enumerate}

After the successful sampling of $k$, \Rind{GBREMM} generates the polar
angles of the radiated photon with respect to an axis defined along the parent
muon's momentum calling \Rind{GBTETH}. For more information on the sampling
procedure see {\tt [PHYS341]}.
