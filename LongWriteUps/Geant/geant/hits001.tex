%%%%%%%%%%%%%%%%%%%%%%%%%%%%%%%%%%%%%%%%%%%%%%%%%%%%%%%%%%%%%%%%%%%
%                                                                 %
%  GEANT manual in LaTeX form                              %
%                                                                 %
%  Michel Goossens (for translation into LaTeX)                   %
%  Version 1.00                                                   %
%  Last Mod. Jan 24 1991  1300   MG + IB                          %
%                                                                 %
%%%%%%%%%%%%%%%%%%%%%%%%%%%%%%%%%%%%%%%%%%%%%%%%%%%%%%%%%%%%%%%%%%%
\Documentation{F.Bruyant}
\Submitted{15.08.84}         \Revised{17.12.93}
\Version{Geant 3.16}         \Routid{HITS001}
\Makehead{The detector response package}
\section{Introduction}
In the context of {\tt GEANT}:
\begin{itemize}
\item {\bf hit} is the user-defined {\it information} recorded at 
tracking time to keep track of the interaction between one
particle and a given detector,
and necessary to compute the digitisations later.
\item  {\bf digitisation} ({\it digit}) is the user-defined
{\tt information} simulating the response
of a given detector element, usually estimated
after tracking a complete event.
\end{itemize}
The detector response package consists of tools to store,
retrieve or print the information relevant to hits and
digitisation which is in the data structures
{\tt JSET, JHITS} and {\tt JDIGI}.
A few subroutines which may
help the user to solve some of the usual digitisation problems
in simple detectors
have been added to the package, e.g. the intersection of a
track with a plane or a cylinder
and the digitisation of conventional drift and {\tt MWP} chambers.

For complex setups with different types of detectors
the user has normally to define
several types of hits and digitisations.
In addition to the hits generated by all particles of the
current event, computing the digitisations
requires usually some information about the intrinsic
characteristics and performance of the detectors.
The information to be recorded for the hits and digitisations
is highly experiment dependent, therefore only a framework
can be proposed to store it.
 
Two remarks can be made:
\begin{itemize}
\item during the life of an experiment, 
the stability of the format and content of the information to be stored
is usually reached much earlier for the hits than for the digitisations.
Therefore the user may save computing time
by designing an intermediate event output at the hits level.
\item  the scheme proposed for storing the digitisations
should in any case be considered as
an intermediate stage, a further processing of the data being necessary
if the user wants to
simulate more closely the specific format of the real
data-acquisition system.
\end{itemize}

\section{{\tt SET}s and {\tt DET}ectors}

The reader is assumed to be familiar with the way the
geometrical setup is described ({\tt [GEOM]}), in particular
with the concepts of logical and physical volume tree structure.

The user is required to classify into sets all sensitive detectors
(defined as those volume defined as detector via \Rind{GSDET}/\Rind{GSDETV})
for which he wants to store hits in the data structure {\tt JHITS}.
The 4-character names which identify the sets are user defined,
and the list of sets
which the user wants to activate for a given run can be entered
through the data record {\tt SETS}.
The user can group together in one or in several sets
detectors of the same or different types. For convenience,
it is recommended to have at least one set for
each main component of the setup, e.g. hadronic calorimeters,
electromagnetic calorimeters, vertex chamber, etc.

A volume can be declared as a sensitive detector through the tracking
medium parameter {\tt ISVOL},
and allocated to a set through the subroutine \Rind{GSDET} or
\Rind{GSDETV}.
Each (logical) sensitive detector is identified by the 4-character
name of the corresponding volume. As a given volume
may describe several similar detectors in the physical setup,
some additional information is needed for associating
correctly the hits with the physical detectors.

When using \Rind{GSDET} the user has to enter the (shortest) list of volume
{\bf names} (the vector {\tt CHNMSV}), which permits unambiguous
identification of the path through the physical tree,
even in the presence of multiple copies.
This identification is obtained by specifying a list of volume
{\bf numbers} (the vector {\tt NUMBV}), in a one to one
correspondence with the list of volume names.
This list, after packing, will constitute the
identifier of the physical detector.

If \Rind{GSDETV} is used instead of
\Rind{GSDET} then the routine
\Rind{GGDETV} (called by \Rind{GGCLOS}) constructs the lists
{\tt CHNMSV} automatically and stores them in the structure {\tt JSET}.

\section{The user tools}
 
The data structure {\tt JSET} is built through
calls to the routine  \Rind{GSDET} or \Rind{GSDETV}
which assign detectors to
sets and define their parameters. After this, the
following routines can be called, for each detector, to complete the structure:
\begin{DLtt}{MMMMMMMM}
\item[\Rind{GSDETH}] provides the parameters required for the storage of the
hit elements in the data structure {\tt JHITS},
such as the packing and scaling conventions;
\item[\Rind{GSDETD}] provides the parameters required for the storage of
the digitisations in the structure {\tt JDIGI},
such as the packing conventions;
\item[\Rind{GSDETU}] adds the user parameters, which may consist,
for instance, of the intrinsic detector
characteristics needed for computing the digitisations.
\end{DLtt}

To permit storage of more than one type of hit for a given sensitive
detector, or to provide additional detector entries,
detector {\it aliases} can be defined through calls to the routine
\Rind{GSDETA}. They are entered in the {\tt JSET} structure as new detectors,
with the same geometrical characteristics as the original one.
The user has the possibility to call appropriate routines
\Rind{GSDETH}, \Rind{GSDETD} and \Rind{GSDETU} for this new detector.

During the tracking, for each step inside the
sensitive detectors, under control of the subroutine
\Rind{GUSTEP}, the hits can be stored in the data structure
{\tt JHITS} through the subroutine \Rind{GSAHIT} (or \Rind{GSCHIT}, more
appropriate for calorimetry).
For each hit the information consists of:
\begin{itemize}
\item the reference to the track in the structure {\tt JKINE};
\item the packed identifier of the physical detector;
\item the packed data for the different elements of the hit.
\end{itemize}

When the tracking has been completed for the whole
event the digitisations can be
computed in the user subroutine \Rind{GUDIGI} which
may extract the hits with the subroutine \Rind{GFHITS} and
store the digitisations in the data structure {\tt JDIGI}, with
the subroutine \Rind{GSDIGI}.
For each digitisation the information should at least consist of:
\begin{itemize}
\item the reference to the track(s);
\item the packed identifier of the physical detector;
\item the packed data for the digitisation itself.
\end{itemize}

\section{Retrieval of geometrical information}
 
The packed identifier of a physical detector, stored as part of the hit
(or digitisation) information, is returned unpacked by the routine
\Rind{GFHITS} (or \Rind{GFDIGI}) which extracts the
information from the {\tt JHITS} or {\tt JDIGI} structures,
and may be used to retrieve the geometrical characteristics
of the given detector.

If the detectors have been defined by the routine \Rind{GSDETV}, the
geometrical information can be retrieved by the routines \Rind{GFPATH}
and \Rind{GLVOLU}. \Rind{GFPATH} prepares the lists {\tt CHNAM} and {\tt LNUM} 
required by \Rind{GLVOLU} ({\tt [GEOM001]}), from the information
preprocessed at initialisation time by the routine \Rind{GGDETV} and
stored in the structure {\tt JSET}
