%%%%%%%%%%%%%%%%%%%%%%%%%%%%%%%%%%%%%%%%%%%%%%%%%%%%%%%%%%%%%%%%%%%
%                                                                 %
%  GEANT manual in LaTeX form                              %
%                                                                 %
%  Michel Goossens (for translation into LaTeX)                   %
%  Version 1.00                                                   %
%  Last Mod. Jan 24 1991  1300   MG + IB                          %
%                                                                 %
%%%%%%%%%%%%%%%%%%%%%%%%%%%%%%%%%%%%%%%%%%%%%%%%%%%%%%%%%%%%%%%%%%%
\Origin{R.Brun}
\Submitted{01.06.83}        \Revised{15.12.93}
\Version{Geant 3.10}        \Routid{TRAK400}
\Makehead{Handling of track space points}

\Shubr{GSXYZ }{}
Stores the position of the current particle from common \FCind{/GCTRAK/} into
the data structure {\tt JXYZ}.
 
\Shubr{GPJXYZ}{(NUMB)}
 
\begin{DLtt}{MMMMMMMM}
\item[NUMB] ({\tt INTEGER}) track number, all tracks if =0;
\end{DLtt}
Prints space points stored in the data structure {\tt JXYZ}
for track number {\tt NUMB}.
 
\Shubr{GPCXYZ}{}
Prints tracking and physics parameters after
the current step. This routine can be called from \Rind{GUSTEP}.
 
Some of these routines are called by \Rind{GDEBUG}, see {\tt [BASE400]}
for more information.
