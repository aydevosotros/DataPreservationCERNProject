%%%%%%%%%%%%%%%%%%%%%%%%%%%%%%%%%%%%%%%%%%%%%%%%%%%%%%%%%%%%%%%%%%%
%                                                                 %
%  GEANT manual in LaTeX form                                     %
%                                                                 %
%  Michel Goossens (for translation into LaTeX)                   %
%  Version 1.00                                                   %
%  Last Mod. Jan 24 1991  1300   MG + IB                          %
%                                                                 %
%%%%%%%%%%%%%%%%%%%%%%%%%%%%%%%%%%%%%%%%%%%%%%%%%%%%%%%%%%%%%%%%%%%
\Origin{M.S. Dixit}
\Revision{G.Lynch} 
\Documentation{F.Carminati} 
\Submitted {17.06.85}  \Revised {16.12.93}
\Version{Geant 3.16}\Routid{PHYS325}
\Makehead {Moli\`ere  scattering}
\section{Subroutines}
\subsection{Initialisation}

\Shubr{GMULOF}{}
\Rind{GMULOF} calculates the table of maximum
steps corresponding to the
energies in the array {\tt ELOW} for electrons and muons.
It is called during initialisation by \Rind{GPHYSI}.
For hadrons and ions the computation is performed during 
tracking in the routines \Rind{GTHADR} and \Rind{GTHION}.

\Shubr{GMOLI}{(AC,ZC,WMAT,NLM,DENS,OMC*,CHC*)}
\begin{DLtt}{MMMMMMMM}
\item[AC] {\tt REAL} array of dimenstion {\tt NLM} containing
the mass numbers;
\item[ZC] {\tt REAL} array  of dimenstion {\tt NLM} 
containing the atomic numbers;
\item[WMAT] {\tt REAL} array containing the proportions by
weight of the component of a mixture/compound. {\tt 1} in case
of a single component;
\item[NLM] {\tt INTEGER} number of the elements in the mixture/compound;
\item[DENS] {\tt REAL} density in g cm$^{-3}$;
\item[OMC] {\tt REAL} $b_{c}$ constant of Moli\`{e}re theory;
\item[CHC] {\tt REAL} $\chi_{cc}$ constant of Moli\`{e}re theory.
\end{DLtt}

\Rind{GMOLI} calculates the two material-dependent constants 
({\tt OMC} and {\tt CHC}). These are 
needed during initialisation by \Rind{GMULOF} and during tracking
to sample the scattering angle, as explained below.
It is called at initialisation time by the routine
\Rind{GPROBI} which initialises some material constants and computes
the probabilities for various processes.

\subsection{Sampling of the multiple scattering angle}

\Shubr{GMULTS}{}
\Rind{GMULTS} is the steering routine for the sampling
of the multiple scattering angle when the variable {\tt IMULTS}
in common \FCind{/GCPHYS/} is {\tt 1} or {\tt 2}. This variable
can be controlled by the {\tt MULT} data record.
The routine decides if the material-dependent constant {\tt OMC}
computed by \Rind{GMOLI} during initialisation has to
be corrected due to its dependence on $\beta$ and $Z_{inc}$, and
selects Moli\`ere theory (\Rind{GMOLIE}) 
or single Coulomb scattering (\Rind{GMCOUL}).
It updates the direction of the particle in the array 
{\tt VECT} (COMMON \FCind{GCTRAK}).
It is called by the tracking routines \Rind{GTELEC}, \Rind{GTHADR},
\Rind{GTHION} and \Rind{GTMUON}.

\Shubr{GMOLIO}{(AC,ZC,WMAT,NLM,DENS,BETA2,CHARG2,OMC*)}
\begin{DLtt}{MMMMMMMM}
\item[BETA2] {\tt REAL} $\beta^2$ of the particle;
\item[CHARG2] {\tt REAL} charge squared of the particle in electron
charge unit ($Z_{inc}^2$ in the following).
\end{DLtt}

All the other arguments are routine \Rind{GMOLI}.
\Rind{GMOLIO} re-evaluates the material-dependent constant {\tt OMC}.
It is called by \Rind{GMULTS}, if needed. 

\Shubr{GMOLIE}{(OMEGA,BETA2,DIN*)}
\begin{DLtt}{MMMMMMMM}
\item[OMEGA] {\tt REAL} $\Omega_0$ of the Moli\`{e}re theory;
\item[BETA2] {\tt REAL} $\beta^2$ of the particle;
\item[DIN]   {\tt REAL} array of dimension 3 containing the
new direction cosines with respect to the original direction of the
particle.
\end{DLtt}
\Rind{GMOLIE} samples the multiple scattering angle according to
Moli\`{e}re theory, as explained in the next section.
It is called by the multiple scattering steering routine \Rind{GMULTS},

\Shubr{GMOL4}{(Y*,X,VAL,ARG,EPS,IER*)}
\begin{DLtt}{MMMMMMMM}
\item[Y] {\tt REAL} value of the interpolated function;
\item[X] {\tt REAL} value at which the function should be interpolated;
\item[VAL] {\tt REAL} array of four values of the function to
interpolate;
\item[ARG] {\tt REAL} array of four values of the argument of the
function to which the values contained in {\tt VAL} correspond;
\item[EPS]  {\tt REAL} required precision for the interpolation;
\item[IER] {\tt INTEGER} error flag;
\end{DLtt}
\Rind{GMOL4} inverts the integral of Moli\`ere distribution
function using four-points continued-fraction interpolation.
It is called by \Rind{GMOLIE}.

\section{Method} 
\subsection{Introduction}
When traversing ordinary matter, elementary particles undergo deflection
from their originary trajectory due to the interaction with the atoms.
This effect is rather large for charged particles, which are deflected
by the electric field of nuclei and electrons via a large number of 
small elastic {\it collisions}. To simulate precisely the transport of
particles in matter, it is important to provide a precise description 
of this effect. Considerable effort has gone over the years into
the formulation of a theory of Coulomb multiple scattering. In 
{\tt GEANT} we follow the theory originally formulated by Moli\`{e}re
\cite{bib-MOLI} \cite{bib-MOL1} and then elaborated by Bethe 
\cite{bib-BET1}. A review of this theory can be found in \cite{bib-SCOT}.

It should be noted that this is not the only theory which describes
multiple Coulomb scattering, and a thorough analysis of the problem has
been performed by Goudsmit and Saunderson \cite{bib-GOUD} \cite{bib-GOU1}.

A complete account of the Moli\`{e}re theory is outside the scope of this
manual. For more information the interested reader could refer to the
works cited above and to the EGS4 manual
\cite{bib-EGS4}, from which the notation used in the present chapter
has been derived. 

The principal limitations of Moli\`{e}re theory are the following:
\begin{enumerate}
\item the angular deflection is {\it small}. Effectively this
constraint provides the upper limit for the step-size;
\item the theory is a {\it multiple scattering} theory; that is, many
atomic collisions participate in causing the incident particle to be
deflected. In some implementations, this constraints provides a lower 
limit on the step size. In {\tt GEANT}, when the number of scatters is
below the limit of applicability of Moli\`{e}re single Coulomb scatterings
are simulated, so that there is no effective minimum step size below which
multiple scattering is {\it switched off};
\item the theory applies only in semi-infinite homogeneous media. This 
constraint calls for a very careful handling of particles near to a
medium's boundary;
\item there is no energy loss built into the theory. This again introduces
the necessity to take small steps;
\end{enumerate}

In Moli\`{e}re theory corrected for finite angle
scattering ($\sin\theta \neq \theta$) as described by 
Bethe~\latexhtml{\cite{bib-BET1,bib-SCOT}}%
                {\cite{bib-BET1} \cite{bib-SCOT}},
the angular distribution is given by:
\begin{eqnarray}
      f(\theta) \: \theta d\theta & = & \sqrt{\frac{\sin\theta}{\theta}} \:
f_r(\eta) \: \eta d\eta,
\end{eqnarray}
 
where for $f_r(\eta)$ we use the first three terms of Bethe's expansion:
\begin{eqnarray}
\label{eq:phys325-1}
f_r(\eta) & = & f_{r}^{(0)}(\eta)+
f_{r}^{(1)}(\eta)B^{-1}+f_{r}^{(2)}(\eta)B^{-2} \\
\eta & = & \frac{\theta}{\chi_c \sqrt{ B }} \nonumber
\end{eqnarray}
$\eta$ is called the {\it reduced angle}.

\subsection{Calculation of the reduced angle}
In formula (\ref{eq:phys325-1}) above $\chi_c$ the {\it critical
scattering angle} defined as:
\begin{equation}
\chi^2_c = \frac{4 \pi e^4  Z_s \rho N_{Av} Z_{inc}^2 t}{W c^2 p^2\beta^2} 
= \chi^2_{cc} Z_{inc}^2 \frac{t}{E^2\beta^4}
\end{equation}
with
\[
\begin{array}{ll}
Z_{inc} & \mbox{incident particle charge}               \\
Z_{s}   & \mbox{defined below}                           \\
N_{Av}  & \mbox{Avogadro's number}                      \\
\rho  & \mbox{density}                                  \\
W     & \sum_{i=1}^{N} n_i A_i \mbox{\hspace{0.5cm}molecular weight}  \\
E     & \mbox{particle energy in $MeV$}                    \\
r_0   & e^2/(m c^2) \mbox{\hspace{0.5cm} classical electron radius} \\
t     & \mbox{total path length in the scatterer (and not its thickness).}
\end{array}
\]

The variable $\chi_c$ is factorised into two parts.
The first part, $\chi_{cc}$, is a constant of the medium for 
a given incident particle.
The second part depends on the incident particle energy and on the
path length in the medium. 

\begin{equation}
\chi^2_{cc} = \frac{4 \pi e^{4} Z_s \rho N_{Av}}{W} = \frac{4 \pi c^4 r_0^2
m_e^2 Z_s \rho N_{Av}}{W} \approx (0.39612 \: 10^{-3})^2
Z_s \frac{\rho}{W}
\left [ \: GeV^2 \: cm^{-1} \right ] 
\end{equation}

The variable {\tt CHC} or {\tt CHCMOL} stored in {\tt Q(JPROB+25)} corresponds
to $\chi_{cc}$. 
$B$ is defined by the equation
\begin{equation}
\label{eq:phys325-2}
B-\ln B =\ln\Omega_0, 
\end{equation}

where $\Omega_0$ can be interpreted as the number of collisions in the step:

\begin{equation}
\Omega_0 =\frac{ \chi^2 _c}{e^{2\gamma-1}\chi_\alpha^2} 
= b_c Z_{inc}^2\frac{t}{\beta^2}
\end{equation}

Here $\gamma$ is Euler's constant ($\gamma =0.57721\dots$)
and $\chi_\alpha$ the atomic electron
screening angle. For a single element this is given by:

\[
\chi_\alpha^2 = \left ( \frac{\lambda_0}{2 \pi r_{TF}} \right ) ^2
\left [ 1.13 + 3.76 \left ( \frac{\alpha Z Z_{inc}}{\beta} \right ) ^2 \right ]
\]

where

\begin{eqnarray*}
\frac{\lambda_0}{2 \pi} & = & \hbar/p 
\mbox{\hspace{0.5cm}Compton wavelength of the electron} \\
r_{TF} & = & 0.885 a_0 Z^{-\frac{1}{3}} \mbox{\hspace{0.5cm}Thomas-Fermi
radius of the atom} \\
a_0 & = & \frac{\hbar^2}{m e^2} \mbox{\hspace{0.5cm}Bohr's radius} \\
\alpha & = & 1/137.035 \dots \mbox{\hspace{0.5cm}the fine-structure constant}
\end{eqnarray*}

so that we have

\[
\chi_\alpha^2 = \frac{m^2 e^4 1.13}{p^2 \hbar^2 (0.885)^2} Z^{\frac{2}{3}}
\left [ 1 + 3.34 \left ( \frac{\alpha Z Z_{inc}}{\beta} \right ) ^2 \right ].
\]

For a compound/mixture the following rule holds:

\begin{eqnarray*}
\ln(\chi_\alpha^2) & = & \frac{\displaystyle \sum_{i=1}^N n_i Z_i (Z_i+1) \ln
( \chi_{\alpha i}^2)}{\displaystyle \sum_{i=1}^N n_i Z_i (Z_i+1)} = 
\frac{\displaystyle \sum_{i=1}^N \frac{p_i}{A_i} Z_i (Z_i+1) \ln
( \chi_{\alpha i}^2) }{\displaystyle \sum_{i=1}^N \frac{p_i}{A_i} Z_i (Z_i+1)}\\
& = & \frac{\displaystyle \sum_{i=1}^N \frac{p_i}{A_i} Z_i (Z_i+1) \ln \left \{
\frac{m^2 e^4 1.13}{p^2 \hbar^2 (0.885)^2} Z_i^{\frac{2}{3}}
\left [ 1 + 3.34 \left ( \frac{\alpha Z_i Z_{inc}}{\beta} \right ) ^2 \right ]
\right \} }{\displaystyle \sum_{i=1}^N \frac{p_i}{A_i} Z_i (Z_i+1)} \\
& = & \ln \left ( \frac{m^2 e^4 1.13}{p^2 \hbar^2 (0.885)^2} \right ) -
\frac{\displaystyle \sum_{i=1}^N \frac{p_i}{A_i} Z_i (Z_i+1) \ln 
Z_i^{-\frac{2}{3}}}{\displaystyle \sum_{i=1}^N \frac{p_i}{A_i} Z_i (Z_i+1)} \\
 & & + \frac{\displaystyle \sum_{i=1}^N \frac{p_i}{A_i} Z_i (Z_i+1) 
\left [ 1 + 3.34 \left ( \frac{\alpha Z_i Z_{inc}}{\beta} \right ) ^2 \right ]}
{\displaystyle \sum_{i=1}^N \frac{p_i}{A_i} Z_i (Z_i+1)} \\
\end{eqnarray*}

To understand the transformations in the above formulae, the following 
should be noted. Let $n_i$ be the number of atoms
of type $i$ in a compound/mixture, that is the number of moles
of element $i$ in a mole of material, and $p_i$ the proportion by weight
of that element. We have the following relation:

\[
p_i = \frac{n_i A_i}{\sum_{j=1}^{N} n_j A_j} \mbox{\hspace{2cm} that
is\hspace{2cm}} n_i = \left ( \sum_{j=1}^{N} n_j A_j \right ) 
\frac{p_i}{A_i} = W  \frac{p_i}{A_i}
\]

so that we can simplify expressions in the following way:

\[
\frac{\sum_{i=1}^{N} n_i \dots}{W} = \sum_{i=1}^{N} (p_i/A_i) \dots
\]

If now we set

\[
\begin{array}{L@{}c@{}L@{}c@{}L@{}c@{}L}
Z_s & = & \sum^{N}_{i = 1} n_i Z_i(Z_i+1)  & =  &
W \sum^{N}_{i = 1} \frac{p_i}{A_i} Z_i(Z_i+1) & = & W Z_s'  \\
Z_E & = & \sum^{N}_{i = 1} n_i Z_i(Z_i + 1 )\ln Z_i^{-2/3}  & = &
W\sum \frac{p_i}{A_i} Z_i(Z_i + 1 )\ln Z_i^{-2/3} & = & W Z_E' \\
Z_x & = & \sum^{N}_{i = 1} n_i Z_i(Z_i + 1 ) \ln \left [ 1+3.34
\left( \frac{\alpha Z_i Z_{inc}}{\beta}\right)^2 \right ] \\
 & = & W \sum^{N}_{i = 1} \frac{p_i}{A_i} Z_i(Z_i + 1 ) \ln \left [ 1+3.34
\left( \frac{\alpha Z_i Z_{inc}}{\beta}\right)^2 \right ] & &  & = & W Z_x'
\end{array}
\]

we can write:

\begin{eqnarray*}
\ln(\chi_{\alpha}^2) & = & 
\ln \left ( \frac{m^2 e^4 1.13}{p^2 \hbar^2 (0.885)^2} \right )
+ \frac{Z_x-Z_E}{Z_s} \\
\chi_{\alpha}^2 & = & \frac{m^2 e^4 1.13}{p^2 \hbar^2 (0.885)^2} 
e^{(Z_x  - Z_E )/Z_s} = \frac{m^2 e^4 1.13}{p^2 \hbar^2 (0.885)^2}
e^{(Z_x'  - Z_E' )/Z_s'}
\end{eqnarray*}

and finally:

\begin{eqnarray*}
\Omega_0 & = & \frac{ \chi^2 _c}{e^{2\gamma-1}\chi_\alpha^2} \\
 & = & \frac{1}{1.167} \:
\frac{4 \pi e^4  Z_s \rho N_{Av} Z_{inc}^2 t}{W c^2 p^2\beta^2} \:
\frac{p^2 \hbar^2 (0.885)^2}{1.13 m^2 e^4}e^{-(Z_x'  - Z_E' )/Z_s'} \\
 & = & b_c Z_{inc}^2\frac{t}{\beta^2} \\
b_c & = & \frac{4 \pi N_{Av} \hbar^2}{m^2 c^2}
\frac{(0.885)^2}{1.13 \times 
1.167} \rho Z_s' e^{(Z_E'  - Z_x' )/Z_s'} \approx 6702.33 \rho Z_s'
e^{(Z_E'  - Z_x' )/Z_s'}
\end{eqnarray*}

The quantity $b_c$ is calculated during initialisation by
the routine \Rind{GMOLI} setting $\beta=1$ and $Z_{inc}=1$. 
This variable is called
{\tt OMC} or {\tt OMCMOL} and it is stored in {\tt Q(JPROB+21)}.
It has, via the atomic electron screening angle $\chi_{\alpha}^2$,
a small dependence
on $\beta$  and $Z_{inc}$ in the term $Z_x$. The quantity $b_c \rho Z_s'$
is re-evaluated during tracking by the routine \Rind{GMOLIO}, called by
\Rind{GMULTS}, only when necessary.

The dependence of $Z_x$ on $\beta$ and $Z_{inc}$ is via a term of the form:
$1+3.34\left(\alpha Z Z_{inc}\beta^{-1} \right )^2$, so it is not 
necessary to recalculate it as long as
$1 + 3.34\left(\alpha Z Z_{inc}\beta^{-1} \right )^2 \approx
1 + 3.34\left(\alpha Z \right )^2$ that is:

\[
\begin{array}{RcLcLcL}
3.34 (Z \alpha)^2  \left [ \left( \frac{Z_{inc}}{\beta} \right )^2
-1 \right] & \ll & 1 & \Rightarrow &
Z^2 \frac{Z_{inc}^2 - \beta^2}{\beta^2} & \ll & 3.34 \alpha^{-2} 
\approx 5500 \\ [.2cm]
 Z^2 ( Z_{inc}^2 - \beta^2 ) & \leq & 50 \beta^2
\end{array}
\]

For mixtures/compounds this condition should be checked for every component
and this would be unacceptable from the point of view of computer time. So
we simply make the condition more restrictive multiplying the first term
by the number of elements in the mixture/compound.

\subsection{Sampling of the distribution function}

The component distribution functions (\ref{eq:phys325-1}) are given by
 
\[
\begin{array}{lcccr}
f_{r}^{(0)}(\eta)   & = & 2e^{-\eta^2}  & = &  2D_0(1,1,-\eta^2)         \\
f_{r}^{(1)}(\eta)   &   &               & = &  2D_1(2,1,-\eta^2)         \\
f_{r}^{(2)}(\eta)   &   &               & = &   D_2(3,1,-\eta^2),
\end{array}
\]
 
where $ D_n(a,b,z) = d^n[\Gamma(a)M(a,b,z)]/d a^n $, with
$M(a,b,z)$ the Kummers hypergeometric function.
 
Integrals of the functions $f_{r}^{(1)}$ and $f_{r}^{(2)}$ needed for the
Monte Carlo can be written down directly in terms of the $D$ functions
\begin{eqnarray*}
\int_{\eta}^{\infty}\eta f_{r}^{(1)} \: \eta  d\eta & = &
     D_1 ( 2,1, - \eta^2)-D_1( 1,1, - \eta^2 ) - D_0 ( 1,1,-\eta^2)\\
\int_{\eta}^{\infty}\eta f_{r}^{(2)} \: \eta d\eta    & = &
\frac{1}{2} D_2 (3,1,-\eta^2) - D_2 (2,1,-\eta^2) - D_1(2,1,-\eta^2)
\end{eqnarray*}
 
We note that the first term is just the Gaussian part and it dominates
for large $B$, that is, for large number of scatters, as it has to
be expected. Recalling the definition of $\eta$ in (\ref{eq:phys325-1})
we can say that the half-width of the Gaussian part of the Moli\`{e}re
distribution is $\sigma^{2} = \chi_{c}^2 B/2$.
Routine \Rind{GMOLIE} performs the sampling of the Moli\`{e}re 
distribution via the following steps:

\begin{enumerate}
\item the value of $B$ is calculated recursively. If we set $f(B) = 
B$ with $f(B) = \ln \Omega_0  + \ln B$ the relation used is
$B_n = B_{n-1} + \Delta B = B_{n-1} + (f(B_{n-1}) - B_{n-1})/
(1-1/B_{n-1})$. Convergence is assumed when $|\Delta B| \leq 10^{-4}$.

\item a random number $r_1$ is sampled and
a table of four values $F_{r}(\eta_i)$ is built with
$i = j-2,j-1,j,j+1$ and $F_{r}(\eta_j) \geq r_1$. $F$ is the distribution
function derived from (\ref{eq:phys325-1}):
\begin{eqnarray*}
F_{r}(\eta) & = & \int_{0}^{\eta}{f_{r}(t) \: dt}
= \int_{0}^{\eta}{f_{r}^{(0)}(t)+f_{r}^{(1)}(t)B^{-1}+f_{r}^{(2)}(t)B^{-2} \: dt} \\
& = & \int_{0}^{\eta}{f_{r}^{(0)}(t) \: dt} + 
B^{-1} \int_{0}^{\eta}{f_{r}^{(1)}(t) \: dt} +
B^{-2} \int_{0}^{\eta}{f_{r}^{(2)}(t) \: dt} \\
& = & F_{r}^{(0)}(\eta) +
B^{-1} F_{r}^{(1)}(\eta) + B^{-2} F_{r}^{(2)}(\eta)
\end{eqnarray*}

40 values of the functions $F_{r}^{(n)}$ are tabulated.

\item using a four-points continued-fraction interpolation method 
(\Rind{GMOL4}) a table of $\eta_i^2$ is interpolated using
the values of $r_1$ and $F$ tabulated in the previous step. 
This is similar
to the inversion of the distribution function, but instead of
obtaining directly the random variable, we interpolate a table
of its squares.
This provides a better fit to the inverse function;

\item the value of $\theta = \chi_{c} \sqrt{B \eta^2}$ is calculated;

\item a random number $r_2$ is extracted, and the rejection function
$g(\theta) = \sqrt{\sin\theta/\theta}$ is computed. If $r_2 > g(\theta)$
resampling begins from step 2, otherwise the value is accepted.
\end{enumerate}

The Moli\`ere distribution gives the space scattering angle. A similar
expression may be written for the lateral displacement of the
scattered particles. However, the problem of joint angle lateral displacement
in the Moli\`ere approximation has not been solved, and,  for small
step size, lateral displacement is of second order and may be neglected.
This introduces a problem when large step sizes are taken. The step
size can be artificially limited via the use of the variable {\tt
STEMAX}, which is an argument to the routine \Rind{GSTMED}. For more
information on this point the user is invited to consult chapter
{\tt [CONS200]}.

\section{Path and step length}
\subsection{Restrictions on the step length}
Restrictions on the length of the step arise from:
\begin{enumerate}
\item the number of scatters
$\Omega_0 \geq 20 $ to stay within the multiple scattering regime. When
$\Omega_0 < 20$, an appropriate number of {\it single} scatterings is
used. See routine \Rind{GMCOUL} {\tt [PHYS328]}.
\item
$\chi_c^2B \leq 1 $ i.e. the width of the Gaussian part of the distribution
should be less than one radian. This
condition induces a maximum step length for the multiple scattering called
$t_{Bethe}$. In order to find this value we write the limiting condition
as $\chi_c^2(t_{Bethe})B(t_{Bethe}) = 1 $, that is 
$B(t_{Bethe}) = 1/\chi_c^2(t_{Bethe})$. Now using equation 
(\ref{eq:phys325-2}) we take the exponential of both members 
$B(t_{Bethe}) \Omega_0(t_{Bethe}) = \exp(1/\chi_c^2(t_{Bethe}))$. 
Dividing the two last equalities we obtain:
\begin{eqnarray*}
\Omega_0(t_{Bethe}) = \frac{b_c Z_{inc}^2}{\beta^2} t_{Bethe} & = & 
\chi_c^2 (t_{Bethe})\exp(1/\chi_c^2(t_{Bethe})) \\
&  = & \frac{\chi_{cc}^2 Z_{inc}^2 t_{Bethe} 
\exp{[(E^2\beta^4)/(\chi_{cc}^2 Z_{inc}^2 t_{Bethe})]}}
{E^2\beta^4} \\
\exp{[(E^2\beta^4)/(\chi_{cc}^2 Z_{inc}^2 t_{Bethe})]} & = & 
\frac{b_c E^2 \beta^2}{\chi_{cc}^2} \\
t_{Bethe} & = & \frac{E^2 \beta^4}{\chi_{cc}^2 \: Z_{inc}^2 \: \ln
\left [b_c E^2 \beta^2/\chi_{cc}^2 \right]}
\end{eqnarray*}
 
For electrons and muons this constraint on the step-length is tabulated
at initialisation time in the routine \Rind{GMULOF} {\tt [PHYS201]}.
For hadrons this formula can be approximated as:
\[
t_{Gauss}
\approx \left(\frac{1}{14.1 \: 10^{-3}} \: \frac{E^2 \beta}{Z_{inc}}
\right )^2 X_0,
\]
where $E$ is in GeV and $X_0$ is the radiation length in centimeters and
the formula has been taken from the Gaussian approximation to multiple
scattering (see {\tt [PHYS320]}). This condition is more restrictive, 
because it is equivalent to require that the width of the Gaussian
part of the distribution be less than $0.5$, but it has been found that
the two conditions are numerically equivalent;
\item limitation due to the path length correction algorithm used (see
below).
\end{enumerate}
\subsection{Path Length Correction}
 
A path length correction may be applied in an approximate manner.
We have from the Fermi-Eyges theory \cite{bib-EGS4}
\begin{equation}
\label{eq:phys325-3}
t = S + \frac{1}{2}\int_{0}^{t}{\bar{\theta}^{2}\bar{(t)} \: dt}
\end{equation}
where
\[
\begin{array}{LL}
\bar{\theta}^2\bar{(t)}  & \mbox{the mean square angle of scattering;} \\
S                     & \mbox{step size along a straight line;} \\
t                      & \mbox{actual path length.}
\end{array}
\]
 
We have further:
\begin{equation}
\label{eq:phys325-4}
\bar{\theta}^2\bar{(t)} = \left( 0.0212 \frac{Z_{inc}}{p\beta} \right)^2
\frac{t}{X_0}
\end{equation}
$X_0$ is the radiation length. From (\ref{eq:phys325-3}) and 
(\ref{eq:phys325-4}) we get
\begin{equation}
\label{eq:phys325-5}
S = t - K t^2
\makebox[4cm]{with}
K = 1.12 \times 10^{-4} \: \frac{Z_{inc}^2}{p^2\beta^2X_0}
\end{equation}
Equation (\ref{eq:phys325-5}) 
may be used to make the path length correction. Solving equation
(\ref{eq:phys325-5}) with respect to $t$ implies that, in order to
obtain real solutions:
\begin{equation}
\label{eq:phys325-6}
S \leq \frac{1}{4K} \makebox[3cm]{i.e.} S \leq 2232 \:
\frac{X_0 p^2 \beta^2}{Z_{inc}^2}
\end{equation}
This condition provides an additional constraint to the maximum step
length for multiple scattering (variable {\tt TMXCOR} in routine \Rind{GMULOF}).
The corrected step can be approximated as:
\[
t \approx S(1 + KS) = S ( 1 + \mbox{\tt CORR} )
\]
where {\tt CORR} $ \leq 0.25$ due to condition (\ref{eq:phys325-6}).
