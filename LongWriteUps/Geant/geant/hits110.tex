%%%%%%%%%%%%%%%%%%%%%%%%%%%%%%%%%%%%%%%%%%%%%%%%%%%%%%%%%%%%%%%%%%%
%                                                                 %
%  GEANT manual in LaTeX form                                     %
%                                                                 %
%  Michel Goossens (for translation into LaTeX)                   %
%  Version 1.00                                                   %
%  Last Mod. Jan 24 1991  1300   MG + IB                          %
%                                                                 %
%%%%%%%%%%%%%%%%%%%%%%%%%%%%%%%%%%%%%%%%%%%%%%%%%%%%%%%%%%%%%%%%%%%
\Origin{R.Brun}
\Submitted{01.11.83}     \Revised{17.12.93}
\Version{Geant 3.16}     \Routid{HITS110}

\Makehead{DETector hit parameters}

\Shubr{GSDETH}{(CHSET,CHDET,NH,CHNAMH,NBITSH,ORIG,FACT)}
\begin{DLtt}{MMMMMMMM}
\item[CHSET] ({\tt CHARACTER*4}) set name;
\item[CHDET] ({\tt CHARACTER*4}) detector name;
\item[NH] ({\tt INTEGER}) number of the components of a hit;
\item[CHNAMH] ({\tt CHARACTER*4}) array of {\tt NH} names for the hit components;
\item[NBITSH] ({\tt INTEGER}) array of {\tt NH}, 
{\tt NBITSH(I)} ({\tt I=1,...NH}) is the number of bits in which to pack
the {\tt I$^{th}$} component of the hit;
\item[ORIG] ({\tt REAL}) array of {\tt NH} offset applied before packing
the hits values;
\item[FACT] ({\tt REAL}) array of {\tt NH} scale factors applied before packing
the hits values;
\end{DLtt}

Defines hit parameters for detector {\tt CHDET} of set {\tt CHSET}.
The routine must be called at initialisation time once the
geometrical volumes have been defined to describe
the hit elements and the way to pack them in the data structure {\tt JHITS}.
The value of the hit before packing is transformed in the following way:

\begin{center}
\tt VAL(I) = (HIT(I)+ORIG(I)) $\times$ FACT(I)
\end{center}

\section*{Example}

Assume an electromagnetic calorimeter {\tt ECAL} divided into
40 {\tt PHI} sections called {\tt EPHI}. Each {\tt EPHI} division is in turn
divided along the $z$ axis in 60 sections called {\tt EZRI}. Each {\tt EZRI}
is finally divided into 4 lead glass blocks called {\tt BLOC}.
The geometrical information to describe one hit will then be:

\begin{itemize}
\item the {\tt EPHI} section number (between 1 and 40);
\item the {\tt EZRI} division number (between 1 and 60);
\item the {\tt BLOC} number (1 to 4).
\end{itemize}

The quantities which should be stored for each hit are:
 
\begin{DLtt}{MMMM}
\item[X] $x$ position of the hit in the lead glass block ($-1000<x<1000$);
\item[Y] $y$ position of the hit in the lead glass block ($-1000<y<1000$);
\item[Z] $z$ position of the hit in the lead glass block ($-1000<z<1000$);
\item[E]energy of the particle;
\item[ELOS]the energy deposited;
\end{DLtt}

In this scheme a hit could look like:

\begin{center}
\begin{tabular}{l@{\hspace{2cm}}l}
Element & Value \\ \hline
\tt EPHI  &  12  \\
\tt EZRI  &  41  \\
\tt BLOC  &   3  \\
\tt X  &     7.89 cm  \\
\tt Y  &     -345.6 cm  \\
\tt Z  &     1234.8 cm  \\
\tt E  &     12 Gev  \\
\tt ELOS  &   11.85 Gev  \\
\end{tabular}
\end{center}

The code to define the {\tt SET/DET/HIT} information could be:
\begin{verbatim}
      CHARACTER*4 CHNMSV(3),CHNAMH(5)
      DIMENSION   NBITSV(3),NBITSH(5)
      DIMENSION   ORIG(5),FACT(5)
*---
      DATA CHNMSV/'EPHI','EZRI','BLOC'/
      DATA NBITSV/     6,     6,     3/
*---
      DATA CHNAMH /'X   ','Y   ','Z   ','E   ','ELOS'/
      DATA NBITSH /    16,    16,    16,    16,    16/
      DATA ORIG   / 1000., 1000., 1000.,    0.,    0./
      DATA FACT   /   10.,   10.,   10.,  100.,  100./
*---
      CALL GSDET ('ECAL','BLOC',3,CHNMSV,NBITSV,2,100,100,ISET,IDET)
      CALL GSDETH('ECAL','BLOC',5,CHNAMH,NBITSH,ORIG,FACT)
\end{verbatim}

\Shubr{GFDETH}{(CHSET,CHDET,NH*,CHNAMH*,NBITSH*,ORIG*,FACT*)}
Returns the hit parameters for detector {\tt CHDET} of set {\tt CHSET}.
All arguments are as explained above.
 
