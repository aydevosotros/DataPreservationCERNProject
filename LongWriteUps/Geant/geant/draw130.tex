%%%%%%%%%%%%%%%%%%%%%%%%%%%%%%%%%%%%%%%%%%%%%%%%%%%%%%%%%%%%%%%%%%%
%                                                                 %
%  GEANT manual in LaTeX form                              %
%                                                                 %
%  Michel Goossens (for translation into LaTeX)                   %
%  Version 1.00                                                   %
%  Last Mod. Jan 24 1991  1300   MG + IB                          %
%                                                                 %
%%%%%%%%%%%%%%%%%%%%%%%%%%%%%%%%%%%%%%%%%%%%%%%%%%%%%%%%%%%%%%%%%%%
\Origin {R.Brun, P.Zanarini}
\Documentation{P.Zanarini}
\Submitted{15.05.84}             \Revised{11.12.92}
\Version{Geant 3.16}\Routid{DRAW130}
\Makehead{Draw Particle Trajectories}
\section{Visualisation of particle trajectories}
The tracks generated by the tracking package, and optionally stored
in the data structure {\tt JXYZ}, can be displayed by the routine \Rind{GDXYZ}
(corresponding to the interactive command {\tt DXYZ}).
 
The names of the particles and or the track numbers can be added
by the routine \Rind{GDPART} (corresponding to the interactive command 
{\tt DPART}).
 
A special routine has been provided to visualise the tracks during the
transport process
(\Rind{GDCXYZ}), that could be called for instance by \Rind{GUSTEP}.
That routine shows the tracks while they are transported
by the tracking package of {\tt GEANT}, providing a useful interactive
debugging tool.

\Shubr{GDXYZ}{(ITRA)}
 
Draws track number {\tt ITRA} for which space points have been stored in bank
{\tt JXYZ} via calls to the routine \Rind{GSXYZ}. The view parameters are 
taken from \FCind{/GCDRAW/}.
\begin{DLtt}{MMM}
\item[ITRA] ({\tt INTEGER}) track number (if 0 all tracks are taken)
\end{DLtt}
The colour and line style corresponds to the track type :
\begin{center}
\newcommand{\HSP}{\hspace{3mm}}
\newcommand{\HDD}{\makebox[2mm][c]{.}}
\newcommand{\HHH}{\makebox[2mm][c]{-}}
\newcommand{\HSPD}{\HSP\HDD\HSP\HDD\HSP\HDD\HSP\HDD\HSP\HDD\HSP\HDD}
\newcommand{\HSPH}{\HSP\HHH\HSP\HHH\HSP\HHH\HSP\HHH\HSP\HHH\HSP\HHH}
\begin{tabular}{llp{3cm}}
(blue)  & dotted line for gammas
                                                    & \dotfill \\
(red)   & solid line for charged particles (except muons)
                                                    & \hrulefill \\
(black) & blank/dotted line for neutral hadrons or neutrinos
                                                    & \HSPD \\
(green) & dashed line for muons
                                                    & \HSPH \\
(yellow) & dotted line for \v{C}erenkov photons
                                                    & \dotfill
\end{tabular}\end{center}
\Shubr{GDPART}{(ITRA,ISEL,SIZE)}
Draws the particle names and/or the track numbers
of track {\tt ITRA}, supposing that its space points had been stored in 
the bank {\tt JXYZ} via the routine \Rind{GSXYZ}.
At present only primary tracks are displayed by \Rind{GDPART}
and their name or number is written at the end the track trajectory.
The view parameters are taken from \FCind{/GCDRAW/}.
\begin{DLtt}{MMMM}
\item[ITRA]  ({\tt INTEGER}) Track number (if 0 all tracks are taken)
\item[ISEL]  ({\tt INTEGER}) \\
             {\tt ISEL=x1} draws the track number, \\
             {\tt ISEL=1x} draws the particle name, \\
             {\tt ISEL=11} draws both.
\item[SIZE]  ({\tt REAL}) Character size in cm.
\end{DLtt}
\Shubr{GDCXYZ}{}
If \Rind{GDCXYZ} is called at tracking time (for instance by
\Rind{GUSTEP}), it draws the tracks while tracking is performed,
at the same time as
they are generated by the tracking package of {\tt GEANT}.
This is a very
useful interactive debugging tool. The line style is the same as for
\Rind {GDXYZ}.
The view parameters are taken from \FCind{/GCDRAW/}.

Here we give an example of the use of \Rind{GDPART}:
 
\begin{verbatim}
 CALL GSATT('HB','FILL',3)
 CALL GSATT('HE','FILL',1)
 - - - - -
 - - - - -
 
 CALL GDRAWC('OPAL',2,5.,10.,10.,0.013,0.013)
 CALL GDXYZ(0)
 CALL GDPART(0,11,0.25)
\end{verbatim}
 
