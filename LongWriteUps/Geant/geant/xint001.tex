%%%%%%%%%%%%%%%%%%%%%%%%%%%%%%%%%%%%%%%%%%%%%%%%%%%%%%%%%%%%%%%%%%%
%                                                                 %
%  GEANT manual in LaTeX form                              %
%                                                                 %
%  Michel Goossens (for translation into LaTeX)                   %
%  Version 1.00                                                   %
%  Last Mod. Jan 24 1991  1300   MG + IB                          %
%                                                                 %
%%%%%%%%%%%%%%%%%%%%%%%%%%%%%%%%%%%%%%%%%%%%%%%%%%%%%%%%%%%%%%%%%%%
\Documentation{R.Brun}
\Submitted{15.06.84}   \Revised{19.12.93}
\Version{Geant 3.16}    \Routid{XINT001}
\Makehead{The interactive version of GEANT}
 
The interactive version is an essential tool for users who are
in charge of the design of a detector.
In addition to all the batch tools which are also available, the
interactive user can call one by one and in any order all the basic
functions of {\tt GEANT} to:
\begin{itemize}
\item design or modify the geometry of the setup;
\item exploit the drawing package in a more efficient way;
\item change the running conditions on a event by event basis.
\end{itemize}
The system is based on the {\tt KUIP}~\cite{bib-KUIP}
command processor.
The user has to know a minimum about {\tt KUIP} and should read
at least the first chapter of the manual.

A set of encapsulated Postscript files have been collected in
a new section XINT010. They illustrate the visualization possibilities
of the interactive version of {\tt GEANT}. 

\section{Invoking the interactive version}
Instead of writing a MAIN program to initialize the Geant batch
version, it is possible to use the MAIN interactive program (GXINT321.F)
provided in /cern/pro/lib exactly as the library LIBGEANT321.A. The user
has simply to insert in his link file the file GXINT321.F as the main
program, followed by the user object code and by the library LIBGEANT321.A.
\\[.5em] Example: \\[.1em]
       PROGRAM GXINT \\[.1em]
* \\[.1em]
*     GEANT main program. To link with the MOTIF user interface \\[.1em]
*     the routine GPAWPP(NWGEAN,NWPAW) should be called, whereas \\[.1em]
*     the routine GPAW(NWGEAN,NWPAW) gives access to the basic \\[.1em]
*     graphics version. \\[.1em]
* \\[.1em]
      PARAMETER (NWGEAN=3000000,NWPAW=1000000) \\[.1em]
      COMMON/GCBANK/GEANT(NWGEAN) \\[.1em]
      COMMON/PAWC/PAW(NWPAW) \\[.1em]
* \\[.1em]
      CALL GPAWPP(NWGEAN,NWPAW) \\[.1em]
* \\[.1em]
      END \\[.5em]
The user has to set the desired value of NWGEAN and NWPAW for the
GEANT and PAW Zebra stores, and to call the desired initialization routine:
\begin{itemize}
\item GPAW to initialize, besides GEANT, also HIGZ and to include the full
      functionality of PAW;
\item GPAWPP to initialize, besides GEANT and HIGZ, also the Motif version 
      and to include the full functionality of Paw++;
\item USER initialization routine, to do anything else (for example a UGINIT-
      like routine or a gxint315.f--like routine). 
\end{itemize}
The interactive version, after the initialization, gives the control to the
user at the prompt GEANT $>$ ; then it is possible to type and execute 
commands (corresponding to batch routines) to edit the geometry, the materials
or the tracking media at run time. It is also possible to execute commands
to visualize the detectors, to set the kinematics and to run events. Again
interactively, one can spy the histograms, change the kinematics, and run
more events (visualizing the tracks and the hits, for example). The GXINT
chapter contains in the following pages a full description of the available
GEANT commands. See the PAW, KUIP, DZDOC, HIGZ manuals for a description of
the relative commands executable from GEANT. All the commands are also
documented by an on-line help. (Try to type HELP at the first GEANT $>$ prompt).
In the interactive version, a COMIS interface is also available: COMIS is a
FORTRAN interpreter which allows:
\begin{itemize}
\item to edit at run time important routines like UGEOM, GUSTEP, GUKINE, etc.
\item to `execute' them from the interactive session, without having to 
      recompile and relink, by typing commands like CALL UGEOM.F.
\end{itemize}
Of course the interpreter is slower than the compiled object code,
then, since GEANT321, it is also possible to invoke the native compiler and
to link dinamically to the executable the compiled routine (see the COMIS 
manual for further details).


The following write ups describe individual commands which can be typed
one by one at the terminal, or grouped into macros which can be edited
and saved in the {\tt KUIP} environment.

The commands are listed in subsection 1 - 13:

\begin{center}\begin{tabular}{ll}
1. &  General GEANT \\
2. & Clipping commands GEANT/CVOL \\
3. & Drawing commands GEANT/DRAWING \\
4. & Graphics control commands GEANT/DRAWING \\
5. & Geometry commands GEANT/GEOMETRY \\
6. & Volume creation commands GEANT/CREATE \\
7. & Control commands GEANT/CONTROL \\
8. & ZEBRA/RZ commands GEANT/RZ \\
9. & ZEBRA/FZ commands GEANT/FZ \\
10. & Data structure commands GEANT/DZ \\
11. & Scanning commands GEANT/SCAN \\
12. & Physics parameter commands GEANT/PHYSICS \\
13. & List commands GEANT/LISTS \\
\end{tabular}\end{center}

\section{The Motif Interface}
The interactive version 3.21 contains an object oriented Motif-based user
interface. It can be accessed specifying `m' as workstation type.
The full functionality of the X11 version remains available, while new 
Motif-specific features have been added. \\[1em]
\section{The main ideas} 
The {\tt GEANT} data structures are 
considered as KUIP browsable classes and their
contents as `objects' on which one can perform actions 
(the {\tt GEANT} commands).
According to the Motif conventions, an object can be selected clicking the
left button of the mouse (when the cursor is on the icon representing that
object). Clicking then on the right button of the mouse, a menu of possible
commands will appear (double clicking on the left button the first action of
this list will be executed); 
the selected command will perform the relative actions
on the selected object. Such actions (like drawing, for example) can be
executed either directly or via the use of an automatically opened Motif panel.
Objects drawn in the graphics window can be `picked' as well (for example,
volumes, tracks, hits); clicking the right button when the cursor is on the
selected object, a menu of possible actions on that object is displayed.
Users can finally define Motif panels containing buttons corresponding to the
most frequently used commands. An on-line help is available for any specific
subject. \\[1em]

\section {The Geant++ Executive Window}
It replaces the normal dialog window; it contains a Transcript Pad, where the
text output of the executed commands is displayed, and an Input Pad, where the
user can still type the desired commands in the old style. \\[1em]
The {\tt Geant++} Main File Browser \\[1em]
On the left side it displays a list of the {\tt GEANT} data structures, of the 
available commands, file, macros and Zebra divisions used. Selecting one of
them, the full list of icons representing the objects of that class is shown
in the main area of the browser. Proceeding as described before, it is 
possible to perform actions on the classes (like create a new object) or on
the objects belonging to them. It is possible to create menus of commands
just clicking on the string `commands' at the top line of the browser. \\[1em]
The Geant++ Graphics Window \\[1em]
Any object to be drawn in the graphics window can be stored in the current
picture file (automatically opened after each NEXT command) via a call to 
IGPID (see Higz manual). It can be afterwards `picked' as described before.
In the case of commands executed via the use of Motif panels, some input values
can be set with a slider ranging in the specifed range for the relative
variable; moving the slider (after having clicked on the right-hand `activating
box') the relative action is performed in the graphics
window when releasing the button of the mouse; when in `drag mode', the 
action is performed {\it while} moving the slider (keeping the left button
pressed): especially when double buffering has been selected, this can be
useful for real time manipulations.\\[1em]
\section {An Example}
Start your {\tt GEANT321} executable module 
(linked with GXINT321 and Motif1.2);
\\[.5em] type `m' as workstation type;
\\[.5em] click the left button of the mouse after positioning the cursor on the 
string VOLU in the browser;
\\[.5em] click the left button of the mouse after positioning the cursor on 
any icon in the main area of the browser;
\\[.5em] click now the right button of the mouse and keep it pressed;
\\[.5em] move the mouse to select the action `Tree' and release the button;
\\[.5em] the drawing of the logical tree will be displayed in the graphics
window;
\\[.5em] position the cursor on the drawing of a box (containing a volume name)
in the graphics window, click the right button and keep it pressed;
\\[.5em] release the button selecting the action `Dspec';
\\[.5em] the command DSPEC for that volume will be executed in a separate
window;
\\[.5em] repeat the exercise selecting this time the action `Dspe3d';
\\[.5em] the DSPEC will be executed in the first window, the volume 
specifications will be printed in a separate window and a Motif panel will
appear;
\\[.5em] click the left button of the mouse positioning the cursor in the Motif
panel on the `Value changed' button, and select the DRAG option;
\\[.5em] click now the left button on the `activating box' on the right of
the `Theta' slider;
\\[.5em] click on the `Theta' slider and, keeping pressed the left button of the
mouse, move it right-wards;
\\[.5em] the drawing in the graphics window will rotate;
\\[.5em] release the button and type `igset 2buf 1' in the executive window;
\\[.5em] restart moving the slider as before. 
