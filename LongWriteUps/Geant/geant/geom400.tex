%%%%%%%%%%%%%%%%%%%%%%%%%%%%%%%%%%%%%%%%%%%%%%%%%%%%%%%%%%%%%%%%%%%
%                                                                 %
%  GEANT manual in LaTeX form                              %
%                                                                 %
%  Michel Goossens (for translation into LaTeX)                   %
%  Version 1.00                                                   %
%  Last Mod. Jan 24 1991  1300   MG + IB                          %
%                                                                 %
%%%%%%%%%%%%%%%%%%%%%%%%%%%%%%%%%%%%%%%%%%%%%%%%%%%%%%%%%%%%%%%%%%%
\Origin{R.Brun,F.Bruyant,A.C.McPherson}
\Revision{S.Egli}
\Submitted{16.12.83}          \Revised{15.12.93}
\Version{Geant 3.16}          \Routid{GEOM400}

\Makehead{Pseudo-division of a volume}

\Shubr{GSORD}{(CHNAME,ICORD)}
\begin{DLtt}{MMMMMMMM}
\item[CHNAME]  ({\tt CHARACTER*4}) name of the volume;
\item[ICORD]  ({\tt INTEGER}) direction of the pseudo-divisions:
\begin{DLtt}{MMMM}
\item[1] $x$;
\item[2] $y$;
\item[3] $z$;
\item[4] cylindrical $R$ ($\sqrt{x^2+y^2}$);
\item[5] spherical $\rho$ ($\sqrt{x^2+y^2+z^2}$);
\item[6] $\phi$, azimuthal angle;
\item[7] $\theta$, polar angle with respect to the $z$ axis.
\end{DLtt}
\end{DLtt}

This routine sets the search flag ({\tt Q(JVO+1)}) of volume {\tt CHNAME} 
to {\tt -ICORD}. When the definition of the geometry is complete and 
\Rind{GGCLOS} is called, this flag is interpreted as a request to order 
the content
of {\tt CHNAME} along {\it axis} {\tt ICORD}. This operation is 
performed by the routine \Rind{GGORD}.
\Rind{GGORD} computes the limits of each of the contents along the given
coordinate axis (see {\tt [GEOM001]}),
and prepares the lists of contents in each of the sections
defined by the neighbouring coordinate. The {\tt JVOLUM} structure 
is extended, for
the mother volume, with banks which contains the list of boundaries and the
lists of contents, so as to permit a binary search to access the contents
of interest. The coordinates are in the local system of the mother volume.
The routine \Rind{GGORD} will not be called if the number of contents 
exceeds 500.

The actual effect of this routine depends on the setting of the {\tt IOPTIM}
variable in the common \FCind{/GCOPTI/}. {\tt IOPTIM} is controlled by the
data record {\tt OPTI} or the interactive command with the same name.
The meaning of the different values of {\tt IOPTIM} is the following:
\begin{DLtt}{MMMM}
\item[$<$0] no call to \Rind{GGORD} will be made, irrespective of the
value of {\tt ISEARCH};
\item[~0] \Rind{GGORD} will be called for those volumes for which \Rind{GSORD}
has been called;
\item[~1] for all volumes with contents for which neither \Rind{GUSEAR} nor
\Rind{GSORD} has been called, the routine \Rind{GGORDQ} will be called;
\item[~2] \Rind{GGORDQ} is called for all volumes with contents
for which \Rind{GUSEAR} has not been called.
\end{DLtt}
 
\Rind{GGORDQ} orders the contents along all the possible axes and choses the
ordering which provides the lowest number of volumes per division.
