%%%%%%%%%%%%%%%%%%%%%%%%%%%%%%%%%%%%%%%%%%%%%%%%%%%%%%%%%%%%%%%%%%%
%                                                                 %
%  GEANT manual in LaTeX form                                     %
%                                                                 %
%  Michel Goossens (for translation into LaTeX)                   %
%  Version 1.00                                                   %
%  Last Mod. Jan 24 1991  1300   MG + IB                          %
%                                                                 %
%%%%%%%%%%%%%%%%%%%%%%%%%%%%%%%%%%%%%%%%%%%%%%%%%%%%%%%%%%%%%%%%%%%
%\Authors {F.Bruyant}      \Origin{GEANT}                         %
%%%%%%%%%%%%%%%%%%%%%%%%%%%%%%%%%%%%%%%%%%%%%%%%%%%%%%%%%%%%%%%%%%%
\Version{Geant 3.21}\Routid{AAAA002}
\Submitted{01.10.84}       \Revised{10.03.94}
\Makehead{Introduction to the manual}
The present documentation is divided into
sections which follow the structure of
{\tt GEANT} and its major functions. Each section is identified by a
{\it keyword} which indicates its content. Sections are in alphabetical
order:

\begin{DLtt}{MMMMMMMM}
\item[AAAA]       introduction to the system;
\item[BASE]       {\tt GEANT} framework and user interfaces to be read first;
\item[CONS]       particles, materials and tracking medium parameters;
\item[DRAW]       the drawing package, interfaced to {\tt HIGZ};
\item[GEOM]       the geometry package;
\item[HITS]       the detector response package;
\item[IOPA]       the I/O package;
\item[KINE]       event generators and kinematic structures;
\item[PHYS]       physics processes;
\item[TRAK]       the tracking package;
\item[XINT]       interactive user interface;
\item[ZZZZ]       appendix.
\end{DLtt}

Within each section, the principal system functions or the details of
subroutines are described in a series of {\it papers} numbered from 001 to 999.
In the upper left corner it is indicated in which {\tt Geant} release 
the subroutines
were introduced and left unchanged.
The authors of the conceptual ideas or/and of the early versions of the
code are acknowledged under the item {\bf Origin}, while {\bf Revision}
contains the contributors to any important upgrade. {\bf Documentation}
is essential, but sometime implies a not negligeable amount of work. When
relevant these contributions are acknowledged here. 
In addition all reported bugs, accepted suggestions...etc...are mentioned in
the history part of the source code and correction cradle.

Subroutines which are not necessary for an understanding of
the program flow and which are not intended to be called directly by
the user have been omitted.

The notation {\tt [<KEYW>nnn]} is used whenever
additional information can be found in the quoted section.
In the description of subroutine calling sequences,
the arguments used both on input and on output
are preceded by a * and the output arguments are followed by a * .

For convenience, two more sections have been added:
the section {\tt AAAA}, for general introductory information
at the beginning, and
the section {\tt ZZZZ}, for various appendices and indexed lists, at
the end.

A table of contents is available in {\tt AAAA000}.
To ease access to this documentation an index
appears in {\tt ZZZZ999}. It gives in alphabetic order the names of all
documented {\tt GEANT} subroutines with references
to the appropriate write up(s).

A short write up of {\tt GEANT} can be obtained by collecting the papers
numbered 001 to 009 in each section.
 
