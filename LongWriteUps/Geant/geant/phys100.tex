%%%%%%%%%%%%%%%%%%%%%%%%%%%%%%%%%%%%%%%%%%%%%%%%%%%%%%%%%%%%%%%%%%%
%                                                                 %
%  GEANT manual in LaTeX form                              %
%                                                                 %
%  Michel Goossens (for translation into LaTeX)                   %
%  Version 1.00                                                   %
%  Last Mod. Jan 24 1991  1300   MG + IB                          %
%                                                                 %
%%%%%%%%%%%%%%%%%%%%%%%%%%%%%%%%%%%%%%%%%%%%%%%%%%%%%%%%%%%%%%%%%%%
\Origin{R. Brun}
\Revision{F.Carminati}
\Submitted{30.05.86}\Revised{15.12.93}
\Version{Geant 3.16}\Routid{PHYS100}
\Makehead{Steering routine for physics initialisation}
 
\Shubr{GPHYSI}{}
 
The routine \Rind{GPHYSI} is called at initialisation time by the user
usually from the \Rind{UGINIT} routine (see example in {\tt [BASE100]}). 
The purpose of \Rind{GPHYSI} is twofold:
\begin{itemize}
\item print the parameters of the current run
\item compute the cross-section and energy loss tables and fill the
{\tt JMATE} data structure {\tt [CONS199]}.
\end{itemize}
\section {Print the run definitions }
 
\Rind{GPHYSI} prints the {\tt GEANT} version number and the 
{\tt ZEBRA}~\cite{bib-ZEBRA} version number stored
in the {\tt JRUNG} data structure {\tt [BASE299]} 
 
\Rind{GPHYSI} also prints the tracking and physics
parameters for the current run, which are stored in
the {\tt JTMED} data structure.
See the description of the {\tt JTMED} 
data structure in {\tt [CONS210]} and {\tt [CONS299]}.
 
A summary table like below will be output by \Rind{GPHYSI}
\begin{verbatim}
 ************************************************************
 *                                                          *
 *    G E A N T  Version 3.1600      DATE/TIME 930727/1135  *
 *                                                          *
 *                      R U N      1                        *
 *                                                          *
 ************************************************************
 *                                                          *
 *      Data structure   Date   Time    GVERSN    ZVERSN    *
 *      --------------   ----   ----    ------    ------    *
 *                                                          *
 *           INIT       930727  1135    3.1600     3.71     *
 *                                                          *
 *           KINE       930727  1135    3.1600     3.71     *
 *                                                          *
 *           HITS       930727  1135    3.1600     3.71     *
 *                                                          *
 *           DIGI       930727  1135    3.1600     3.71     *
 *                                                          *
 *     Random number seeds:          9876        54321      *
 *     --------------------                                 *
 *                                                          *
 *----------------------------------------------------------*
 *                                                          *
 *              Standard TPAR for this run are              *
 *              ------------------------------              *
 *                                                          *
 *  CUTGAM=100.00 keV  CUTELE=100.00 keV  CUTNEU=100.00 keV *
 *  CUTHAD=100.00 keV  CUTMUO=  1.00 MeV                    *
 *  BCUTE =100.00 keV  BCUTM =100.00 keV                    *
 *  DCUTE =100.00 keV  DCUTM =100.00 keV  PPCUTM= 10.00 MeV *
 *  IPAIR =        1.  ICOMP =        1.  IPHOT =        1. *
 *  IPFIS =        1.  IDRAY =        1.  IANNI =        1. *
 *  IBREM =        1.  IHADR =        1.  IMUNU =        1. *
 *  IDCAY =        1.  ILOSS =        1.  IMULS =        1. *
 *  IRAYL =        0.  ILABS =        0.  ISYNC =        0. *
 *  ISTRA =        0.                                       *
 *                                                          *
 *                                                          *
 *     Special TPAR for TMED   5   GAS                      *
 *     -------------------------                            *
 *  CUTGAM=100.00 keV  CUTELE=100.00 keV  CUTNEU=100.00 keV *
 *  CUTHAD=100.00 keV  CUTMUO=  1.00 MeV                    *
 *  BCUTE =100.00 keV  BCUTM =100.00 keV                    *
 *  DCUTE = 10.00 GeV  DCUTM =100.00 keV  PPCUTM= 10.00 MeV *
 *  IPAIR =        1.  ICOMP =        1.  IPHOT =        1. *
 *  IPFIS =        1.  IDRAY =        1.  IANNI =        1. *
 *  IBREM =        1.  IHADR =        1.  IMUNU =        1. *
 *  IDCAY =        1.  ILOSS =        1.  IMULS =        1. *
 *  IRAYL =        0.  ILABS =        0.  ISYNC =        0. *
 *  ISTRA =        1.                                       *
 *                                                          *
 *                                                          *
 ************************************************************
\end{verbatim}
The meaning of the printed values is described in the documentation of the
commons \FCind{/GCCUTS/} and \FCind{/GCPHYS/} in {\tt [BASE030]}.
Different values of the parameters can be assigned by the user to each
tracking medium (see {\tt [BASE100]}). The parameter definition in 
{\tt GEANT} proceeds as follows:
\begin{enumerate}
\item
a default set of {\it global} values of the parameters is defined for 
all tracking media by the routine {\tt GINIT};
\item
the default values can be changed by the user via the data records read 
by the routine \Rind{GFFGO}. A summary of all valid data records is given 
in {\tt [BASE040]};
\item
after having defined tracking media and materials, the user can
redefine the parameters for
{\bf each medium} via the routine \Rind{GSTPAR} {\tt [CONS210]};
\item
if the data structures are read from an external file
(see {\tt IOPA} section),
all the parameters are taken from this file. In this case the user can
still modify them, but this should be done {\bf before} calling the
routine \Rind{GPHYSI}.
\end{enumerate}
\section{Compute cross-section and energy loss tables}
 
\Rind{GPHYSI} is the steering routine to compute the cross-section
and energy loss tables for all materials used as tracking media.
\Rind{GPHYSI} builds and fills the {\tt JMATE} data structure described in
{\tt [CONS199]}. Here, we give the flow chart of this process.
The description of the specialised routines can be found in the
rest of the section {\tt PHYS}.
 
{\bf Note}:
     if  several tracking media are using the same material
(for instance a calorimeter and a chamber support can be both in steel)
the energy cuts must be the same. If this is not possible,
the user must define materials with different numbers.
%
\begin{tabbing}
MM \= M \= M \= MMMMMMMMMMM \= \kill
\Rind{GPHYSI}     \> \> \> \> \parbox[t]{10cm}{Initialisation of physics
processes. This routine should be called by the user before the tracking
starts, but after all the material, {\it mechanism} flags and energy
cuts have been defined. \Rind{GPHYSI} should also be called whenever
a new initialisation data structure is read from disk.} \rule{0cm}{.45cm} \\
 \>  \Rind{GRNDMQ}   \> \> \> 
\parbox[t]{10cm}{Initialises the value of the
seeds of the random number generator. See {\tt [BASE420]} for more 
information} \rule{0cm}{.45cm} \\
 \>  \Rind{GEVKEV}   \> \> \> 
\parbox[t]{10cm}{Routine to format energy
values for printout. See {\tt [BASE410]} for more details.} \rule{0cm}{.45cm} \\
 \>  \Rind{GPHINI}   \> \> \> 
\parbox[t]{10cm}{Initialisation of the
constants for photoelectric effect (see {\tt [PHYS230]}).} \rule{0cm}{.45cm} \\
\rule[-.1cm]{6cm}{.05cm} Loop on tracking media 
\rule[-.1cm]{6cm}{.05cm} \rule{0cm}{.45cm} \\
 \>  \Rind{GPHXSI}   \> \> \> \parbox[t]{10cm}{Initialisation of the cross
section coefficients for the photoelectric effect in a tracking medium, see
{\tt [PHYS230]}.}
\rule{0cm}{.45cm} \\
 \>  \Rind{GPROBI}   \> \> \> \parbox[t]{10cm}{Initialisation
of constants for various physical effects.} \rule{0cm}{.45cm} \\
 \> \>  \Rind{GMOLI}    \> \> \parbox[t]{10cm}{Initialises constants for
Moli\`ere scattering, see {\tt [PHYS325]}.} \rule{0cm}{.45cm} \\
\rule[-.1cm]{6cm}{.05cm} Loop on energy bins 
\rule[-.1cm]{6cm}{.05cm} \rule{0cm}{.45cm} \\
 \>  \Rind{GDRELA}   \> \> \> \parbox[t]{10cm}{Initialises the ionisation
energy loss tables
$dE/dx$ for protons, $\mu$ and \Pep, \Pem, see {\tt [PHYS430]}.} 
\rule{0cm}{.45cm} \\
 \> \>  \Rind{GDRELP}   \> \> \parbox[t]{10cm}{Calculates ionisation
energy loss for protons, see {\tt [PHYS430]}.} \rule{0cm}{.45cm} \\
 \> \>  \Rind{GDRELM}   \> \> \parbox[t]{10cm}{Calculates ionisation
energy loss for $\mu$, see {\tt [PHYS430]}.} \rule{0cm}{.45cm} \\
 \> \>  \Rind{GDRELE}   \> \> \parbox[t]{10cm}{Calculates ionisation
energy loss for \Pem/\Pep, see {\tt [PHYS330]}.} \rule{0cm}{.45cm} \\
 \>  \Rind{GBRELA}   \> \> \> \parbox[t]{10cm}{Adds the contribution of
bremsstrahlung to continuous energy loss tables for protons, $\mu$
and \Pem/\Pep, see {\tt [PHYS440]}.} \rule{0cm}{.45cm} \\
 \>  \Rind{GPRELA}   \> \> \> \parbox[t]{10cm}{Adds the contribution
of direct pair production and $\mu$-nucleus interactions to the
$\mu$ continuous energy loss tables, see {\tt [PHYS450]}.} \rule{0cm}{.45cm} \\
 \>  \Rind{GPHOTI}   \> \> \> \parbox[t]{10cm}{Calculates the
cross section for photoelectric effect, see {\tt [PHYS230]}.}\rule{0cm}{.45cm}\\
 \>  \Rind{GRAYLI}   \> \> \> \parbox[t]{10cm}{Initialises the tables
of cross sections for the Rayleigh effect, see {\tt [PHYS250]}.} 
\rule{0cm}{.45cm} \\
 \>  \Rind{GANNII}   \> \> \> \parbox[t]{10cm}{Initialises the cross 
section for positron annihilation, see {\tt [PHYS350]}.} \rule{0cm}{.45cm} \\
 \>  \Rind{GCOMPI}   \> \> \> \parbox[t]{10cm}{Initialises the cross section 
tables for Compton effect, see {\tt [PHYS220]}.} \rule{0cm}{.45cm} \\
 \>  \Rind{GBRSGA}   \> \> \> \parbox[t]{10cm}{Initialises the cross
section tables for discrete bremsstrahlung of \Pem/\Pep and
$\mu$, see {\tt [PHYS340]}, {\tt [PHYS440]}.} \rule{0cm}{.45cm} \\
 \>  \Rind{GPRSGA}   \> \> \> \parbox[t]{10cm}{Initialises cross
section tables for pair production by photons and direct
pair production by muons, see {\tt [PHYS210]}, {\tt [PHYS450]}.} 
\rule{0cm}{.45cm} \\
 \>  \Rind{GDRSGA}   \> \> \> \parbox[t]{10cm}{Initialises the
cross section tables for delta rays production by $\mu$ and
electrons, see {\tt [PHYS330]}, {\tt [PHYS430]}.} \rule{0cm}{.45cm} \\
 \>  \Rind{GMUNUI}   \> \> \> \parbox[t]{10cm}{Initialises the cross section
tables for $\mu$-nucleus interactions, see {\tt [PHYS460]}.} 
\rule{0cm}{.45cm} \\
 \>  \Rind{GPFISI}   \> \> \> \parbox[t]{10cm}{Initialises the cross section
tables for photo-fission and photo-absorption, see {\tt [PHYS240]}.} 
\rule{0cm}{.45cm} \\
\rule[-.1cm]{6cm}{.05cm} End of loop on energy bins 
\rule[-.1cm]{6cm}{.05cm} \rule{0cm}{.45cm} \\
 \>  \Rind{GRANGI}   \> \> \> \parbox[t]{10cm}{Calculates the stopping
range integrating the $dE/dx$ tables, see {\tt [PHYS010]}.} \rule{0cm}{.45cm} \\
 \>  \Rind{GCOEFF}   \> \> \> \parbox[t]{10cm}{Calculation of the coefficients
of the interpolating parabolas for the range tables, see {\tt [PHYS010]}.} 
\rule{0cm}{.45cm} \\
 \>  \Rind{GSTINI}   \> \> \> \parbox[t]{10cm}{Initialisation of the
energy loss tables for the thin layers, see {\tt [PHYS334]}.} 
\rule{0cm}{.45cm} \\
 \> \>  \Rind{GFCNRM}   \> \> \parbox[t]{10cm}{Calculation of
mean ionisation potentials and normalisation factors for
oscillator functions.} \rule{0cm}{.45cm} \\
 \> \>  \Rind{GSREE0}   \> \> \parbox[t]{10cm}{Calculation of the real
part of the refractive index.} \rule{0cm}{.45cm} \\
 \> \>  \Rind{GSDNDX}   \> \> \parbox[t]{10cm}{Calculation of number of
collisions for a given value of $\beta$.} \rule{0cm}{.45cm} \\
 \>  \Rind{GMULOF}   \> \> \> \parbox[t]{10cm}{Calculation of the
{\tt STMIN tables}, see {\tt [PHYS325]}.} \rule{0cm}{.45cm} \\
 \> \>  \Rind{GMOLIO}   \> \> \parbox[t]{10cm}{Calculation of the
constants of Moli\`ere scattering to estimate the step limitation
due to multiple scattering simulation, see {\tt [PHYS325]}.} 
\rule{0cm}{.45cm} \\
\rule[-.1cm]{5.5cm}{.05cm} End of loop on tracking media 
\rule[-.1cm]{5.5cm}{.05cm} 
\rule{0cm}{.45cm} \\
\end{tabbing}
{\bf Notes}:
\begin{enumerate}
\item
The $\mu$-nucleus interactions are treated
{\bf either} as a continuous energy loss by the muon ({\tt IMUNU}$=0$)
{\bf or} as a discrete process ({\tt IMUNU}$\geq$1),
{\bf in an exclusive way} i.e.:
\begin{itemize}
\item If {\tt IMUNU}$=0$, the $dE/dx$ due to the interactions is added
to that coming from direct ($ e^+/e^-$) pair production in the routine 
\Rind{GPRELA}.
\item
If {\tt IMUNU}$\geq 1$, the total cross section is computed in 
\Rind{GMUNUI}.
\end{itemize}
\item
The total cross sections for hadronic interactions cannot be tabulated at
initialisation time, as they are dependent on the nature
of the incident particle. They are computed at tracking time by the function
\Rind{FLDIST} in case of {\tt FLUKA} {\tt [PHYS520]}, \Rind{GHESIG} 
in case of {\tt GHEISHA} {\tt [PHYS510]} or \Rind{FMDIST} in case of {\tt MICAP}
{\tt [PHYS530]}.
\end{enumerate}
 
