%%%%%%%%%%%%%%%%%%%%%%%%%%%%%%%%%%%%%%%%%%%%%%%%%%%%%%%%%%%%%%%%%%%
%                                                                 %
% GEANT manual in LaTeX form                                      %
%                                                                 %
% Michel Goossens (for translation into LaTeX)                    %
% Version 1.00                                                    %
% Last Mod. Jan 24 1991 1300  MG + IB                             %
%                                                                 %
%%%%%%%%%%%%%%%%%%%%%%%%%%%%%%%%%%%%%%%%%%%%%%%%%%%%%%%%%%%%%%%%%%%
\Origin{L.Urb\'{a}n}
\Revision{J. Chwastowski}
\Submitted{26.10.84}\Revised{16.12.93}
\Version{GEANT 3.16}\Routid{PHYS231}
\Makehead{Simulation of photoelectric Effect}
\section{Subroutines}
\Shubr{GPHOT}
{\tt GPHOT} simulates the photoelectric effect. It uses the following
input and output:
\begin{DLtt}{MMMMMMM}
\item[Input]{via common \Cind{GCTRAK}}
\item[Output]{via common \Cind{GCKING}}
\end{DLtt}
It calls the functions \Rind{GPHSG1}, \Rind{GPHSGP}, \Rind{GAVRL2}
and \Rind{GAVRL3} for the cross-section and the functions \Rind{GHPHAK},
\Rind{GHPHAL1}, \Rind{GHPHAL2}, \Rind{GPHAL3} for the cosine distribution 
of the photoelectron. \Rind{GPHOT} is called by the tracking routine
\Rind{GTGAMA}, if, and when, the photon reaches its interaction point.
\Sfunc{GAVRL2}{VALUE = GAVRL2(GAMAL2,BETAL2,X)}
\Rind{GAVRL2} calculates approximation of the cross-section for the 
photoelectric effect from $L_{II}$ shell. {\tt GAMAL2},
{\tt BETAL2} are the Lorentz gamma and beta factors of the emitted 
photoelectron and $X = m_e/E_\gamma$ where $m_e$ is
the electron mass and $E_\gamma$ is the incident radiation energy.
\Rind{GAVRL2} is called by \Rind{GHPOT}.
\Sfunc{GAVRL3}{VALUE = GAVRL3(GAMAL3,BETAL3,X)}
\Rind{GAVRL3} calculates approximation of the cross-section for the 
photoelectric effect from $L_{III}$ shell.
{\tt GAMAL3}, {\tt BETAL3} are the Lorentz gamma and beta factors of the 
emitted photoelectron and $X = m_e/E_\gamma$ where $m_e$ is the electron 
mass and $E_\gamma$ is the  incident radiation energy.
{\tt GAVRL3} is called by \Rind{GPHOT}.
\Sfunc{GPHAK}{VALUE = GPHAK(BETAK)}
\Sfunc{GPHAL1}{VALUE = GPHAL1(BETAL1)}
{\tt GPHAK} and its entry point {\tt GPHAL1} poll the cosine distribution 
(w.r.t. the incident photon direction) of the photoelectron emitted from 
$K$ or $L_{I}$ shell, respectively. They are called by \Rind{GPHOT}.
\Sfunc{GPHAL2}{VALUE = GPHAL2(BETAL2)}
\Rind{GPHAL2} polls the cosine distribution (w.r.t. the incident photon
direction) of the photoelectron emitted from $L_{II}$ shell.
\Sfunc{GPHAL3}{VALUE = GPHAL3(BETAL3)}
{\tt GPHAL3} polls the cosine distribution (w.r.t. the incident photon
direction) of the photoelectron emitted from $L_{III}$ shell.
It is called by \Rind{GPHOT}.
\Sfunc{GPHSGP}{VALUE = GPHSGP(EGAM)}
\Rind{GPHSGP} calculates the photoelectric effect total cross-section for
a particular atom of a mixture taking into account weights. It is called by 
\Rind{GPHOT} for ${\tt Z} \leq  100$.
 
\section{Method}
 
If the energy of the radiation incident on an atom is $E_{\gamma}$, then
the quanta can be absorbed if $E_{\gamma} > E_{shell}$.
The photoelectron is emitted with total energy:
\begin{equation}
E_{photoelectron} = E_{\gamma}-E_{shell}+m_e.
\end{equation}
In the above equation it is assumed that the atom has infinite mass. One should
note that there exists a correction term (see \cite{bib-BETHE} and
references therein) which becomes more and more important with
increasing $E_{\gamma}$ \cite{bib-HALL} \cite{bib-PRATT} \cite{bib-PRATT1}.

\subsection{Probability of Interaction with an Atom}
Probability of the interaction with an atom is calculated taking into
account partial cross-sections of atoms of a mixture
or a compound.
\subsection{Probability of Interaction with a Shell}
To calculate the probability of the interaction with a particular shell we use
the jump ratios defined as:
\begin{equation}
J_{shell} = \frac{\sigma(E_{shell}+\delta E)}{\sigma(E_{shell}-\delta E)}
\label{eq:jumprat}
\end{equation}
where $\delta E \rightarrow 0$.
In addition we assume that the jump ratio is also valid away from the edges.\\
From (\ref{eq:jumprat}) it follows that the probability $p_{shell}$ 
to interact with a shell is:
\begin{equation}
p_{shell} = 1-\frac{1}{J_{shell}}
\label{eq:probrat}
\end{equation}
We use the following parametrisation of the jump ratios for $K$ and
$L_{I}$ shells\cite{bib-VEIGELE}:
\begin{equation}
J_K = \frac{125}{Z} + 3.5
\end{equation}
\begin{equation}
J_{L_{I}} = 1.2
\label{eq:jumpl1}
\end{equation}
For the $L_{II}$ and $L_{III}$ shells we adopt approximation of the formulae
calculated by Gavrila \cite{bib-GAVRILA-L}:
\begin{eqnarray}
\sigma_{L_{II}} &=& \gamma \beta \frac{m_e}{E_\gamma}
              \left\{
               \gamma^3-5\gamma^2+24\gamma-16
               -(\gamma^2+3\gamma-8)\frac{\log(\gamma(1+\beta))}{\gamma\beta}
              \right\}
\label{eq:sigmal2}
\end{eqnarray}
and
\begin{eqnarray}
\sigma_{L_{III}} &=& \gamma \beta \frac{m_e}{E_\gamma}
             \left\{
                4\gamma^3-6\gamma^2+5\gamma+3
               -(\gamma^2-3\gamma+4)\frac{\log(\gamma(1+\beta))}{\gamma\beta}
             \right\}
\label{eq:sigmal3}
\end{eqnarray}
where\\
 
\begin{tabular}[t]{l@{\hspace{1cm}}p{.7\textwidth}}
$ \gamma, \beta $    & are the emitted photoelectron  Lorentz gamma and beta
                       factors;\\
$ E_\gamma      $    & is the incident radiation energy;  \\
$ m_e           $    & is the electron mass.  \\
\end{tabular}\\
Below an example of the shell interaction probability calculations for 
$E_\gamma > E_K$ is given.\\
If
\begin{eqnarray*}
   \Sigma_{II,III} & = &  \sigma_{L_{II}}+\sigma_{L_{III}} \\
   r_{L_{II}} & = & \frac{\sigma_{L_{II}}}{\Sigma_{II,III}}  \\
   r_{L_{III}} & = & \frac{\sigma_{L_{III}}}{\Sigma_{II,III}}  
\end{eqnarray*}
then from (\ref{eq:probrat}) one can find that
\begin{eqnarray*}
       p_K & = & 1-\frac{1}{J_K} \\
       p_{L_1} & = & (1-p_K)\cdot (1 - \frac{1}{J_{L_1}}) \\
       p_{L_{II}} & = & (1-p_K-p_{L_I})\cdot r_{L_{II}} \\
       p_{L_{III}} & = & (1-p_K-p_{L_I})\cdot r_{L_{III}} 
\end{eqnarray*}
After simple calculations one obtains the probability distribution function
which is used to select the shell.
\subsection{Angular distributions of photoelectrons}
 
The angular distributions of photoelectrons should be calculated using the
partial wave analysis for the outgoing electron. Since this method is very time
consuming we use approximations of the angular distributions calculated by
F. von Sauter \cite{bib-SAUTER1} \cite{bib-SAUTER2} (K shell) and Gavrila
\cite{bib-GAVRILA-K} \cite{bib-GAVRILA-L} 
(K and L shells). We use the cross-section
which is correct only to zero order in $\alpha Z$. This approximation foresees
zero scattering amplitude in the direction of incident photon while
experimentally the cross-section at $0$ angle is non-vanishing \cite{bib-NAGEL}.
If
\begin{eqnarray*}
           X &=& 1-\beta\cos\Theta
\end{eqnarray*}
then for $K$ and $L_I$ shells we use:
\begin{eqnarray}
           \frac{d\sigma_{K,L_{I}}}{d(\cos\Theta)}
           &\sim& \frac{\sin^2\Theta}{X^4}
                \left\{1+ \frac{1}{2}\gamma(\gamma-1)(\gamma-2)\right\}
\label{eq:angkl}
\end{eqnarray}
 and for $L_{II}$  and $L_{III}$ shells we have:
\begin{eqnarray}
 \frac{d\sigma_{L_{II}}}{d(\cos\Theta)}
           &\sim & \frac{(\gamma^2-1)^{\frac{1}{2}}}{\gamma^5(\gamma-1)^5}
           \left\{ \frac{\gamma(3\gamma+1)}{2 X^4}
      -\frac{\gamma^2(9\gamma^2+30\gamma-7)}{8 X^3} \right.  \nonumber \\
      \ & \ &      +\frac{\gamma^3(\gamma^3+6\gamma^2+11\gamma-2)}{4 X^2}
            -\frac{\gamma^4(\gamma-1)(\gamma+7)}{8 X} \\
         \ & \ &    +\sin^2\Theta
                  \left.    \left(
                         \frac{(\gamma+1)}{X^5}
                        -\frac{\gamma(\gamma+1)}{X^4}
                        -\frac{\gamma^2(3\gamma+1)(\gamma^2-1)}{X^3}
                      \right)
           \right\} \nonumber
\label{eq:angl2}
\end{eqnarray}
 
\begin{eqnarray}
           \frac{d\sigma_{L_{III}}}{d(\cos\Theta)}
           &\sim & \frac{(\gamma^2-1)^{\frac{1}{2}}}{\gamma^5(\gamma-1)^5}
           \left\{ -\frac{\gamma(3\gamma-1)}{2 X^4}
                   +\frac{\gamma^2(3\gamma^2-1)}{X^3} \right. \nonumber \\
        \ & \ & \frac{\gamma^3(\gamma^3-3\gamma^2+2\gamma+1)}{X^3}
                   -\frac{\gamma^4(\gamma-2)(\gamma-1)}{2 X} \\
        \ & \ & +\sin^2\Theta
                  \left.    \left(
                   \frac{(\gamma+1)}{X^5}
                  -\frac{\gamma(\gamma+1)(3\gamma-1)}{X^4}
                  +\frac{\gamma^2(\gamma^2-1)}{X^3}
                      \right)
           \right\} \nonumber
\label{eq:angl3}
\end{eqnarray}
The azimuthal angle distribution is uniform.
 
\subsection{Final State}
 
After the photoelectron emission the atom is left in excited state.
The excitation energy equal to the binding energy $E_i$ of the shell in which the interaction took place. Subsequently the atom emits a fluorescent photon
or Auger or Coster-Kronig electron. The selection of radiative or
non-radiative transition is based on compilation by Krause \cite{bib-KRAUSE}.\\
The Auger or Coster-Kronig transitions are represented by the most probable
line for a given vacancy \cite{bib-CULLEN}. The emitted electron energy $E_e$ is
\begin{equation}
 E_e = E_i-(E_j+E_k)
\label{eq:augere}
\end{equation}
where
$E_i, E_j, E_k$ are the subshell binding energies and $E_j > E_k$.\\
In case of fluorescence we use transition rates of Scofield \cite{bib-SCOFIELD}.
We use only those transitions for which the occurrence probability is not less
than 1\%. The fluorescent photon is emitted with energy $E_\gamma$
\begin{equation}
E_\gamma = E_i-E_j
\label{eq:fluore}
\end{equation}
for transition between the subshells $i$ and $j$.\\
In addition to the above, to fulfill the energy conservation law,
emission of an additional photon is simulated. For non-radiative transitions 
its energy is $E_k$ (see formula \ref{eq:augere}). In case of fluorescent
transition this photon has energy $E_j$ (see equation \ref{eq:fluore}). 
The angular distribution of the emitted particle is isotropic.
