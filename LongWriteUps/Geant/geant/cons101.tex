%%%%%%%%%%%%%%%%%%%%%%%%%%%%%%%%%%%%%%%%%%%%%%%%%%%%%%%%%%%%%%%%%%%
%                                                                 %
%  GEANT manual in LaTeX form                                     %
%                                                                 %
%  Michel Goossens (for translation into LaTeX)                   %
%  Version 1.00                                                   %
%  Last Mod. Jan 24 1991  1300   MG + IB                          %
%                                                                 %
%%%%%%%%%%%%%%%%%%%%%%%%%%%%%%%%%%%%%%%%%%%%%%%%%%%%%%%%%%%%%%%%%%%
\Origin{R.Brun, M.Maire}
\Version{Geant 3.16}\Routid{CONS101}
\Submitted{23.04.86}         \Revised{03.11.93}
\Makehead{Retrieve material cross-sections and stopping power}
 
\Shubr{GFTMAT}{(IMATE,IPART,CHMECA,KDIM,TKIN,VALUE*,PCUT*,IXST*)}
 
{\bf Retrieve} and {\bf interpolate} the $dE/dx$ and cross-sections
tabulated in material banks corresponding to a given array of kinetic
energy bins. 

\begin{DLtt}{MMMMMMMM}
\item[IMATE]     ({\tt INTEGER}) material number;
\item[IPART]     ({\tt INTEGER}) particle number;
\item[CHMECA] ({\tt CHARACTER*4}) name of the mechanism bank to be retrieved;
\item[KDIM]  ({\tt INTEGER}) dimension of the arrays {\tt TKIN} and {\tt VALUE};
\item[TKIN]   ({\tt REAL}) array of kinetic energies in GeV;
\item [VALUE]    ({\tt REAL}) array of energy losses in MeV cm$^{-1}$, 
or macroscopic cross sections in cm$^{-1}$;
\item[PCUT]({\tt REAL}) array containing the 5 energy thresholds for the material
in GeV ({\tt CUTGAM}, {\tt CUTELE}, {\tt CUTNEU}, {\tt CUTHAD}, {\tt CUTMUO});
\item[IXST] ({\tt INTEGER}) return code:
\begin{DLtt}{MMM}
\item[0]   the table was not filled:
\item[1]   normal return;
values are returned into the {\tt VALUE} array;
\end{DLtt}
\end{DLtt}

The mechanisms have the following
conventional name and a code number ({\tt IMECA}):
 
\begin{DLtt}{MMMMMMMM}
\item[HADF (~1)] total hadronic interaction cross-section according
to {\tt FLUKA};
\item[INEF (~2)] hadronic inelastic cross-section according
to {\tt FLUKA};
\item[ELAF (~3)] hadronic elastic cross-section according
to {\tt FLUKA};
\item[NULL (~4)] unused;
\item[NULL (~5)] unused;
\item[HADG (~6)] total hadronic interaction cross-section according
to {\tt GHEISHA};
\item[INEG (~7)] hadronic inelastic cross-section according
to {\tt GHEISHA};
\item[ELAG (~8)] hadronic elastic cross-section according
to {\tt GHEISHA};
\item[FISG (~9)] nuclear fission cross-section according
to {\tt GHEISHA};
\item[CAPG (10)] neutron capture cross-section according
to {\tt GHEISHA};
\item[LOSS (11)] stopping power;
\item[PHOT (12)] photoelectric cross-section;
\item[ANNI (13)] positron annihilation cross-section;
\item[COMP (14)] Compton effect cross-section;
\item[BREM (15)] bremsstrahlung cross-section;
\item[PAIR (16)] photon and muon direct- \Pem\Pep pair cross-section;
\item[DRAY (17)] $\delta$-rays cross-section;
\item[PFIS (18)] photo-fission cross-section;
\item[RAYL (19)] Rayleigh scattering cross-section;
\item[MUNU (20)] muon-nuclear interaction cross-section;
\item[RANG (21)] range in cm;
\item[STEP (22)] maximum step in cm.
\end{DLtt}
 
{\bf Note:}
Common \FCind{/GCMULO/} contains an array {\tt ELOW(200)}
(see {\tt [CONS199]})
with {\tt NEK1} kinetic energy values
ranging from {\tt EKMIN} to {\tt EKMAX}. \Rind{GPHYSI} initialises by
default the first 91 locations of {\tt ELOW} with values of energy
from $10^{-5}$ GeV (10 keV) to $10^{4}$ GeV (10 TeV) equally spaced on
logarithmic scale. This can be controlled by the user via the data
record or the interactive command {\tt ERAN} ({\tt [BASE040]}).
{\tt ELOW} can be used as the input argument {\tt TKIN}, e.g.:
\begin{verbatim}
CALL GFTMAT (10, 1,'PHOT',NEK1, ELOW ,VALUE, PCUT, IXST)
\end{verbatim}
will return in array {\tt VALUE} the photoelectric cross-section
in material number 10.

\Shubr{GPLMAT}{(IMATE,IPART,CHMECA,KDIM,TKIN,IDM)}
 
{\bf Plot} and {\bf interpolate} the $dE/dx$ and cross-sections
tabulated in material banks corresponding to a given array of kinetic
energy bins. The mechanisms are the same as the previous routine. 
If the mechanism name is {\tt 'ALL'}, then all the mechanisms will
be plotted.
 
The first five parameters have the same meaning as in the previous routine
and:
\begin{DLtt}{MMMMMMMM}
\item[IDM]({\tt INTEGER}) treatment of the created histogram(s):
\begin{DLtt}{MMM}
\item[$>0$]fill, print, keep histogram(s);
\item[$=0$]fill, print, delete histogram(s);
\item[$<0$]fill, noprint, keep histogram(s).
\end{DLtt}
\end{DLtt}
The result is not returned to the user as is the case in the previous
routine but it is put in an {\tt HBOOK} histogram. The {\tt HBOOK}
package needs to be initialised by the user via the routine \Rind{HLIMIT},
see {\tt [BASE100]} for more information.
The histogram identifier is calculated as: {\tt 10000*IMATE+100*IPART+IMECA}
where {\tt IMECA} is the mechanism code listed above.
 
\Shubr{GPRMAT}{(IMATE,IPART,CHMECA,KDIM,TKIN)}
 
{\bf Print} and {\bf interpolate} the $dE/dx$ and cross-sections
tabulated in material banks corresponding to a given array of kinetic
energy bins. The mechanisms are the same as the previous routine. 
If the mechanism name is {\tt 'ALL'}, then all the mechanisms will
be printed.

The result is printed in the form of a table.
 
The five parameters have the same meaning as in the previous routine.
