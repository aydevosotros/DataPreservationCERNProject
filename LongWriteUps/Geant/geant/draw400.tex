%%%%%%%%%%%%%%%%%%%%%%%%%%%%%%%%%%%%%%%%%%%%%%%%%%%%%%%%%%%%%%%%%%%
%                                                                 %
%  GEANT manual in LaTeX form                                     %
%                                                                 %
%  Michel Goossens (for translation into LaTeX)                   %
%  Version 1.00                                                   %
%  Last Mod. Jan 24 1991  1300   MG + IB                          %
%                                                                 %
%%%%%%%%%%%%%%%%%%%%%%%%%%%%%%%%%%%%%%%%%%%%%%%%%%%%%%%%%%%%%%%%%%%
\Origin{P.Zanarini}
\Revision{S.Giani}
\Documentation{P.Zanarini, S.Giani, F. Carminati}
\Submitted {15.08.83}          \Revised{16.12.93}
\Version{Geant 3.16}\Routid{DRAW400}
\Makehead{Utility routines of the drawing package}


\Shubr{GDOPT}{(CHOPT,CHVAL)}
 
\begin{DLtt}{MMMMMMMM}
\item[CHOPT] ({\tt CHARACTER*4}) option to be set;
\item[CHVAL] ({\tt CHARACTER*4}) value to be assigned to the option.
\end{DLtt}

Drawing options can be set via the routine \Rind{GDOPT} 
(corresponding to the {\tt DOPT} interactive command).
The possible value of {\tt CHOPT} and of {\tt CHVAL} are the following:
\begin{DLtt}{MMMMMMMM}
\item[PROJ] selects the type of projection, it can have the following
values:
\begin{DLtt}{MMMMMM}
\item[PARA] orthographic parallel projection (the default);
\item[PERS] perspective projection.
\end{DLtt}
\item[THRZ] acronym for {\tt T}racks and {\tt H}its in a {\tt R}adius 
versus {\tt Z} projection, where {\tt R} is the {\it vertical} axis
of the drawing and
{\tt Z} the horizontal one. If a {\it cut view} has been drawn with the
following parameters:
\begin{DLtt}{MMMMMMM}
\item[X-Z cut] {\tt CUTTHE=90} and {\tt CUTPHI=90} in \Rind{GDRAWX} or
{\tt icut=2} in \Rind{GDRAWC};
\item[Y-Z cut] {\tt CUTTHE=90} and {\tt CUTPHI=180} in \Rind{GDRAWX} or
{\tt icut=1} in \Rind{GDRAWC}.
\end{DLtt}
the different values of the option {\tt THRZ} have the following effect
on the drawing of hits and tracks:
\begin{DLtt}{MMMMMM}
\item[360] hits and tracks will be rotated around {\tt Z} onto the
{\tt Z-R, R$>$0} half plane;
\item[180] (equivalent to {\tt ON}) hits and tracks will be rotated
around {\tt Z} onto the {\tt Z-R} plane by the smaller $\phi$ angle,
so to conserve the sign of their {\tt R}-ordinate (where {\tt R} is
either {\tt X} or {\tt Y};
\item[OFF] (the default) tracks and hits will be projected from their
position in space onto the plane {\tt R-Z}.
\end{DLtt}
 
{\bf Note:} when the {\tt THRZ} option is set to {\tt ON} or {\tt 180},
particles originating in one half space and crossing the horizontal axis
will be reflected back into the original half space. If the option 
{\tt 360} can be 
chosen, all tracks and hits are on the top (positive) half space.
\item[TRAK] selects the way tracks are drawn. Possible values:
\begin{DLtt}{MMMMMM}
\item[LINE] (the default) the lines joining track points are drawn by 
{\tt GDXYZ};
\item[POIN] only the track points are drawn.
\end{DLtt}
\item[HIDE] selects the hidden line removal technique for the
drawings. Possible values are:
\begin{DLtt}{MMMMMM}
\item[OFF] (the default) hidden lines are not removed from the drawing:
the drawing is usually rather fast, but the result can be
quite messy if a large number of volumes is visible;
\item[ON] hidden lines are not drawn: if this value is selected, the 
detector can be exploded (interactive command {\tt BOMB}), cut with 
different shapes (interactive command menu {\tt CVOL}), and some parts 
can be shifted from their original position (interactive command 
{\tt SHIFT}). Drawing with hidden line removal can be demanding in
terms of CPU time and memory. At the end of each drawing
the program displays the memory used and the statistics on the 
volume visibility. When the drawing requires more memory 
than available, the program will evaluate and display the number of 
missing words, so that the user can increase accordingly the
size of the \Rind{ZEBRA} store in the common \FCind{/GCBANK/}.
\end{DLtt}
\item[SHAD] when {\tt HIDE} is {\tt ON}, selects the 
shading and filling option for the detector's surfaces that are drawn:
\begin{DLtt}{MMMMMM}
\item[OFF] (the default) no shading of the surfaces will be performed;
\item[ON]  the volume surfaces will be filled according to the value of 
their {\tt FILL} and {\tt LWID} attribute, set via {\tt GSATT}:
\begin{DLtt}{MMMMMM}
\item[FILL$=$1] the surfaces are filled with the current colour assigned
to the volume via \Rind{GSATT}, 166 different colours are available;
\item[FILL$>$1] the surfaces are filled with a shade of the basic colour
assigned to the volume according to their angle with respect to the line
of illumination, the resolution
increases with the value of {\tt FILL} (1$<$ {\tt FILL} $\leq$ 7);
\item[LWID$>$0] just as the number of scan-lines used to fill a surface
depends on the value of {\tt FILL}, their width depends on the value of 
{\tt LWID}: increasing the value of {\tt LWID} will result in thicker lines.
\end{DLtt}
\end{DLtt}
\item[EDGE] control drawing of the volume edges in {\tt HIDE=ON, FILL=ON}
mode:
\begin{DLtt}{MMMMMM}
\item[ON] the edges of the faces are not drawn,
so that they will be visible only as intersections of filled faces
with different luminosity; 
\item[OFF] the edges of the faces are drawn.
For the volumes which are not filled, the normal
edges will be drawn irrespective of the value of this option.
\end{DLtt}
\end{DLtt}

\Shubr{GDZOOM}{(ZFU,ZFV,UZ0,VZ0,U0,V0)}
\begin{DLtt}{MMMMMMMM}
\item[ZFU]    ({\tt REAL}) zoom factor for {\tt U} coordinate;
\item[ZFV]    ({\tt REAL}) zoom factor for {\tt V} coordinate;
\item[UZ0]    ({\tt REAL}) {\tt U} coordinate of centre of zoom;
\item[VZ0]    ({\tt REAL}) {\tt V} coordinate of centre of zoom;
\item[U0]     ({\tt REAL}) {\tt U} coordinate of centre of zoom 
in the resulting picture;
\item[V0]     ({\tt REAL}) {\tt V} coordinate of centre of zoom 
in the resulting picture.
\end{DLtt}
This routine sets the zoom parameters (part of the viewing parameters 
in common \FCind{/GCDRAW/}) that define how objects 
will be displayed by the subsequent drawing operations.
The zoom is computed around {\tt UZ0}, {\tt VZ0} (user coordinates).
This point will be moved to {\tt U0}, {\tt V0} in the resulting picture.
{\tt ZFU}, {\tt ZFV} are the zoom factors (positive). If {\tt ZFU} 
or {\tt ZFV} are zero, the zoom parameters are reset, and the 
original viewing is restored.

\Shubr{GDAXIS}{(X0,Y0,Z0,AXSIZ)}
\begin{DLtt}{MMMMMMMM}
\item[X0]    ({\tt REAL}) {\tt X} coordinate of origin;
\item[Y0]    ({\tt REAL}) {\tt Y} coordinate of origin;
\item[Z0]    ({\tt REAL}) {\tt Z} coordinate of origin;
\item[AXSIZ] ({\tt REAL}) axis size in centimeters.
\end{DLtt}
Draws the axes of the {\tt MA}ster {\tt R}eference {\tt S}ystem,
corresponding to the current viewing parameters.
All the arguments refer to the {\tt MARS}.

\Shubr{GDSCAL}{(U0,V0)}
\begin{DLtt}{MMMMMMMM}
\item[U0]    ({\tt REAL}) {\tt U} coordinate of the centre of the scale;
\item[V0]    ({\tt REAL}) {\tt V} coordinate of the centre of the scale.
\end{DLtt}
Draws a scale corresponding to the current viewing parameters.
Seven kinds of units are available, from 1 micron to 100 cm.
The best unit is automatically selected.

\Shubr{GDMAN}{(U0,V0)}
\Shubr{GWOMA1}{(U0,V0)}
\Shubr{GDWMN2}{(U0,V0)}
\Shubr{GDWMN3}{(U0,V0)}
\begin{DLtt}{MMMMMMMM}
\item[U0]     ({\tt REAL}) {\tt U} coordinate of the centre of the figure;
\item[V0]     ({\tt REAL}) {\tt V} coordinate of the centre of the figure.
\end{DLtt}
Draws the profile of a man, or three different female figures
in 2D user coordinates, with the current scale factors.
The figure is approximately 160 cm high.

\Shubr{GDRAWT}{(U,V,CHTEXT,SIZE,ANGLE,LWIDTH,IOPT)}
\begin{DLtt} {MMMMMMMM}
\item[U]    ({\tt REAL}) {\tt U} position of text string;
\item[V]   ({\tt REAL}) {\tt V}   position of text string;
\item[CHTEXT] ({\tt CHARACTER*(*)}) text string;
\item[SIZE]   ({\tt REAL}) character size (cm);
\item[ANGLE]  ({\tt REAL}) rotation angle (degrees);
\item[LWIRTH] ({\tt INTEGER}) line width (1,2,3,4,5);
\item[IOPT]  ({\tt INTEGER}) centering option for the text:
\begin{DLtt}{MMMM}
\item[-1] left adjusted;
\item[~0] centered;
\item[~1] right adjusted.
\end{DLtt}
\end{DLtt}
Draws text with software characters. It has the same
arguments of the {\tt HPLOT}\cite{bib-HPLOT} routine \Rind{HPLSOF} 
(a call to \Rind{HPLSOF} is actually performed).

\Shubr{GDRAWV}{(U,V,NP)}
\begin{DLtt}{MMMMMMMM}
\item[U]    ({\tt REAL}) array of {\tt U} coordinates;
\item[DT]   ({\tt REAL}) array of {\tt V} coordinates;
\item[NP]   ({\tt INTEGER}) number of points.
\end{DLtt}
Draws 2D polylines in user coordinates.
The routine \Rind{GDFR3D} can be called to transform 3D points 
in 2D user coordinates.

\Shubr{GDHEAD}{(ISEL,CHNAME,CHRSIZ)}
\begin{DLtt}{MMMMMMMM}
\item[ISEL]  ({\tt INTEGER}) option to be selected for the title name 
(decimal integer):
\begin{DLtt}{MMMMMMM}
\item[~~~~~0] only the header lines;
\item[xxxxx1] add the text {\tt CHNAME} centered on top of header;
\item[xxxx1x] add global detector name (first volume) on left;
\item[xxx1xx] add date on right;
\item[xx1xxx] select thick characters for text in top of head (i.e. 
with larger line width);
\item[x1xxxx] add the text {\tt EVENT NR x} on top of header;
\item[1xxxxx] add the text {\tt RUN NR x} on top of header.
\end{DLtt}

{\bf Note:} {\tt ISEL=x1xxx1} or {\tt ISEL=1xxxx1} are illegal choices,
i.e. they generate overwritten text.
\item[CHNAME] ({\tt CHARACTER*4}) title string;
\item[CHRSIZ] ({\tt REAL}) character size (cm) of text {\tt CHNAME}.
\end{DLtt}
This routine draws a frame header.

\Shubr{GDCOL}{(ICOL)}

{\bf This routine has been kept for backward compatibility. The appropriate
{\tt HIGZ} routines should be called instead.}

\begin{DLtt}{MMMMMMMM}
\item[ICOL] ({\tt INTEGER}) colour code (1,2,\ldots,8), 
it can be positive, negative, 
or zero:
\begin{DLtt}{MMMM}
\item[$>$0] to set the colour permanently;
\item[$<$0] to set the colour temporarily;
\item[$=$0] to restore the permanent colour value.
\end{DLtt}
\end{DLtt}
This routine sets the colour code.
For example we can set {\tt ICOL=-1} to start with colour  1,
then change to {\tt ICOL=3} to set colour  3, and at the end
restore the original colour by setting {\tt ICOL=0} that
takes colour  1 again.

\Shubr{GDLW}{(LW)}

{\bf This routine has been kept for backward compatibility. The appropriate
{\tt HIGZ} routines should be called instead.}

\begin{DLtt}{MMMMMMMM}
\item[ LW] ({\tt INTEGER}) line width code (1,2,\ldots,5), it can be positive, 
negative, or zero:
\begin{DLtt}{MMMM}
\item[$>$0] to set the line width permanently;
\item[$<$0] to set the line width temporarily;
\item[$=$0] to restore the permanent line width value.
\end{DLtt}
\end{DLtt}
This routine sets the line width.
For example we can set {\tt LW=-1} to start with line width  1,
then change to {\tt LW=3} to set line width 3, and at the end
restore the original line width by setting {\tt LW=0} that
takes line width  1 again.

\Shubr{GDCURS}{(U0*,V0*,ICHAR*)}
\begin{DLtt}{MMMMMMMM}
\item[U0]   ({\tt REAL}) {\tt U} coordinate of the graphics cursor;
\item[V0]   ({\tt REAL}) {\tt V} coordinate of the graphics cursor;
\item[ICHAR] ({\tt INTEGER}) {\tt ASCII} value of the key pressed.
\end{DLtt}
A graphics input is provided by this routine to fetch the 2D user coordinates
of the graphics cursor on the screen.
Interactive commands to zoom, 
measure, or pick tracks or hits make use of this routine.
When the routine is called in the interactive
version of {\tt GEANT},
the present position of the graphics cursor on the screen becomes visible and
it can be moved with the mouse. The user then positions the 
cursor and presses the left button. 
The routine returns then in {\tt U0}, {\tt V0} the user coordinates
of the graphics cursor.
On terminals which do not have the mouse, the graphic cursor is moved
with the arrows, and the point is acquired when any key, except {\it carriage
return}, is pressed. In this case the routine returns in {\tt ICHAR} the 
{\tt ASCII} value of the key pressed. If the user types the carriage return, 
the previous value is retained.

\Shubr{GDFR3D}{(X,NPOINT,U*,V*)}
\begin{DLtt}{MMMMMMMM}
\item[X]      ({\tt REAL}) array {\tt X(3,NPOINT)} of space points;
\item[NPOINT] ({\tt INTEGER}) number of points:
\begin{DLtt}{MMMM}
\item[$>$0] {\tt X} is in the current {\tt DRS};
\item[$<$0] {\tt X} is in {\tt MARS};
\end{DLtt}
\item[U]  ({\tt REAL}) array of {\tt NPOINT U} coordinates;
\item[V]  {(\tt REAL}) array of {\tt NPOINT V} coordinates.
\end{DLtt}
Converts from 3D space coordinates (either in {\tt MA}ster or 
{\tt D}aughter {\tt R}eference {\tt S}ystem) to the corresponding 2D 
user coordinates.
 
This routine maps a space point ({\tt X, Y, Z} in a right-handed reference 
system) onto a plane perpendicular to the observer's line of sight,
defined by the spherical angles $\theta$, $\phi$:
\begin{DLtt}{MMMM}
\item[${\tt \theta}$] the angle between the line of sight and the {\tt Z} 
axis ($0 \leq \theta \leq 180$);
\item[${\tt \phi}$] the angle between the {\tt X} axis and the projection 
onto the {\tt X-Y} plane of the line of sight ($0 \leq \phi \leq 360$).
\end{DLtt}
The plane onto which the point is mapped is actually the {\tt X-Y} plane 
of the {\tt P}rojection {\tt R}eference {\tt S}ystem, and
the observer's line of sight is the {\tt Z} axis of {\tt PRS}. The vertical 
axis on
this plane ({\tt Y} axis on {\tt PRS}) is obtained by intersecting this plane
with the one of {\tt MARS} or {\tt DRS} containing the line of sight and 
the {\tt Y} axis.

\Shubr{GD3D3D}{(XIN,NPOINT,XOUT*)}
\begin{DLtt}{MMMMMMMM}
\item[XIN]   ({\tt REAL}) array {\tt XIN(3,NPOINT)} of input points in 
{\tt MARS};
\item[NPOINT] ({\tt INTEGER}) number of points;
\item[XOUT]   ({\tt REAL}) array {\tt XOUT(3,NPOINT)} of output points 
in {\tt PRS}.
\end{DLtt}
Converts from 3D space coordinates in {\tt MARS}
to 3D coordinates in {\tt P}rojection {\tt R}eference 
{\tt S}ystem (whose {\tt Z} axis is along
the line of sight given by $\theta$ and $\phi$ angles).
