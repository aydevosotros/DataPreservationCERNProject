%%%%%%%%%%%%%%%%%%%%%%%%%%%%%%%%%%%%%%%%%%%%%%%%%%%%%%%%%%%%%%%%%%%
%                                                                 %
%  GEANT manual in LaTeX form                                     %
%                                                                 %
%  Michel Goossens (for translation into LaTeX)                   %
%  Version 1.00                                                   %
%  Last Mod. Jan 24 1991  1300   MG + IB                          %
%                                                                 %
%%%%%%%%%%%%%%%%%%%%%%%%%%%%%%%%%%%%%%%%%%%%%%%%%%%%%%%%%%%%%%%%%%%
\Origin{A.Ferrari, K.Lassila-Perini}
\Documentation{K.Lassila-Perini}
\Submitted{25.10.91} \Revised{17.12.93}
\Version{Geant 3.16}\Routid{PHYS520}
\Makehead{The GEANT/FLUKA Interface}
\section{Subroutines}

\Shubr{FLINIT}{}
\Rind{FLINIT} initialises the {\tt FLUKA} variables and
reads data from a file (flukaaf.dat) which is automatically
opened.
\Rind{FLINIT} is called from \Rind{FLDIST} when a hadron 
enters there first time in the run.


\Shubr{FLDIST}{}
\Rind{FLDIST} computes the distance to the next interaction
point. It calls the {\tt FLUKA} routines to compute the
cross-sections for all particles except neutrons with
kinetic energy below 20 MeV for which {\tt GHEISHA}
routines are called.
\Rind{FLDIST} is called from the user routine \Rind{GUPHAD}
where the hadronic package can be chosen.

\Shubr{FLUFIN}{}
\Rind{FLUFIN} calls the {\tt FLUKA} routines to
generate the hadronic interaction. It passes the
particle to {\tt FLUKA} interaction routines
and puts the eventual secondary particles to
the {\tt GEANT} stack. \Rind{FLUFIN} is called from
the user routine \Rind{GUHADR}.

\section{Method}
{\tt FLUKA}
\cite{bib-FLUK,bib-FLU1,bib-FLU2,bib-FLU3,bib-FLU4,bib-FLU5,bib-FLU6}
is a simulation program
which as a standalone code contains transport and the
physical processes for hadrons and leptons and
tools for geometrical description. 
In {\tt GEANT}, only the hadronic interaction part
is included.

The total cross-section of the hadronic processes
is computed by {\tt FLUKA} routines called from
\Rind{FLDIST} (the cross-section for neutrons below
20 MeV is computed in {\tt GHEISHA}). If hadronic
intercation is chosen in {\tt GEANT} tracking routine,
the particle is passed to \Rind{FLUFIN}.

If particles are stopping (i.e. their energy is below
the cut-off energy), their kinetic energy is deposited.
However, if the particle can decay ($\pi^+$, $K^{\pm}$)
it is forced to decay, or if it is an annihilating
particle ($\pi^-$, $\bar{n}$, $\bar{p}$), it is sent
to {\tt FLUKA} routines for annihilation.
The neutrons with kinetic energy below 20 MeV are
passed to {\tt GHEISHA}.

If the particle is not stopping, a sampling is done
between elastic and inelastic processes. The cross-sections
have been computed in \Rind{FLDIST} in the same time as the total 
cross-section. The particle is sent correspondingly
to the elastic or inelastic interaction routines.
After the interaction, the eventual secondary particles
are written to {\tt GEANT} stack.
The program flow is shown in figures
\ref{fg:phys520-1} and \ref{fldist}.

When the tracking media is a mixture or a
compound material (defined by \Rind{GSMIXT}, see
{\tt [CONS110]}), the atom with which the interaction
is taking place is chosen by sampling on the basis
of the cross-sections. This is important especially
in hydrogenous materials.

\begin{figure}
\normalsize{
\begin{picture}(460,280)(0,100)

\put(180,400){\framebox(60,30)}
\put(210,420){\makebox(0,0){GTHADR/}}
\put(210,410){\makebox(0,0){GTNEUT}}
\put(210,400){\vector(0,-1){20}}
\put(180,360){\framebox(60,20){GUPHAD}}

\put(210,360){\vector(0,-1){40}}
\put(40,220){\dashbox{3}(220,120)[tl]{Interface routines}}
\put(180,300){\framebox(60,20){FLDIST}}
\put(210,300){\line(0,-1){20}}
\put(90,280){\line(1,0){310}}
\put(90,285){First time only}
\put(320,285){n, $E_{kin} <$ 20 MeV}
\put(90,280){\vector(0,-1){20}}
\put(60,240){\framebox(60,20){FLINIT}}

\put(210,280){\vector(0,-1){100}}
\put(290,280){\vector(0,-1){100}}
\put(160,120){\dashbox{3}(180,80)[tl]
{Distance to the interaction (FLUKA)}}
\put(180,140){\framebox(60,40)}
\put(210,170){\makebox(0,0){SIGEL}}
\put(210,160){\makebox(0,0){\small{Elastic}}}
\put(210,150){\makebox(0,0){\small{processes}}}
\put(260,140){\framebox(60,40)}
\put(290,170){\makebox(0,0){NIZLNW}}
\put(290,160){\makebox(0,0){\small{Inelastic}}}
\put(290,150){\makebox(0,0){\small{processes}}}

\put(400,280){\vector(0,-1){100}}
\put(370,160){\framebox(60,20)}
\put(400,170){\makebox(0,0){GPGHEI}}


\put(90,240){\vector(0,-1){60}}
\put(55,140){\dashbox{3}(70,40){}}
\put(90,170){\makebox(0,0){FLUKA}}
\put(90,160){\makebox(0,0){initialisation}}
\put(90,150){\makebox(0,0){routines}}
\end{picture}}
\parbox{\textwidth}{
\begin{minipage} [b]{\textwidth} {
\caption{\label{fldist}Program flow for calculation of the distance to
the next interaction point}}
\end{minipage}}

\end{figure}

\begin{figure}
\normalsize{
\begin{picture}(380,200)(-60,140)

\put(130,360){\framebox(60,30)}
\put(160,380){\makebox(0,0){GTHADR/}}
\put(160,370){\makebox(0,0){GTNEUT}}
\put(160,360){\vector(0,-1){20}}
\put(130,320){\framebox(60,20){GUHADR}}
\put(160,320){\vector(0,-1){20}}
\put(130,280){\framebox(60,20){FLUFIN}}
\put(160,280){\line(0,-1){20}}

\put(20,160){\dashbox{3}(200,80)[tl]{FLUKA interaction routines}}

\put(80,260){\line(1,0){200}}
\put(160,260){\vector(0,-1){40}}
\put(130,180){\framebox(60,40)}
\put(160,210){\makebox(0,0){NUCREL}}
\put(160,200){\makebox(0,0){\small{Elastic}}}
\put(160,190){\makebox(0,0){\small{interactions}}}

\put(80,260){\vector(0,-1){40}}
\put(50,180){\framebox(60,40)}
\put(80,210){\makebox(0,0){EVENTV}}
\put(80,200){\makebox(0,0){\small{Inelastic}}}
\put(80,190){\makebox(0,0){\small{interactions}}}

\put(200,265){n, $E_{kin} <$ 20 MeV}
\put(280,260){\vector(0,-1){40}}
\put(250,200){\framebox(60,20)}
\put(280,210){\makebox(0,0){GHEISH}}
\end{picture}}
\parbox{\textwidth}{
\begin{minipage} [b]{\textwidth} {
\vspace{.5cm}
\caption{\label{fg:phys520-1}Program flow for generating secondary particles}}
\end{minipage}}

\end{figure}
