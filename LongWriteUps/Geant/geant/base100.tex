%%%%%%%%%%%%%%%%%%%%%%%%%%%%%%%%%%%%%%%%%%%%%%%%%%%%%%%%%%%%%%%%%%%
%                                                                 %
%  GEANT manual in LaTeX form                                     %
%                                                                 %
%  Michel Goossens (for translation into LaTeX)                   %
%  Version 1.00                                                   %
%  Last Mod. Jan 24 1991  1300   MG + IB                          %
%                                                                 %
%%%%%%%%%%%%%%%%%%%%%%%%%%%%%%%%%%%%%%%%%%%%%%%%%%%%%%%%%%%%%%%%%%%
\Documentation{R.Brun, S.Ravndal}  
\Submitted{01.10.84}    \Revised{10.03.94}
\Version{Geant 3.21}\Routid{BASE100}
\Makehead{Examples of GEANT application}
This section contains a skeleton of a standard user program {\tt GEXAMP}
to use the
{\tt GEANT} system. More detailed examples can be found in the
standard examples {\tt GEXAM1 - 6}. The recommended user routines
are indicated in bold characters and will be explained more in detail
in the following.

\begin{multicols}{2}
\footnotesize
\begin{XMP}
      PROGRAM GEXAMP
      PARAMETER (NGBANK=50000, NHBOOK=5000)
      COMMON/GCBANK/Q(NGBANK)
      COMMON/PAWC  /H(NHBOOK)
C--> {\sl Initialises {\tt HBOOK} and {\tt GEANT} memory}
      CALL GZEBRA( NGBANK)
      CALL HLIMIT(-NHBOOK)
C--> {\sl Open graphics system}
      CALL HPLINT(0)
      CALL IGMETA(8,0)
C--> {\sl {\tt GEANT} initialisation}
      CALL {\bf UGINIT}
C--> {\sl Start events processing}
      CALL GRUN
C--> {\sl End of Run}
      CALL {\bf UGLAST}
      END
*-----------------------------------------------
      SUBROUTINE {\bf UGINIT}
C--> {\sl Initialise {\tt GEANT}}
      CALL GINIT
C--> {\sl Read data records}
      OPEN(4,FILE='gcards.dat',STATUS='UNKNOWN')
      CALL GFFGO
C--> {\sl Initialise data structure}
      CALL GZINIT
C--> {\sl Initialise graphics}
      CALL GDINIT
      IF(NRGET.GT.0) THEN
C--> {\sl Read data structures from file}
         CALL GRFILE(1,'mygeom.dat','I')
      ELSE
C--> {\sl Particle table initialisation}
         CALL GPART
C--> {\sl Geometry and materials description}
         CALL UGEOM
      ENDIF
C--> {\sl Energy loss and cross-sections tables}
      CALL GPHYSI
      IF(NRSAVE .GT. 0) THEN
C--> {\sl Save permanent data structures} 
         CALL GRFILE(2,'mysave.dat','NO')
      ENDIF
C--> {\sl Print banks}
      CALL GPRINT('MATE',0)
      CALL GPRINT('TMED',0)
      CALL GPRINT('VOLU',0)
C--> {\sl Book histograms} 
      END
*-----------------------------------------------
      SUBROUTINE {\bf UGEOM}
C--> {\sl Defines material, tracking media }
C--> {\sl and geometry.}
C--> {\sl Close geometry banks.}
      CALL GGCLOS
      END
*-----------------------------------------------
      SUBROUTINE {\bf GUKINE}
C--> {\sl Generates kinematics}
C--> {\sl Data card {\tt KINE} itype x y z px py pz}
+SEQ, GCFLAG.
+SEQ, GCKINE.
      CALL GSVERT(PKINE,0,0,0,0,NVERT)
      CALL GSKINE(PKINE(4),IKINE,NVERT,0,0,NT)
C--> {\sl Print kinematic}
      IF (IDEBUG.NE.0) THEN
         CALL GPRINT('VERT',0)
         CALL GPRINT('KINE',0)
      END IF
      END
*-----------------------------------------------
      SUBROUTINE {\bf GUSTEP}
C--> {\sl Called at the end of each tracking }
C--> {\sl step.}
+SEQ, GCKINE.
C--> {\sl Debug event}
      CALL GDEBUG
C--> {\sl Store the created particles} 
      IF (NGKINE.GT.0) CALL GSKING (0)
      END
*-----------------------------------------------
      SUBROUTINE {\bf UGLAST}
C--> {\sl Termination routine} 
C--> {\sl Print histograms and statistics}
      CALL GLAST
C--> {\sl Close {\tt HIGZ/GKS} file}
      CALL IGEND
      END
\end{XMP}
\end{multicols}

\section{Notes}
\begin{itemize}
\item Whenever possible {\tt GEANT} makes use of the {\tt ZEBRA} store
for large data structures. This allows it to adapt the size of the program
data portion
to the size of the problem. The total amount of space required depends
on the application. {\tt GEANT} can run with as little as 50,000 words
or less, but for large detectors it is not uncommon to 
declare stores of several million words. The call to \Rind{GZEBRA}
initialises the common \FCind{/GCBANK/} to receive the {\tt GEANT}
data structures. This call is necessary before any other routine of
the {\tt GEANT} system is called.
\item The call to \Rind{HLIMIT} initialises the {\tt ZEBRA} system
to use the \FCind{/PAWC/} common block for the {\tt HBOOK} histogram
package. The size of the common depends on the number and size of the
plot requested. The {\tt ZEBRA} system must be initialised only once,
and the negative argument to \Rind{HLIMIT} prevents a second
initialisation of the system.
The \Rind{HLIMIT} call has to be placed {\bf after} the call
to \Rind{GZEBRA} and the argument has to be the dimension of the 
\FCind{/PAWC/} common block with a negative sign in front. 
\item
The main program is intended for {\it batch} applications,
while to run the simulation interactively, the interactive main program
called \Rind{GXINT} should be linked in front of the user code.
\item The program shown will require the graphic libraries in the
link sequence. Often, for batch production or for
small tests, graphics is not needed, and not loading the graphics
code makes the program smaller. To avoid loading graphic routines
the calls to \Rind{IGINIT}, \Rind{IGMETA}, \Rind{IGEND}, \Rind{GDINIT}
and \Rind{GDEBUG} should be removed. 

If, on the other hand, the user is interested in including the routine
\Rind{GDEBUG} and in excluding graphics at the same time, then the
following routine should be included in the code:
\begin{verbatim}
      SUBROUTINE GDCXYZ
      ENTRY IGSA
      ENTRY GDTRAK
      END
\end{verbatim}

which will avoid every reference to the graphics routines from \Rind{GDEBUG}.

\item The user code to define the {\it tracking media} and the geometry
of the setup should be inside the routine \Rind{UGEOM}. The pre-initialised
data structured can be read from disk, but it is recommended to call 
\Rind{GPHYSI} in any case, to initialise the cross-section tables. An 
example of a full material, geometry and detector design is given below
and has been extracted from the example {\tt GEXAM3}. Here only major
calls are shown, the redundant parts can be found in the source code
of \Rind{UGEOM} in {\tt GEXAM3}. 

The example shows the basic concept in 
{\tt GEANT}. First material parameters are defining the properities of
a detector material calling the subroutine \Rind{GSMATE}. 
Here in addition to the 16 predefined materials, the material definition of
Calcium is examplary shown. More information towards the predefined materials
and further use of material definition routines can be found in the 
section {\tt CONS001 - CONS101}.
Then tracking parameters are associated to the materials, defining
a so called tracking medium. Each {\tt GEANT}  volume must be 
associated to an existing tracking medium. Here in the example the
tracking medium {\tt 'TARGET'} is defined to exist of Calcium.
 
In the example shown below several
detector volumes are defined using the subroutine \Rind{GSVOLU}.
The defined volume have associated parameters of name, shape,
tracking medium and shape parameters.
In this example the volume {\tt 'TGT '} consists of the 
previously defined tracking medium {\tt 'TARGET'}.The 
volumes (and if necessary identical copies of them)
are then positioned according to the detector geometry. The 
volumes are positioned on the same level, or inside each other. By
setting the parameter {\tt ONLY} or  {\tt MANY} in the call of \Rind{GSPOS} 
the user has the opportunity to tell either {\tt GEANT} the logical
volume structure and to apply boolean operations (cutting, joining and
intersection) between two positioned volumes. More information about
the concept in defining volumes and positioning can be retrieved from
the section {\tt GEOM}. 

Finally the user is required to classify into sets all
sensitive detectors (defined as those volume defined as
detector via \Rind{GSDET} and other related routines, for which 
he wants to store hits in the hit data structure {\tt JHITS}.
\begin{multicols}{2}
\footnotesize
\begin{XMP}
      SUBROUTINE {\bf UGEOM}
+SEQ,GCLIST
+SEQ,GCONSP
      COMMON/DLSFLD/ISWFLD,FLDVAL
C--> {\sl Defining material parameters}
C--> {\sl Defining geometry parameters}
C--> {\sl Defining positioning parameters}
C--> {\sl Data statements, left out here, to}
C--> {\sl Define materials and mixtures}
      CALL GSMATE(17,'CALCIUM$',
     + 40.08,20.,1.55,10.4,23.2,0,0)
C--> {\sl .......}
C--> {\sl further material an mixture definitions}
C--> {\sl .......}
C--> {\sl Defining tracking media}
      CALL GSTMED( 2,'TARGET    $',
     + 17,0,0,0.,10.,.2,.1,.001,.5,0,0)
C--> {\sl .......}
C--> {\sl defining further media}
C--> {\sl .......}
C--> {\sl Define the reference frame}
      CALL GSVOLU
     +     ('CAVE','BOX ',1,CAVPAR,3,ICAVE)
C--> {\sl The targe box is shifted by 100 cm}
C--> {\sl in the cave.}
      CALL GSVOLU
     +     ('TGT ','BOX ',2,TGTPAR,3,ITGT )
      CALL GSVOLU
     +     ('TBIN','TRD1',3,TBIPAR,4,ITBIN)
      CALL GSVOLU
     +     ('TBOU','TRD1',4,TBOPAR,4,ITBOU)
      CALL GSVOLU
     +     ('ARM ','TRD1',1,ARMPAR,4,IARM)
      CALL GSVOLU
     +     ('FDIN','BOX ',9,FDIPAR,3,IFDIN)
      CALL GSVOLU
     +     ('FDOU','BOX ',4,FDOPAR,3,IFDOU)
C--> {\sl Define drift wire planes}
      CALL GSVOLU
     +     ('FSP ','BOX ',13,FDIPAR,3,IFSP)
C--> {\sl .......}
C--> {\sl further geometry definitions}
C--> {\sl .......}
C--> {\sl Positioning the daughter volumes in}
C--> {\sl their mother volume.}
      CALL GSPOS
     + ('TGT ',1,'TBIN', 0., 0.,-5.08,0,'ONLY')
      CALL GSPOS
     + ('TGT ',2,'TBIN', 0., 0.,-2.54,0,'ONLY')
      CALL GSPOS
     + ('TGT ',3,'TBIN', 0., 0., 0.  ,0,'ONLY')
      CALL GSPOS
     + ('TGT ',4,'TBIN', 0., 0., 2.54,0,'ONLY')
      CALL GSPOS
     + ('TGT ',5,'TBIN', 0., 0., 5.08,0,'ONLY')
      CALL GSPOS
     + ('TBIN',1,'TBOU', 0., 0.,   0.,0,'ONLY')
      CALL GSPOS
     + ('TBOU',1,'CAVE', 0., 0.,  ZTG,0,'ONLY')
      CALL GSPOS
     + ('ARM ',1,'CAVE',XLARM,0.,ZLARM,1,'ONLY')
      CALL GSPOS
     + ('ARM ',2,'CAVE',XRARM,0.,ZRARM,2,'ONLY')
      CALL GSPOS
     + ('FDOU',1,'ARM ',0.,0., DFDO  ,0,'ONLY')
      CALL GSPOS
     + ('FDIN',1,'FDOU',0.,0., 0.    ,0,'ONLY')
      CALL GSPOS
     + ('FSP ',1,'FDIN',0.,0.,-2.9975,0,'ONLY')
C--> {\sl .......}
C--> {\sl positioning of further volumes}
C--> {\sl .......}
C--> {\sl Print the stored definitions}
      CALL GLOOK('VOLU',LPRIN,NPRIN,ILOOK)
      IF(ILOOK.NE.0) CALL GPVOLU(0)
      CALL GLOOK('ROTM',LPRIN,NPRIN,ILOOK)
      IF(ILOOK.NE.0) CALL GPROTM(0)
      CALL GLOOK('TMED',LPRIN,NPRIN,ILOOK)
      IF(ILOOK.NE.0) CALL GPTMED(0)
      CALL GLOOK('MATE',LPRIN,NPRIN,ILOOK)
      IF(ILOOK.NE.0) CALL GPMATE(0)
      CALL GLOOK('PART',LPRIN,NPRIN,ILOOK)
      IF(ILOOK.NE.0) CALL GPPART(0)
C--> {\sl Clean up volume banks and perform}
C--> {\sl optimization}
      CALL GGCLOS
C--> {\sl Define sensitive detector parts}
      CALL GSDET
     &('DRFT','FSP ',2,NAFD ,NBITSV,1,100,
     &100,IDRFT,IFD )
C--> {\sl Define hit parameters}
      CALL GSDETH('DRFT','FSP ',9,NAMESH,
     &NBITSH,ORIG,FACT)
      RETURN
      END
\end{XMP}
\end{multicols}





\item It is convenient to store the input data records (see {\tt [BASE040]})
in an auxiliary file ({\tt gcards.dat} in the example). This allows to
have a standard input file and to overwrite selected input data records
as needed.
If, for instance, the standard {\tt gcards.dat} file contains the record
{\tt TRIG 1000} and a short test run is requested this can be obtained
with the following input:
\begin{verbatim}
READ 4
TRIG 10
STOP
\end{verbatim}
the first line instructs {\tt FFREAD} to open and process the file
connected with logical unit 4, and the second line (re-)defines the
number of events to be processed. The {\tt STOP} command ends the 
{\tt FFREAD} processing of the input.
\item 
In the above
example the common blocks have not been expanded in the code. The notation
used is the one of the {\tt PATCHY}/{\tt CMZ}~\cite{bib-PATCHY,bib-CMZ}
code management systems. These products, among other things, can
run as pre-processors, replacing the {\tt +SEQ,...}
instructions with the corresponding code fragments. Users are strongly
recommended to use these systems to include {\tt GEANT} common blocks
in their code. 

Long experience in supporting {\tt GEANT} users has shown that, as the
user program grows, typing errors in the insertion of the common blocks
{\it by hand} become very common, but difficult to find. The investment
needed to learn a code management system at the user level is usually 
negligible compared with the time and energy needed in hunting a
problem introduced by a mistyped common.
\end{itemize}
