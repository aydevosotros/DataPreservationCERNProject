\documentstyle{book}
\oddsidemargin 0.54cm
\evensidemargin 0.0cm
\topmargin -50pt
\textheight 22.5cm \textwidth 15.cm

\begin{document}
\pagestyle{plain}
\rm
\Large
{\bf The Motif Interface}
\\[2em]
\large
\rm
The interactive version 3.21 contains an object oriented Motif-based user
interface. It can be accessed specifying `m' as workstation type.
The full functionality of the X11 version remains available, while new 
Motif-specific features have been added. \\[1em]
The main ideas \\[1em]
The GEANT data structures are considered as KUIP browsable classes and their
contents as `objects' on which one can perform actions (the GEANT commands).
According to the Motif conventions, an object can be selected clicking the
left button of the mouse (when the cursor is on the icon representing that
object). Clicking then on the right button of the mouse, a menu of possible
commands will appear (double clicking on the left button the first action of
this list will be executed); 
the selected command will perform the relative actions
on the selected object. Such actions (like drawing, for example) can be
executed either directly or via the use of an automatically opened Motif panel.
Objects drawn in the graphics window can be `picked' as well (for example,
volumes, tracks, hits); clicking the right button when the cursor is on the
selected object, a menu of possible actions on that object is displayed.
Users can finally define Motif panels containing buttons corresponding to the
most frequently used commands. An on-line help is available for any specific
subject. \\[1em]
The Geant++ Executive Window \\[1em]
It replaces the normal dialog window; it contains a Transcript Pad, where the
text output of the executed commands is displayed, and an Input Pad, where the
user can still type the desired commands in the old style. \\[1em]
The Geant++ Main File Browser \\[1em]
On the left side it displays a list of the GEANT data structures, of the 
available commands, file, macros and Zebra divisions used. Selecting one of
them, the full list of icons representing the objects of that class is shown
in the main area of the browser. Proceeding as described before, it is 
possible to perform actions on the classes (like create a new object) or on
the objects belonging to them. It is possible to create menus of commands
just clicking on the string `commands' at the top line of the browser. \\[1em]
The Geant++ Graphics Window \\[1em]
Any object to be drawn in the graphics window can be stored in the current
picture file (automatically opened after each NEXT command) via a call to 
IGPID (see Higz manual). It can be afterwards `picked' as described before.
In the case of commands executed via the use of Motif panels, some input values
can be set with a slider ranging in the specifed range for the relative
variable; moving the slider (after having clicked on the right-hand `activating
box') the relative action is performed in the graphics
window when releasing the button of the mouse; when in `drag mode', the 
action is performed {\it while} moving the slider (keeping the left button
pressed): especially when double buffering has been selected, this can be
useful for real time manipulations.\\[1em]
An Example \\[1em]
Start your GEANT321 executable module (linked with GXINT321 and Motif1.2);
\\[.5em] type `m' as workstation type;
\\[.5em] click the left button of the mouse after positioning the cursor on the 
string VOLU in the browser;
\\[.5em] click the left button of the mouse after positioning the cursor on 
any icon in the main area of the browser;
\\[.5em] click now the right button of the mouse and keep it pressed;
\\[.5em] move the mouse to select the action `Tree' and release the button;
\\[.5em] the drawing of the logical tree will be displayed in the graphics
window;
\\[.5em] position the cursor on the drawing of a box (containing a volume name)
in the graphics window, click the right button and keep it pressed;
\\[.5em] release the button selecting the action `Dspec';
\\[.5em] the command DSPEC for that volume will be executed in a separate
window;
\\[.5em] repeat the exercise selecting this time the action `Dspe3d';
\\[.5em] the DSPEC will be executed in the first window, the volume 
specifications will be printed in a separate window and a Motif panel will
appear;
\\[.5em] click the left button of the mouse positioning the cursor in the Motif
panel on the `Value changed' button, and select the DRAG option;
\\[.5em] click now the left button on the `activating box' on the right of
the `Theta' slider;
\\[.5em] click on the `Theta' slider and, keeping pressed the left button of the
mouse, move it right-wards;
\\[.5em] the drawing in the graphics window will rotate;
\\[.5em] release the button and type `igset 2buf 1' in the executive window;
\\[.5em] restart moving the slider as before. 














   


   






\end{document}


