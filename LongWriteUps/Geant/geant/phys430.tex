%%%%%%%%%%%%%%%%%%%%%%%%%%%%%%%%%%%%%%%%%%%%%%%%%%%%%%%%%%%%%%%%%%%
%                                                                 %
%  GEANT manual in LaTeX form                              %
%                                                                 %
%  Michel Goossens (for translation into LaTeX)                   %
%  Version 1.00                                                   %
%  Last Mod. Jan 24 1991  1300   MG + IB                          %
%                                                                 %
%%%%%%%%%%%%%%%%%%%%%%%%%%%%%%%%%%%%%%%%%%%%%%%%%%%%%%%%%%%%%%%%%%%
\Origin{G.N.Patrick, L.Urb\'{a}n, D.Ward}
\Version{Geant 3.16}\Routid{PHYS430}
\Submitted{12.03.82}  \Revised{16.12.93}
\Makehead{Ionisation processes for muons and protons}

\section{Subroutines}

\Shubr{GDRELA}{}

\Rind{GDRELA} initialises the ionisation energy-loss tables for different
materials and particles.
The energy binning is in the array
{\tt ELOW} (\FCind{/GCMULO/}) initialised in the routine \Rind{GPHYSI}.
The tables are filled with the quantity $dE/dx$ in GeV cm$^{-1}$,
for elements, mixtures and compounds. A
temporary table is also filled, containing 
additional points for integration, to obtain the range-energy
table (see {\tt [PHYS010]}).

\Rind{GDRELP} computes the ionisation energy loss for protons,
and the values are stored in the energy loss table.
This value is used for other charged hadrons with scaled energy bin.

\Rind{GDRELM} computes the ionisation energy loss for muons,
and sums it in the muon table
with the energy loss due to the other processes
(bremsstrahlung, pair production and nuclear interactions). 

The following pointers are used:

\begin{DLtt}{MMMMMMMMMMMMMMMMMM}
\item[JMA = LQ(JMATE-I)]    I$^{th}$ material bank;
\item[JEL1 = LQ(JMA-1)]     $dE/dx$ for \Pep/\Pem (see {\tt [PHYS330]});
\item[JEL2 = LQ(JMA-2)]     $dE/dx$ for $\mu^+/\mu^-$;
\item[JEL3 = LQ(JMA-3)]     $dE/dx$ for protons (used also 
for other charged particles).
\end{DLtt}

\Rind{GDRELA} is called during initialisation by \Rind{GPHYSI}.

\Shubr{GDRELP}{(A,Z,DENS,T,DEDX*)}

\Rind{GDRELP} computes the ionisation energy loss for 
protons. It is called by the routine \Rind{GDRELA}. See \Rind{GDRELX}
below for the meaning of the parameters.

\Shubr{GDRELM}{(A,Z,DENS,T,DEDX*)}

\Rind{GDRELM} computes the energy loss due to ionisation for 
muons. It is called by the routine \Rind{GDRELA}. See \Rind{GDRELX}
below for the meaning of the parameters.

\Shubr{GDRELX}{(A,Z,DENS,T,HMASS,DEDX*)}

\begin{DLtt}{MMMMMMMM}
\item[A] ({\tt REAL}) mass number of the material;
\item[Z] ({\tt REAL}) atomic number of the material;
\item[DENS] ({\tt REAL}) density of the material in g cm$^{-3}$; it is only used 
to calculate the density correction in case of high values of {\tt T};
\item[T] ({\tt REAL}) kinetic energy of the particle in GeV;
\item[HMASS] ({\tt REAL}) mass of the particle;
\item[DEDX] ({\tt REAL}) energy loss of the particle in GeV cm$^{2}$ g$^{-1}$.
\end{DLtt}

This routine calculates the ionisation energy loss for a muon or a
proton. It is called by \Rind{GDRELM} and \Rind{GDRELP}.


\Shubr{GDRSGA}{}

\Rind{CDRSGA} calculates the total cross-section for the production
of $\delta$-rays by electrons (M\"{o}ller scattering), positrons
(Bhabha scattering) and muons.
For hadrons, this value is calculated
at tracking time. The mean free path as a function of the energy
is stored for every medium. The following pointers are used: 

\begin{DLtt}{MMMMMMMMMMMMMMMMMM}
\item[JMA = LQ(JMATE-I)]  pointer to the I$^{th}$ material;
\item[JDRAY = LQ(JMA-11)]  $\delta$-ray cross-section for electrons;
\item[JDRAY+NEK1]  $\delta$-ray cross-section for positrons;
\item[JDRAY+2*NEK1]  $\delta$-ray cross-section for muons.
\end{DLtt}

The routine is called during initialisation by {\tt GPHYSI}.

\section{Method}

In {\tt GEANT}, $\delta$-rays are generated only above a threshold energy
$T_{cut}$ which corresponds to the variable {\tt DCUTE} for electrons and 
positrons, {\tt DCUTM} for all other charged particles. The total cross-section 
for the production of a $\delta$-ray electron of kinetic
energy $T > T_{cut}=${\tt DCUTE, DCUTM} by a particle of kinetic energy $E$
is:
\begin{equation}
\label{eq:phys430-3}
\sigma (E,T_{cut}) = \int_{T_{cut}}^{T_{max}} \frac{d \sigma (E,T)}{dT} dT
\end{equation}
where $T_{max}$ is the maximum energy transferable to the free electron:
\begin{equation}
T_{max} =\frac{2m(\gamma^2 -1)}
      {1+2\gamma\frac{m}{M}+
      \left(\frac{m}{M} \right)^2 }
\end{equation}

The energy lost in 
ionising collisions producing $\delta$-rays below ${T_{cut}}$
are included in the continuous energy loss. The mean value of the
energy lost due to these {\it soft} collisions is:
\begin{equation}
\label{eq:phys430-1}
E_{loss}(E,T_{cut}) = \int_{0}^{T_{cut}} 
T \frac{d \sigma (E,T)} {dT} dT
\end{equation}
where $m$ is the electron mass and $M$ is the mass of the incident particle.

In this chapter, the method of calculation of the continuous energy loss
and the total cross-section are
explained. The generation of $\delta$-rays is explained in 
chapter {\tt [PHYS331]}.

\subsection{Continuous energy loss}
The integration of (\ref{eq:phys430-1}) leads to the Bethe-Block
stopping power or to the restricted energy 
loss formula \cite{bib-PDGB}:
\begin{equation}
\label{eq:phys430-2}
\frac{1} {\rho} \left(\frac{dE}{dx} \right) =
\left \{
\begin{array}{LcL}
D \frac{Z Z_{inc}^{2}}{A \beta^{2}} \left [
\ln \left ( 
\frac{2 m_{e} \beta^{2} \gamma^{2} T_{max} }{I^{2}} 
\right ) - 2 \beta^{2} - 
\delta -
\frac{ 2 C_{e}}{Z} \right ] 
\hfill & \mbox{if} & T_{cut} \geq T_{max} \\ [0.5cm]
D \frac{Z Z_{inc}^{2}}{A \beta^{2}} \left [
\ln \left ( 
\frac{ 2 m_{e} \beta^{2} \gamma^{2} T_{c} }{I^{2}} 
\right ) - 
\beta^{2} \left ( 1 + \frac{T_{c}}{T_{max}} \right )
 - \delta
-\frac{2 C_{e}}{Z} \right ] \hfill & \mbox{if} & T_{cut} < T_{max}
\end{array} 
\right .
\end{equation}

where,
\[
\begin{array}{LL@{\hspace{2cm}}LL}
D & 2\pi N_A r_e^2 m c^2 = 0.000153537 \: \mbox{GeV cm$^2$ g$^{-1}$}
& Z & \mbox{atomic number of medium}  \\
A   & \mbox{mass number} & \rho & \mbox{density of medium} \\
T_c &  \min(T_{cut},T_{max}) & m_e & \mbox{electron mass}
\end{array}
\]

and $I$ is the average ionisation potential of the atom in question.
There exists a variety of phenomenological approximations for this.
In former versions of {\tt GEANT} the formula quoted by
\cite{bib-BRIC} was used ($I=16\:Z^{0.9}$ eV). At present the
values recommended by~\cite{bib-ANZI} are used. The ionisation
potential $I$ only enters into the logarithmic term of the energy
loss formula, but it has been verified that the new parameterisation
gives better accuracy especially in the case of high $Z$.
It should be noted that this is not the value of $I$ which
is stored in the material data structure by \Rind{GPROBI}, 
which is still calculated as $I=16\:Z^{0.9}$ eV.

{\bf Note:} the ionisation potential $I$ must not be changed blindly,
hoping that the most up-to-date values give the better results. The
value of $I$ is closely connected to the shell correction term (see
later). 

$\delta$ is a correction term which takes account of the reduction
in energy loss due to the so-called density effect. This becomes
important at high energy because media have a tendency to become
polarised as the incident particle velocity increases. As a consequence,
the atoms in a medium can no longer be considered as isolated. To correct
for this effect the formulation of Sternheimer~\cite{bib-STE1,bib-STE2} 
is used:
 
\begin{equation}
\delta = \left \{ \begin{array}{L@{\hspace{1cm}}lL} 
            0                   &    \mbox{if} & X  < X_0      \\
            4.606X + C + a(X_1-X)^m & \mbox{if} & X_0 \leq X < X_1 \\
            4.606X + C  & \mbox{if}  & X  \geq X_1
            \end{array} \right .
\end{equation}
 
where the medium-dependent constants are calculated as follows:
\[
\begin{array}{LcL@{\hspace{1.5cm}}LcL}
X & = & \log_{10}(\gamma \beta) = \ln(\gamma^{2} \beta^{2})/4.606 & 
\nu_{p} & = & \sqrt{ \frac{N_{el} e^2 }{\pi m} }
\mbox{\hspace{0.25cm}s$^{-1}$ plasma frequency} \\
N_{el} & = & \frac{\rho Z N_{Av}}{A} \mbox{\hspace{0.25cm}electrons cm$^{-3}$}
& C & = & -2 \ln \left ( \frac{I}{h\nu_{p}} \right ) - 1 \\
a & = & \frac{4.606(X_a - X_0)}{(X_1 - X_0)^m} & 4.606 \: X_{a} & = & -C
\end{array}
\]

For condensed media we have:
 
\[
\begin{array}{LL}
I < 100 \: \mbox{eV} & \left \{
\begin{array}{LcL}
X_{0} & = & \left \{
\begin{array}{LlL}
0.2 & \mbox{if} & -C \leq 3.681 \\
-0.326C-1.0 & \mbox{if} & -C > 3.681
\end{array} \right . \\
X_{1} & = & 2.0 \\
m & = & 3.0
\end{array} \right . \\ [1.0cm]
I \geq 100 \: \mbox{eV} & \left \{
\begin{array}{LcL}
X_{0} & = & \left \{
\begin{array}{LlL}
0.2 & \mbox{if} & -C \leq 5.215 \\
-0.326C-1.5 & \mbox{if} & -C > 5.215
\end{array} \right . \\
X_{1} & = & 3.0 \\
m & = & 3.0
\end{array} \right .
\end{array}
\]

and in the case of gaseous media $m=3$ and:

\[
\begin{array}{LLcr@{}c@{}l}
X_0 = 1.6 & X_1 = 4 & \mbox{for} &       & C & \leq 9.5 \\
X_0 = 1.7 & X_1 = 4 & \mbox{for} & 9.5 < & C & \leq 10 \\
X_0 = 1.8 & X_1 = 4 & \mbox{for} & 10  < & C & \leq 10.5 \\
X_0 = 1.9 & X_1 = 4 & \mbox{for} & 10.5 < & C & \leq 11 \\
X_0 = 2.  & X_1 = 4 & \mbox{for} & 11  < & C & \leq 12.25 \\
X_0 = 2.  & X_1 = 5 & \mbox{for} & 12.25 < & C & \leq 13.804 \\
X_0 = 0.326C-2.5 & X_1 = 5 & \mbox{for} & 13.804 < & C &  \\
\end{array}
\]

$C_e/Z$ is a so-called {\it shell correction term}
which accounts for the fact that, at
low energies for light elements and at all energies for heavy ones, the
probability of collision with the electrons of the inner atomic shells
(K, L, etc.) is negligible. The semi-empirical formula used in
{\tt GEANT}, applicable to all materials, is due to Barkas \cite{bib-BARK}:
\begin{eqnarray*}
 C_e(I,\eta) & = & (0.42237\eta^{-2}+0.0304\eta^{-4}-0.00038\eta^{-6})
10^{-6} I^2 \\
& +& (3.858\eta^{-2}-0.1668\eta^{-4}+0.00158\eta^{-6})10^{-9}I^3 \\
\eta & = &  \gamma \beta
\end{eqnarray*}
$C_e$ is a dimensionless constant, but as $I$ in the original
article was expressed in eV and in {\tt GEANT} it is expressed in GeV,
the exponent of ten in the $I^2$-term is $10^{-6+2 \times 9} = 10^{12}$, and
that of the $I^3$-term is $10^{-9+3 \times 9} = 10^{18}$.)
This formula breaks down at low energies, and it only applies for values 
of $\eta > 0.13$ (i.e. $T > 7.9$ MeV for a proton). For $\eta \leq
0.13$ the shell correction term is calculated as:

\[
\left . C_{e}(I,\eta) \rule{0mm}{5mm} \right |_{\eta \leq
0.13} = C_{e}(I,\eta=0.13)\frac{\ln \left ( \frac{T}{T_{2l}} \right )}
{\ln \left ( \frac{7.9 \: 10^{-3} \: \rm GeV}{T_{2l}} \right )}
\]
 
i.e. the correction is {\it switched off} logarithmically from $T=7.9$
MeV to $T=T_{2l}=2$ MeV.
\Rind{GDRELX} has been tested for protons against energy loss tables 
\cite{bib-SERR,bib-SER1}
for various materials in the energy range 50 MeV-25 GeV. 
Typical discrepancies are as follows:
 
\begin{flushleft}
\begin{tabular}{lll}
Beryllium: &   1.1\%  at 0.05 GeV      &  0.02\%  at  25 GeV\\
Hydrogen : &   1.5\%  at 0.05 GeV      &  12.1\%  at  25 GeV\\
Water   :  &   8.1\%  at 0.05 GeV      &  4.4\%  at  6  GeV\\
\end{tabular}
\end{flushleft}

The mean energy loss can be described by the Bethe-Bloch formula 
(\ref{eq:phys430-2}) only if the projectile velocity is larger than that
of orbital electrons. In the low-energy region where this is not 
verified, a different kind of parameterisation has been chosen \cite{bib-ANZI}:
\[
\frac{1}{\rho} \frac{dE}{dx} =  \left \{
\begin{array}{rLcL}
I: & C_{1} \tau^{\frac{1}{2}} & \mbox{for} & 0 \leq T \leq T_{1l} = 10 \:
\mbox{keV} \\
II: & \frac{S_{L} \times S_{H}}{S_{L} - S_{H}} & \mbox{for} & T_{1l}
< T \leq T_{2l} = 2 \: \mbox{MeV} \\
III: & \left \{ \mbox{Bethe-Bloch} \right \} \left ( 1+\frac{\nu}{T} \right )
& \mbox{for} & T_{2l} < T
\end{array} \right .
\]

where

\[
\begin{array}{LcL@{\hspace{1.5cm}}LcL}
S_{L} & = & C_{2} \tau^{0.45} & S_{H} & = & \frac{C_{3}}{\tau}
\ln \left [ 1+\frac{C_{4}}{\tau}+C_{5} \tau \right ] \\
\tau & = & \frac{T}{M_{p}} & M_{p} & = & \mbox{proton mass}
\end{array}
\]

\begin{table}
\begin{centering}
\begin{tabular}{l|c|c|c}
\multicolumn{4}{c}{$\displaystyle 
\frac{dE}{dx}^{calculated} - \frac{dE}{dx}^{measured}$}\\ [0.3cm]
\hline 
experiment & A~\cite{bib-REYN} & B~\cite{bib-GREE} & C~\cite{bib-SAKA} \\
\hline
projectile & p & p & p \\
T (MeV) & 0.03-0.6 & 0.4-1 & 55,65,73 \\ [0.2cm]
material 
& \parbox{3cm}{H$_2$, He, N$_2$, O$_2$, Ne, Ar, Xe}
& \parbox{3cm}{Cu, Ge, Sn, Pb}
& \parbox{3cm}{Al, Ti, Cu, Sn, Pb} \\
exp err (\%) & $\sim$ 3 & $\sim$ 2.5 & $\sim$ 0.7 \\
\hline\hline
tot N$^{\circ}$ of pts & 121 & 52 & 15 \\
\hline 
N$^{\circ}$ of pts \hfill & & & \\
\hfill with $|\Delta| < 1 \sigma$ & 94 & 50 & 8  \\
\hfill      $|\Delta| < 2 \sigma$ & 114 & 52 & 13  \\
\hfill      $|\Delta| < 3 \sigma$ & 119 & 52 & 14  \\
\hfill      $|\Delta| < 4 \sigma$ & 121 & 52 & 15  \\
\hline
\end{tabular}
\caption{Test of {\tt GDRELP} with low energy protons.}
\end{centering}
\label{tb:phys430-1}
\end{table}

The formula used in the region $T > T_{2l}$ ensures the continuity of the
energy loss function at $T = T_{2l}$ when the Bethe-Bloch formula and
the parameterisation meet. The parameter $\nu$ is chosen in such a
way that:

\[
\frac{1}{\rho} \left . \frac{dE}{dx}^{II} \right |_{T=T_{2l}} =
\frac{1}{\rho} \left . \frac{dE}{dx}^{III} \right |_{T=T_{2l}}
\]
 
The routine {\tt GDRELP} calculates the stopping power or restricted
energy loss only for $T > T_{2l}$; below this kinetic energy it gives
the stopping power (i.e. total energy loss) irrespectively of the 
value of $T_{cut}$. This approximation does not introduce a serious
source of error since, in the case of a proton, at $T=T_{2l}$ the 
maximum energy transferable to the atomic electron is $T_{max}
\approx 4$ keV, and the restricted loss should be calculated only if
$T_{cut} < T_{max}$.

{\tt GDRELP} has been tested against experimental data and energy loss
tables. Some of the test results are summarised in tables \ref{tb:phys430-1},
\ref{tb:phys430-2} and \ref{tb:phys430-3}.

\begin{table}
\begin{centering}
\begin{tabular}{rl|c|c}
\multicolumn{4}{c}{1 keV $ \leq T \leq $ 100 MeV,
31 pts per element~\cite{bib-ANZI}}\\
\hline
Z & element & \parbox{2.2cm}{mean (r.m.s.) deviation in \%} &
\parbox{2.2cm}{max (r.m.s.) deviation in \%} \\ [0.2cm]
\hline
1 & H$_2$ & 0.4 & 1.1 \\
6 & O & 0.5 & 1.6 \\
13 & Al & 0.6 & 2.1 \\
29 & Cu & 0.7 & 2.0 \\
82 & Pb & 0.7 & 2.3 \\
\hline
\end{tabular}
\caption{Test of {\tt GDRELP} against stopping power tables.
Stated accuracy of the tables is $\sim 5\%$ for $T > 0.5$ MeV and
$\sim 10\%$ for $T<0.5$ MeV}
\end{centering}
\label{tb:phys430-2}
\end{table}
The energy lost due to the soft $\delta$-rays is tabulated during
initialisation as a function of the medium and of the energy by routine
\Rind{GDRELA}

\begin{table}
\begin{centering}
\begin{tabular}{|l|r||l|r||l|r|}
\multicolumn{6}{c}{$\displaystyle
\Delta = \frac{\frac{dE}{dx}^{min}_{calc} - \frac{dE}{dx}^{min}_{table}}
{\frac{dE}{dx}^{min}_{table}} (\%)$} \\[0.5cm]
\multicolumn{6}{c}{Data from~\cite{bib-BAR1}} \\
\hline
material & $\Delta$ (\%) & material & $\Delta$ (\%) 
& material & $\Delta$ (\%)\\
\hline
H$_{2}$ & -0.4 & Na & -0.4 & Sn & 1.7 \\
D$_{2}$ & -0.3 & Al & -0.6 & Xe & 1.0 \\
He & 0.0 & Si & 0.5 & W & 1.1 \\
Li & 3.1 & Ar & 0.4 & Pt & 0.3 \\
Be & -0.2 & Ti & -1.4 & Pb & 1.5 \\
C & -0.1 & Fe & -0.4  & U & 1.4 \\
N$_{2}$ & -0.2 & Cu & -0.9 & & \\
O$_{2}$ & -1.2 & Ge & -0.8  & & \\
\hline
\end{tabular}
\caption{The minimum stopping power
calculated from the formula used is compared
with values from the tables.}
\end{centering}
\label{tb:phys430-3}
\end{table}

The tables are filled with the quantity $dE/dx$ in GeV cm$^{-1}$
(formula (\ref{eq:phys430-2})
above). For a molecule or a mixture the following formula
is used:
\begin{equation}
\frac{dE} {dx} = \rho \sum_{i} p_i
      \left( \frac{dE}{dX} \right)_i
\end{equation}
where $x$ is in cm, $X$ in g cm$^{-2}$ and $p_{i}$ is the proportion
by weight of the I$^{th}$ element.

The energy loss of all charged particles other than electrons,
positrons and muons is obtained from that of protons 
by calculating the kinetic energy of a proton with the same
$\beta$, and using this value to interpolate the tables:
\begin{equation}
 T_{proton} =\frac{M_{\rm p}}{M}T
\end{equation}

\subsection{Total cross-section}
The integration of formula (\ref{eq:phys430-3}) gives the total cross-section :
\begin{equation}
\label{eq:phys430-4}
\sigma (Z,E,T_{cut}) = \left \{
\begin{array}{LL}
2 \pi r^2_0 m  \frac{Z}{\beta^2}
   \frac{1-Y+\beta^2 Y\ln Y}{T_{cut}} & \mbox{for spin-0 particles} \\ [.5cm]
2 \pi r^2_0 m  \frac{Z}{\beta^2} \left (\frac{1-Y+\beta^2 Y\ln Y}{T_{cut}}
+ \frac{T_{max}-T_{cut}} {2E^2} \right) & 
\mbox{for spin-}\frac{1}{2}\mbox{~particles}
\end{array} \right .
\end{equation}

The mean free path is tabulated during initialisation 
as a function of the medium and of the energy by the routine
\Rind{GDRSGA}
{\it for leptons only}.
 
The cross-section (\ref{eq:phys430-4}) is strongly dependent on the mass of 
the incident
particle, and cannot be tabulated in a general way for any charged hadrons.
Therefore, for such particles, the cross-section is computed
at tracking time in the routine \Rind{GTHADR}.
