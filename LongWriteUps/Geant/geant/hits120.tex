%%%%%%%%%%%%%%%%%%%%%%%%%%%%%%%%%%%%%%%%%%%%%%%%%%%%%%%%%%%%%%%%%%%
%                                                                 %
%  GEANT manual in LaTeX form                              %
%                                                                 %
%  Michel Goossens (for translation into LaTeX)                   %
%  Version 1.00                                                   %
%  Last Mod. Jan 24 1991  1300   MG + IB                          %
%                                                                 %
%%%%%%%%%%%%%%%%%%%%%%%%%%%%%%%%%%%%%%%%%%%%%%%%%%%%%%%%%%%%%%%%%%%
\Origin{R.Brun}
\Submitted{01.11.83}             \Revised{17.12.93}
\Version{Geant 3.16}             \Routid{HITS120}

\Makehead{DETector Digitisation parameters}

\Shubr{GSDETD}{(CHSET,CHDET,ND,CHNMSD,NBITSD)}
\begin{DLtt}{MMMMMMMM}
\item[CHSET] ({\tt CHARACTER*4}) set name;
\item[CHDET] ({\tt CHARACTER*4}) detector name;
\item[ND] ({\tt INTEGER}) number of elements per digitisation;
\item[CHNMSD] ({\tt CHARACTER*4}) array of {\tt ND} names for the digitisation 
elements;
\item[NBITSD] ({\tt INTEGER}) array of {\tt ND}, {\tt NBITSD(I)} 
({\tt I=1,...,ND})
is the number of bits into which to pack the {\tt I$^{th}$} element of the
digit.
\end{DLtt}
Defines digitisation parameters for detector {\tt CHDET} of set {\tt CHSET}.
The routine must be called at initialisation, after the
geometrical volumes have been defined, to describe
the digitisation elements and the way to pack them in the data structure
{\tt JDIGI}.

\section*{Example}
Following on from the example in {\tt [HITS110]}, we want to add the 
digitisation information to detector {\tt EPHI} of set {\tt ECAL}. The
information which should be stored for each digitisation is 
the ADC pulse height in the lead glass block, so that a digitisation
in this scheme could look like:

\begin{center}
\begin{tabular}{l@{\hspace{2cm}}l}
Element & Value \\ \hline
\tt EPHI  &  12  \\
\tt EZRI  &  41  \\
\tt BLOC  &   3  \\
\tt ADC  &   789 \\
\end{tabular}
\end{center}

The code to define the digitisation information could be:
\begin{verbatim}
     CHARACTER*4 CHNMSD
     DATA CHNMSD/'ADC '/ 
     DATA NBITSD/16/    
 
     CALL GSDETD('ECAL','BLOC',1,CHNMSD,NBITSD)
\end{verbatim}

\Shubr{GFDETD}{(CHSET,CHDET,ND*,CHNMSD*,NBITSD*)}
Returns the digitisation parameters for detector {\tt CHDET} of set {\tt
CHSET}.  All arguments as explained in \Rind{GSDETD}.
 
