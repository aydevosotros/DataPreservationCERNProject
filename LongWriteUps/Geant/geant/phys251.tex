%%%%%%%%%%%%%%%%%%%%%%%%%%%%%%%%%%%%%%%%%%%%%%%%%%%%%%%%%%%%%%%%%%%
%                                                                 %
%  GEANT manual in LaTeX form                              %
%                                                                 %
%  Michel Goossens (for translation into LaTeX)                   %
%  Version 1.00                                                   %
%  Last Mod. Jan 24 1991  1300   MG + IB                          %
%                                                                 %
%%%%%%%%%%%%%%%%%%%%%%%%%%%%%%%%%%%%%%%%%%%%%%%%%%%%%%%%%%%%%%%%%%%
\Origin{G.Tromba, P.Bregant}
\Submitted{10.10.89}\Revised{16.12.93}
\Version{Geant 3.16}\Routid{PHYS251}
\Makehead{Simulation of Rayleigh scattering}
\Shubr{GRAYL}{}
 
\Rind{GRAYL} generates Rayleigh scattering of a photon using 
the random-number
composition and rejection technique to sample the momentum
of the scattered photon and the scattering angle, according to the
form-factor distribution. In this reaction no new particles are
generated and the kinematical quantities of the scattered photon
replace the original ones in the \FCind{/GCTRAK/} common block.
 
Activation of the Rayleigh scattering is done via the {\tt FFREAD} data
record {\tt RAYL}. If this process is activated, \Rind{GRAYL} is called
by \Rind{GTGAMA} when a Rayleigh scattering occurs.
\section{ Method }
 
The Rayleigh differential cross section as a function of $ q^2 $ is given
by \cite{bib-NELS}:
\begin{equation}
\frac{d \sigma_R \left( q^2 \right) }{ d \Omega } =
 \frac{ \pi r_0^2}{ k^2 } \left( \frac{1 + \mu^2 }{ 2 } \right)
 \left| F_T \left( q \right) \right| ^2
\end{equation}
where:\\
 
\[\begin{array}{LL}
r_0       & \mbox{electron radius} \\
k         & \mbox{incident wave vector} \\
q=2k sin \frac{ \theta }{2}  & \mbox{momentum of scattered
              photon ($\theta$ is the scattering angle)} \\
\mu = cos \theta  & = 1 - \frac{q^2 }{ 2 k^2} \\
F_T \left( q \right)     & \mbox{molecular form factor}
\end{array} \]
 
Under the assumption that the atoms of a molecule are completely independent,
$ \left| F_T \left( q \right) \right| ^2 $ is given by:
\begin{equation}
 \left| F_T \left( q \right) \right| ^2 = \sum_{i=1}^{N}
 \frac{W_i}{A_i}  \left| F_i \left( q_i , Z_i  \right) \right| ^2
 \sigma_{c_i} \left( Z_i , E \right)
\end{equation}
where the index $i$ runs on the $N$ atoms in the molecule and:
 
\[\begin{array}{LL}
 W_i        & \mbox{proportion by weight} \\
 Z_i , A_i  & \mbox{atomic number and weight} \\
 F_i        & \mbox{form factor} \\
\sigma_{c_i}& \mbox{total atomic cross section for coherent scattering}\\
\end{array} \]
 
Using the combined composition and rejection sampling method described in
\Rind{GPAIRG} ({\tt [PHYS211]}) we may set:
\begin{equation}
f \left( q \right) = \sum_{i=1}^{N} \alpha_i f_i \left( q \right)
g_i \left( q \right) = \sum_{i=1}^{N} A \left( q_i^2 \right)
\frac{ \left| F_T \left( q \right) \right| ^2 }{ A \left( q_n^2 \right) }
\left( \frac{1+\mu^2}{2} \right)
\end{equation}

where:
 
\[\begin{array}{LL}
 n          & \mbox{number of energy bins} \\
 q_i        & \mbox{momentum of the photon with energy $ E_i $ of the
                 $i^{th}$  bin} \\
 q_n        & \mbox{upper limit for the momentum of the scattered photon} \\
 \alpha_i   & A \left( q_i^2 \right)  \\
 f_i \left( q \right )  &
 \frac{\left| F_T \left( q \right) \right| ^2}{A \left( q_n^2 \right)}  \\
 g_i \left( q \right )  & = \frac{1+\mu^2}{2} \mbox{\hspace{0.5cm} 
rejection function.}
\end{array} \]
 
 
Therefore, for given values of the random numbers $r_1$ and $r_2$, \Rind{GRAYL}
samples the momentum of the scattered photon and the scattering angle
$\theta$ via the following steps:
 
\begin{enumerate}
\item sample $ A \left( q^2 \right) = r_1 A \left( q_n^2 \right) $
\item find the $ \left (  q_{i-1} , q_i  \right] $ interval
which gives $ A \left( q_{i-1}^2 \right) \leq A \left( q^2 \right)
\leq  A \left( q_i^2 \right) $
\item calculate the linear extrapolation:
\begin{equation}
q = q_{i-1} + \left( A \left( q^2 \right) -  A \left( q_{i-1}^2 \right) \right)
\frac{q_i - q_{i-1} }
{ A \left( q_i^2 \right) -  A \left( q_{i-1}^2 \right)} \nonumber
\end{equation}
\item calculate $\mu = cos \theta  = 1 - q^2/(2 k^2) $
\item calculate $g_i \left( q \right) = (1 + \mu^2 )/2 $
\item if $g_i \left( q \right) > r_2 $ the event is accepted, otherwise
go back to 1.
\end{enumerate}
