The final state of elastic scattering is described by sampling the differential
scattering cross-sections
${{\rm d} \sigma \over {\rm d} \Omega}$. 
Two representations
are supported
for the normalised differential cross-section for 
elastic scattering.
The first is a tabulation of the differential cross-section, 
 as a 
function of the cosine of the scattering angle $\theta$ and the kinetic energy
$E$ of the incoming
neutron.
$${{\rm d} \sigma \over {\rm d} \Omega}~=~
{{\rm d} \sigma \over {\rm d} \Omega}\left(\cos{\theta,~E}\right)$$
The tabulations used are normalised by $\sigma/(2\pi)$ so the integral of the
differential cross-sections over the scattering angle yields unity.

In the second representation, the normalised cross-section are represented
as a series of legendre polynomials $P_l(\cos{\theta})$, and the 
legendre-coefficients $a_l$ are
tabulated as a function of the incoming energy of the neutron.
$${2\pi\over\sigma (E)}{{\rm d} \sigma \over {\rm d} \Omega}\left(\cos{\theta,~E}\right)~=~
\sum_{l=0}^{n_l} {2l+1\over 2}a_l(E)P_l(\cos{\theta})$$

Describing the details of the sampling procedures is outside the scope
of this paper.

An example of the result we show in figure \ref{elastic} for the elastic
scattering of 15~MeV neutrons off Uranium
a comparison of the simulated angular distribution of the scattered neutrons
with evaluated data.
The points are the evaluated data,
the histogram is the Monte Carlo prediction. 

In order to provide full
test-coverage for the algorithms, similar tests have been performed for 
${\rm^{72}Ge}$, 
${\rm^{126}Sn}$, 
${\rm^{238}U}$, 
${\rm^{4}He}$, and
${\rm^{27}Al}$ for a set of neutron kinetic energies.
The agreement is very good for all values of scattering angle and neutron 
energy investigated.
\begin{figure}[b!] % fig 1
\centerline{\epsfig{file=hadronic/lowEnergyNeutron/neutrons/plots/elastic.u238.14mev.costh.epsi,height=5.5in,width=3.5in}}
\vspace{10pt}
\caption{Comparison of data and Monte Carlo for the angular distribution of 
15~MeV neutrons scattered
elastically off Uranium ($^{238}U$). The points are evaluated data, and the histogram is
the Monte Carlo prediction.}
\label{elastic}
\end{figure}
