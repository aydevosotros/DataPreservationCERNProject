For inelastic scattering, the currently supported final states are (nA$\rightarrow$)
n$\gamma$s (discrete and continuum), np, nd, nt, n$^3$He, n$\alpha$, nd2$\alpha$,
nt2$\alpha$, n2p, n2$\alpha$, np$\alpha$, n3$\alpha$, 2n, 2np,
2nd, 2n$\alpha$, 2n2$\alpha$, nX, 3n, 3np, 3n$\alpha$, 4n, p,
pd, p$\alpha$, 2p d, d$\alpha$, d2$\alpha$, dt, t, t2$\alpha$,
$^3$He, $\alpha$, 2$\alpha$, and 3$\alpha$.

The photon distributions are again described as in the case of 
radiative capture. 

The 
possibility to describe the angular and energy distributions of the final 
state particles as in the case of fission is maintained, except that normally
only the arbitrary tabulation of secondary energies is applicable. 

In addition, we support the possibility to describe the energy angular
correlations explicitly, in analogy with the ENDF/B-VI data formats. 
In this case, the production cross-section for
reaction product n can be written as
$$\sigma_n(E, E', \cos(\theta))~=~\sigma(E)Y_n(E)p(E, E', \cos(\theta)).$$
Here $Y_n(E)$ is the product multiplicity, $\sigma(E)$ is the inelastic
cross-section, and $p(E, E', \cos(\theta))$ is the distribution probability.
Azimuthal symmetry is assumed.

The representations for the distribution probability supported are isotropic
emission, discrete two-body kinematics, N-body phase-space distribution, 
continuum energy-angle distributions, and continuum angle-energy distributions 
in the laboratory system.

The description of isotropic emission and discrete two-body kinematics is
possible without further information. In the case of N-body phase-space 
distribution, tabulated values for the number of particles being treated by the
law, and the total mass of these particles are used.
For the continuum energy-angle distributions, several options for representing
the angular dependence are available. Apart from the already introduced methods
of expansion in terms of legendre polynomials, and tabulation (here in
both the incoming neutron energy, and the secondary energy), the Kalbach-Mann
systematic is available.
In the case of the continuum angle-energy distributions 
in the laboratory system, only the tabulated form in incoming neutron energy,
product energy, and product angle is implemented.

First comparisons for product yields, energy and angular distributions have
been performed for a set of incoming neutron energies, but full test coverage
is still to be achieved.
In all cases currently investigated, the agreement between evaluated data and 
Monte Carlo is very good.
