\section{Reaction initial state simulation.}

\subsection{Allowed projectiles and bombarding energy range for interaction 
with nucleon and nuclear targets.}
\hspace{1.0em}
The GEANT4 parton string model is capable to predict final states (produced 
hadrons 
which belong to the scalar and vector meson nonets and the baryon (antibaryon) 
octet and decuplet) of reactions on nucleon and nuclear targets 
with a large variety of hadron (the same types as final state hadrons) and 
nuclear projectiles. The allowed bombarding energy in the hadron--nucleon 
collision should be above 2-pion production threshold (exception is for 
baryon--antibaryon annihilation, which can be performed even at rest of baryons).
In the case of hadron--nucleus or nucleus--nucleus collisions the initial
energy $\sqrt{s} > 5$ \ AGeV is recommended. This model is also able to predict
final states in the photon--nucleon and photon--nucleus inelastic 
collisions at the initial
energy $\sqrt{s} > 5$ \ AGeV. The last case is described in the 
separate chapter.
 
\subsection{ MC initialization procedure for a nucleus.}
\hspace{1.0em}The initialization of each nucleus, consisting from $A$
nucleons and $Z$ protons ($N=A-Z$ neutrons) 
with coordinates $\mathbf{r}_i$ and momenta
$\mathbf{p}_i$, where $i = 1,2,...,A$ is performed.
We use the standard initialization Monte Carlo procedure, which
is realized in the most of the high energy nuclear interaction models:
\begin{itemize}
\item Nucleon radii $r_i$ are selected randomly in the rest of nucleus according 
to proton or neutron density $\rho(r_i)$. 
For heavy nuclei with $A > 16$ \cite{GLMP91} normalized on unity 
nucleon density is
\begin{equation}
\label{NIS1}\rho(r_i) = 
 \frac{\rho_0}{1 + \exp{[(r_i - R)/a]}}
\end{equation}
where
\begin{equation}
\label{NIS2}\rho_0 \approx \frac{3}{4\pi R^3}(1+\frac{a^2\pi^2}{R^2})^{-1}.
\end{equation} 
Here $R=r_0 A^{1/3}$ \ fm and $r_0=1.16(1-1.16A^{-2/3})$ \ fm and $a
\approx 0.545$ fm.  For light nuclei with $A < 17$ normalized on unity 
nucleon density is
given by a harmonic oscillator shell model \cite{Elton61}, e. g.
\begin{equation}
\label{NIS2a} \rho(r_i) = (\pi R^2)^{-3/2}\exp{(-r_i^2/R^2)},
\end{equation}
where $R^2 = 2/3<r^2> = 0.8133 A^{2/3}$ \ fm$^2$ or 
\begin{equation}
\label{NIS3} \rho(r_i) =
\frac{4}{\pi^{3/2}R^3}[1+\frac{A-4}{6}(\frac{r_i}{R})^2]
\exp{(-r_i^2/R^2)},
\end{equation}
where
\begin{equation}
\label{NIS4} R^2=(\frac{5}{2}-\frac{4}{A})^{-1}(<r^2_{ch}>_{A}- 
<r^2_{ch}>_{p})
\end{equation}
The mean squared charge radii of the nucleus $<r^2_{ch}>_{A}$ and proton
$<r^2_{ch}>_{p}$ are taken from the measurements of lepton-nucleus
scattering experiments \cite{BJ77}.  To take into account nucleon
repulsive core it is assumed that internucleon distance $d > 0.8$ \ fm;

\item The initial momenta of the nucleons are randomly choosen between $0$ and 
$p^{max}_F$, where 
the maximal momenta of nucleons (in the local Thomas-Fermi 
approximation \cite{DF74}) depends from
the proton or neutron density $\rho_{Z,N}$ according to 
\begin{equation}
\label{NIS5} p^{max}_F = \hbar c(3\pi^2 \rho_{Z,N})^{1/3}
\end{equation}
with $\hbar c = 197.327$ MeVfm. 

\item To obtain coordinate and momentum components, it
 is assumed that nucleons are distributed isotropicaly in configuration
 and momentum spaces;

\item Then perform shifts of nucleon coordinates ${\bf r_i^{\prime}}
= {\bf r_i} - 1/A \sum_i {\bf r_i}$ and momenta ${\bf p_i^{\prime}}
= {\bf p_i} - 1/A \sum_i {\bf p_i}$ 
of nucleon momenta. The nucleus must be centered in configuration space around
$\mathbf{0}$, \textit{i. e.} $\sum_i {\mathbf{r}_i} = \mathbf{0}$ and
 the nucleus must be at rest, i. e. $\sum_i {\bf p_i} = \bf 0$ and
$\sum_i {\bf r_i} \times {\bf p_i}={\bf 0}$;

\item We compute energy per nucleon $e = E/A = m_{N} + B(A,Z)/A$, 
where $m_N$ is nucleon mass and the nucleus binding energy $B(A,Z)$ is given  
by the Bethe-Weizs\"acker formula\cite{BM69}:
\begin{equation}
\begin{array}{c}
\label{NIS6} B(A,Z) = \\
= -0.01587A + 0.01834A^{2/3} + 0.09286(Z- \frac{A}{2})^2 +
0.00071 Z^2/A^{1/3},
\end{array}
\end{equation} 
 and find the effective mass of each nucleon $m^{eff}_i = 
\sqrt{(E/A)^2 - p^{2\prime}_i}$.
\end{itemize}

\subsection{Lorentz boost of nucleon longitudinal momenta and energies.}

\hspace{1.0em}In the case of fast moving nucleus with initial momentum
per nucleon ${\bf P_0}= \{0, 0, P_{z0}\}$ one should perform Lorentz
transformation of the nucleon longitudinal momenta
\begin{equation}
\label{NIS7}  p_{zi} \rightarrow \gamma_i (p_{zi} - \beta_i e_{i})
\end{equation}
and the nucleon energies 
\begin{equation}
\label{NIS8} e_{i} \rightarrow \gamma_i (e_{i} - \beta_i p_{zi}),
\end{equation}
where $\beta_{i}$ is defined as
\begin{equation}
\label{NIS9} \beta_{i} = \frac{P_{z0}}{\sqrt{P_{z0}^2 + m^{eff2}_i}}
\end{equation}
and $\gamma_i$ is given by 
\begin{equation}
\label{NIS10}\gamma_i = \frac{1}{\sqrt{1 - \beta_i^2}}.
\end{equation}

\subsection{Random choice of the impact parameter.}

\hspace{1.0em}The impact parameter $0 \leq b \leq R_p + R_t$ is randomly
selected according to the probability:
\begin{equation}
\label{NIS11}P({\bf b})d{\bf b} = b d{\bf b},
\end{equation}
where $R_p$ and $R_t$ are the target and projectile radius,
respectively. In the case of nuclear projectile or target the nuclear radius is
determined from condition:
\begin{equation}
\label{NIS12}\frac{\rho(R)}{\rho(0)} = 0.01.
\end{equation}
 Then one should update the transversal components of
nucleon coordinates:
\begin{equation}
\label{NIS13} r_{xi} \rightarrow r_{xi} + b_x
\end{equation}
and
\begin{equation}
\label{NIS14} r_{yi} \rightarrow r_{yi} + b_y.
\end{equation}
Finaly, for target nucleus, if a projectile is centered around $0$, one
should perform a shift of the nucleon longitudinal coordinates:
\begin{equation}
\label{NIS15} r_{zi} \rightarrow r_{zi} + \Delta r_z/\gamma_{i},
\end{equation}
where $\Delta r_z = R_p + 1.5 \ fm + R_t + 1.5 \ fm$ is taken.

In the case of hadron--nucleus collision we determine the initial 
angular momentum 
of target nucleus:
\begin{equation}
\label{NIS16} \vec{L}= \vec{p}_0 \times \vec{r}_0,
\end{equation}
where $\vec{p}_0$ and $\vec{r}_0$ are initial momentum and enter radius vector 
of incoming hadron in the target nucleus rest frame.
