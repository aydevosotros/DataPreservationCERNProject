\section{Sample of collision participants
in nuclear collisions.}

\subsection{MC procedure to define collision participants.}
\hspace{1.0em} The inelastic
hadron--nucleus or 
nucleus--nucleus interactions at ultra--relativistic energies are considered 
as
independent hadron--nucleon or 
nucleon-nucleon inelastic collisions.  It was shown long
time ago \cite{CK78} for the hadron--nucleus collision that such a
picture can be obtained starting from the Regge--Gribov
approach \cite{BT76}, when one assumes that the hadron-nucleus elastic
scattering amplitude is a result of reggeon exchanges between the
initial hadron and  nucleons from target--nucleus. This result can be
extended for the nucleus--nucleus collision case and leads to 
simple and efficient MC procedure \cite{Am86} to define
the interaction cross sections and the number of the nucleons
participating in the inelastic nucleus--nucleus collision:
\begin{itemize}
\item We should randomly distribute $A$ nucleons from the projectile-nucleus and $B$
nucleons from the target-nucleus on the impact parameter plane according
to the weight functions $T([\vec{b}^{A}_{i}])$ and
$T([\vec{b}^{B}_{j}])$, respectively. These functions represent
probability densities to find sets of the nucleon impact parameters
$[\vec{b}^{A}_{i}]$, where $i=1,2,...,A$ and $[\vec{b}^{B}_{j}]$, where
$j=1,2,...,B$.
\item For each pair of nucleons $i$ and $j$ with choosen impact parameters $\vec{b}^{A}_{i}$ and
$\vec{b}^{B}_{j}$ we should check whether they interact inelastically or
not using the probability $p_{ij}(\vec{b}^{A}_{i}-\vec{b}^{B}_{j},s)$,
where $s_{ij}=(p_{i}+p_{j})^2$ is the squared total c.m.  energy of the
given nucleons with the $4$--momenta $p_{i}$ and $p_{j}$, respectively.
\end{itemize}
 
The described MC procedure is based on the probability
$P(\vec{B},s)$ given impact parameter $\vec{B}$ and at given squared total
c.m. nucleon--nucleon energy $s$ ( to simplify notations, we assume, that
all interacting nucleon pairs have the same $s$) 
of a such configuration, when several pairs of nucleons
from the projectile and target nuclei interact inelastically and the
rest of the nucleons do not participate in collisions:
\begin{equation}
\label{SP1}P(\vec B,s)=<\prod\limits_{i,j=1}p_{ij}(\vec b_i^A-\vec
b_j^B,s)\prod\limits_{k,l=1}(1-p_{kl}(\vec b_k^A-\vec b_k^B,s))>, 
\end{equation}
which can be rewritten
 more explicitly as
\begin{equation}
\label{SP2} 
\begin{array}{c}
P(\vec B,s)=  
\int \prod\limits_{i,j=1}p_{ij}(\vec b_i^A-\vec
b_j^B,s)\prod\limits_{k,l=1}(1-p_{kl}(\vec b_k^A-\vec b_l^B,s))\\ T_A(\vec
b_1^A)T_A(\vec b_2^A)...T_A(\vec B-\vec b_B^B)d\vec b_1^Ad\vec b_2^A...d\vec
b_B^B. 
\end{array}
\end{equation}

In the Regge--Gribov approach\cite{BT76} the probability for an inelastic
collision of nucleons $i$ and $j$ as a function at the squared impact
parameter difference $b_{ij}^2=(\vec{ b}_i^A-\vec{ b}_j^B)^2 $ and $s$
is given by
\begin{equation}
\label{SP3}
 p_{ij}(\vec{ b}_i^A-\vec{ b}_j^B,s)=
 c^{-1}[1-\exp{\{-2u(b_{ij}^2,s)\}}] = 
\sum_{n=1}^{\infty}p^{(n)}_{ij}(\vec{ b}_i^A-\vec{ b}_j^B,s), 
\end{equation}
where
\begin{equation}
\label{SP4}
 p^{(n)}_{ij}(\vec{ b}_i^A-\vec{ b}_j^B,s)
=c^{-1}\exp{\{-2u(b_{ij}^2,s)\}}
 \frac{[2u(b_{ij}^2,s)]^{n}}{n!}.
\end{equation}
is the probability to find the $n$ cut Pomerons (or the probability for
$2n$ string produced in an inelastic nucleon-nucleon collision).  These
probabilities are defined in terms of the (eikonal) amplitude of
nucleon--nucleon elastic scattering with Pomeron exchange:
\begin{equation}
\label{SP5}u(b_{ij}^2,s)=\frac{z(s)}{2}\exp (-b_{ij}^2/4\lambda (s)). 
\end{equation}
The quantities $z(s)$ and $\lambda (s)$ are expressed through the
parameters of the Pomeron trajectory, $\alpha _P^{^{\prime }}=0.25$
$GeV^{-2}$ and $\alpha _P(0)=1.0808$, and the parameters of the
Pomeron-nucleon vertex $R_P^2=3.56$ $GeV^{-2}$ and $\gamma _P=3.96$
$GeV^{-2}$:
\begin{equation}
\label{SP6}z(s)=\frac{2c\gamma _P}{\lambda (s)}(s/s_0)^{\alpha _P(0)-1} 
\end{equation}
\begin{equation}
\label{SP7}\lambda (s)=R_P^2+\alpha _P^{^{\prime }}\ln (s/s_0), 
\end{equation}
respectively, where $s_{0} = 3.0$ $GeV^{2}$ is a dimensional parameter.

In Eqs. (\ref{SP3},\ref{SP4}) the so--called shower enhancement
coefficient $c=1.4$ can be introduced to determine the contribution of
diffractive dissociation\cite{BT76}.  Thus, the total interaction
probability and the probability for diffractive dissociation of a pair
of nucleons can be computed as
\begin{equation}
\label{SP8}p_{ij}^{tot}(\vec b_i^A-\vec b_j^B,s)=(2/c)[1-\exp
\{-u(b_{ij}^2,s)\}]. 
\end{equation}
\begin{equation}
\label{SP9}p_{ij}^d(\vec b_i^A-\vec b_j^B,s)=\frac{c-1}{c}[p_{ij}^{tot}(\vec
b_i^A-\vec b_j^B,s)-p_{ij}(\vec b_i^A-\vec b_j^B,s)]. 
\end{equation}
The Pomeron parameters are found from a global fit of the total,
elastic, differential elastic and diffractive cross sections of the
nucleon--nucleon interaction at different energies.

For the pion-nucleon and kaon-nucleon collisions the Pomeron vertex
parameters and shower enhancement coefficient should be changed, e. g.
$R^{\pi}_{P^2} = 2.36$ $GeV^{-2}$, $\gamma^{\pi}_P = 2.17$ $GeV^{-2}$,
$s^{\pi}_{0} = 1.5$ $GeV^{2}$, $c^{\pi}=1.6$ and $R^{K}_{P^2} = 1.96$
$GeV^{-2}$, $\gamma^{K} _P = 1.92$ $GeV^{-2}$, $s^{K}_{0} = 2.3$
$GeV^{2}$, $c^{\pi}=1.8$ can be used to describe properly the total,
elastic and diffractive cross sections.

\subsection{High energy hadron--nucleon interaction cross sections.}

\hspace{1.0em}The hadron--nucleon  cross sections can be calculated from
the corresponding interaction probabilities. Particularly, the
hadron--nucleon total and inelastic cross section can be written as
follows:
\begin{equation}
\label{SP10}\sigma_{tot} = 2\pi \int_0^{\infty}bdbp_{ij}^{tot}(b^2_{ij},s) 
= \sigma_{P} f(\frac{z}{2})
\end{equation}
and
\begin{equation}
\label{SP11}\sigma_{in} = 2\pi \int_0^{\infty}bdbp_{ij}^{in}(b^2_{ij},s) 
= \sigma_{P} f(z),
\end{equation}
where
\begin{equation}
\label{SP12}\sigma_{P}=4\pi z(s)\lambda(s)
\end{equation}
and
\begin{equation}
\label{SP13} f(z)= \sum_{\nu = 1}^{\infty}\frac{(-z)^{\nu -1}}{\nu \nu !}.
\end{equation}


\subsection{Separation of hadron diffraction excitation.}

\hspace{1.0em}For each pair of nucleons $i$ and $j$ with choosen impact
parameters $\vec{b}^{A}_{i}$ and $\vec{b}^{B}_{j}$ we should check
whether they interact inelastically or not using the probability
\begin{equation}
\label{SP14}
p^{in}_{ij}(\vec{b}^{A}_{i}-\vec{b}^{B}_{j},s)=
p_{ij}(\vec{b}^{A}_{i}-\vec{b}^{B}_{j},s)
+ p_{ij}^d(\vec b_i^A-\vec b_j^B,s).
\end{equation}
 If interaction will be realized, then 
we have to consider it to be diffractive or nondiffractive with probabilities
\begin{equation}
\label{SP15}
\frac{p_{ij}^d(\vec b_i^A-\vec b_j^B,s)}{p^{in}_{ij}
(\vec{b}^{A}_{i}-\vec{b}^{B}_{j},s)}
\end{equation}
and
\begin{equation}
\label{SP16}
\frac{p_{ij}(\vec b_i^A-\vec b_j^B,s)}{p^{in}_{ij}
(\vec{b}^{A}_{i}-\vec{b}^{B}_{j},s)}.
\end{equation}
