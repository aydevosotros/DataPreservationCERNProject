\section{Longitudinal string excitation.}

\subsection{Hadron--nucleon inelastic collision.}

\hspace{1.0em}Let us consider collision of two hadrons with their c. m. momenta 
$P_1 = \{E^{+}_1,m^2_1/E^{+}_1,{\bf 0}\}$ and $P_2 =
\{E^{-}_2,m^2_2/E^{-}_2,{\bf 0}\}$, where the light-cone variables
$E^{\pm}_{1,2} = E_{1,2} \pm P_{z1,2}$ are defined through hadron
energies $E_{1,2}=\sqrt{m^2_{1,2} + P^2_{z1,2}}$, hadron longitudinal
momenta $P_{z1,2}$ and hadron masses $m_{1,2}$, respectively. Two
hadrons collide by two partons with momenta $p_1 = \{x^{+}E^{+}_1,0,{\bf
0}\}$ and $p_2 = \{0, x^{-}E^{-}_2,{\bf 0}\}$, respectively.

\subsection{The diffractive string excitation.} 

In the diffractive string excitation (the Fritiof approach \cite{FRITIOF87})
 only momentum can be transferred:
\begin{equation}
\label{LSE1}
\begin{array}{cc}
P^{\prime}_1 = P_1 + q\\
P^{\prime}_2 = P_2 -q,
\end{array}
\end{equation}
where 
\begin{equation}
\label{LSE2}q=\{-q^2_t/(x^{-}E^{-}_2),q^2_t/(x^{+}E^{+}_1),\bf q_t \}
\end{equation}
 is parton 
momentum transferred and ${\bf q_t}$ is its transverse component.
We use the Fritiof approach to simulate the diffractive excitation of 
particles.
 
\subsection{The string excitation by parton exchange.}

\hspace{1.0em}For this case the parton exchange (rearrangement) and
the momentum exchange are allowed \cite{QGSM82},\cite{DPM94}:
\begin{equation}
\label{LSE3}
\begin{array}{cc}
P^{\prime}_1 = P_1 - p_1 + p_2 + q \\
P^{\prime}_2 = P_2 + p_1 - p_2 - q,
\end{array}
\end{equation}
where $q= \{0,0, {\bf q_t}\}$ is parton momentum transferred, i. e. only
its transverse components ${\bf q_t}$ is taken into account \cite{Am86}.

\subsection{Transverse momentum sampling.}

\hspace{1.0em}The transverse component of the parton momentum
transferred is generated according to probability
\begin{equation}
\label{LSE4}P({\bf q_t})d{\bf q_t} = 
\sqrt{\frac{a}{\pi}} \exp{(-aq^2_t)}d{\bf q_t},
\end{equation}
where  parameter $a = 1.4$ GeV$^{-2}$.

{\bf $x^{+}$ and $x^{-}$ samplings.}

Light cone parton quantities $x^{+}$ and $x^{-}$ are generated
independently and according to distribution:
\begin{equation}
\label{LSE5} u(x) \sim x^{\alpha}(1 - x)^{\beta},
\end{equation}
where $x=x^{+}$ or $x=x^{-}$.
Parameters $\alpha =-1$ and $\beta = 0$ are chosen for the FRITIOF approach 
\cite{FRITIOF87}. In the case of the QGSM approach \cite{Am86} $\alpha = -0.5$ 
and $\beta = 1.5$ or $\beta = 2.5$.  Masses of the excited strings
should satisfy the kinematical constraints:
\begin{equation}
\label{LSE6} P^{\prime +}_1 P^{\prime -}_1 \geq m^2_{h1} + q^2_t
\end{equation}
and
\begin{equation}
\label{LSE7} P^{\prime +}_2 P^{\prime -}_2 \geq m^2_{h2} + q^2_t,
\end{equation}
where hadronic masses $m_{h1}$ and $m_{h2}$ (model parameters) are
defined by string quark contents.  Thus, the random selection of the
values $x^{+}$ and $x^{-}$ is limited by above constraints.

\subsection{The string excitation by quark or diquark annihilation.}

\hspace{1.0em}
We consider also hadron-hadron inelastic processes when antiquark or 
antidiquark from hadron projectile annihilate with corresponding quark 
or diquark from hadron target.
In this case excitation of one baryonic (string with quark and diquark 
ends) or mesonic (string with quark and antiquark ends) is created, 
respectively. These processes in the Regge theory correspond to cut 
reggeon exchange diagrams. Initial energy $\sqrt{s}$ 
dependences of these processes 
cross sections are defined by  intercepts of reggeon exchange trajectories.
For example $\sigma_{\pi^{+}p\rightarrow S(s)} \sim s^{\alpha_{\rho}(0)-1}$, 
$S$ notes string and $\alpha_{\rho}(0)$ is the intercept of $\rho$ reggeon 
trajectory. Thus $\sigma_{\pi^{+}p\rightarrow S(s)}
$ decreases with energy 
rise. Cross sections for other quark and diquark proccesses have simiar 
as $\sigma_{\pi^{+}p\rightarrow S(s)}$ initial energy dependences. 
Thus quark and diquark annihilation processes are important at 
relative low initial energies. Another example of these processes is 
$\bar{p}p \rightarrow S$, which is used in the kinetic model to describe 
final state of $\bar{p}p$ annihilation.
Simulation of such kind process is rather simple. We should randomly 
(according to weight calculated using hadron wave function)
choose quark (antiquark) or diquark (antidiquark) from projectile and 
find suitable (with the same flavor content) partner for annihilation 
from target. The created string four-momentum will be equal total reaction 
four-momentum since annihilated system has small neglected momentum (only 
low momenta quarks are able to annihilate).
 
To determine statistical weights for 
 quark annihilation processes are leading to a string production 
and separate them from processes, when two or more strings can be produced we 
use the Regge motivated total cross section parametrization suggested by
Donnachie and Landshoff \cite{DL92}. Using their parametrization the
statistical weight for the one string production process is given by
\begin{equation}
\label{OSE1} W_{1} = \frac{Y_{hN}s^{-\eta}}{\sigma^{tot}_{hN}(s)}
\end{equation}
and statistical weight to produce two and more strings is given by 
\begin{equation}
\label{OSE2} W_{2} = \frac{X_{hN}s^{\epsilon}}{\sigma^{tot}_{hN}(s)},
\end{equation}
where hadron-nucleon total cross sections  $\sigma^{tot}_{hN}(s)$ and its 
fit parameters $Y_{hN}$, $X_{hN}$, which do not depend 
from the total c.m. energy squared $s$ and depend on type of
projectile hadron $h$ and target nucleon $N$ can be found in \cite{PDG96}. 
The reggeon intercept $\eta \approx 
0.45$ and the pomeron intercept $\epsilon \approx 0.08$.


\subsection{Hadron--nucleus or nucleus--nucleus inelastic collisions.}

\hspace{1.0em}Hadron-nucleus or nucleus-nucleus collisions in the both
approaches (diffractive and parton exchange) are considered 
as a set of the independent hadron-nucleon or
nucleon-nucleon collisions.  However, the string excitation procedures
in these approaches are rather different.

\subsection{The diffractive string excitation.} 

\hspace{1.0em}In the diffractive string excitation (the 
FRITIOF approach \cite{FRITIOF87}) for each
inelastic hadron--nucleon or nucleon--nucleon collision we have to select
randomly the transverse momentum transferred ${\bf q_t}$ (in accordance
with the probability given by Eq. ($\ref{LSE4}$)) and select randomly
the values of $x^{\pm}$ (in accordance with distribution defined by
Eq. ($\ref{LSE5}$)). Then we have to calculate the parton momentum
transferred $q$ using Eq. ($\ref{LSE2}$) and update scattered hadron
and nucleon or scatterred nucleon and nucleon momenta using
Eq. ($\ref{LSE3}$). For each collision we have to check the constraints
($\ref{LSE6}$) and ($\ref{LSE7}$), which can be written more
explicitly:
\begin{equation}
\label{LSE8} [E_1^{+} -\frac{q^2_t}{x^{-}E^{-}_2}][\frac{m_1^2}{E^{+}_1} + 
\frac{q^2_t}{x^{+}E^{+}_1}]\geq m^2_{h1} + q^2_t
\end{equation}
and
\begin{equation}
\label{LSE9} [E_2^{-} +\frac{q^2_t}{x^{-}E^{-}_2}][\frac{m_2^2}{E^{-}_2} - 
\frac{q^2_t}{x^{+}E^{+}_1}]\geq m^2_{h1} + q^2_t.
\end{equation}

\subsection{The string excitation by parton rearrangement.} 

\hspace{1.0em}In this approach \cite{Am86} strings (as result of parton
rearrangement) should be spanned not only between valence quarks of
colliding hadrons, but also between valence and sea quarks and between
sea quarks.  The each participant hadron or nucleon should be splitted
into set of partons: valence quark and antiquark for meson or valence
quark (antiquark) and diquark (antidiquark) for baryon (antibaryon) and
additionaly the $(n-1)$ sea quark-antiquark pairs (their flavours are
selected according to probability ratios $ u:d:s = 1:1:0.35$), if hadron
or nucleon is participating in the $n$ inelastic collisions.  Thus for
each participant hadron or nucleon we have to generate a set of light
cone variables $x_{2n}$, where $x_{2n}=x^{+}_{2n}$ or
$x_{2n}=x^{-}_{2n}$ according to distribution:
\begin{equation}
\label{LS10} f^{h}(x_1,x_2,...,x_{2n})=f_{0}\prod_{i=1}^{2n}u^h_{q_i}(x_i)
\delta{(1-\sum_{i=1}^{2n}x_i)},
\end{equation}
where $f_0$ is the normalization constant.
Here, the quark structure functions $u_{q_i}^h(x_i)$ for valence quark 
(antiquark) $q_v$, 
sea quark and antiquark $q_s$ and valence diquark (antidiquark) $qq$ are:
\begin{equation}
\label{LS11}
u^h_{q_v}(x_v)=x_v^{\alpha_v},\ u^h_{q_s}(x_s)=x_s^{\alpha_s},\ u^h_{qq}(x_{qq})
=x_{qq}^{\beta_{qq}},
\end{equation}
where $\alpha_v = -0.5$ and $\alpha_s = -0.5$ \cite{QGSM82} 
 for the non-strange quarks (antiquarks) and $\alpha_v =
0$ and $\alpha_s = 0$ for strange quarks (antiquarks), $\beta_{uu} =
1.5$ and $\beta_{ud} = 2.5$ for proton (antiproton) and $\beta_{dd} =
1.5$ and $\beta_{ud} = 2.5$ for neutron (antineutron).  Usualy $x_i$ are
selected between $x^{min}_i \leq x_i \leq 1$, where model parameter
$x^{min}$ is a function of initial energy, to prevent from production of
strings with low masses (less than hadron masses), when whole selection
procedure should be repeated.  Then the transverse momenta of partons
${\bf q_{it}}$ are generated according to the Gaussian probability
Eq. ($\ref{LSE4}$) and under the constraint: $\sum_{i=1}^{2n}{\bf
q_{it}}=0$. The partons are considered as the off-shell partons,
i. e. $m^2_i \neq 0$.

\subsection{Barion and meson splitting.}
\hspace{1.0em}To perform
 a simulation of the string excitation we need to split hadron and
choose valence quark (antiquark) and diquark (antidiquark). In the case 
of a meson we split it into valence quark and antiquark (for neutral mixed 
mesons this sampling is performed according to mixing probabilities.)
In the case of barion we determine probabilities of baryon (antbarion) 
state (a quark (antiquark) and 
diquark (antidiquark) with given spin and isospin)
 from $SU(6)$ symmetric wave barion functions. These probabilities are 
 given in the Table 3.1. 
\hspace{1.0em}
\begin{table}
\begin{center}
\begin{tabular}{|c|c|}
\hline
Baryon type & Quark content \\
\hline
$p$         &$1/3uu_{11}d+1/6(1/12)ud_{11}u+1/2(7/12)ud_{00}u$    \\
$n$         &$1/6(1/12)ud_{11}d+1/2(7/12)ud_{00}d+1/3dd_{11}u$    \\
$\Sigma^{+}$&$1/3uu_{11}s+1/6us_{11}u+1/2us_{00}u$    \\
$\Sigma^{0}$&$1/3ud_{11}s+1/12us_{11}d+1/4us_{00}d+1/12ds_{11}u+1/4ds_{00}u$\\
$\Sigma^{-}$&$1/3dd_{11}s+1/6ds_{11}d+1/2ds_{00}d$    \\
$\Xi^{-}$   &$1/6ds_{11}s+1/2ds_{00}s+1/3ss_{11}d$    \\
$\Xi^{0}$   &$1/6us_{11}s+1/2us_{00}s+1/3ss_{11}u$    \\
$\Lambda^{0}$&$1/3ud_{00}s+1/4us_{11}d+1/12us_{00}d+1/4ds_{11}u+1/12ds_{00}u$\\
$\Delta^{++}$&$uu_{11}u$    \\
$\Delta^{+}$ &$1/3uu_{11}d+2/3ud_{11}u$    \\
$\Delta^{0}$ &$2/3ud_{11}d+1/3dd_{11}u$    \\
$\Delta^{-}$ &$dd_{11}d$    \\
$\Sigma^{*+}$&$1/3uu_{11}s+2/3us_{11}u$    \\
$\Sigma^{*0}$&$1/3ud_{11}s+1/3us_{11}d+1/3ds_{11}u$\\
$\Sigma^{*-}$&$1/3dd_{11}s+2/3ds_{11}d$    \\
$\Xi^{*0}$   &$1/3us_{11}s+2/3ss_{11}u$    \\
$\Xi^{*-}$   &$2/3ds_{11}s+1/3ss_{11}d$    \\
$\Omega^{-}$   &$ss_{11}s$    \\
\hline
\end{tabular}
\end{center}
Table 3.1: Baryon quark contents. Diquark indices indicate spin-isospin states.
\end{table}
We also  use these probability to sample barion or barion resonance in 
the string fragmentation, assuming that a valence diquark (antidiquark) 
keeps its spin and isospin during whole reaction.
