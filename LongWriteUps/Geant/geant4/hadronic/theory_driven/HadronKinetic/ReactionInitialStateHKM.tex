\section{Reaction initial state simulation.}

\subsection{Allowed projectiles and bombarding energy range for interaction 
with nucleon and nuclear targets.}
\hspace{1.0em}
The GEANT4 kinetic model is capable to predict final states (produced 
hadrons 
which belong to the pseudoscalar meson nonet and the baryon (antibaryon) 
octet) of reactions on nucleon and nuclear targets 
with a large variety of hadron (the same types as final state hadrons) and 
nuclear projectiles. The allowed bombarding kinetic energy in 
the hadron--nucleon, 
  hadron--nucleus and nucleus--nucleus collisions 
 is recommended to be more than 
 20 MeV in the laboratory frame.
  The upper limit of initial 
  kinetic energy is approximately $10$ GeV per initial particle. 
   
  This model is also able to predict
final states in the photon--nucleon and photon--nucleus inelastic 
collisions  as well as the nuclear absorption of the 
stopping $\pi^{-}$, $K^{-}$ and $\bar{p}$. 
 The last cases are described in the 
separate chapters.

  There are also needs
 to simulate hadron
kinetics after the "parton string" stage of nuclear interaction, taking
account secondary hadron rescatterings.

\subsection{MC initialization procedure for a nucleus}
\hspace{1.0em}The initialization of each nucleus, consisting from $A$
nucleons and $Z$ protons ($N=A-Z$ neutrons) 
with coordinates $\mathbf{r}_i$ and momenta
$\mathbf{p}_i$, where $i = 1,2,...,A$ is performed.
For the hadron kinetic model the MC initialization procedure for a nucleus is 
similar as for the 
parton string model except of some peculiarities, which are taken into 
account:
\begin{itemize}
\item Nucleon radii $r_i$ are selected randomly in the rest of nucleus according 
to proton or neutron density $\rho(r_i)$. 
For heavy nuclei with $A > 16$ \cite{GLMP91} normalized on unity 
nucleon density is
\begin{equation}
\label{KNIS1}\rho(r_i) = 
 \frac{\rho_0}{1 + \exp{[(r_i - R)/a]}}
\end{equation}
where
\begin{equation}
\label{KNIS2}\rho_0 \approx \frac{3}{4\pi R^3}(1+\frac{a^2\pi^2}{R^2})^{-1}.
\end{equation} 
Here $R=r_0 A^{1/3}$ \ fm and $r_0=1.16(1-1.16A^{-2/3})$ \ fm and $a
\approx 0.545$ fm.  For light nuclei with $A < 17$ normalized on unity 
nucleon density is
given by a harmonic oscillator shell model \cite{Elton61}, e. g.
\begin{equation}
\label{KNIS3} \rho(r_i) = (\pi R^2)^{-3/2}\exp{(-r_i^2/R^2)},
\end{equation}
where $R^2 = 2/3<r^2> = 0.8133 A^{2/3}$ \ fm$^2$ or 
\begin{equation}
\label{KNIS4} \rho(r_i) =
\frac{4}{\pi^{3/2}R^3}[1+\frac{A-4}{6}(\frac{r_i}{R})^2]
\exp{(-r_i^2/R^2)},
\end{equation}
where
\begin{equation}
\label{KNIS5} R^2=(\frac{5}{2}-\frac{4}{A})^{-1}(<r^2_{ch}>_{A}- 
<r^2_{ch}>_{p})
\end{equation}
The mean squared charge radii of the nucleus $<r^2_{ch}>_{A}$ and proton
$<r^2_{ch}>_{p}$ are taken from the measurements of lepton-nucleus
scattering experiments \cite{BJ77}.  To take into account nucleon
repulsive core it is assumed that internucleon distance $d > 0.8$ \ fm;

\item The initial momenta of the nucleons are randomly choosen between $0$ and 
$p^{max}_F$, where 
the maximal momenta of nucleons (in the local Thomas-Fermi 
approximation \cite{DF74}) depends from
the proton or neutron density $\rho_{Z,N}$ according to 
\begin{equation}
\label{KNIS6} p^{max}_F = \hbar c(3\pi^2 \rho_{Z,N})^{1/3}
\end{equation}
with $\hbar c = 197.327$ MeVfm. 

\item To obtain coordinate and momentum components, it
 is assumed that nucleons are distributed isotropicaly in configuration
 and momentum spaces;

We assume \cite{URQMD97}
that nucleons are not points in configuration space
and they are represented by Gaussian shaped density distributions (c = 1):
\begin{equation}
\label{KNIS7}\phi({\bf x_i}, t) = (\frac{2\alpha}{\pi})^{3/4}
\exp{\{-\alpha ({\bf x_i}
- {\bf r_i} (t))^2 + \frac{i}{\hbar}{\bf p_i} (t) {\bf x_i}\}}, 
\end{equation}
where $\alpha=0.25$ \ fm$^{-2}$ is a model parameter.
It gives us a possibility to take into account 
the Pauli blocking (the Pauli blocking procedure is decsribed below), 
i. e. if the phase space of $i$-th nucleon is already occupied by other
nucleons, its sampling is deserted;

\item Then perform shifts of nucleon coordinates ${\bf r_i^{\prime}}
= {\bf r_i} - 1/A \sum_i {\bf r_i}$ and momenta ${\bf p_i^{\prime}}
= {\bf p_i} - 1/A \sum_i {\bf p_i}$ 
of nucleon momenta. The nucleus must be centered in configuration space around
$\mathbf{0}$, \textit{i. e.} $\sum_i {\mathbf{r}_i} = \mathbf{0}$ and
 the nucleus must be at rest, i. e. $\sum_i {\bf p_i} = \bf 0$ and
$\sum_i {\bf r_i} \times {\bf p_i}={\bf 0}$;

\item We compute energy per nucleon $e = E/A = m_{N} + B(A,Z)/A$, 
where $m_N$ is nucleon mass and the nucleus binding energy $B(A,Z)$ is given  
by the Bethe-Weizs\"acker formula\cite{BM69}:
\begin{equation}
\begin{array}{c}
\label{KNIS8} B(A,Z) = \\
= -0.01587A + 0.01834A^{2/3} + 0.09286(Z- \frac{A}{2})^2 +
0.00071 Z^2/A^{1/3},
\end{array}
\end{equation} 
 and find the effective mass of each nucleon $m^{eff}_i = 
\sqrt{(E/A)^2 - p^{2\prime}_i}$.
\end{itemize}

\subsection{Lorentz boost of nucleon longitudinal momenta and energies.}

\hspace{1.0em}In the case of fast moving nucleus with initial momentum
per nucleon ${\bf P_0}= \{0, 0, P_{z0}\}$ one should perform Lorentz
transformation of the nucleon longitudinal momenta
\begin{equation}
\label{KNIS9}  p_{zi} \rightarrow \gamma_i (p_{zi} - \beta_i e_{i})
\end{equation}
and the nucleon energies 
\begin{equation}
\label{KNIS10} e_{i} \rightarrow \gamma_i (e_{i} - \beta_i p_{zi}),
\end{equation}
where $\beta_{i}$ is defined as
\begin{equation}
\label{KNIS11} \beta_{i} = \frac{P_{z0}}{\sqrt{P_{z0}^2 + m^{eff2}_i}}
\end{equation}
and $\gamma_i$ is given by 
\begin{equation}
\label{KNIS12}\gamma_i = \frac{1}{\sqrt{1 - \beta_i^2}}.
\end{equation}

\subsection{Lorentz contraction of the fast moving nucleus.}
\hspace{1.0em}For this case one need also to perform the Lorentz
contraction of the nucleus, \textit{i. e.}
\begin{equation}
\label{KNIS13} r_{zi} \rightarrow r_{zi}/ \gamma_{i}.
\end{equation}
\subsection{Random choice of the impact parameter.}

\hspace{1.0em}The impact parameter $0 \leq b \leq R_p + R_t$ is randomly
selected according to the probability:
\begin{equation}
\label{KNIS14}P({\bf b})d{\bf b} = b d{\bf b},
\end{equation}
where $R_p$ and $R_t$ are the target and projectile radius,
respectively. In the case of nuclear projectile or target the nuclear radius is
determined from condition:
\begin{equation}
\label{KNIS15}\frac{\rho(R)}{\rho(0)} = 0.01.
\end{equation}
 Then one should update the transversal components of
nucleon coordinates:
\begin{equation}
\label{KNIS16} r_{xi} \rightarrow r_{xi} + b_x
\end{equation}
and
\begin{equation}
\label{KNIS17} r_{yi} \rightarrow r_{yi} + b_y.
\end{equation}
Finaly, for target nucleus, if a projectile is centered around $0$, one
should perform a shift of the nucleon longitudinal coordinates:
\begin{equation}
\label{KNIS18} r_{zi} \rightarrow r_{zi} + \Delta r_z/\gamma_{i},
\end{equation}
where $\Delta r_z = R_p + 1.5 \ fm + R_t + 1.5 \ fm$ is taken.

In the case of hadron--nucleus collision we determine the initial 
angular momentum 
of target nucleus:
\begin{equation}
\label{KNIS19} \vec{L}_0= \vec{p}_0 \times \vec{r}_0,
\end{equation}
where $\vec{p}_0$ and $\vec{r}_0$ are initial momentum and  radius vector 
of incoming hadron enter point in the target nucleus rest frame.
