\section{Residual nucleus parameters in the hadron-nucleus 
collision.}

\hspace{1.0em}
The number of nucleons $A_{res}$ as well as the number of protons
$Z_{res}$ and total momentum $P_{res}$ of the residual nucleus are
determined by the relations:
\begin{equation}
\label{FR1}A_{res}=A+q _p-\sum_{i=1}^{N_{out}}q_i^c, 
\end{equation}
\begin{equation}
\label{FR2}Z_{res}=Z+e_p-\sum_{i=1}^{N_{out}}e_i^c, 
\end{equation}
\begin{equation}
\label{FR3}{\bf P}_{res}={\bf p}_p-\sum_{i=1}^{N_{out}}{\bf p_i^c}, 
\end{equation}
\begin{equation}
\label{FR4}{\bf L}_{res}={\bf L}_0-\sum_{i=1}^{N_{out}}{\bf l_i^c}. 
\end{equation}
Here $A$, $Z$ are the numbers of target nucleons and protons, 
$q_p,e_p,{\bf p}_p$ are the baryon number, the charge and momentum of incoming 
particle, respectively, ${\bf L}_0$ is initial particle-nucleus 
angular momentum and 
$q_i,e_i,{\bf p}_i,{\bf l_i^c}$  denote, correspondingly, the baryon number, 
charge,
momentum and angular carried away by the $i$th 
outgoing particle and $N_{out}$ is 
total number of the outgoing nucleons.
In the cascade model angular momentum of emitted particle is treated as a 
classical vector 
${\bf l_i^c} = [{\bf p_ir_i}]$, where ${\bf r_i^c}$ is the radius vector at the exit of the
cascade particle $i$ from the nucleus and ${\bf p_i^c}$ is its momentum.
