\section{Fermi break-up simulation for light nuclei.}

\hspace{1.0em}For light nuclei ($A \leq 16$) the values of excitation
energy per nucleon are often comparable with nucleon binding
energy. Thus a light excited nucleus breaks into two or more fragments
with branching given by available phase space.  To describe a process of
nuclear disassembling the so-called Fermi break-up model is used
\cite{Fermi50}, \cite{Kretz61}, \cite{EG67}.  This statistical approach
was first used by Fermi \cite{Fermi50} to describe the multiple
production in high energy nucleon collision.

The initial information for calculation of break-up stage consists from
the atomic mass number $A$, charge $Z$ and number of neutrons $N$ of
residual (e. g. after cascade or fission) nucleus and its excitation
energy $U$. The total energy of nucleus in the rest system will be
$E=U+M(A,Z)$.

\subsection{Masses of nuclei.} 

\hspace{1.0em}The tabulated values (calculated according to the liquid drop
model) \cite{CAM57} of mass defects  $\Delta M(A,Z)$
are used to calculate the masses of nuclei $M(A,Z)$ in ground states.

\subsection{ Allowed channel.} 

\hspace{1.0em}The channel will be allowed for decay, if the total
kinetic energy $E_{kin}$ of all fragments of the given channel at the
moment of break-up is positive. This energy can be calculated according
to equation:
\begin{equation}
\label{FBS1}E_{kin} = U+M(A,Z)-E_{Coulomb} - \sum_{b=1}^{n}(m_b+\epsilon_{b}),
\end{equation} 
$m_{b}$ and $\epsilon_{b}$ are masses and excitation energies of fragments, 
respectively, $E_{Coulomb}$ is the Coulomb barrier for the given channel. It 
is approximated by
\begin{equation}
\label{FBS2}E_{Coulomb} = \frac{3}{5} \frac{e^2}{r_{0}}(1 + 
\frac{V}{V_{0}})^{-1/3}
(\frac{Z^2}{A^{1/3}}-\sum_{b=1}^{n}\frac{Z^2}{A_b^{1/3}}),
\end{equation}
where $V_0$ is the volume of the system corresponding to the normal
nuclear matter density and $\kappa = \frac{V}{V_0}$ is a parameter (
$\kappa = 1$ is used).

\subsection{Break-up probability.} 

\hspace{1.0em}The total  probability per unit time for nucleus to
break-up into $n$ componets in the final state (i.e. residual nucleus,
if it will be, nucleons, deutrons, tritons, alphas etc) is given by
\begin{equation}
\label{FBS3}W(E,n) = (V/\Omega)^{n-1}\rho_{n}(E),
\end{equation}
where $\rho_{n}(E)$ is the density of a number of final states, $V$ is
the volume of decaying system and $\Omega = (2\pi h)^{3}$ is the
normalization volume.  The density $\rho_{n}(E)$ can be defined a
product of three factors:
\begin{equation}
\label{FBS4}\rho_{n}(E)=M_{n}(E)S_nG_n.
\end{equation}
The first one is the phase space factor defined as
\begin{equation}
\label{FBS5}M_{n} = \int_{-\infty}^{+\infty}...\int_{-\infty}^{+\infty}
\delta(\sum_{b=1}^{n} {\bf p_{b}}) \delta(E-\sum_{b=1}^{n}\sqrt{p^2+m^2_b})
\prod_{b=1}^{n} d^3p_b,
\end{equation}
where ${\bf p_b}$ are fragments momenta. The second one is the spin
factor
\begin{equation}
\label{FBS6} S_n = \prod_{b=1}^{n}(2s_b+1),
\end{equation}
which gives the number of states with different spin orientations.  The
last one is the permutation factor
\begin{equation}
\label{FBS7}G_n = \prod_{j=1}^{k}\frac{1}{n_j !},
\end{equation}
which takes into account identity of components in final state ($n_j$ is
a number of components of $j$- type particles and $k$ is defined by $n =
\sum_{j=1}^{k}n_{j}$). E.g. if in final state we have $n = 6$ particles
and from them there are $2$-alphas, $3$-nucleons and $1$-deutrons, then
$G_{6} = 1/(2! 3! 1!) = 1/12$.

In non-relativistic case (Eq. ($\ref{FBS10}$) the integration in
Eq. ($\ref{FBS5}$) can be evaluated analiticaly (see e. g. \cite{BBB58})
and the probability for a nucleus with energy $E$ disassembling into $n$
fragments with masses $m_b$, where $b = 1,2,3,...,n$ equals
\begin{equation}
\label{FBS8} W(E_{kin},n) = 
S_nG_n (\frac{V}{\Omega})^{n-1}(\frac{1}{\sum_{b=1}^{n}m_b}
\prod_{b=1}^{n}
m_{b})^{3/2}
 \frac{2\pi^{3(n-1)/2}}{\Gamma(3(n-1)/2)}E_{kin}^{3n/2-5/2}, 
\end{equation}
where $\Gamma(x)$ is the gamma function ($\Gamma(3(n-1)/2 =  (3(n-1)/2
- 1)!$).

\subsection{Fermi break-up model parameter.} 

\hspace{1.0em}Thus the Fermi break-up model has only one free parameter
$V$ is the volume of decaying system, which can be calculated as
following:
\begin{equation}
\label{FBS9} V = 4\pi R^3/3 = 4\pi r_{0}^3 A/3,
\end{equation}
where $r_{0} = 1.4 $ fm is used.

\subsection{ Fragment characteristics.}

We take into account the formation of fragments in their ground and
low-lying excited states, which are stable for nucleon
emission. However, several unstable fragments with large lifetimes:
$^{5}He$, $^{5}Li$, $^{8}Be$, $^{9}B$ etc are also considered.  Fragment
characteristics $A_b$, $Z_b$, $s_b$ and $\epsilon_b$ are taken from
\cite{AS81}.


\subsection{ MC procedure.} 

\hspace{1.0em}The nucleus break-up is described by the Monte Carlo (MC)
procedure. We randomly (according to probability Eq. ($\ref{FBS8}$) and
condition Eq. ($\ref{FBS1}$)) select decay channel. Then for given
channel we calculate kinematical quantities of each fragment according
to $n$-body phase space distribution:
\begin{equation}
\label{FBS10}M_{n} = \int_{-\infty}^{+\infty}...\int_{-\infty}^{+\infty}
\delta(\sum_{b=1}^{n} {\bf p_{b}}) \delta(\sum_{b=1}^{n}
\frac{p^2_b}{2m_b}-E_{kin})
\prod_{b=1}^{n} d^3p_b.
\end{equation}
The Kopylov's sampling procedure \cite{Kopylov70} is applied.  The angular
distributions for emitted fragments are considered as isotropical.

We take into account that the chargeless fragments are not affected by
Coulomb field. When fragment fly away to infinity its total kinetic
energy can be approximated by the sum of its translational motion and
the contribution to the energy erasing from its Coulomb repulsion.
