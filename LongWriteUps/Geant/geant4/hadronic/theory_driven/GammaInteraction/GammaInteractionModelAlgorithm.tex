\section{Gamma interaction model algorithm.}

\hspace{1.0em}
At intermediate energies $\gamma$-nucleon and $\gamma$-nucleus interactions are 
performed within the hadron kinetic model similarly as the hadron-nucleon and 
hadron-nucleus interactions. 

At high energies the Monte Carlo procedure
 in the case of $\gamma$--nucleon collision can be 
outlined as following:
\begin{itemize}
\item At given c.m. energy squared and at given virtuality $Q^2$ sample 
mass $M^2$ of 
hadronic $q\bar{q}$ fluctuation according to ($\ref{HEGI2}$) 
and sample its flavor 
according to statistical weights: $\omega_{u\bar{u}}= 1/2$, 
$\omega_{d\bar{d}}= 1/4$ and $\omega_{s\bar{s}}= 1/4$ are derived from 
($\ref{HEGI3}$);
\item Sample the momentum fraction $x$ of a valence quark inside
 a hadronic fluctuation 
according to 
\begin{equation}
\label{GIMA1} \rho(x) \sim \frac{1}{\sqrt{x(1-x)}}
\end{equation}
and transverse momentum of a quark according to the Gaussian 
distribution as for hadrons;
\item Split nucleon into quark and diquark as it was described
 for hadron-nucleon 
interaction;
\item Create two strings spanned between quark from a hadronic fluctuation and 
diquark from nucleon and between antiquark from a hadronic
 fluctuation and quark from nucleon;
\item Decay string into hadrons as it was described for
 hadron-nucleon interactions.
\end{itemize}

In the case of $\gamma$--nucleus collision the MC procedure is following:
\begin{itemize}
\item At given c.m. energy squared and at given virtuality
 $Q^2$ sample mass $M^2$ of 
hadronic $q\bar{q}$ fluctuation and sample its flavor as it is done for 
$\gamma$--nucleon collision;

\item Calculate coherence length $d$;

\item If coherence length less than internucleon distance 
then simulate inelastic 
hadron fluctuation-nucleon collission at choosen impact 
parameter $B$ as was described 
above;

\item If coherence length more than internucleon distance then 
 perform simulation of hadron fluctuation-nucleus collision at choosen 
impact parameter $B$ using parton string model similarly as for meson-nucleus 
interactions. For this case the probability of inelastic collision of
a  hadron fluctuation with nucleon
$i$  at given impact parameter ${\bf b}_i$ is calculated according to
\begin{equation}
\label{GIMA3} p_{VN}(s,b^2) = 1 - exp{[-2u(s, b^2)]};
\end{equation}
with  the eikonal $u(s,b^2)$ defined by Eq. ($\ref{HEGI7}$) at 
$Q^2 = 0$ and $M^2=M_{\rho}$.
\end{itemize}



