\section{Fission model algorithm.}

\hspace{1.0em}The Monte Carlo procedure of calculation of characteristics of
fission fragments can be outlined as following:
\begin{itemize}
\item Select fission mode (symmetric or asymmetric). 
Sample atomic number $A_f$ of a fission 
fragment according to experimentaly defined distribution, which consists 
from the "symmetrical" and "asymmetrical" parts; 
\item For choosen $A_f$ randomly in accordance with 
Gaussian distribution and the experimentaly defined 
dispersion and average
 select the fragment charge $Z_f$;
\item For 
choosen $A_f, Z_f$ sample
the kinetic energy of fragments 
 according to the Gaussian 
distribution 
 with experimentaly defined average values 
and dispersions;
\item Applying energy conservation and using fragment ground state masses 
calculate excitation energy of fragments and share it between fragments
assuming that fragments have equal temperatures;
\item Calculate absolute value of the c.m. 
fragment momentumeach(non-relativistic kinematics 
is used) and sample fragment flay off angles assuming isotropical 
angular distribution of fragments;
\item Perform evaporation for the excited fragments.
\end{itemize}
