\section{Fission process simulation.}

\subsection{Atomic number distribution of fission products.}

\hspace{1.0em}As follows from experimental data \cite{VH73} mass
distribution of fission products consists of the symmetric and the
asymmetric components:
\begin{equation}
\label{FPS1} F(A_f) = F_{sym}(A_f) + \omega F_{asym}(A_f),
\end{equation}
where $\omega(U,A,Z)$ defines relative contribution of each component
and it depends from excitation energy $U$ and $A,Z$ of fissioning
nucleus.  It was found in \cite{ABIM93} that experimental data can be
approximated with a good accuracy, if one take
\begin{equation}
\label{FPS2} F_{sym}(A_f) = \exp{[-\frac{(A_f - A_{sym})^2}{2\sigma_{sym}^2}]}
\end{equation}
and
\begin{equation}
\begin{array}{c}
\label{FPS3} F_{asym}(A_f) = \exp{[-\frac{(A_f - A_{2})^2}{2\sigma_{2}^2}]} + 
\exp{[-\frac{{A_f - (A - A_{2})}^2}{2\sigma_{2}^2}]} + \\
+ C_{asym}\{\exp{[-\frac{(A_f - A_{1})^2}{2\sigma_{1}^2}]} + 
\exp{[-\frac{{A_f - (A - A_{1})}^2}{2\sigma_{2}^2}]}\},
\end{array}
\end{equation}
where $A_{sym} = A/2$, $A_1$ and $A_2$ are the mean values and
$\sigma^2_{sim}$, $\sigma^2_1$ and $\sigma^2_2$ are dispertions of the
Gaussians respectively.  From an analysis of experimental data
\cite{ABIM93} the parameter $C_{asym} \approx 0.5$ was defined and the
next values for dispersions:
\begin{equation}
\label{FPS4} \sigma^2_{sym} = \exp{(0.00553U + 2.1386)},
\end{equation} 
where $U$ in MeV, 
\begin{equation}
\label{FPS5} 2\sigma_1 = \sigma_2 = 5.6 \ MeV
\end{equation}
for $A \leq 235$ and 
\begin{equation}
\label{FPS6} 2\sigma_1 = \sigma_2 = 5.6 + 0.096 (A - 235) \ MeV
\end{equation}
for $A > 235$ were found.

The weight $\omega(U,A,Z)$ was approximated as follows
\begin{equation}
\label{FPS7} \omega = \frac{\omega_{a} - F_{asym}(A_{sym})}
{1 - \omega_a F_{sym}((A_1 + A_2)/2)}.
\end{equation}
The values of $\omega_a$ for nuclei with $96 \geq Z \geq 90$ were 
approximated by
\begin{equation}
\label{FPS8} \omega_a(U) = \exp{(0.538U - 9.9564)}
\end{equation}
for $U \leq 16.25$ MeV,
\begin{equation}
\label{FPS9} \omega_a(U) = \exp{(0.09197U - 2.7003)}
\end{equation}
for $U > 16.25$ MeV and 
\begin{equation}
\label{FPS10} \omega_a(U) = \exp{(0.09197U - 1.08808)}
\end{equation}
for $z = 89$. 
For nuclei with $Z \leq 88$ the authors of \cite{ABIM93} constracted 
the following approximation:
\begin{equation}
\label{FPS11}\omega_a(U) = 
\exp{[0.3(227 - a)]} \exp{ \{0.09197[U - (B_{fis} - 7.5)] 
- 1.08808 \}},
\end{equation}
where for $A > 227$ and $U < B_{fis} - 7.5$ the corresponding factors occuring
in exponential functions vanish.

\subsection{Charge distribution of fission products.}

\hspace{1.0em}At given mass of fragment $A_f$ the 
experimental data \cite{VH73} on the charge $Z_f$ distribution of
fragments are well approximated by Gaussian with dispertion
$\sigma^2_{z} = 0.36$ and the average $<Z_f>$ is described by
expression:
\begin{equation}
\label{FPS12} <Z_f> = \frac{A_f}{A}Z + \Delta Z, 
\end{equation}
when parameter $\Delta Z = -0.45$ for $A_f \geq 134$, $\Delta Z = -
0.45(A_f -A/2)/(134 - A/2)$ for $ A - 134 < A_f < 134$ and $\Delta Z =
0.45$ for $A \leq A - 134$.

After sampling of fragment atomic masses numbers and fragment charges, 
we have to check that fragment ground state masses do not exceed initial 
energy and calculate the maximal fragment kinetic energy 
\begin{equation}
\label{FPS13a}T^{max} < U + M(A,Z) - M_1(A_{f1}, Z_{f1}) - M_2(A_{f2}, Z_{f2}),
\end{equation}
where $U$ and $M(A,Z)$ are the excitation energy and mass of initial
nucleus,  $M_1(A_{f1},
Z_{f1})$,  and $M_2(A_{f2}, Z_{f2})$ are masses
of the first and second fragment, respectively.

  
\subsection{Kinetic energy distribution of fission products.}

\hspace{1.0em}We use the empiricaly defined \cite{VKW85} dependence of 
the average kinetic energy $<T_{kin}>$ (in MeV) of fission fragments on
the mass and the charge of a fissioning nucleus:
\begin{equation}
\label{FPS13}<T_{kin}> = 0.1178 Z^2/A^{1/3} + 5.8.
\end{equation}
This energy is distributed differently in cases of symmetric and
asymmetric modes of fission.  It follows from the analysis of data
\cite{ABIM93} that in the asymmetric mode, the average kinetic energy of
fragments is higher than that in the symmetric one by approximately
$12.5$ MeV. To approximate the average numbers of kinetic energies
$<T_{kin}^{sym}$ and $<T_{kin}^{asym}>$ for the symmetric and asymmetric
modes of fission the authors of \cite{ABIM93} suggested empirical
expressions:
\begin{equation}
\label{FPS14} <T_{kin}^{sym}> = <T_{kin}> - 12.5 W_{asim}, 
\end{equation}
\begin{equation}
\label{FPS15} <T_{kin}^{asym}> = <T_{kin}> + 12.5 W_{sim},
\end{equation} 
where 
\begin{equation}
\label{FPS16} W_{sim} = \omega \int F_{sim}(A)dA/\int F(A)dA
\end{equation}
and
\begin{equation}
\label{FPS17} W_{asim} = \int F_{asim}(A)dA/\int F(A)dA,
\end{equation} 
respectively. In the symmetric fission the experimental data for the
ratio of the average kinetic energy of fission fragments
$<T_{kin}(A_f)>$ to this maximum energy $<T^{max}_{kin}>$ as a function
of the mass of a larger fragment $A_{max}$ can be approximated by
expressions
\begin{equation}
\label{FPS18} <T_{kin}(A_f)>/<T^{max}_{kin}> = 
1 - k [(A_f - A_{max})/A]^2
\end{equation}
for $A_{sim} \leq A_f \leq A_{max} + 10$ and 
\begin{equation}
\label{FPS19} <T_{kin}(A_f)>/<T^{max}_{kin}> = 
1 - k(10/A)^2 - 2 (10/A)k(A_f - A_{max} - 10)/A
\end{equation}
for $A_f > A_{max} + 10$, where $A_{max} = A_{sim}$ and $k = 5.32$ and
$A_{max} = 134$ and $k = 23.5$ for symmetric and asymmetric fission
respectively.  For both modes of fission the distribution over the
kinetic energy of fragments $T_{kin}$ is choosen Gaussian with their own
average values $<T_{kin}(A_f)>= <T_{kin}^{sym}(A_f)>$ or
$<T_{kin}(A_f)>=<T_{kin}^{asym}(A_f)>$ and dispersions $\sigma^2_{kin}$
equal $8^2$ MeV or $10^2$ MeV$^2$ for symmetrical and asymmetrical
modes, respectively. 

\subsection{Calculation of the excitation energy of fission products.}

\hspace{1.0em}The total excitation energy of fragments $U_{frag}$ 
can be defined according to equation:
\begin{equation}
\label{FPS21} U_{frag} = U + M(A,Z) - M_1(A_{f1}, Z_{f1}) - M_2(A_{f2}, Z_{f2}) - 
T_{kin},
\end{equation}
where $U$ and $M(A,Z)$ are the excitation energy and mass of initial
nucleus, $T_{kin}$ is the fragments kinetic energy, $M_1(A_{f1},
Z_{f1})$,  and $M_2(A_{f2}, Z_{f2})$ are masses
of the first and second fragment, respectively.

The value of excitation energy of fragment $U_f$ determines the fragment
temperature ($T = \sqrt{U_f/a_f}$, where $a_f \sim A_f$ is the parameter
of fragment level density).  Assuming that after disintegration
fragments have the same temperature as initial nucleus than the total
excitation energy will be distributed between fragments in proportion to
their mass numbers one obtains
\begin{equation}
\label{FPS22} U_f = U_{frag} \frac{A_f}{A}.
\end{equation}

\subsection{Excited fragment momenta.}

\hspace{1.0em}Assuming that fragment kinetic energy $T_f= 
P^2_f/(2(M(A_{f},Z_{f}+U_f)$ we are 
able to calculate the absolute value of fragment c.m. momentum 
\begin{equation}
\label{FPS23}
P_f=\frac{(M_1(A_{f1},Z_{f1}+U_{f1})(M_2(A_{f2},Z_{f2}+U_{f2})}{
M_1(A_{f1},Z_{f1})+U_{f1} + M_2(A_{f2},Z_{f2})+U_{f2}}T_{kin}.
\end{equation}
and its components, assuming fragment isotropical distribution.
