\chapter{Adding a new package to CERNLIB}

The following information is normally only required by members
of the CERN Program Library team. 

To add a new package to CERNLIB, the following steps must be
performed.

\section{Unix systems}

\section{VMS systems}

\chapter{Changing the version of an existing package}

\index{Changing the version of ! isajet}
\index{Changing the version of ! jetset}
\index{Changing the version of ! pythia}
\index{Changing the version of ! herwig}
\index{Changing the version of ! geant}
\index{isajet}
\index{jetset}
\index{pythia}
\index{herwig}
\index{geant}

Certain packages, notably the Monte Carlo libraries, contain
the version number in the source and library file names. When
a new version is received from the authors, the following 
must be done:

\begin{itemize}
\item
The new source file must be installed on asis.
\item
The {\it make} files must be updated appropriately.
\end{itemize}

\section{VMS}


On VMS systems, the {\bf CERNLIB} command must be modified and
rebuilt, if the new version is to become the default.

\section{Unix}

On Unix systems, the {\bf CERNLIB} script must be modified and
rebuilt, if the new version is to become the default.

