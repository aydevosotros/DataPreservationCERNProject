\chapter{Description of the VMS DCL command procedures}
\label{sect-DCL}

\section{BACKUP\_CERNLIB}

Command file to build a {\bf BACKUP} saveset containing material
from the CERN:[PRO.DOC], [PRO.LIB], [PRO.EXE], [PRO.MGR], [PRO.SRC.COM]
directories. The saveset is written to 
CERN:[PRO.BCK]CRNLIB.BCK.~\footnote{{\bf N.B. Note that VAX and Alpha
specific savesets are to be found in the {\tt CERNVAX} and {\tt CERNAXP}
trees respectively.}}

\section{BACKUP\_CMZ}

Command file to build a {\bf BACKUP} saveset of the current
CERN:[PRO.CMZ] tree. The saveset is written to
CERN:[PRO.BCK]CRNCMZ.BCK.

\section{BACKUP\_PATCHY}

Command file to build a {\bf BACKUP} saveset of the current
PATCHY PRO tree. The saveset is written to CERN:[PRO.BCK]CRNPAT.BCK.

\section{cernstart}

{\bf CERNSTART.COM} is a command file that is invoked at system startup
time to perform functions such as

\begin{itemize}
\item
Define system logical names
\item
Install software using the VMS {\bf INSTALL} utility
\item
Perform NFS mounts
\item
\ldots
\end{itemize}

{\bf N.B. The command file on the VXCERN cluster is CERN specific
but may be used as an example for other systems.}

\section{enable\_staging}

This command file is part of the {\bf VAXTAP} package~\cite{bib-VAXTAP}
and is not required unless you wish to use the {\bf STAGE} component
of {\bf VAXTAP}.

\section{make}

{\bf MAKE.COM} is a partial emulation of the Unix {\it make} command.
Together with {\bf MAKEPACK.COM}, it is used to build the various
components of the CERN Program Library.

The syntax is given below:

\begin{verbatim}
make [-d] [-f] [-i] [-l] [-n] [-s] [-t] target1 ... targetn
\end{verbatim}
The options are as follows:
\begin{DLtt}{12}
\item[-d]Debug mode
\item[-f]Define makefile (dummy)
\item[-i]Interactive run
\item[-l]Library only
\item[-n]Noexecute mode - implies -i
\item[-s]Run job as subprocess (default)
\item[-t]Testpack flag
\end{DLtt}

\section{makepack}

This command file is auxiliary to {\bf MAKE.COM} and is used
to build components of the CERN Library. 

\section{nfsdir}

This command file can be used to sort the output of a directory
command in a manner similar to the Unix {\tt ls -ltr} command.

\begin{XMPt}{Example of using the NFSDIR command file}

vxcrna:/cernlib > @cern:[new.mgr]nfsdir cern:[new.src.car]kern*.car

Total of 40 files, 9383/9383 blocks.
Directory CERN:[NEW.SRC.CAR]
KERNCDC.CAR;1           800/800      7-SEP-1988 20:33:44.00
KERNHYW.CAR;1            91/91      21-DEC-1989 12:41:02.00
KERNUNI.CAR;1           574/574     21-DEC-1989 12:41:45.00
KERNNOR.CAR;1           146/146     21-DEC-1989 16:03:21.00
KERNTMO.CAR;1            35/35      21-DEC-1989 16:03:41.00
KERNPDP.CAR;1           257/257      3-MAY-1990 02:24:10.00
KERNDGE.CAR;1            63/63       3-MAY-1990 02:24:26.00
KERNAMX.CAR;1            10/10      11-JUL-1991 04:38:43.00
KERNCVX.CAR;1           119/119     16-AUG-1991 16:08:48.00
KERNALI.CAR;1            23/23       7-OCT-1991 22:29:35.00
KERNHPX.CAR;1            14/14      22-MAY-1992 16:49:29.00
KERNIBX.CAR;1           326/326     22-MAY-1992 16:49:31.00
KERNA10.CAR;1           159/159      6-OCT-1992 12:02:09.00
KERNAPO.CAR;1           159/159      6-OCT-1992 12:02:09.00
KERNMPW.CAR;1            21/21       9-OCT-1992 01:55:41.00
KERNCRY.CAR;1           117/117     20-OCT-1992 18:29:12.00
KERNCRU.CAR;1           117/117     20-OCT-1992 18:29:12.00
KERNSGI.CAR;1            13/13      20-JAN-1993 12:39:13.00
KERNNXT.CAR;1            31/31      29-JAN-1993 12:02:29.00
KERNIBM.CAR;1           653/653     10-FEB-1993 00:38:04.00
KERNCMS.CAR;1           402/402     27-FEB-1993 02:24:52.00
KERNOS9.CAR;1            14/14       4-MAY-1993 21:35:21.00
KERNALT.CAR;1            61/61       4-JUN-1993 17:38:26.00
KERNDEC.CAR;1            38/38       4-JUN-1993 17:38:27.00
KERNOSF.CAR;1            38/38       4-JUN-1993 17:38:27.00
KERNVMI.CAR;1            38/38       4-JUN-1993 17:38:27.00
KERNDOS.CAR;1           153/153     23-JUL-1993 20:55:15.00
KERNGEN.CAR;1             4/4       12-AUG-1993 11:41:41.00
KERNNUM.CAR;1          1834/1834    25-AUG-1993 18:47:04.00
KERNNUMT.CAR;1          631/631     25-AUG-1993 18:48:28.00
KERNLNX.CAR;1            56/56       6-SEP-1993 15:34:22.00
KERNVAX.CAR;1           385/385     10-NOV-1993 10:09:46.33
KERNSOL.CAR;1             1/1       14-JAN-1994 15:48:56.77
KERNSUN.CAR;1             1/1       14-JAN-1994 15:48:56.77
KERNIRS.CAR;1            56/56      17-JAN-1994 17:16:11.98
KERNIRT.CAR;1            56/56      17-JAN-1994 17:16:11.98
KERNBIT.CAR;1           637/637     18-JAN-1994 18:43:03.85
KERNFOR.CAR;1           929/929     21-JAN-1994 19:56:29.15
KERNGENT.CAR;1          308/308     21-JAN-1994 19:57:34.87

\end{XMPt}

\section{plienv}

This command file is used to define the environment expected for
installation of the libraries. This includes the definition
of various logical names and symbols.

Of particular importance are the following:

\begin{DLtt}{1234567890}
\item[PLINAME]Symbol used by the installation jobs to select the correct
version of the various packages. See section \ref{sect-PLINAME} for more
details.
\item[MAKE]Runs CERN\_ROOT:[EXE]MAKE.COM
\item[MAKEPACK]Runs CERN\_ROOT:[EXE]MAKEPACK.COM
\item[TESTPACK]Runs CERN\_ROOT:[EXE]TESTPACK.COM
\item[PLIENV]Runs CERN\_ROOT:[EXE]PLIENV.COM
\item[PLILOG]Runs CERN\_ROOT:[EXE]PLILOG.COM\\
\item[SAY]Defined as {\tt WRITE SYS\$OUTPUT}
\item[XTYPE]Runs CERN\_ROOT:[EXE]XTYPE.COM
\end{DLtt}

\section{plilog}

Command file to edit a specified installation log file.

\section{release}

Command file to issue the appropriate {\bf SET FILE/ENTER} and
{\bf SET FILE/REMOVE} commands to equivalence specific versions
of the library with {\bf OLD}, {\bf PRO} and {\bf NEW}.
See section \ref{sect-VXCERN} for more information on this 
VXCERN specific command file.

\section{stagelist}

This command file is part of the {\bf VAXTAP} package~\cite{bib-VAXTAP}
and is not required unless you wish to use the {\bf STAGE} component
of {\bf VAXTAP}.

\section{setenv}

Command file to define a global symbol or a logical name.

\begin{XMPt}{Examples of using the {\bf SETENV} command}

setenv baggins frodo

show symbol baggins

BAGGINS == "FRODO"

setenv -l baggins frodo

show logical baggins

   "BAGGINS" = "[.FRODO]" (LNM$PROCESS_TABLE)

setenv display zfatal:0

show display

   Device:    WS404:  [super]
   Node:      ZFATAL:0
   Transport: TCPIP
   Server:    0
   Screen:    0

\end{XMPt}

\section{testpack}

Command file auxiliary to {\bf MAKE} to build the test
jobs for the various CERNLIB components.

\section{ytocmz}

Command file to convert a {\bf PATCHY} card file into {\bf CMZ}
format.
