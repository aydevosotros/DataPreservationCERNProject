\section{Porting {\bf PATCHY} to a new platform}
 
Should you need to install {\bf PATCHY} on a machine to which
it has not already been ported, the following tips may prove
useful. One should start from the installation kit for a similar
system or compiler (e.g. the VAX/VMS kit for the Alpha/VMS
version - an Alpha/VMS version is, of course, available).

\begin{itemize}
\item
file extensions. Although most Unix systems use {\bf .f} for Fortran files,
some, such as Apollo, use {\bf .ftn}.
\item
compiler name and options. The Fortran compiler
is normally invoked using the {\bf f77} command, although
the RS6000 uses {\bf xlf} and the Convex {\bf fc}.
If it is necessary to modify the compiler and/or options, one
should also remove the check of the file {\bf p4boot.sh} against
{\bf p4boot.sh0}. If there is a mismatch, the installation procedure
will exit.
\item
Fortran installation packages. It may be necessary to make modifications
to the files {\bf rceta.f} or {\bf fcasplit.f}
\end{itemize}
