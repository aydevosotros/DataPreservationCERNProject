%%%%%%%%%%%%%%%%%%%%%%%%%%%%%%%%%%%%%%%%%%%%%%%%%%%%%%%%%%%%%%%%%%%
%                                                                 %
%  CERNLIB manual in LaTeX form 	                          %
%                                                                 %
%  Run one routine from the manual                                %
%                                                                 %
%  Michel Goossens (for translation into LaTeX)                   %
%                                                                 %
%  Last Mod. 15 May 1996  0730   MG                               %
%                                                                 %
%%%%%%%%%%%%%%%%%%%%%%%%%%%%%%%%%%%%%%%%%%%%%%%%%%%%%%%%%%%%%%%%%%%
\documentclass[11pt,fleqn]{cernman}
\usepackage{amssymb,amsmath}
\usepackage{cernlib}
\usepackage{html}

% PDF/A packages
\usepackage[pdftex, pdfa, linktoc=none]{hyperref}

% ----------------------------------------------
% Add metadata
\hypersetup{
	pdftitle={Hyperbolic Arcsine - CERN Program Library},
	pdfauthor={K.S. Kölbig},
	pdfsubject={Cern Library Documentation},
	pdfkeywords={ARSINH HYPERBOLIC ARCSIN},
	pdflang={en},
	bookmarksopen=true,
	bookmarksopenlevel=3,
	hypertexnames=false,
	linktocpage=true,
	plainpages=false,
	breaklinks
}

\newcommand{\binomv}[2]{\genfrac{}{}{0pt}{}{#1}{#2}}
\newcommand{\binomg}[2]{\genfrac{\{}{\}}{0pt}{}{#1}{#2}}
\newcommand{\binoms}[2]{\genfrac{[}{]}{0pt}{}{#1}{#2}}
%\usepackage{screen}
\newcommand{\Title}{CERN Program Library}%       Title for document
\renewcommand{\rmdefault}{ptm}
\newcommand{\OBSOLETE}{\makebox[114mm][c]{\textbf{OBSOLETE}}\\}
\newcommand{\POBSOLETE}{\makebox[114mm][c]{\textbf{PARTIALLY OBSOLETE}}\\}
\renewcommand{\Rind}[2]{\texttt{#1} (#2)\Inref{#1}}%Special for shortwrups
\begin{document}

\let\LARGE\large
\let\Large\large

% Here comes the name to run

%%%%%%%%%%%%%%%%%%%%%%%%%%%%%%%%%%%%%%%%%%%%%%%%%%%%%%%%%%%%%%%%%%%
%                                                                 %
%  CERNLIB manual in LaTeX form 	                          %
%                                                                 %
%  Run one routine from the manual                                %
%                                                                 %
%  Michel Goossens (for translation into LaTeX)                   %
%                                                                 %
%  Last Mod. 15 May 1996  0730   MG                               %
%                                                                 %
%%%%%%%%%%%%%%%%%%%%%%%%%%%%%%%%%%%%%%%%%%%%%%%%%%%%%%%%%%%%%%%%%%%
\documentclass[11pt,fleqn]{cernman}
\usepackage{amssymb,amsmath}
\usepackage{cernlib}
\usepackage{html}

% PDF/A packages
\usepackage[pdftex, pdfa, linktoc=none]{hyperref}

% ----------------------------------------------
% Add metadata
\hypersetup{
	pdftitle={Hyperbolic Arcsine - CERN Program Library},
	pdfauthor={K.S. Kölbig},
	pdfsubject={Cern Library Documentation},
	pdfkeywords={ARSINH HYPERBOLIC ARCSIN},
	pdflang={en},
	bookmarksopen=true,
	bookmarksopenlevel=3,
	hypertexnames=false,
	linktocpage=true,
	plainpages=false,
	breaklinks
}

\newcommand{\binomv}[2]{\genfrac{}{}{0pt}{}{#1}{#2}}
\newcommand{\binomg}[2]{\genfrac{\{}{\}}{0pt}{}{#1}{#2}}
\newcommand{\binoms}[2]{\genfrac{[}{]}{0pt}{}{#1}{#2}}
%\usepackage{screen}
\newcommand{\Title}{CERN Program Library}%       Title for document
\renewcommand{\rmdefault}{ptm}
\newcommand{\OBSOLETE}{\makebox[114mm][c]{\textbf{OBSOLETE}}\\}
\newcommand{\POBSOLETE}{\makebox[114mm][c]{\textbf{PARTIALLY OBSOLETE}}\\}
\renewcommand{\Rind}[2]{\texttt{#1} (#2)\Inref{#1}}%Special for shortwrups
\begin{document}

\let\LARGE\large
\let\Large\large

% Here comes the name to run

%%%%%%%%%%%%%%%%%%%%%%%%%%%%%%%%%%%%%%%%%%%%%%%%%%%%%%%%%%%%%%%%%%%
%                                                                 %
%  CERNLIB manual in LaTeX form 	                          %
%                                                                 %
%  Run one routine from the manual                                %
%                                                                 %
%  Michel Goossens (for translation into LaTeX)                   %
%                                                                 %
%  Last Mod. 15 May 1996  0730   MG                               %
%                                                                 %
%%%%%%%%%%%%%%%%%%%%%%%%%%%%%%%%%%%%%%%%%%%%%%%%%%%%%%%%%%%%%%%%%%%
\documentclass[11pt,fleqn]{cernman}
\usepackage{amssymb,amsmath}
\usepackage{cernlib}
\usepackage{html}

% PDF/A packages
\usepackage[pdftex, pdfa, linktoc=none]{hyperref}

% ----------------------------------------------
% Add metadata
\hypersetup{
	pdftitle={Hyperbolic Arcsine - CERN Program Library},
	pdfauthor={K.S. Kölbig},
	pdfsubject={Cern Library Documentation},
	pdfkeywords={ARSINH HYPERBOLIC ARCSIN},
	pdflang={en},
	bookmarksopen=true,
	bookmarksopenlevel=3,
	hypertexnames=false,
	linktocpage=true,
	plainpages=false,
	breaklinks
}

\newcommand{\binomv}[2]{\genfrac{}{}{0pt}{}{#1}{#2}}
\newcommand{\binomg}[2]{\genfrac{\{}{\}}{0pt}{}{#1}{#2}}
\newcommand{\binoms}[2]{\genfrac{[}{]}{0pt}{}{#1}{#2}}
%\usepackage{screen}
\newcommand{\Title}{CERN Program Library}%       Title for document
\renewcommand{\rmdefault}{ptm}
\newcommand{\OBSOLETE}{\makebox[114mm][c]{\textbf{OBSOLETE}}\\}
\newcommand{\POBSOLETE}{\makebox[114mm][c]{\textbf{PARTIALLY OBSOLETE}}\\}
\renewcommand{\Rind}[2]{\texttt{#1} (#2)\Inref{#1}}%Special for shortwrups
\begin{document}

\let\LARGE\large
\let\Large\large

% Here comes the name to run

%%%%%%%%%%%%%%%%%%%%%%%%%%%%%%%%%%%%%%%%%%%%%%%%%%%%%%%%%%%%%%%%%%%
%                                                                 %
%  CERNLIB manual in LaTeX form 	                          %
%                                                                 %
%  Run one routine from the manual                                %
%                                                                 %
%  Michel Goossens (for translation into LaTeX)                   %
%                                                                 %
%  Last Mod. 15 May 1996  0730   MG                               %
%                                                                 %
%%%%%%%%%%%%%%%%%%%%%%%%%%%%%%%%%%%%%%%%%%%%%%%%%%%%%%%%%%%%%%%%%%%
\documentclass[11pt,fleqn]{cernman}
\usepackage{amssymb,amsmath}
\usepackage{cernlib}
\usepackage{html}

% PDF/A packages
\usepackage[pdftex, pdfa, linktoc=none]{hyperref}

% ----------------------------------------------
% Add metadata
\hypersetup{
	pdftitle={Hyperbolic Arcsine - CERN Program Library},
	pdfauthor={K.S. Kölbig},
	pdfsubject={Cern Library Documentation},
	pdfkeywords={ARSINH HYPERBOLIC ARCSIN},
	pdflang={en},
	bookmarksopen=true,
	bookmarksopenlevel=3,
	hypertexnames=false,
	linktocpage=true,
	plainpages=false,
	breaklinks
}

\newcommand{\binomv}[2]{\genfrac{}{}{0pt}{}{#1}{#2}}
\newcommand{\binomg}[2]{\genfrac{\{}{\}}{0pt}{}{#1}{#2}}
\newcommand{\binoms}[2]{\genfrac{[}{]}{0pt}{}{#1}{#2}}
%\usepackage{screen}
\newcommand{\Title}{CERN Program Library}%       Title for document
\renewcommand{\rmdefault}{ptm}
\newcommand{\OBSOLETE}{\makebox[114mm][c]{\textbf{OBSOLETE}}\\}
\newcommand{\POBSOLETE}{\makebox[114mm][c]{\textbf{PARTIALLY OBSOLETE}}\\}
\renewcommand{\Rind}[2]{\texttt{#1} (#2)\Inref{#1}}%Special for shortwrups
\begin{document}

\let\LARGE\large
\let\Large\large

% Here comes the name to run

\include{b102}

\end{document}


\end{document}


\end{document}


\end{document}
