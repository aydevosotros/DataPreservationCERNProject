%\newcount\higzdraft \higzdraft=1

\def\listing#1{
  \parindent=0pt \par
  \begingroup
    \def\par{\leavevmode\endgraf}
    \footnotesize \tt
    \obeylines \obeyspaces \frenchspacing
    \def\do##1{ \catcode`##1=12 } \dospecials
    \vobeyspaces
    \input#1
  \endgroup
}
{\catcode`\ =\active\gdef\vobeyspaces{\catcode`\ \active \let \xobeysp}}
\def\xobeysp{\leavevmode{} }

\documentstyle[12pt]{article}

\begin{document}

\title{Producing \LaTeX{} metafiles with HIGZ}
\author{A.~Nathaniel \\
        Univ. of Munich}
\date{}
\maketitle

The graphics interface package HIGZ~\cite{HIGZ} is now able to write 
metafiles which are ready to be included in \LaTeX{}~\cite{LATEX} documents.
These metafiles make use of the \verb'\picture' environment.
Compared to other possibilities of merging graphics
into documents \LaTeX{} metafiles have a number of advantages:

\begin{itemize}
\item 
It is not necessary to specify the space which has to be reserved for
the picture.
The picture size can be changed at formatting time.
\item
Because \verb'\special' commands are not used
the {\tt dvi} file is fully transportable and can be output on any device 
for which a driver exists.
Documents can be written, formatted, and previewed on workstations
while the {\tt dvi} file can be sent by network to a central server for
printing.
\item
The metafile can eventually be merged into the \LaTeX{} file to keep the 
full document in a single file.
\LaTeX{} metafiles tend to be smaller than Postscript metafiles
for pictures involving many line drawing commands.
\end{itemize}

\section{\LaTeX{} metafile capabilities}

The capabilities of the \verb'\picture' environment are basically
limited to drawing straight horizontal or vertical lines.
Slanted lines do exist but only in a limited number of slopes and 
a minimum length of $\approx 4 {\rm mm}$.
Therefore slanted lines have to be approximated by small steps of
straight lines where the step size should be close to the printer resolution.

The workstation type for \LaTeX{} metafiles is -777.
Coordinates written to the metafile are integral numbers assuming
a grid spacing of $0.1 {\rm mm}$.
Therefore the settings for {\tt XSIZE} and {\tt YSIZE} should approximately
correspond to the final picture size.

\subsection{Line and marker types}

Supported are line types 1 through 4 and marker types 1 through 5.

\subsection{Text fonts}

Apart from software characters the font numbers -1 through~-8 at 
precision~0 can be used.
They map to the \TeX{} fonts
{\rm Roman}, {\em Emphatic}, {\bf Bold}, {\it Italic}, {\sl Slanted},
{\sf Sans Serif}, {\sc Small Caps}, and {\tt Typewriter}, respectively.

\TeX{} fonts look much nicer and are faster to generate than
software characters.
But the disadvantage is that they are available in horizontal orientation only
and the character size does not scale with the picture size.

When using \TeX{} fonts the {\tt IGTEXT} control characters 
``\verb'<>[]"#^?!''' are interpreted to obtain superscripts, greek letters,
and other special characters.
If a text string contains a ``\verb'\''' or ``\verb'{''' the remaining part
is written verbatim into the metafile.
This allows to use \TeX{} formatting commands for elaborate displays.
Of course ``\verb'{''' and ``\verb'}''' must be properly matched.

The whole text is typeset in math mode which does not allow a change of
fontsize in between.
In order to format a formula at a larger size the formula text must be
preceded by ``\verb'{}$\large$'''.

\section{Configuration parameters}

To some extent the apperance of a picture can be changed at formatting time
by defining configuration parameters which have the following default values:

\begin{verbatim}
\newdimen\higzunit \higzunit=0pt
\newcount\higzstep \higzstep=2
\newcount\higzdraft \higzdraft=0
\end{verbatim}

\subsection{\tt higzunit}

By default the picture is automatically scaled to fill the full page width.
The picture size can be changed by setting \verb'\higzunit'
it to the wanted grid spacing, e.g. to get the true {\tt XSIZE}

\begin{verbatim}
\newdimen\higzunit \higzunit=0.1mm
\end{verbatim}

\subsection{\tt higzstep}

Slanted lines are approximated by straight lines along the major axis.
The step size along the minor axis is 
\verb'\higzstep'$\times$\verb'\unitlength'.
By setting \verb'\higzstep=1' curves will look smoother 
but if line segments come too close to the printer resolution the 
{\tt dvi} driver may choose not to display them.
A larger value will result in faster formatting requiring less \TeX{} memory.

\subsection{\tt higzdraft}

Setting \verb'\higzdraft=1' replaces the actual picture by an empty box 
of the same size to save formatting time during drafting.

\section{Transfering {\tt dvi} files to CERNVM}

The {\tt dvi} driver {\tt dvi38pp} for {\tt IBM3812} printers tests whether
the last record of the {\tt dvi} file is padded with {\tt 0xDF} characters.
This condition is normally not met by files which were formatted on another
computer and a small program is needed to patch it up:

\hbox{}
\listing{dvipad.fortran}
\hbox{}

Then a {\tt dvi} file can be fetched over the network and printed by the
following command sequence:

\begin{verbatim}
tcpipibm ftp host
bin f 1024
get file.tex
quit
dvipad file
xprint file list38pp
\end{verbatim}

\section{{\tt tex} running out of memory}

Here are some hints what to do if you cannot format the document
because \TeX{} runs out of memory.

\begin{enumerate}
\item 
Convince your system administrator to build a bigger {\tt tex} program.
\item 
Find another computer system which runs a bigger {\tt tex} program.
\item 
Use \TeX{} fonts instead of software characters.
\item 
Place figures such that each picture gets shipped out before the next one
is formatted.
\item 
Fill areas with horizontal or vertical lines only.
\item 
Do not use fill areas.
\item 
Draw curves with dotted instead of solid lines.
\item 
Do not use marker types 3 or 5.
\item 
Use a larger value for \verb'\higzstep'.
\item 
Make the picture less complex.
\item
You lose.
\end{enumerate}

\begin{thebibliography}{1}
\bibitem{HIGZ}
R.Bock, R.Brun, O.Couet, R.Nierhaus, N.Cremel-Somon, C.Vandoni, P.Zanarini,
{\em HIGZ - High level Interface to Graphics and Zebra},
CERN program library Q120
\bibitem{LATEX}
L.Lamport,
{\em \LaTeX{} user's guide and reference manual},
Addison-Wesley Publishing Company
\bibitem{PAW}
R.Brun, O.Couet, N.Cremel-Somon, P.Zanarini,
{\em PAW users guide},
CERN program library Q121
\end{thebibliography}

\appendix

\section{Examples}

Here are some examples of \LaTeX{} metafiles produced with PAW~\cite{PAW}.
The lego plot needed 250000~words of {\tt tex}~memory at \verb'\higzstep=4'.
The formatting of this document took 150~seconds of CPU time
on a DECstation~3100.

\hbox{} 
\listing{examples.kumac}

\begin{figure}
\include{example1}
\listing{example1.kumac}
\end{figure}
\clearpage

\begin{figure}
\include{example2}
\listing{example2.kumac}
\end{figure}
\clearpage

\begin{figure}
\include{example3}
\listing{example3.kumac}
\end{figure}
\clearpage

\begin{figure}
\newcount\higzstep \higzstep=4
\include{example4}
\listing{example4.kumac}
\end{figure}

\end{document}
