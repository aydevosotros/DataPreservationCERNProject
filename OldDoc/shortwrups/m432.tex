%24 mar 94 ksk
\Version{CHPACK}                        \Routid{M432}
\Keywords{SUBROUTINE PACKAGE}
\Keywords{CKRACK CCOPYL CCOPYR}
\Keywords{CFILL CLEFT CRIGHT CSQMBL CSQMCH CLTOU CUTOL}
\Keywords{CSETDI CSETHI CSETOI CTRANS}
\Keywords{CCOPIV CSETVI CSETVM CCOSUB CENVIR}
\Keywords{ICDECI ICHEXI ICOCTI NCDECI NCHEXI NCOCTI}
\Keywords{ICFIND ICFILA ICEQU ICINQ ICINQL ICINCU ICNUMU}
\Keywords{ICFMUL ICFNBL ICLOC ICLOCL ICLOCU ICLUNS ICNEXT}
\Keywords{ICNTH ICNTHL ICNTHU ICNUM ICNUMA ICTYPE}
\Keywords{LNBLNK SLATE}
\Keywords{CHARACTER MANIPULATION STRING-PARSING HANDLING PARSE}
\Author{J. Zoll}                        \Library{KERNLIB}
\Submitter{}                            \Submitted{02.06.1989}
\Language{Fortran}                \Revised{01.04.1994}
\Cernhead {Utility Routines for Character String Parsing and
Construction}
The routines of this package analyse and manipulate Fortran
{\tt CHARACTER} strings.
\Structure
{\tt SUBROUTINE} and {\tt FUNCTION} subprograms \\
User Entry Names:
\begin{htmlonly}
\begin{tabular}{llllllll}
\end{htmlonly}
\begin{latexonly}
\begin{tabular}[t]{l*{7}{@{\hspace{4pt}}l}}
\end{latexonly}
\Rdef{CKRACK}, & \Rdef{CCOPYL}, & \Rdef{CCOPYR}, & \Rdef{CCOPIV}, &
\Rdef{CCOSUB}, & \Rdef{CENVIR}, & \Rdef{CFILL},  & \Rdef{CLEFT},  \\
\Rdef{CRIGHT}, & \Rdef{CSQMBL}, & \Rdef{CSQMCH}, & \Rdef{CLTOU},  &
\Rdef{CUTOL},  & \Rdef{CSETDI}, & \Rdef{CSETHI}, & \Rdef{CSETOI}, \\
\Rdef{CSETVI}, & \Rdef{CSETVM}, & \Rdef{CTRANS}, & \Rdef{ICDECI}, &
\Rdef{ICHEXI}, & \Rdef{ICOCTI}, & \Rdef{ICEQU},  & \Rdef{ICFIND}, \\
\Rdef{ICFILA}, & \Rdef{ICFMUL}, & \Rdef{ICFNBL}, & \Rdef{ICLOC},  &
\Rdef{ICLOCL}, & \Rdef{ICLOCU}, & \Rdef{ICLUNS}, & \Rdef{ICNEXT}, \\
\Rdef{ICNTH},  & \Rdef{ICNTHL}, & \Rdef{ICNTHU}, & \Rdef{ICINQ},  &
\Rdef{ICINQL}, & \Rdef{ICINQU}, & \Rdef{ICNUM},  & \Rdef{ICNUMA}, \\
\Rdef{ICNUMU}, & \Rdef{ICTYPE}, & \Rdef{LNBLNK}, & \Rdef{NCDECI}, &
\Rdef{NCHEXI}, & \Rdef{NCOCTI}  \\
\end{tabular} \\
{\tt COMMON} Block Names and Lengths: {\tt /SLATE/ 40}\\[3mm]
{\bf Summary:}
\begin{tabular}[t]{l@{\qquad}l}
{\tt CALL CKRACK}  & Read integer or real number from character\\
{\tt CALL CCOPYL}  & Copy string, left shift allowed if overlap\\
{\tt CALL CCOPYR}  & Copy string, right shift allowed if overlap\\
{\tt CALL CCOPIV}  & Copy string, with characters front-to-back\\
{\tt CALL CCOSUB}  & Copy string, replacing a token by text\\
{\tt CALL CENVIR}  & Copy string, replacing environment variables\\
{\tt CALL CFILL}   & Fill\\
{\tt CALL CLEFT}   & Left justify\\
{\tt CALL CRIGHT}  & Right justify\\
{\tt CALL CSQMBL}  & Squeeze multiple blanks\\
{\tt CALL CSQMCH}  & Squeeze multiple character\\
{\tt CALL CLTOU}   & Convert low to up\\
{\tt CALL CUTOL}   & Convert up to low\\
{\tt CALL CSETDI}  & Set decimal integer to character\\
{\tt CALL CSETHI}  & Set hexadecimal integer to character\\
{\tt CALL CSETOI}  & Set octal integer to character\\
{\tt CALL CSETVI}  & Set a vector of integers to character\\
{\tt CALL CSETVM}  & Set a series of generated integers to character\\
{\tt CALL CTRANS}  & Character translation\\
{\tt IX = ICDECI}  & Read decimal integer from character\\
{\tt IX = ICHEXI}  & Read hexadecimal integer from character\\
{\tt IX = ICOCTI}  & Read octal integer from character\\
{\tt IX = ICEQU}   & Compare two strings for equality\\
{\tt JX = ICFIND}  & Find first occurrence, single\\
{\tt JX = ICFILA}  & Find last occurrence, single\\
{\tt JX = ICFMUL}  & Find first occurrence, multiple\\
{\tt JX = ICFNBL}  & Find first non-blank\\
\end{tabular}
\newpage
{\bf continue:}
\begin{tabular}[t]{l@{\qquad}l}
{\tt JX = ICLOC}   & Locate case sensitive\\
{\tt JX = ICLOCL}  & Locate case insensitive, up to low\\
{\tt JX = ICLOCU}  & Locate case insensitive, low to up\\
{\tt JX = ICLUNS}  & Locate unseen characters\\
{\tt JX = ICNEXT}  & Delimit next word\\
{\tt JX = ICNTH}   & Identify choice case sensitive\\
{\tt JX = ICNTHL}  & Identify choice case insensitive, up to low\\
{\tt JX = ICNTHU}  & Identify choice case insensitive, low to up\\
{\tt JX = ICINQ}   & Inquire presence in a list, case sensitive\\
{\tt JX = ICINQL}  & Inquire presence in a list, case insensitive,
                     up to low\\
{\tt JX = ICINQU}  & Inquire presence in a list, case insensitive,
                     low to up\\
{\tt JX = ICNUM}   & Verify numeric\\
{\tt JX = ICNUMA}  & Verify alpha-numeric\\
{\tt JX = ICNUMU}  & Verify alpha-numeric or underscore\\
{\tt JX = ICTYPE}  & Identify type\\
{\tt NX = LNBLNK}  & Find last non-blank character\\
{\tt IX = NCDECI}  & Read decimal integer from character\\
{\tt IX = NCHEXI}  & Read hexadecimal integer from character\\
{\tt IX = NCOCTI}  & Read octal integer from character\\
\end{tabular}
\Usage
{\bf General Remarks:} \\[2mm]
For what follows, let the {\tt CHARACTER} variable {\tt LINE} contain a
string of $n$ characters assuming the following declaration:
\begin{verbatim}
       CHARACTER LINE*(n),COL(n)*1
       EQUIVALENCE(LINE,COL)
\end{verbatim}
thus {\tt COL(j)}  is the {\tt j}-th character {\tt LINE(j:j)}.
A sub-string of {\tt LINE} is specified by {\tt JL} and {\tt JR},
where {\tt COL(JL)} is the first or left-most, and {\tt COL(JR)} is the
last or right-most character.
\begin{verbatim}
    COMMON /SLATE/ ND,NE,NF,NG,NUM(2),DUMMY(34)
\end{verbatim}
 returns certain search parameters, which are set by some of the
 routines.
\par
The types of all variables and functions follow from the Fortran
default typing convention, except that {\tt LINE, COL}, and
variables starting with the letters {\tt CH} are
of type {\tt CHARACTER}.  \\[3mm]
{\bf Convention}\\[2mm]
Typing rules for data to be interpreted by {\tt CKRACK}:
\begin{DLtt}{1234567890}
\item[Binary:] String of zeros or ones prefixed by {\tt \#B} or
{\tt \#b}.
\item[Octal:] String of octal digits prefixed by {\tt \#0} or {\tt \#O}
or {\tt \#o}.
\item[Hex:] String of hexadecimal digits prefixed by {\tt \#X} or
{\tt \#x}.
\item[Integer:] String of decimal digits optionally prefixed by {\tt +}
or {\tt -}.
\item[Real:] {\tt [+|-] [int] [.] [fract] [E] [+|-] [exp]}\\
\parbox[t]{142mm}{
{\tt int, fract, exp} are strings of decimal digits, either the decimal
dot or the letter {\tt E} (or {\tt e}) must be present.}
\item[Double:] {\tt [+|-] [int] [.] [fract] D [+|-] [exp]}\\
the letter {\tt D} (or {\tt d}) must be present.
\end{DLtt}
\newpage
{\bf Read integer or real number from character:}
\begin{verbatim}
    CALL CKRACK(LINE,JL,JR,IFLD)
\end{verbatim}
reads the number whose character representation starts with
the first non-blank character at or after {\tt COL(JL)}
and ends at {\tt COL(JR)} or at the first blank after the number
({\it normal termination}),
or at the first character after the number which cannot be part of it
({\it special termination}).
\par
{\tt CKRACK} detects the type of the number (bit-pattern, integer,
real single, real double) from its representation.
The typing rules for data to be interpreted by {\tt CKRACK} are
given in the note on the previous page.
\par
The number read is returned in {\tt /SLATE/} in {\tt NUM(1)} or
{\tt ANUM(1)} or {\tt DNUM}, for which one will need:
\begin{verbatim}
    REAL ANUM(2)
    DOUBLE PRECISION DNUM
    EQUIVALENCE (ANUM(1),NUM(1)),(DNUM,NUM(1))
\end{verbatim}
\par
The flag in the last parameter is normally given as zero;
$\mathtt{IFLD > 0}$ demands that single-precision real numbers be handled
and returned as double precision numbers;
$\mathtt{IFLD < 0}$ demands that double-precision numbers be
returned in single precision.
\par
Apart from {\tt NUM}, the following parameters are returned in
{\tt /SLATE/}:
\begin{DLtt}{12345678}
\item [ND] Number of numeric digits seen.
\item[COL(NE)] Terminating character in the field; $\mathtt{NE=JR+1}$ if
terminated by end-of-field.
\item [NF] Type of the number read:\\
$\mathtt{<0:}$ error code for bad data;\\
$\mathtt{=0:}$ the whole field is blank;\\
$\mathtt{=1:}$ bit pattern (binary, octal, or hexadecimal);\\
$\mathtt{=2:}$ integer\\
$\mathtt{=3:}$ single precision real;\\
$\mathtt{=4:}$ double precision real.
\item[NG] $\mathtt{=0}$ for normal termination; special termination
otherwise.
\end{DLtt}
{\bf Copy string, left shift allowed if overlap:}
\begin{verbatim}
    CALL CCOPYL (CHFROM,CHTO,NCH)
\end{verbatim}
copies {\tt NCH} characters from {\tt CHFROM(1:NCH)} to
{\tt CHTO(1:NCH)}; the characters are copied in order, thus the end of
the target may overlap the beginning of the source. \\[2mm]
{\bf Copy string, right shift allowed if overlap:}
\begin{verbatim}
    CALL CCOPYR (CHFROM,CHTO,NCH)
\end{verbatim}
copies {\tt NCH} characters from {\tt CHFROM(1:NCH)} to
{\tt CHTO(1:NCH)}; the characters are copied in reverse order,
thus the beginning of the target may overlap the end of the source.
These two routines are useful to copy strings from or into a
very large array of type {\tt CHARACTER*1}. \\[2mm]
{\bf Copy string, with characters front-to-back:}
\begin{verbatim}
    CALL CCOPIV(CHFROM,CHTO,NCH)
\end{verbatim}
copies {\tt NCH} characters from {\tt CHFROM(1:NCH)} to
{\tt CHTO(1:NCH)} inverting the order of the characters such that the
last becomes the first, etc. \\[2mm]
\newpage
{\bf Copy string, replacing a token by text:}
\begin{verbatim}
    CALL CCOSUB(CHFROM,NFR,LINE,JL,JR,CHTOKEN,CHSUB)
\end{verbatim}
copies {\tt CHFROM(1:NFR)} to {\tt LINE} starting at {\tt COL(JL)}
and not going beyond {\tt COL(JR)}, substituting each occurrence
of {\tt CHTOKEN} by {\tt CHSUB}.
\par
The following parameters are returned in {\tt /SLATE/}:
\begin{DLtt}{12345678}
\item[ND] Number of characters stored;
\item[COL(NE)] is the first character after the last stored;
\item[NF] Non-zero if {\tt LINE(JL:JR)} too small to receive
the complete copy;
\item[NG] Zero if no substitution was done, i.e.
{\tt CHTOKEN} did not occur.
\end{DLtt}
{\bf Copy string, replacing environment variables:}
\begin{verbatim}
    CALL CENVIR(CHFROM,NFR,LINE,JL,JR,IFLAG)
\end{verbatim}
copies {\tt CHFROM(1:NFR)} to {\tt LINE} starting at {\tt COL(JL)}
and not going beyond {\tt COL(JR)}, substituting each occurrence of
{\tt \$\{name\}} by the value of the
environment variable {\tt "name"}
obtained by calling {\tt GETENVF} (Z 265);
on machines running UNIX the form {\tt "\$name"} is also recognized.
The handling of undefined environment variables is defined by
{\tt IFLAG}: if zero the string {\tt \$\{name\}} is skipped from the
copy; if
non-zero the string is copied through as is.
\par
The following parameters are returned in {\tt /SLATE/}:
\begin{DLtt}{12345678}
\item[ND] Number of characters stored;
\item[COL(NE)] is the first character after the last stored;
\item[NF] Bit 1 is set if undefined env/v have been encountered;
\item[]   Bit 2 is set if syntax error (missing closing bracket);
\item[]   Bit 3 is set if {\tt LINE(JL:JR)} is too small to
receive the copy;
\item[NG] Zero if no substitution was done.
\end{DLtt}
{\bf Fill:}
\begin{verbatim}
    CALL CFILL(CHI,LINE,JL,JR)
\end{verbatim}
fills {\tt LINE(JL:JR)} with as many copies of {\tt CHI} as possible;
if $\mathtt{JL+1-JR}$ is not a multiple of {\tt LEN(CHI)} as many
characters of {\tt CHI} as necessary to fill up to {\tt JR} will be
taken for the last copy. \\[2mm]
{\bf Left justify:}
\begin{verbatim}
    CALL CLEFT(LINE,JL,JR)
\end{verbatim}
left-justifies {\tt LINE(JL:JR)} squeezing blanks to the right.
\begin{DLtt}{12345678}
\item [ND] Number of non-blank characters in the field.
\item[COL(NE)] First blank character after left-justifying
(or $\mathtt{NE=JR+1}$ if there are no blanks).
\end{DLtt}
{\bf Right justify:}
\begin{verbatim}
    CALL CRIGHT(LINE,JL,JR)
\end{verbatim}
 right-justifies {\tt LINE(JL:JR)} squeezing blanks to the left.
\begin{DLtt}{12345678}
\item [ND] Number of non-blank characters in the field.
\item[COL(NE)] Last blank character after right-justifying
(or $\mathtt{NE=JL-1}$ if there are no blanks).
\end{DLtt}
\newpage
{\bf Squeeze multiple blanks:}
\begin{verbatim}
    CALL CSQMBL(LINE,JL,JR)
\end{verbatim}
left-justifies {\tt LINE(JL:JR)} replacing any string of several
consecutive blanks by a single blank.
\begin{DLtt}{12345678}
\item[ND] Number of characters retained (vacated trailing characters,
if any, are blanked).
\item[COL(NE)] First blank character after (or $\mathtt{NE=JR+1}$ if
there are no multiple blanks).
\end{DLtt}
{\bf Squeeze multiple characters:}
\begin{verbatim}
    CALL CSQMCH(CHIS,LINE,JL,JR)
\end{verbatim}
left-justifies {\tt LINE(JL:JR)} reducing any multiple occurrence
of the character {\tt CHIS} to this character just once.
{\tt CHIS} is of type {\tt CHARACTER*1}.
\begin{DLtt}{12345678}
\item[ND] Number of characters retained (vacated trailing characters,
if any, are blanked).
\item[COL(NE)] First character after the squeezed string
(or $\mathtt{NE=JR+1}$ if there are no multiple occurrences).
\end{DLtt}
{\bf Convert low to up:}
\begin{verbatim}
    CALL CLTOU(LINE(JL:JR))
\end{verbatim}
converts lower case letters in {\tt LINE(JL:JR)} to upper case. \\[2mm]
{\bf Convert up to low:}
\begin{verbatim}
    CALL CUTOL(LINE(JL:JR))
\end{verbatim}
 converts upper case letters in {\tt LINE(JL:JR)} to lower case. \\[2mm]
{\bf Set decimal integer to character:}
\begin{verbatim}
    CALL CSETDI(INT,LINE,JL,JR)
\end{verbatim}
writes the integer {\tt INT} into {\tt LINE(JL:JR)} right-justified.
If {\tt INT} is too large, the most significant characters are lost.
Unused positions are not cleared to blank, so that they may be
pre-loaded with default characters, such as leading zeros.
(One normally clears the whole of {\tt LINE} initially with
$\mathtt{LINE}=${\tt ' '}, one could clear the substring with
{\tt LINE(JL:JR)}$=${\tt ' '} or preset it before calling
{\tt CSETDI}).
\begin{DLtt}{1234567890}
\item[ND] Number of digits which have been set.
\item[COL(NE+1)] Most significant digit set.
\item[COL(NF+1)] Most significant character set. $\mathtt{NF=NE}$ if
{\tt INT} is positive, $\mathtt{NF=NE-1}$ if {\tt INT} is negative and no
overflow.
\item[NG] $\mathtt{=0}$ normally, non-zero if field too small.
\end{DLtt}
{\bf Set hexadecimal integer to character:}
\begin{verbatim}
    CALL CSETHI(INT,LINE,JL,JR)
\end{verbatim}
acts like {\tt CSETDI}, except that the hexadecimal rather than
the decimal representation of {\tt INT} is stored.
\\[2mm]
{\bf Set octal integer to character:}
\begin{verbatim}
    CALL CSETOI(INT,LINE,JL,JR)
\end{verbatim}
acts like {\tt CSETDI}, except that the octal rather than
the decimal representation of {\tt INT} is stored. \\[2mm]
\newpage
{\bf Set a vector of integers to character:}
\begin{verbatim}
    CALL CSETVI(NI,INTV,NBIAS,LINE,JL,JR,NCOL,IFLSQ)
\end{verbatim}
sets the {\tt NI} integers $\mathtt{INTV(J)+NBIAS}$ into
{\tt LINE(JL:JR)} in decimal representation,
every {\tt NCOL} columns, each right-justified
within its field of $\mathtt{NCOL-1}$ columns; squeeze multiple blanks
to single blanks in the resulting {\tt LINE(JL:JR)} if {\tt IFLSQ}
non-zero. Like the other {\tt CSETxx} routines, this routine does not
clear unused positions to blank.
\begin{DLtt}{1234567}
\item[COL(NE)] Last character of the last integer stored.
\item[NG] $\mathtt{=0}$ normally, $\mathtt{=N>0}$ if there is
not enough room to store {\tt INTV(N)}.
\end{DLtt}
{\bf Set a series of generated integers to character:}
\begin{verbatim}
    CALL CSETVM(NI,INC,IGO,LINE,JL,JR,NCOL,IFLSQ)
\end{verbatim}
acts like {\tt CSETVI}, but the  {\tt NI} integers are
$\mathtt{IGO}+n*\mathtt{INC}$, $n=0,1,\ldots,\mathtt{NI-1}$. \\[2mm]
{\bf Character translation:}
\begin{verbatim}
    CALL CTRANS(CHO,CHN,LINE,JL,JR)
\end{verbatim}
replaces each occurrence in {\tt LINE(JL:JR)} of the character {\tt CHO}
by the character {\tt CHN}. {\tt CHO} and {\tt CHN} are of type
{\tt CHARACTER*1}.  \\[2mm]
{\bf Read decimal integer from character:}
\begin{verbatim}
    IX = ICDECI(LINE,JL,JR)
\end{verbatim}
reads the decimal integer whose character representation starts
at {\tt COL(JL)} and stops on the first non-numeric character or at
{\tt COL(JR+1)}, returning its value in {\tt IX}. Leading blanks are
ignored, a leading minus or plus sign is recognized. Note that a blank
after the number, or after {\tt '+'} or {\tt '-'}, is taken as
terminator.
\begin{DLtt}{12345678}
\item[ND] Number of digits read ({\tt '-'} or {\tt '+'} do not count).
\item[COL(NE)] Terminating character in the field; $\mathtt{NE=JR+1}$ if
pure numeric or if the whole field is blank (in which case
$\mathtt{ND=0}$).
\item[NG] $\mathtt{=0}$ if the number is terminated by 'blank'
or by end-of-field, non-zero otherwise.
\end{DLtt}
{\bf Read hexadecimal integer from character:}
\begin{verbatim}
    IX = ICHEXI(LINE,JL,JR)
\end{verbatim}
acts like {\tt ICDECI},
but reads a hexadecimal rather than a decimal representation. \\[2mm]
{\bf Read octal integer from character:}
\begin{verbatim}
    IX = ICOCTI(LINE,JL,JR)
\end{verbatim}
acts like {\tt ICDECI},
but reads an octal rather than a decimal representation. \\[2mm]
{\bf Compare two strings for equality:}
\begin{verbatim}
    IX = ICEQU(CHA,CHB,N)
\end{verbatim}
checks that {\tt CHA(1:N)} and {\tt CHB(1:N)} are identical
and returns zero if so, otherwise the ordinal number of the first
non-matching character is returned.
\par
Note: this and many other routines of this package are handy when
manipulating text stored in an area declared with
{\tt CHARACTER TEXT(big)*1}, which will explain some of
the maybe unexpected calling sequences.\\[2mm]
\newpage
{\bf Find first occurrence, single:}
\begin{verbatim}
    JX = ICFIND(CHIS,LINE,JL,JR)
\end{verbatim}
returns in {\tt JX} the position in {\tt LINE} of the first occurrence
of the single character {\tt CHIS} in {\tt LINE(JL:JR)}.
\begin{DLtt}{1234}
\item[JX] $\mathtt{=JR+1}$ if {\tt CHIS} is not contained in
{\tt LINE(JL:JR)},
or if $\mathtt{JL > JR}$.
\item[NG] $\mathtt{=0}$ if not found, $\mathtt{=JX}$ otherwise.
\end{DLtt}
{\bf Find last occurrence, single:}
\begin{verbatim}
    JX = ICFILA(CHIS,LINE,JL,JR)
\end{verbatim}
returns in {\tt JX} the position in {\tt LINE} of the last occurrence of
the single character {\tt CHIS} in {\tt LINE(JL:JR)}.
\begin{DLtt}{1234}
\item[JX] $\mathtt{=JR+1}$ if {\tt CHIS} is not contained in
{\tt LINE(JL:JR)},
or if $\mathtt{JL > JR}$.
\item[NG] $\mathtt{=0}$ if not found, $\mathtt{=JX}$ otherwise.
\end{DLtt}
{\bf Find first occurrence, multiple:}
\begin{verbatim}
    JX = ICFMUL(CHI,LINE,JL,JR)
\end{verbatim}
returns in {\tt JX} the position in {\tt LINE} of the first occurrence
in {\tt LINE(JL:JR)} of any one of the characters {\tt CHI(j:j)},
where $\mathtt{j = 1,2,\ldots,n=LEN(CHI)}$.
\begin{DLtt}{1234}
\item[JX] $\mathtt{=JR+1}$ if none of {\tt CHI} is found in
{\tt LINE(JL:JR)},
or if $\mathtt{JL > JR}$.
\item[ND] $=j$, i.e. {\tt COL(JX)} is {\tt CHI(j:j)} if found.
\item[NG] $\mathtt{=0}$ if not found, $\mathtt{= JX}$ otherwise.
\end{DLtt}
{\bf Find first non-blank:}
\begin{verbatim}
    JX = ICFNBL(LINE,JL,JR)
\end{verbatim}
returns in {\tt JX} the position in {\tt LINE} of the first non-blank
character in {\tt LINE(JL:JR)}.
\begin{DLtt}{1234}
\item[JX] $\mathtt{=JR+1}$ if {\tt LINE(JL:JR)} is all blank,
or if $\mathtt{JL > JR}$.
\item[NG] $\mathtt{=0}$ if all blank, $\mathtt{= JX}$ otherwise.
\end{DLtt}
{\bf Locate, case sensitive:}
\begin{verbatim}
    JX = ICLOC(CHI,NI,LINE,JL,JR)
\end{verbatim}
locates the first occurrence of the complete string {\tt CHI(1:NI)}
within {\tt LINE(JL:JR)}, it returns in {\tt JX} the position in
{\tt LINE} of the first character of the string found.
$\mathtt{JX=0}$ if {\tt CHI} is not contained in {\tt LINE(JL:JR)}.
\\[2mm]
{\bf Locate, case insensitive, up to low:}
\begin{verbatim}
    JX = ICLOCL(CHI,NI,LINE,JL,JR)
\end{verbatim}
acts like {\tt ICLOC}, but upper case characters from {\tt LINE} are
converted to lower case for the comparison. \\[2mm]
{\bf Locate, case insensitive, low to up:}
\begin{verbatim}
    JX = ICLOCU(CHI,NI,LINE,JL,JR)
\end{verbatim}
acts like {\tt ICLOC}, but lower case characters from {\tt LINE} are
converted to upper case for the comparison. \\[2mm]
{\bf Locate unseen characters:}
\begin{verbatim}
    JX = ICLUNS(LINE,JL,JR)
\end{verbatim}
returns in {\tt JX} the position in {\tt LINE} of the first
'unseen' character in {\tt LINE(JL:JR)}, i.e. any character which
will not show on the terminal, except 'blank'.
$\mathtt{JX=0}$ if {\tt LINE(JL:JR)} does not contain unseen
characters. \\[2mm]
\newpage
{\bf Delimit next word:}
\begin{verbatim}
    JX = ICNEXT(LINE,JL,JR)
\end{verbatim}
returns in {\tt JX} the position in {\tt LINE} of the first non-blank
character in {\tt LINE(JL:JR)} and in {\tt NE} the position of the first
blank character after {\tt COL(JX)}, if any.
\begin{DLtt}{1234}
\item[JX] Position of the first character of the 'word'.
\item[NE] Position of the first 'blank' after the 'word' or
$\mathtt{NE=JR+1}$.
\item[ND] Number of characters in the 'word'. \\
$\mathtt{JX=NE=JR+1}$, $\mathtt{ND=0}$  if {\tt LINE(JL:JR)}
is all blank.
\end{DLtt}
{\bf Identify choice, case sensitive:}
\begin{verbatim}
    JX = ICNTH(CHACT,CHPOSS,NPOSS)
\end{verbatim}
compares the character string {\tt CHACT} against the strings
stored in the character array {\tt CHPOSS(NPOSS)},
and returns in {\tt JX} the ordinal number of the first match found,
or $\mathtt{JX=0}$ if no match.
Neither the strings of {\tt CHPOSS} nor of {\tt CHACT}
may contain embedded blanks: the first blank, if any,
is the string terminator.
\par
To allow matching a shortened key-word given in {\tt CHACT}
one may insert (\a`a la VAX) a {\tt '*'} in the text of
{\tt CHPOSS(J)} to mark the separation between the obligatory
and further possible characters;
a second {\tt '*'} may be given to signal that {\tt CHACT}
may have any other characters beyond this point,
this is implied if the string in {\tt CHPOSS(J)} is
not blank terminated. \\
For example:
\begin{verbatim}
    PARAMETER  (NPOSS=6)
    CHARACTER*8 CHPOSS(NPOSS)
    DATA CHPOSS /'del*ete ', 'add     ', 'adb*efor',
   +             'rep*lace', 'ch*ange ', 'c*ol*   '/
\end{verbatim}
Calling the above with the following strings will give these results:
\begin{verbatim}
     CHACT = 'add'       JX = 2   exact match
             'delete'         1   exact match
             'del'            1   short match
             'del  '          1
             'delphi'         0   wrong optional characters
             'deleted'        0   CHPOSS(1) is terminated
             'replaced'       4   CHPOSS(4) is not terminated
             'chan'           5   short match
             'channel'        0   wrong optional characters
             'c'              6   short match
             'columns'        6   abritrary trailing characters allowed
             'cols'           6
\end{verbatim}
{\bf Identify choice, case insensitive, up to low:}
\begin{verbatim}
    JX = ICNTHL(CHACT,CHPOSS,NPOSS)
\end{verbatim}
acts like {\tt ICNTH} converting upper case characters
from {\tt CHACT} to lower case for the comparison,
hence the {\tt CHPOSS} array must be given in lower case.
\\[2mm]
{\bf Identify choice, case insensitive, low to up:}
\begin{verbatim}
    JX = ICNTHU(CHACT,CHPOSS,NPOSS)
\end{verbatim}
acts like {\tt ICNTH} converting lower case characters
from {\tt CHACT} to upper case for the comparison,
hence the {\tt CHPOSS} array must be given in upper case. \\[2mm]
\newpage
{\bf Inquire presence in a list, case sensitive:}
\begin{verbatim}
    JX = ICINQ(CHLOOK,CHHAVE,NHAVE)
\end{verbatim}
like {\tt ICNTH} this compares the character string {\tt CHLOOK}
against the strings stored in the character array {\tt CHHAVE(NHAVE)},
and returns in {\tt JX} the ordinal number of the first match found,
or $\mathtt{JX=0}$ if no match.
Again, neither the strings of {\tt CHHAVE} nor of {\tt CHLOOK}
may contain embedded blanks: the first blank, if any,
is the string terminator.
\par
As opposed to {\tt ICNTH}, a {\tt '*'} may be given in {\tt CHLOOK},
but not in {\tt CHHAVE(J)}, to allow wild-card checking on the presence
of a string in the list of {\tt CHHAVE(J)}.
The {\tt '*'} divides the string into the characters which
must be present in the looked-for object of {\tt CHHAVE(J)},
and additional restricting characters which can be absent,
but if present they must be right.
Again a second {\tt '*'} can be given in {\tt CHLOOK},
but this is not useful, since any characters beyond the
string terminator both in {\tt CHLOOK} and in {\tt CHHAVE(J)}
are assumed to be allowed anyway, unlike as with {\tt ICNTH}.
\par
For example:
\begin{verbatim}
    PARAMETER  (NHAVE=7)
    CHARACTER*8 CHHAVE(NHAVE)
    DATA CHHAVE /'apo     ', 'apol    ', 'apollo  ', 'irs6000 ',
   +             'decra1  ', 'decra2  ', 'decra3  '/
\end{verbatim}
Calling the above with the following strings will give these results:
\begin{verbatim}
    CHLOOK = 'apo'       JX = 1
             'apo*'           1
             'ap*ollo'        1
             'ap*'            1
             'ap'             0
             'apol'           2
             'apoll'          0
             'apoll*'         3
             'ir*s60'         4
             'ir*s70'         0
             'dec*'           5
             'dec*ra'         5
             'dec*ra*'        5
             'dec*ra3'        7
\end{verbatim}
In spite of the similarity, the operations of {\tt ICINQ}
and {\tt ICNTH} serve really very different functions:
\par
With {\tt ICNTH} we have a key word {\tt CHACT} which we
try to identify; {\tt CHPOSS(N)} is most likely a fixed
table built into the program which gives the possible
key words and allowed abbreviations \`a la VAX.
The return value from {\tt ICNTH} tells us which key word  we have.
\par
With {\tt ICINQ} we inspect a table {\tt CHHAVE(N)}, which
most likely has been built up at execution time, to see
whether it contains an object according to the specifications
given in {\tt CHLOOK}.
The interesting thing about the return value from {\tt ICINQ}
is mainly whether it is zero or not, the position of the
found object in the table is of secondary importance.\\[2mm]
{\bf Inquire presence in a list, case insensitive, up to low:}
\begin{verbatim}
    JX = ICINQL(CHLOOK,CHHAVE,NHAVE)
\end{verbatim}
acts like {\tt ICINQ} converting upper case characters from {\tt CHLOOK}
to lower case for the comparison, hence {\tt CHHAVE} must be held
in lower case. \\[2mm]
\newpage
{\bf Inquire presence in a list, case insensitive, low to up:}
\begin{verbatim}
    JX = ICINQU(CHLOOK,CHHAVE,NHAVE)
\end{verbatim}
acts like {\tt ICINQ} converting lower case characters from {\tt CHLOOK}
to upper case for the comparison, hence {\tt CHHAVE} must be held
in upper case. \\[2mm]
{\bf Verify numeric:}
\begin{verbatim}
    JX = ICNUM(LINE,JL,JR)
\end{verbatim}
returns in {\tt JX} the position in {\tt LINE} of the first non-numeric
character in {\tt LINE(JL:JR)};  blanks are ignored. Note that
{\tt '+', '-'} or {\tt '.'} are not considered numeric.
\begin{DLtt}{1234}
\item [JX] $\mathtt{=JR+1}$ if {\tt LINE(JL:JR)} is all numeric.
\item [ND] Number of digits seen in {\tt LINE(JL:JX-1)}.
\item [NG] $\mathtt{=0}$ if all numeric, $\mathtt{=JX}$ otherwise.
\end{DLtt}
{\bf Verify alpha-numeric:}
\begin{verbatim}
    JX = ICNUMA(LINE,JL,JR)
\end{verbatim}
returns in {\tt JX} the position in {\tt LINE} of the first
non-alphanumeric character in {\tt LINE(JL:JR)}; blanks are ignored.
Note that {\tt '+', '-'} or {\tt '.'} are not considered alpha-numeric.
\begin{DLtt}{1234}
\item [JX] $\mathtt{=JR+1}$ if {\tt LINE(JL:JR)} is all alpha-numeric.
\item[ND] Number of alpha-numeric characters seen in
{\tt LINE(JL:JX-1)}.
\item[NE] Position of the first numeric character, $\mathtt{=0}$ if none.
\item[NF] Position of the first alphabetic character, $\mathtt{=0}$ if
none.
\item[NG] $\mathtt{=0}$ if all alpha-numeric, $\mathtt{= JX}$ otherwise.
\end{DLtt}
{\bf Verify alpha-numeric or underscore:}
\begin{verbatim}
    JX = ICNUMU(LINE,JL,JR)
\end{verbatim}
acts like {\tt ICNUMA}, but the character "underscore" is considered
alphabetic. \\[2mm]
{\bf Identify type:}
\begin{verbatim}
    JX = ICTYPE(CHIS)
\end{verbatim}
returns in {\tt JX} the type of the single character {\tt CHIS}:
\begin{DLtt}{1234}
\item[JX] $\mathtt{= 0:}$ {\it Unseen}, i.e. a character not showing on
an ASCII terminal. \\
$\mathtt{= 1:}$ Anything else. \\
$\mathtt{= 2:}$ Numeric character. \\
$\mathtt{= 3:}$ Lower case character. \\
$\mathtt{= 4:}$ Upper case character.
\end{DLtt}
{\bf Find last non-blank character:}
\begin{verbatim}
    NX = LNBLNK(CHV)
\end{verbatim}
returns the non-blank length of the string in {\tt CHV(1:LEN(CHV))},
i.e. the characters {\tt NX+1} to {\tt LEN(CHV)} are all blank.
(Note that this is an intrinsic function of several compilers.)
If there are many trailing blanks the routine {\tt LENOCC}
of {\tt M507} is faster; depending on the machine the break-even
point with {\tt LENOCC} is around 25 trailing blanks. \\[2mm]
\newpage
{\bf Read decimal integer from character:}
\begin{verbatim}
    IX = NCDECI(CHTEXT)
\end{verbatim}
acts like {\tt ICDECI},
with $\mathtt{JL=1}$ and $\mathtt{JR=LEN(CHTEXT)}$. \\[2mm]
{\bf Read hexadecimal integer from character:}
\begin{verbatim}
    IX = NCHEXI(CHTEXT)
\end{verbatim}
acts like {\tt ICHEXI},
with $\mathtt{JL=1}$ and $\mathtt{JR=LEN(CHTEXT)}$. \\[2mm]
{\bf Read octal integer from character:}
\begin{verbatim}
    IX = NCOCTI(CHTEXT)
\end{verbatim}
acts like {\tt ICOCTI},
with $\mathtt{JL=1}$ and $\mathtt{JR=LEN(CHTEXT)}$. \\[2mm]
$\bullet$
