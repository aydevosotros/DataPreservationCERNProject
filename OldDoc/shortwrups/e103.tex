\Version {AMAXMU}                         \Routid{E103}
\Keywords{ABSOLUTE LARGE NUMBER SCATTER VECTOR}
\Author{J. Zoll}                          \Library{KERNLIB}
\Submitter{C. Letertre}                   \Submitted{01.09.1969}
\Language{Fortran}                   %\Revised{}
\Cernhead {Largest Absolute Number in Scattered Vector}
{\tt AMAXMU} looks for the largest absolute value in a scattered
vector of real numbers.
\Structure
{\tt FUNCTION} subprogram  \\
User Entry Names: \Rdef{AMAXMU}
\Usage
In any arithmetic expression,
\begin{center}
{\tt AMAXMU(A,IDO,IW,NA)}
\end{center}
is set to the largest absolute value of numbers in any of the subsets of
{\tt A} as specified by {\tt IDO}, {\tt IW} and {\tt NA}.
\begin{DLtt}{123456}
\item[A] {\tt (REAL)} One-dimensional array, containing a number of
subsets of real numbers.
\item[IDO] {\tt (INTEGER)} Number of subsets to be examined.
\item[IW] {\tt (INTEGER)} Number of words in each subset.
\item[NA] {\tt (INTEGER)} Specifies the distance between the first
elements of consecutive subsets.
\end{DLtt}
\Notes
To find the largest element in a continuous vector, {\tt VMAXA} (F121)
is faster than {\tt AMAXMU}.
\Examples
\begin{verbatim}
    X=AMAXMU(A,4,1,2)
\end{verbatim}
sets {\tt X} equal to the largest absolute value of
{\tt A(1)}, {\tt A(3)}, {\tt A(5)} and {\tt A(7)}.
 
    $\bullet$
