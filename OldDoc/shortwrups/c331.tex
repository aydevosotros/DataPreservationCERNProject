%  20 dec 94  ksk
\Version{RFCONC}                                   \Routid{C331}
\Keywords{CONICAL FUNCTION}
\Author{K.S. K\"olbig}                            \Library{MATHLIB}
\Submitter{}                                      \Submitted{15.02.1989}
\Language{Fortran}                          \Revised{01.12.1994}
\Cernhead{Conical Functions of the First Kind}
Function subprograms {\tt RFCONC} and {\tt DFCONC} calculate the
(real valued) conical function of the first kind
 $$ \displaystyle P^m_{-\frac{1}{2}+i\tau}(x) $$
for real $ x>-1, \tau \ge 0$, and $m=0,1$, where $P^m_\nu(x)$ is the
Legendre (or spherical) function of the first kind and $i=\sqrt{-1}$.
\par
On CDC and  Cray computers, the double-precision version
{\tt DFCONC} is not available.
\Structure
{\tt FUNCTION} subprograms\\
User Entry Names: \Rdef{RFCONC}, \Rdef{DFCONC}\\
Obsolete User Entry Names: \Rdef{FCONC} $\equiv$ \Rdef{RFCONC} \\
Files Referenced: {\tt Unit 6}   \\
External References:
\begin{htmlonly}
\begin{tabular}{llll}
\end{htmlonly}
\begin{latexonly}
\begin{tabular}[t]{l*{3}{@{\hspace{4pt}}l}}
\end{latexonly}
\Rind{CGAMMA}{C305}, & \Rind{WGAMMA}{C305}, &
\Rind{CLGAMA}{C306}, & \Rind{WLGAMA}{C306}, \\
\Rind{BESJO}{C312},  & \Rind{DBESJ0}{C312},  &
\Rind{BESJ1}{C312},  & \Rind{DBESJ1}{C312}, \\
\Rind{BESIO}{C313},  & \Rind{DBESI0}{C313},  &
\Rind{BESI1}{C313},  & \Rind{DBESI1}{C313},  \\
\Rind{RELIKC}{C347}, & \Rind{DELIKC}{C347},  &
\Rind{RELIEC}{C347}, & \Rind{DELIEC}{C347},  \\
\multicolumn{3}{l}{\Rind{MTLMTR}{N002}, \Rind{ABEND}{Z035}}
\end{tabular}
\Usage
For $\mathtt{t=R}$ (type {\tt REAL}), $\mathtt{t=D}$ (type
{\tt DOUBLE PRECISION}),
\begin{verbatim}
    tFCONC(X,TAU,M)}
\end{verbatim}
has, in any arithmetic expression, the value \quad
$P^{\mathtt{M}}_{\frac{1}{2}+i*\mathtt{TAU}}(\mathtt{X})$.
\begin{DLtt}{12345}
\item[X] (type according to {\tt t}) Variable $x$.
\item[TAU] (type according to {\tt t})
The imaginary part of the index, $\tau$.
\item[M] ({\tt INTEGER}) Order $m$. ($\mathtt{M=0}$ or $\mathtt{M=1})$.
\end{DLtt}
\Method
Either (i) series expansions based on the Gaussian
hypergeometric function and evaluated by direct summation
or from rational approximations, or (ii) asymptotic
expressions in terms of Bessel functions. For $\tau =0$ the
conical functions (for $m = 0,1$) can be expressed in terms of
complete elliptic integrals.
For details see Ref. 1.
\Restrict
$\mathtt{X \ge -1}$, $\mathtt{TAU \ge 0}$, $\mathtt{M = 0}$ or {\tt 1}.
\newpage
\Accuracy
{\tt RFCONC} (except on CDC and Cray computers)
has full single-precision accuracy.
For most values of the argument {\tt X}, {\tt DFCONC}
(and {\tt RFCONC} on CDC and Cray computers), an accuracy of
not less than 10 significant digits is usually obtained.
If $x$ and $\tau$ are not too large
the accuracy increases to about 12-13 significant digits.
\Errorh
Error {\tt C331.1}: $\mathtt{X<-1}$ or $\mathtt{TAU<0}$ or
$\mathtt{M \ne 0}$ and $\mathtt{M \ne 1}$. \\
Error {\tt C331.2}: Problems of convergence for a hypergeometric
function. \\
In both cases, the function value is set equal to zero,
and a message is written on
{\tt Unit 6}, unless subroutine {\tt MTLSET} (N002) has been called.
%\newpage
\Notes
This program is an (only formally) modified version of the
CPC Program Library Package {\tt FCONIC} (see Ref.~1).
\Refer
\begin{enumerate}
\item K.S.  K\"olbig, A program for computing the conical
functions of the first kind $P^m_{1/2+i\tau}(x)$ for $m=0$
and $m=1$, Computer Phys. Comm. {\bf 23} (1981) 51--61.
\item M.I. Zhurina and L.N. Karmazina, Tables and
formulae for the spherical functions $P^m_{1/2+i\tau}(z)$,
(Pergamon Press, Oxford 1966).
\end{enumerate}
$\bullet$
