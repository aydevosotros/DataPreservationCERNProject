\Version {BINOMI}                             \Routid{V111}
\Keywords{DISTRIBUTION BINOMIAL NUMBER RANDOM}
\Author{D. Drijard}                           \Library{MATHLIB}
\Submitter{}                               \Submitted{15.09.1978}
\Language{Fortran}                   %\Revised{}
\Cernhead {Binomial Random Numbers}
{\tt BINOMI} generates a random integer $N$ according to the
binomial law:
$$ Prob(N=n) \ = \ {M \choose n}  P^n (1-P)^{M-n} $$
where the 'sample size' $M$ (a positive integer) and the probability
$P$ ($0 \leq P \leq 1$) are specified by the user.
\Structure
{\tt SUBROUTINE} subprogram \\
User Entry Names: \Rdef{BINOMI}\\
External References: \Rind{RNDM} (V104)
\Usage
\begin{verbatim}
    CALL BINOMI(M,P,N,IERR)
\end{verbatim}
\begin{DLtt}{123456}
\item [M] ({\tt INTEGER}) Sample size $M$.
\item [P] ({\tt REAL}) Probability.
\item [N]({\tt INTEGER}) The generated random number, binomially
distributed in the interval $\mathtt{0 \leq N \leq M}$ with mean
{\tt P*M}.
\item [IERR]({\tt INTEGER}) Error flag. \\
$\mathtt{= 0:}$ Normal case, \\
$\mathtt{= 1:}$ $\mathtt{P \leq 0}$ or $\mathtt{P \geq 1}$.
\end{DLtt}
\Notes
{\tt BINOMI} should not be used when {\tt M} is 'large' (say
$ >100 $). The normal approximation is then recommended instead (with
mean $\mathtt{P*M+0.5}$ and standard deviation
$\sqrt{\mathtt{M*P*(1-P)}}$).
\Source
Los Alamos report LA-5061-MS.
\\ $\bullet$
