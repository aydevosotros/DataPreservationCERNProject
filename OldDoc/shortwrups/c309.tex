%  17 nov 95 ksk
\Version{CCLBES}                       \Routid{C309}
\Keywords{COMPLEX COULOMB WAVE BESSEL SPHERICAL FUNCTION}
\Author{I.J. Thompson,  A.R. Barnett}  \Library{MATHLIB}
\Submitter{K.S. K\"olbig}              \Submitted{15.01.1988}
\Language{Fortran}                    \Revised{15.11.1995}
\Cernhead{Coulomb Wave, Bessel, and Spherical Bessel Functions for
Complex Argument(s) and Order}
Subroutine subprograms {\tt CCLBES} and {\tt WCLBES} calculate any
one of the following
sequences of functions:
\begin{enumerate}
\item  Regular and irregular Coulomb wave functions
$F_{\lambda+n}(\eta,z), \ G_{\lambda+n}(\eta,z)$ \\
and their first derivatives with respect to $z$,
$F'_{\lambda+n} (\eta,z), \ G'_{\lambda+n}(\eta,z)$, \\
or simple combination of these;
\item  Spherical Bessel functions
$j_{\lambda+n}(z), \ y_{\lambda+n}(z)$ \\
and their first derivatives with respect to $z$,
$j'_{\lambda +n}(z), \ y'_{\lambda +n}(z)$, \\
or simple combination of these (spherical Hankel functions);
\item  Bessel functions $J_{\lambda +n}(z), \ Y_{\lambda+n}(z)$ \\
and their first derivatives with respect to $z$,
$J'_{\lambda+n}(z), \ Y'_{\lambda+n}(z)$, \\
or simple combination  of these (Hankel functions);
\item Modified Bessel functions
$I_{\lambda+n}(z), \ K_{\lambda+n}(z)$ \\
and their  first derivatives with respect to $z$,
$I'_{\lambda+n}(z), \ K'_{\lambda+n}(z)$;
\end{enumerate}
\par
for complex arguments $\eta,z$, complex order
$\lambda$, and $ n=0,1,\ldots,N.$
\par
The double-precision version {\tt WCLBES} is available only
on computers which support a {\tt COMPLEX*16} Fortran data type.
\Structure
{\tt SUBROUTINE} subprograms\\
User Entry Names: \Rdef{CCLBES}, \Rdef{WCLBES} \\
Internal   Entry Names:
{\tt C309R1}, {\tt C309R2}, {\tt C309R3}, {\tt C309R4}, {\tt C309R5},
{\tt C309R6}, {\tt C309R7}, {\tt C309R8} \\
Files Referenced: {\tt Unit 6} \\
External References: \Rind{CLGAMA}{C306}, \Rind{WLGAMA}{C306},
\Rind{CPSIPG}{C317}, \Rind{WPSIPG}{C317}
\Usage
For $\mathtt{t=C}$ (type {\tt COMPLEX}), $\mathtt{t=W}$ (type
{\tt COMPLEX*16}),
\begin{verbatim}
    CALL tCLBES(Z,ETA,ZLMIN,NL,F,G,FP,GP,SIG,KFN,MODE,JFAIL,JPR)
\end{verbatim}
\begin{DLtt}{1234567890}
\item[Z] (type according to {\tt t}) Argument $ z \ne 0$.
\item [ETA] (type according to {\tt t}) Argument $\eta$
(ignored if $\mathtt{KFN > 0}$).
\item [ZLMIN] (type according to {\tt t})
Order $\lambda_{min}$ of the first function in the computed sequence.
\item[NL] ({\tt INTEGER})
Specifies the order $\lambda_{min}+\mathtt{NL}$ of the last
function in the computed sequence. ($\mathtt{NL \ge 0}$).
\item [F,G,FP,GP] (type according to {\tt t})
One-dimensional arrays with dimension {\tt (0:d)}
where {\tt d} is in each case $\mathtt{\ge NL+1}$.
On exit, each of {\tt F(n),G(n),FP(n),GP(n)} may contain
the value of a function of order $\lambda_{min}+n$, or its first
order derivative, $(n=0,1,\ldots,\mathtt{NL})$, as specified jointly by
{\tt KFN} and {\tt |MODE|}.
\item[SIG] (type according to {\tt t})
One-dimensional array with dimension {\tt (0:d)},
where  $\mathtt{d \ge NL+1}$. On exit, provided $\mathtt{KFN=0}$,
{\tt SIG(n)} contains the Coulomb phase shift $\sigma(\eta)$ for
$\lambda=\lambda_{min}+n, (n=0,1,\ldots, \mathtt{NL})$.
\item [KFN] ({\tt INTEGER}) Specifies, in conjunction with the absolute
value of {\tt MODE}, the type of functions which are stored.
\item[MODE] ({\tt INTEGER})
The absolute value of {\tt MODE} specifies, in conjunction
with {\tt KFN}, the type of function which are stored, and also specifies
which of the arrays {\tt F,G,FP,GP} are in fact set to meaningful values.
The sign of {\tt MODE} specifies whether or not the functions are
multiplied by a scaling factor.
\item[JFAIL] ({\tt INTEGER})
On exit, {\tt JFAIL} is set to zero if no error condition is
detected. Otherwise {\tt JFAIL} is set as described under
{\bf Error handling}.
\item[JPR] ({\tt INTEGER})
\item[] $\mathtt{= 0:}$ Suppress printing of error messages.
\item[] $\mathtt{= 1:}$ Print error messages.
\end{DLtt}
\par
The type of function which is stored in array {\tt F} depends only on
{\tt KFN}, while the type of function which is stored in array {\tt G}
depends both on {\tt KFN} and on {\tt |MODE|}. Arrays {\tt FP} and
{\tt GP} (if set) contain the first order derivatives with respect to $z$
of the functions in {\tt F} and {\tt G}, respectively.
Using the abbreviations ($i = \sqrt{-1}$)
$$\begin{array}{l@{\ \equiv \ }ll@{\ \equiv \ }ll@{\ \equiv \ }l}
F_\lambda & F_\lambda(\eta,z), & G_\lambda & G_\lambda(\eta,z), &
H^\pm_\lambda & G_\lambda \pm iF_\lambda, \\
j_\lambda &  j_\lambda(z), & y_\lambda &  y_\lambda(z),  &
h_\lambda^{(1,2)} &  j_\lambda \pm iy_\lambda, \\
J_\lambda &  J_\lambda(z), & Y_\lambda &  Y_\lambda(z),  &
H_\lambda^{(1,2)} & J_\lambda \pm iY_\lambda,  \\
I_\lambda &  I_\lambda(z), & K_\lambda & K_\lambda(z),
\end{array}$$
\par
the choice of function is given by the following table:
$$\begin{array}{|c|c|c|c|c|c|}
\hline
\mbox{\rm Array} & \mathtt{|MODE|} &
\multicolumn{4}{|c|} {\mathtt{KFN}}\\
\hline
& & -1 \ \mbox{\rm or } 0 & 1 & 2 & 3 \\
\hline
\mathtt{F} & \mbox{\rm all values} & F_{\lambda} & j_{\lambda}
& J_{\lambda} & I_{\lambda}           \\
\mathtt{G} & 1,2,3,4  & G_{\lambda}  & y_{\lambda}
& Y_{\lambda}  & K_{\lambda}      \\ [2mm]
& 11,12 & H^+_{\lambda}& h^{(1)}_{\lambda}
& H ^{(1)}_{\lambda}    &-        \\ [2mm]
& 21,22    & H^-_{\lambda} &h^{(2)}_{\lambda}
& H^{(2)}_{\lambda}&- \\ [2mm]
\hline
\end{array}$$
If {\tt KFN=0} the phase shifts $\sigma(\eta)$ are stored in array
{\tt SIG}. Otherwise {\tt SIG} is not set.
\par
Which of arrays {\tt F,G,FP,GP} are in fact set is determined by {\tt
|MODE|}
according to the following table:
\begin{center}
\begin{tabular}{|c|c|c|c|c|}
\hline
$\mathtt{|MODE|}$ & {\tt F} & {\tt G} & {\tt FP}  & {\tt GP} \\
\hline
1, 11, 21   & set     & set      & set   & set \\
2, 12, 22   & set     & set      & -     & -   \\
3           & set     & -        & set   & -    \\
4           & set     & -        & -     &-    \\
\hline
\end{tabular}
\end{center}
In both the tables above, a dash indicates that the corresponding
array does not contain meaningful values on exit. These arrays are,
however, used internally as working space, and must therefore be
dimensioned correctly.
The sign of {\tt MODE} specifies whether or not the functions are to be
multiplied by a scaling factor, depending only on $z$, which will bring
their values closer to unity when $|z|$ is large, or $\eta$ is small
and $|\lambda|<|z|$. The same scaling factor is applied to the first
order derivatives in {\tt FP} or {\tt GP} as is
applied to the functions in {\tt F} or {\tt G}, respectively.
\begin{DLtt}{1234}
\newpage
\item[MODE] $\mathtt{>0:}$ No scaling factor.
\item[MODE] $\mathtt{<0:}$ Let $S=\mbox{\rm Im} (z)$ if
$\mathtt{KFN<3}$, $S=\mbox{\rm Re} (z)$ if $\mathtt{KFN=3}$;
then the scaling factors for {\tt F} and {\tt G} are
\end{DLtt}
$$ \begin{array}{ll@{\ }l}
 \mathtt{F:} & \exp(-|S|) &  \{F,j,J,I\} \\
 \mathtt{G:} & \exp(-|S|) &  \{G,y,Y\} \\
             & \exp(S)    &  \{H^{+},h^{(1)},H^{(1)},K\} \\
             & \exp(-S)   &  \{H^{-},h^{(2)},H^{(2)}\}.
\end{array} $$
\Method
The method is described in the References.
\Restrict
See Ref. 1, in particular Sect. 4.
\Accuracy
The absolute values of the results are usually
accurate to within two or three decimal digits of the machine
precision. For details of exceptions see Ref. 1, Sect. 4.
\Errorh
If an error condition is detected, {\tt JFAIL} is
set to one of the following values and a message is printed if
$\mathtt{JPR=1}$.
\begin{DLtt}{1234}
\item[$\mathtt{> 0}$]An arithmetic error occurred
during the final recursion. Correct results are available up to and
including subscript value {\tt NL-JFAIL-1}.
\item[$\mathtt{- 1}$] One of the continued fraction
calculations failed or there was an arithmetic error before any results
could be calculated.
\item[$\mathtt{- 2}$] Argument out of range.
\item[$\mathtt{- 3}$] One or more functions
corresponding to $\lambda_{min}$ could not be calculated. Some values
corresponding to $\lambda>\lambda_{min}$ may be correct.
\item[$\mathtt{- 4}$]  Excessive internal cancellation
probably renders the result meaningless.
\end{DLtt}
\Source
This program package is a modified version of the CPC Program Library
package {\tt COULCC} (see Ref. 1). The changes are formal, not
computational.
\Refer
\begin{enumerate}
\item  I.J. Thompson and A.R. Barnett, COULCC: A
continued-fraction algorithm for Coulomb functions of complex order
with complex arguments, Comput. Phys. Comm. {\bf 36} (1985) 363--372.
\item I.J. Thompson and A.R. Barnett, Coulomb and Bessel functions of
complex arguments and order, J. Comput. Phys. {\bf 64} (1986) 490--509.
\end{enumerate}
\Longwr
A copy of Ref. 1 is available in the Program Library Office.
\\ $\bullet$
 
