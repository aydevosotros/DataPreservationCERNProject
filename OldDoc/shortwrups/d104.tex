%  21 oct 94  ksk
\Version {RCAUCH}               \Routid{D104}
\Keywords{ADAPTIVE NUMERICAL INTEGRATION PRINCIPAL VALUE QUADRATURE}
\Author{K.S. K\"olbig}           \Library{MATHLIB}
\Submitter{}                     \Submitted{10.08.1967}
\Language{Fortran}               \Revised{01.12.1994}
\Cernhead{Cauchy Principal Value Integration}
Function subprograms {\tt RCAUCH} and {\tt DCAUCH}
compute the Cauchy principal value integral
$$ I \ = \ P \int_A^B f(x)dx $$
where $f$ has a singularity inside or outside the interval
$[A,B]$ such that the Cauchy principal value exists.
\par
On computers other than CDC or Cray, only
the double-precision version {\tt DCAUCH} is available.
On CDC and  Cray computers, only the single-precision version
{\tt RCAUCH} is available.
\Structure
{\tt FUNCTION} subprograms \\
User Entry Names: \Rdef{RCAUCH}, \Rdef{DCAUCH}\\
Obsolete User Entry Names: \Rdef{CAUCHY} $\equiv$ {\tt RCAUCH} \\
Files Referencend: {\tt Unit 6} \\
External References: \parbox[t]{100mm}{
\Rind{GAUSS}{D103}, \Rind{DGAUSS}{D103},
\Rind{MTLMTR}{N002}, \\
\Rind{ABEND}{Z035}, user-supplied {\tt  FUNCTION} subprogram}
\Usage
For $\mathtt{t=R}$ (type {\tt REAL}), $\mathtt{t=D}$ (type
{\tt DOUBLE PRECISION}),
\begin{verbatim}
    tCAUCH(F,A,B,S,EPS)
\end{verbatim}
has, in any arithmetic expression, the approximate value of the
integral $I$.
\begin{DLtt}{12345}
\item [F] (type according to {\tt t})
Name of a user-supplied {\tt FUNCTION} subprogram, declared
{\tt EXTERNAL} in the calling program. This subprogram must set
$\mathtt{F(X)} = f(\mathtt{X})$.
\item [A,B] (type according to {\tt t}) End-points of the integration
interval. Note that {\tt B} may be less than {\tt A}.
\item [S] (type according to {\tt t}) The absissa of the singularity.
\item [EPS]
(type according to {\tt t}) Accuracy parameter (see under
{\tt Accuracy} in the in short write-up for {\tt GAUSS} (D103)).
\end{DLtt}
\Method
The method described in  Ref. 1 is used for
decomposition of the integral. The resulting integrals are
computed by {\tt GAUSS} (D103).
\Accuracy
See short write-up for {\tt GAUSS} (D103).
\Errorh
Error {\tt D104.1}: $\mathtt{S=A}$ or $\mathtt{S=B}$. \\
Error {\tt D104.2}: The requested accuracy cannot be obtained
(see short write-up for {\tt GAUSS} (D103)). \\
The function value is set equal to zero, and a message is written on
{\tt Unit 6}, unless subroutine {\tt MTLSET} (N002) has been called.
\Refer
\begin{enumerate}
\item I.M. Longman, On the numerical evaluation of Cauchy principal
values of integrals, MTAC (later renamed Math. Comp.)
{\bf 12} (1958) 205--207.
\end{enumerate}
$\bullet$

