\Version {PKCHAR}                      \Routid{M427}
\Keywords{UNPACK PACK CONTINUOUS BYTE STRING}
\Author{J. Zoll}                       \Library{KERNLIB}
\Submitter{}                           \Submitted{01.06.1973}
\Language{Fortran or Assembler}        \Revised{16.09.1991}
\Cernhead {Pack/Unpack Continuous Byte-strings}
{\tt PKCHAR} allows packing of integers into
{\it continuous byte-strings} on zoned memory across word boundaries.
The term {\it continuous byte-string} is used here to designate
$n$-bit bytes, stored from left to right,
as opposed to the objects handled by {\tt PKBYT} (M422),
which are stored right to left within each word.
The inverse unpacking is performed by {\tt UPKCH}.
Leading and trailing bits of each zone can be ignored.
\Structure
{\tt SUBROUTINE} subprograms\\
User Entry Names: \Rdef{PKCHAR}, \Rdef{UPKCH}\\
External References: \Rind{JBYT}{M421}, \Rind {SBYT}{M421}, \Rind{CBYT}{M421}\\
{\tt COMMON} Block Names and Lengths: {\tt /SLATE/ NWU,DUMMY(39)}
\Usage
\begin{verbatim}
    CALL PKCHAR(INT,MPK,N,IPAR)
    CALL UPKCH (MPK,INT,N,IPAR)
\end{verbatim}
{\tt PKCHAR} packs the {\tt N}--word vector {\tt INT} of integers into
the continuous byte-string supported by the vector {\tt MPK} according
to the packing specifications given in {\tt IPAR}.
\par
{\tt UPKCH} is the exact inverse of {\tt PKCHAR}.
\par
The packing parameters are given in the 5-element vector {\tt IPAR}:
\begin{DLtt}{1234567890}
\item [IPAR(1)] Number of bits per byte, must be $\geq 1$.
\item [IPAR(2)] Number of bytes to be used in each
zone (starting with the left-most); \\
if $\mathtt{IPAR(2)=0}$, the maximum
possible number is used.
\item [IPAR(3)] Number of bits per zone. If
$\mathtt{IPAR(3)=0}$, a zone equals 1 word.
\item [IPAR(4)] Number of leading bits of each zone
to be ignored.
\item [IPAR(5)] Each new word handled by {\tt PKCHAR} is preloaded with
{\tt IPAR(5)}.
\end{DLtt}
{\tt MPK} is seen as a continuous string of bits,
starting with the most significant bit of {\tt MPK(1)},
ignoring word boundaries.
This string is divided into a number of consecutive and contiguous zones,
each of {\tt IPAR(3)} bits;
the first zone starts with the most significant bit of {\tt MPK(1)}.
Each zone contains {\tt IPAR(4)} leading bits,
a number of bytes (each of {\tt IPAR(1)} bits), and trailing bits, if any.
\par
On return from either routine, {\tt NWU} in {\tt COMMON} block
{\tt /SLATE/} indicates the number of words in {\tt MPK} actually used.
{\tt PKCHAR} sets to {\tt IPAR(5)} each word of {\tt MPK} before
filling it, but it does not clear any trailing unused words which
logically belong to the last zone.
\newpage
\Examples
\begin{enumerate}
\item To convert, on the CDC 7600, 6-bit Hollerith text to 7-bit
{\tt ASCII}--code, to be held in 36-bit words on the PDP10,
with 5 characters per word.
\begin{verbatim}
    DATA IPACK6 /6,0,0,0,0/
    DATA IPACK7 /7,0,36,0,0/
    CALL UPKCH(HOLL,INT,N,IPACK6)
\end{verbatim}
unpacks the Hollerith string {\tt HOLL} into {\tt INT}, where
{\tt INT(I)} is a small integer giving the display-code value of the
{\tt I}-th character. After conversion to {\tt ASCII}, one may pack:
\begin{verbatim}
    CALL PKCHAR(INT,MPK,N,IPACK7)
\end{verbatim}
giving the vector {\tt MPK} ready to be written out.
If for some reason one required the first and the last (5th)
character in each 36-bit PDP10 word to be zero, one could use:
\begin{verbatim}
    DATA IPACK7 /7,3,36,7,0/
\end{verbatim}
\item To unpack 8-character bytes read with the CDC 7600 from
9-track tapes:
\begin{verbatim}
    DATA IPACK /8,0,120,0,0/
    CALL UPKCH(A,INT,N,IPACK)
\end{verbatim}
\item To unpack on the CRAY 32-bit integers,
read one each into one 64-bit machine word,
into 16-bit integers, one each in one machine word,
right-justified with zero-fill:
\begin{verbatim}
    DATA IPACK /16,2,0,32,0/
    CALL UPKCH(I32,I16,N,IPACK)
\end{verbatim}
The same operation on the Apollo, which has 32-bit words,
could be done with
\begin{verbatim}
    DATA IPACK /16,0,0,0,0/
\end{verbatim}
\item The Fortran implementaion of {\tt BLOW} (M426) executes:
\begin{verbatim}
    IPACK(1) = NBITS
    IPACK(2) = 0
    IPACK(3) = NBYTES*NBITS + 127
    IPACK(4) = 0
    IPACK(5) = 0
    CALL UPKCH(SOURCE,TARGET,NBYTES,IPACK)
\end{verbatim}
\end{enumerate}
$\bullet$
