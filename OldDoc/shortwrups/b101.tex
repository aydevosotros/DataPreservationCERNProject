\Version{ATG}                  \Routid{B101}
\Keywords{ARC TANGENT ATAN2}
\Author{K.S. K\"olbig}         \Library{MATHLIB}
\Submitter{}                   \Submitted{15.02.1989}
\Language{Fortran}             \Revised{15.03.1993}
\Cernhead{Arc Tangent Function}
Function subprogram {\tt ATG} calculates, for real arguments $x_1$ and
$x_2$, $(x_1,x_2) \neq (0.,0.)$, an angle $\alpha$ such that
$$\alpha =\arctan(x_1/x_2)\quad\mbox{and}\quad 0\leq\alpha<2\pi.$$
Note that using the Fortran intrinsic function {\tt ATAN2} instead of
{\tt ATG} would result in $-\pi<\alpha\leq\pi.$
\Structure
{\tt FUNCTION} subprogram
 
User Entry Names: \Rdef{ATG}
 
\Usage
In any arithmetic expression,
\begin{center}
{\tt ATG(X1,X2)}
\end{center}
has the value of $\alpha$ (in radians).
{\tt ATG, X1} and {\tt X2} are of type {\tt REAL}.
\Notes
This function subprogram is equivalent to the statement function
\begin{center}
\tt ATG(X1,X2)=ATAN2(X1,X2)+(PI-SIGN(PI,X1))
\end{center}
where $\mathtt{PI = \displaystyle \pi}$.
\\ $\bullet$
