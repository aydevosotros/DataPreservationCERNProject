\Version{POLROT}                    \Routid{F105}
\Keywords{ROTATION POLAR COORDINATE SYSTEM}
\Author{M. Regler}                 \Library{MATHLIB}
\Submitter{}                      \Submitted{01.03.1968}
\Language{Fortran}                \Revised{27.11.1984}
\Cernhead{Rotate a Three-Dimensional Polar Coordinate System}
{\tt POLROT} calculates the values of
$\theta'$ and $\phi'$ of the coordinate system $S'(\theta',\phi',r)$,
obtained by rotation of the 3-dimensional polar coordinate system
$S(\theta,\phi,r)$ about any axis $(0\leq \theta \leq \pi,
0\leq \phi\leq 2\pi)$.
\Structure
{\tt SUBROUTINE} subprogram \\
User Entry Names: \Rdef{POLROT}
\Usage
\begin{verbatim}
    CALL POLROT(THETA,PHI,THPRIM,PHPRIM,THAX,PHAX,ROTANG)
\end{verbatim}
\begin{DLtt}{1234567890}
\item [THETA] ({\tt REAL}) Angle $\theta$  in the old system
$S(\theta,\phi,r)$.
\item [PHI] ({\tt REAL}) Angle $\phi$  in the old system
$S(\theta,\phi,r)$.
\item [THPRIM] ({\tt REAL}) Angle $\theta'$ in the new system
$S'(\theta',\phi',r)$.
\item [PHPRIM] ({\tt REAL}) Angle $\phi'$ in the new system
$S'(\theta',\phi',r)$.
\item [THAX,PHAX] ({\tt REAL}) Angles defining the axis of rotation in
the old system $S(\theta,\phi,r)$.
\item [ROTANG] ({\tt REAL}) Angle in the old system through which the
system is rotated.
\end{DLtt}
The subroutine calculates from {\tt THETA} and {\tt PHI} the new values
{\tt THPRIM} and {\tt PHPRIM} in a coordinate system obtained by
rotating the old system through an angle {\tt ROTANG} about
an axis defined by {\tt THAX} and {\tt PHAX} in the old system.
\Method
{\tt THETA} and {\tt PHI} are converted to a unit vector in Cartesian
coordinates; {\tt THAX}, {\tt PHAX} and {\tt ROTANG} are converted to a
tensor, which is used to obtain a vector in the new system of axes giving
{\tt THPRIM} and {\tt PHPRIM}.
\Notes
If {\tt THPRIM} is very small, {\tt PHPRIM} is badly defined.
\\ $\bullet$
