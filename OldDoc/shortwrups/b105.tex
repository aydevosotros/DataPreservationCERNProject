% 10.10.94 ksk
\Version {RPLNML}                       \Routid{B105}
\Keywords{POLYNOMIAL HORNER SCHEME}
\Author{K.S. K\"olbig}      \Library{MATHLIB}
\Submitter{}                            \Submitted{01.12.1994}
\Language{Fortran}               %      \Revised{}
\Cernhead {Value of a Polynomial}
Function subprograms {\tt RPLNML}, {\tt DPLNML} calculate the value
of the polynomial
$$p_n(x) \ = \ a_0+a_1x+a_2x^2+\cdots+a_nx^n$$
or
$$q_n(x) \ = \ a_0x^n+a_1x^{n-1}+a_2x^{n-2}+\cdots+a_n$$
for real values $x$, whereas function subprograms {\tt CPLNML},
{\tt WPLNML} calculate the value of the polynomial
$$r_n(z) \ = \ c_0+c_1z+c_2z^2+\cdots+c_nz^n$$
or
$$s_n(x) \ = \ c_0z^n+c_1z^{n-1}+c_2z^{n-2}+\cdots+c_n$$
for complex values $z$.
\par
On CDC and Cray computers, the double-precision versions {\tt DPLNML}
and {\tt WPLNML} are not available.
\Structure
{\tt FUNCTION} subprograms \\
User Entry Names: \Rdef{RPLNML}, \Rdef{DPLNML}, \Rdef{CPLNML},
                  \Rdef{WPLNML}
\Usage
For $\mathtt{t=R}$ (type {\tt REAL}), $\mathtt{t=D}$ (type
{\tt DOUBLE PRECISION}),
\begin{verbatim}
    tPLNML(X,N,A,MODE)
\end{verbatim}
has, in any arithmetic expression, the value $p_n(x)$ or $q_n(x)$;
\par
for $\mathtt{t=C}$ (type {\tt COMPLEX}), $\mathtt{t=W}$ (type
{\tt COMPLEX*16}),
\begin{verbatim}
    tPLNML(Z,N,C,MODE)
\end{verbatim}
has, in any arithmetic expression, the value $r_n(z)$ or $s_n(z)$.
\begin{DLtt}{123456}
\item[X,Z] (type according to {\tt t}) Arguments $x$ or $z$,
respectively.
\item[N] ({\tt INTEGER})
Degree $n$ of $p_n(x),\,q_n(x)$ or $r_n(z),\,s_n(z)$.
\item[A,C] (type according to {\tt t}) One-dimensional arrays of
dimension {\tt (0:d)} where $\mathtt{d \ge N}$, containing the
coefficients $a_k$ or $c_k\,(k=0,\ldots,n)$ in {\tt A(k)} or {\tt C(k)},
respectively.
\item[MODE] ({\tt INTEGER}) Either {\tt +1} for $p_n(x),\,r_n(z)$ or
{\tt -1} for $q_n(x),\,s_n(z)$.
\end{DLtt}
\Method
The Horner scheme is used.
\Notes
A reference with $\mathtt{N<0}$ or {\tt MODE} different from
{\tt +1} or {\tt -1} returns the value zero.
\\ $\bullet$
