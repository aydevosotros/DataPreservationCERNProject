% 20 dec 1994  ksk
\Version {RCHECF}                          \Routid{E406}
\Keywords{CHEBYSHEV COEFFICIENT FUNCTION SERIES APPROXIMATION}
\Author{T. H{\aa}vie}                           \Library{MATHLIB}
\Submitter{K.S. K\"olbig}                    \Submitted{24.01.1986}
\Language{Fortran}                    \Revised{01.12.1994}
\Cernhead {Chebyshev Series Coefficients of a Function}
Subroutine subprograms {\tt RCHECF}, {\tt DCHECF} and {\tt QCHECF}
calculate coefficients for a finite sum of Chebyshev polynomials
approximating a function $f(x)$ over an interval $a\le x\le b$ to
accuracy $\varepsilon$. It returns an integer $n$ and coefficients
$c_0,c_1,\ldots,c_n$ such that the sum
\begin{equation}
f^*(x) = \sum^n_{j=0} c_j T_j(t)
\end{equation}
where $t = (2x - a - b)/(b - a)$ and $T_j(t)$ is the Chebyshev
polynomial of degree $j$, satisfies for $a\le x\le b$ the relation
\begin{equation}
|f^*(x) - f(x)| < \varepsilon.
\end{equation}
Subsequent evaluation of the approximation (1) can be done
by calling  {\tt CHSUM} (E407) with
the appropriate value of its argument {\tt MODE}.
\par
On computers other than CDC and Cray, only the double- and
quadruple-precision versions {\tt DCHECF} and {\tt QCHECF} are
available. On CDC and Cray computers, only the single- and
double-precision versions {\tt RCHECF} and {\tt DCHECF} are available.
\Structure
{\tt SUBROUTINE} subprogram \\
User Entry Names: \Rdef{RCHECF}, \Rdef{DCHECF}, \Rdef{QCHECF} \\
Obsolete User Entry Names: \Rdef{CHECF} $\equiv$ \Rdef{RCHECF} \\
Files Referenced: {\tt Unit 6} \\
External References: \Rind{MTLMTR}{N002}, \Rind{ABEND}{Z035},
user-supplied {\tt FUNCTION} subprogram
\Usage
For $\mathtt{t=R}$ (type {\tt REAL}), $\mathtt{t=D}$ (type
{\tt DOUBLE PRECISION}), $\mathtt{t=Q}$ (type {\tt REAL*16}),
\begin{verbatim}
    CALL tCHECF(F,A,B,EPS,C,N,DELTA)
\end{verbatim}
\begin{DLtt}{123456}
\item[F] (type according to {\tt t}) Name of a user-supplied
{\tt FUNCTION} subprogram, declared {\tt EXTERNAL} in the calling
program.
\item[A,B] (type according to {\tt t})
End-points $a,b$ of the approximation interval.
\item[EPS] (type according to {\tt t}) Requested accuracy.
\item[C] (type according to {\tt t}) One-dimensional array
with dimension {\tt (0:d)}, $\mathtt{d \ge 128}$. On exit,
$\mathtt{C(j)} = c_j,(j = 0,1,\ldots,\mathtt{N})$.
\item[N] ({\tt INTEGER}) On exit, {\tt N} is equal to the subscript
of the last computed coefficient.
\item[DELTA] (type according to {\tt t}) On exit, {\tt DELTA} is such
that the relation $|f^*(x) - f(x)| < \mathtt{DELTA}$
is almost certainly true for $x \in [a,b]$. (See Error Handling.)
\end{DLtt}
\Method
The interval $[a,b]$ is subdivided successively
into sets of subintervals of length  $2^{-k}(b-a),(k = 0,1,2\ldots)$.
After each subdivision the orthogonality properties
of the Chebyshev polynomials with respect to summation
over equally-spaced points are used
to compute two sets of approximate values of the coefficients $c_j$:
one set computed using the end-points of the subintervals, and one
set using the mid-points. The mean of these two values is taken
as the best estimate of the $c_j$, which are then tested to see (a)
whether certain rate-of-convergence criteria are satisfied, (b)
whether there is some $n$ for which the sum for $j>n$ of the available
$c_j$ is less than $\varepsilon$. If both conditions are satisfied
the subroutine terminates.
%\newpage
\Errorh
Error {\tt E406.1}:
If the requested accuracy cannot be obtained with 65 coefficients
(i.e., $\mathtt{N = 64}$) a message is written on
{\tt Unit 6}, unless subroutine {\tt MTLSET} (N002) has been called.
In this case, values of $f^*$ computed from (1) with $\mathtt{N = 64}$
should still be in error by less than {\tt DELTA}.
\Notes
\begin{enumerate}
\item This subroutine is intended for use with functions $f(x)$ which
can be computed to full machine accuracy, and which are sufficiently
smooth to ensure fairly rapid decrease of the $c_j$ with
increasing $j$. Functions defined by experimental data can usually
be approximated better by least-squares methods, using ordinary
polynomials.
\item Note that some authors use a different definition for the
constant term in (1), i.e. $c_0/2$ instead of $c_0$. Here, the
definition of Ref. 1 is used.
\end{enumerate}
\Refer
\begin{enumerate}
\item Y.L. Luke, Mathematical functions and their approximations,
(Academic Press, New York 1975)
\end{enumerate}
$\bullet$
