% 12 may 1995  ksk
\Version{RGAPNC}                          \Routid{C334}
\Keywords{GAMMA FUNCTION INCOMPLETE COMPLEMENT}
\Author{K.S. K\"olbig}                   \Library{MATHLIB}
\Submitter{}                 \Submitted{01.05.1990}
\Language{Fortran}                     \Revised{01.12.1994}
\Cernhead{Incomplete Gamma Functions}
Function subprograms {\tt RGAPNC}, {\tt DGAPNC} and
{\tt RGAGNC}, {\tt DGAGNC} calculate
the incomplete gamma function
$$P(a,x) =\left\{\begin{array}{l@{\qquad}l}
\displaystyle \frac{1}{\Gamma(a)} \int_0^x e^{-t}\,t^{a-1}\,dt &
(a > 0) \\[4mm]
\displaystyle e^{-x}\,x^a \, \frac{M(1,a + 1,x)}{\Gamma(a + 1)} &
(a \le 0), \end{array}\right.$$
and the complementary incomplete gamma function
$$G(a,x) =\left\{\begin{array}{l@{\qquad}l}
\displaystyle \frac{1}{\Gamma(a)} \int_x^\infty e^{-t}\,t^{a-1}\,dt &
(a > 0) \\[4mm]
\displaystyle e^x\,x^{-a} \int_x^\infty\,e^{-t}\,t^{a-1}\,dt &
(a \le 0), \end{array}\right.$$
respectively, for real arguments $x \ge 0$ and $a$.
$M(a,b,x)$ is Kummer's function (see Ref. 3).
\par
On CDC and Cray computers, the double-precision versions
{\tt DGAPNC} and {\tt DGAGNC} are not available.
\Structure
{\tt FUNCTION} subprograms \\
Uses Entry Names:
\Rdef{RGAPNC}, \Rdef{RGAGNC}, \Rdef{DGAPNC}, \Rdef{DGAGNC} \\
Obsolete User Entry Names: \Rdef{GAPNC} $\equiv$ \Rdef{RGAPNC},
                           \Rdef{GAGNC} $\equiv$ \Rdef{RGAGNC} \\
Files Referenced: {\tt Unit 6} \\
External References: \Rind{ALGAMA}{C304}, \Rind{DLGAMA}{C304},
\Rind{MTLMTR}{N002}, \Rind{ABEND}{Z035}
\Usage
In any arithmetic expression,
\begin{center}
{\tt RGAPNC(A,X)} \quad or \quad {\tt DGAPNC(A,X)}
\quad has the value \quad $P(\mathtt{A,X})$, \\
{\tt RGAGNC(A,X)} \quad or \quad {\tt DGAGNC(A,X)}
\quad has the value \quad $G(\mathtt{A,X})$,
\end{center}
where {\tt RGAPNC} and {\tt RGAGNC} are of type {\tt REAL},
{\tt DGAPNC} and {\tt DGAGNC} are of type {\tt DOUBLE PRECISION},
{\tt A} and {\tt X} have the same type as the function name.
\Method
The method is described in Ref. 1.
\Accuracy
{\tt RGAPNC} and {\tt RGAGNC} (except on CDC and Cray computers)
have full single-precision accuracy.
For most values of the arguments, {\tt DGAPNC}, {\tt DGAGNC}
(and {\tt RGAPNC}, {\tt RGAGNC} on CDC and Cray computers) have an
accuracy of
approximately two significant digits less than the machine precision.
\Restrict
For $P(a,x)$: Either (i) $\mathtt{X > 0}$, or (ii) $\mathtt{X = 0}$ and
$\mathtt{A \ge 0}$. \\
For $G(a,x)$: Either (i) $\mathtt{X > 0}$, or (ii) $\mathtt{X = 0}$ and
$\mathtt{A \ne 0}$.
\newpage
\Errorh
Error {\tt C334.1}: $\mathtt{X<0}$. \\
Error {\tt C334.2}: For {\tt RGAPNC} and {\tt DGAPNC:}
$\mathtt{A<0}$ and $\mathtt{X=0}$;
for {\tt RGAGNC} and {\tt DGAGNC:} $\mathtt{A=X=0}$. \\
Error {\tt C334.3}: Problems with convergence (unlikely). \\
In all cases,
the function value is set equal to zero, and a message is written on
{\tt Unit 6}, unless subroutine {\tt MTLSET} (N002) has been called.
\Notes
When speed is more important than accuracy, e.g. for applications in
statistics, use {\tt GAMDIS} (G106) for computing $ P(a,x)$.
Note, however, that in this case the arguments {\tt A} and {\tt X} must
be interchanged.
\Source
The subprograms are based on a Fortran program for the incomplete
gamma functions published in Ref. 2.
\Refer
\begin{enumerate}
\item W. Gautschi, A computational procedure for incomplete gamma
functions, ACM Trans. Math. Software {\bf 5} (1979) 466--481.
\item W. Gautschi, Algorithm 542, Incomplete gamma functions,
Collected Algorithms from CACM (1979).
\item  M. Abramowitz and I.A. Stegun (Eds.), Handbook
of Mathematical Functions, Chapter 13, Confluent Hypergeometric
Functions, 9th printing with corrections, (Dover, New York 1972).
\end{enumerate}
$\bullet$
