\Version{FLOP}                               \Routid{Q902}
\Keywords{FORTRAN SYTAX COMPILER PRETTYPRINT PARSER EDITOR}
\Author{H. Grote}                            \Library{PGMLIB}
\Submitter{}                                 \Submitted{29.11.1988}
\Language{Fortran}                    %\Revised{}
\Cernhead {FLOP - Fortran Language Oriented Parser}
{\tt FLOP} is best described as an "intelligent" editor that recognizes
Fortran ({\tt ANSI 77}) code, with a full coverage of
{\tt ANSI 66} and some of its extensions). To achieve this, {\tt FLOP}
has to perform part of the functions of a compiler, mainly the
declaration and syntax analysis.
The knowledge resulting from this then allows {\tt FLOP} to edit the
Fortran input file in various ways, and to provide useful
information about its contents.
\Structure
Complete {\tt PROGRAM}\\
Files  Referenced: {\tt Unit 11} (input), {\tt Unit 5} (commands),
{\tt Unit 6} (output)\\
External References: \Rdef{TIMEL} (Z007), \Rdef{TIMEX} (Z007)
\Usage
See {\bf Long Write-up}.
\par
Refer also to the interactive help files or to the {\tt FLOP DECKS}
in the various Patches of the {\tt INSTALL} Pam file for examples of
usage.
\par
The source code can be found in the {\tt FLOP} Pam file on the
various machines.
\\ $\bullet$
