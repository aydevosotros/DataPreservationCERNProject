\Version {UBLOW1}                   \Routid{M416}
\Keywords{BIT CONCENTRATE DISPERSE STRING}
\Author{CDC: J. Zoll, IBM: R. Matthews}
\Library{KERNLIB, IBM and CDC only}
\Submitter{}                          \Submitted{01.09.1969}
\Language{Assembler}                   \Revised{27.11.1984}
\Cernhead {Concentrate and Disperse Bit Strings}
\begin{center}
\fbox{\parbox{120mm}{
\begin{center}
{\bf OBSOLETE}
\end{center}
Please note that this routine has been obsoleted in CNL 204. Users are
advised not to use it any longer and to replace it in older programs.
No maintenance for it will take place and it will eventually disappear.
\\[3mm]
Suggested replacement: {\tt PKBYT}, {\tt UPKBYT} (M422)
}}
\end{center}
{\tt UBLOW1} unpacks a continous bit-string (across word boundaries),
into an array with one significant bit per word. {\tt UBNCH1} performs
the inverse operation, packing bits into a continous string.
\Structure
{\tt SUBROUTINE} subprograms \\
User Entry Names: \Rdef{UBLOW1}, \Rdef{UBNCH1}
\Usage
\begin{verbatim}
    CALL UBLOW1(A,K,N)
\end{verbatim}
disperses the string of {\tt N} bits stored in {\tt A} into the vector
{\tt K}: {\tt K(j) = 0} or {\tt 1}, depending on bit {\tt j} being
{\tt 0} or {\tt 1}.
\begin{verbatim}
    CALL UBNCH1(K,A,N)
\end{verbatim}
concentrates into {\tt A} the least significant bits of the elements
in the vector {\tt K} consisting of integers 0 or 1 into a string
of {\tt N} bits. In {\tt A} the bits are numbered from right to left
within words; in {\tt K} the order is reversed such that the least
significant bit of {\tt A(i+1)} immediately follows the sign-bit of word
{\tt A(i)}.
\\ $\bullet$
