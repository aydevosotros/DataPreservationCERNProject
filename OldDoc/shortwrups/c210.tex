\Version{NZERFZ}              \Routid{C210}
\Keywords{NUMBER ZERO COMPLEX ANALYTIC FUNCTION LOGARITHMIC RESIDUE}
\Authors{K.S. K\"olbig}         \Library{MATHLIB}
\Submitter{ }                    \Submitted{07.06.1992}
\Language{Fortran}   %     \Revised{ }
\Cernhead{Number of Zeros of a Complex Function}
Function subprogram {\tt NZERFZ} calculates the number of zeros
of a complex function $f(z)$ inside a closed polygon in the complex
$z$-plane. $f(z)$ must be analytic inside this polygon.
\Structure
{\tt FUNCTION} subprogram \\
User Entry Names: \Rdef{NZERFZ}  \\
Files Referenced : {\tt Unit 6} \\
External References: \Rind{MTLMTR}{N002}, \Rind{ABEND}{Z035},
User-supplied {\tt FUNCTION} subprogram
\Usage
In any arithmetic expression,
\begin{center}
{\tt NZERFZ(F,ZP,N)}
\end{center}
has a value equal to the number of zeros inside the defined polygon.
\begin{DLtt}{1234}
\item[F] Name of a user-supplied {\tt FUNCTION} subprogram, declared
{\tt EXTERNAL} in the calling program. This subprogram must set
$\mathtt{F(Z)}= f(\mathtt{Z})$.
\item[ZP] One-dimensional array of length $\geq \mathtt{N}$ containing
the vertices of the polygon in the $z$-plane.
\item[N] Number of vertices.
\end{DLtt}
{\tt F}, {\tt ZP} and {\tt Z} (in {\tt F}) are of type {\tt COMPLEX*16}
on computers other than CDC or Cray, and of type {\tt COMPLEX}
on CDC and Cray computers.
\Method
The logarithmic residual (winding number) of $f(z)$ is found by
integrating $f'(z)/f(z)$ numerically along the edges of the polygon.
\Notes
No zero or singularity of $f(z)$ should lie on or too near the polygon.
The edges of the polygon should not cross each other.
Numerically unstable functions (e.g. polynomials of high degree)
can result in unreliable values or in timing problems.
\Errorh
Error {\tt C210.1:} The integration is not successful.
This often indicates that the polygon passes through or too
near to a zero or singularity. The function value is set to zero, and
a message is written on {\tt Unit 6}, unless subroutine {\tt MTLSET}
(N002) has been called.
\\ $\bullet$
