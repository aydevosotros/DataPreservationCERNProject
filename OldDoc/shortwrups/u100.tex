\Version{CLEBS}                            \Routid{U100}
\Keywords{ALGEBRA CLEBSCH GORDAN COEFFICIENT}
\Author{H. Yoshiki}                         \Library{MATHLIB}
\Submitter{}                                \Submitted{01.03.1968}
\Language{Fortran}                          \Revised{27.11.1984}
\Cernhead {Clebsch-Gordan Coefficients in Algebraic Form}
{\tt CLEBS} computes Clebsch-Gordan coefficients
$$ C(j_1,j_2,j;m_1,m_2,m) $$
where $j_1,j_2,j$ and $m_1,m_2,m$ are integers or half-integers.
\Structure
{\tt SUBROUTINE} subprogram \\
User Entry Names: \Rdef{CLEBS}\\
Internal Entry Names: {\tt DIVIDE}, {\tt PRMTOD}, {\tt FCTRAL},
{\tt PRIME}\\
Files Referenced: Printer\\
External References: \Rind{UCOPY} (V301),  \Rind{VZERO} (F121)
\Usage
\begin{verbatim}
    CALL CLEBS(J1,J2,J3,JM1,JM2,JM3,K,M)
\end{verbatim}
where
\begin{center}
{\tt J1 }$ = 2j_1$, \ {\tt J2 }$ = 2j_2$, \ {\tt J3 }$ = 2j$,
\ {\tt JM1 }$ = 2m_1$, \ {\tt JM2 }$ = 2m_2$, \ {\tt JM3 }$ = 2m$
\end{center}
and {\tt K} and {\tt M} are {\tt INTEGER} arrays of dimension {\tt 2}
and {\tt 41}, respectively. {\tt K(1)} is set to the signed
numerator of $C^2$, {\tt K(2)} is set to the denominator of $C^2$,
both elements having no common divisor.
\Restrict
The program handles $j_1,j_2,j$ up to about 40. If larger quantum
numbers are required it is necessary to make corresponding changes in
the source.
\Errorh
The program is loaded with the prime numbers up to 173
(40th prime number). A message is printed if an encountered prime number
exceeds this limit. In this case the left-over factor is put into
{\tt M(41)} and $C$ is expressed as
$$
\mathtt{M(41)*2^{M(1)/2}*3^{M(2)/2}*5^{M(3)/2}*7^{M(4)/2}*\cdots *
173^{M(40)/2}}.$$
\Refer
\begin{enumerate}
\item H.E. Rose, Elementary Theory of Angular Momentum.
\item T. Tamura, ORNL - 3877 (1966).
\end{enumerate}
$\bullet$
