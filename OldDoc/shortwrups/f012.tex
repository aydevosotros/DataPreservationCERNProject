\Version {RSINV}                     \Routid{F012}
\Keywords{LINEAR SYMMETRIC POSITIVE DEFINITE SYSTEM}
\Author{H. Lipps}                     \Library{KERNLIB}
\Submitter{}                           \Submitted{01.09.1983}
\Language{Fortran or Assembler or COMPASS} % %\Revised{}
\Cernhead {Symmetric Positive-Definite Linear Systems}
Subroutine {\tt tSINV} (where $\mathtt{t=R}$ or {\tt D} as described
below) computes the inverse of a symmetric positive-definite matrix
{\bf A}.
\par
Subroutine {\tt tSEQN} solves a set of linear equations
$$\mathbf{AX=B} \mbox{\hspace{3cm} (*)}$$
whose coefficient matrix {\bf A} is symmetric and positive-definite.
The determinant det({\bf A}) of {\bf A} may be calculated by subroutine
{\tt tSFACT} described below.
\par
If several systems of the form (*) are to be solved with the same
{\bf A} but differing {\bf B}, a procedure which
is appreciably faster than calling subroutine {\tt tSEQN} repeatedly
is to execute a single call to subroutine {\tt tSEQN} (or subroutine
{\tt tSFACT} if the determinant is required), and then to call subroutine
{\tt tSFEQN} as many times as required. When the last system (*) has
been solved, the inverse matrix $\mathbf{A^{-1}}$, if required, may
be computed by calling {\tt tSFINV}.
\par
Subroutine {\tt tSEQN} and {\tt tSFACT} both replace the matrix
{\bf A} by a lower triangular matrix {\bf L} and an upper triangular
matrix {\bf U} such that $\mathbf{LU=A}$. This {\bf LU} decomposition is
referred to below as {\bf lu(A)}.
\par
Given {\bf lu(A)} and some matrix {\bf B}, subroutine {\tt tSFEQN}
replaces {\bf B} by the solution {\bf X} of equation (*) without
changing {\bf lu(A)}. Subroutine {\tt tSFEQN} may therefore be called
repeatedly with differing {\bf B}.
\par
Given {\bf lu(A)}, subroutine {\tt tSFINV} replaces {\bf lu(A)} by the
inverse $\mathbf{A^{-1}}$ of {\bf A}.
\par
\Structure
{\tt SUBROUTINE} subprograms \\
User Entry Names: \parbox[t]{.6 \textwidth} {
\Rdef{RSFACT}, \Rdef{RSEQN}, \Rdef{RSFEQN}, \Rdef{RSINV}, \Rdef{RSFINV}\\
\Rdef{DSFACT}, \Rdef{DSEQN}, \Rdef{DSFEQN}, \Rdef{DSINV}, \Rdef{DSFINV}
} \\
Files Referenced: Printer\\
External References: \Rind {TMPRNT}{F011}, \Rind{KERMTR}{N001},
\Rind{ABEND}{Z035}
\Usage
For $\mathtt{t=R}$ (type {\tt REAL}), $\mathtt{t=D}$ (type
{\tt DOUBLE PRECISION}):
\begin{verbatim}
    CALL tSINV (N,A,IDIM,IFAIL)
    CALL tSEQN (N,A,IDIM,IFAIL,K,B)
    CALL tSFACT(N,A,IDIM,IFAIL,DET,JFAIL)
    CALL tSFEQN(N,A,IDIM,K,B)
    CALL tSFINV(N,A,IDIM)
\end{verbatim}
\begin{DLtt}{12345678}
\item [N] ({\tt INTEGER}) Order of the matrix {\bf A}.
\item [A] (Type according to {\tt t}) Two-dimensional array
whose first dimension has the value {\tt IDIM}.
\item [IDIM]({\tt INTEGER}) First dimension
of array {\tt A} (and of array {\tt B} if $\mathtt{K > 1}$).
\item [IFAIL] ({\tt INTEGER}) On exit, {\tt IFAIL} will be set to {\tt 0}
if {\bf A} is positive-definite, and to {\tt -1} otherwise.
\item [DET] (Type according to {\tt t}) On exit, {\tt DET} will be set to
the value det({\bf A}) unless {\tt JFAIL} returns a non-zero value.
\item [JFAIL]({\tt INTEGER}) On exit, {\tt JFAIL} will be set to zero if
det({\bf A}) can be safely evaluated. Otherwise {\tt JFAIL} is set as
follows: \\
$\mathtt{= -2}$ if {\bf A} is not positive-definite, \\
$\mathtt{= -1}$ if det({\bf A}) is probably too small, \\
$\mathtt{= +1}$ if det({\bf A}) is probably too large.
\item [K] ({\tt INTEGER}) Number of columns of the matrices {\bf B}
and {\bf X}.
\item [B] (Type according to {\tt t}) In general, a two-dimensional
array whose first dimension has the value {\tt IDIM}. {\tt B} may be
one-dimensional if $\mathtt{K =1}$. {\tt tSEQN} accepts a dummy argument
{\tt B} if $\mathtt{K = 0}$.
\end{DLtt}
The contents of arrays {\tt A} and {\tt B} on entry
and exit are as follows:
\begin{DLtt}{12345678}
\item [tSINV] On entry, {\bf A} must be stored in {\tt A}. On exit,
{\tt A} contains $\mathbf{A^{-1}}$ if $\mathtt{IFAIL = 0}$, or else is
undefined.
\item [tSEQN] On entry, {\bf A} must be stored in {\tt A} and {\bf B}
in {\tt B}. On exit, {\tt A} contains {\bf lu(A)} and {\tt B} contains
{\bf X} if $\mathtt{IFAIL = 0}$, or else {\tt A} is undefined and {\tt B}
is unchanged.
\item [tSFACT] On entry, {\bf A} must be stored in {\tt A}. On exit,
{\tt A} contains {\bf lu(A)} if $\mathtt{IFAIL = 0}$, or else is undefined.
{\tt DET} contains det({\bf A}) if $\mathtt{JFAIL = 0}$, contains zero if
$\mathtt{JFAIL = -1}$, and is undefined otherwise.
\item [tSFEQN] On entry, {\bf lu(A)} must be stored in {\tt A}, and
{\bf B} in {\tt B}. On exit, {\tt A} is unchanged and {\tt B} contains
{\bf X}.
\item [tSFINV] On entry, {\bf lu(A)} must be stored in {\tt A}.
On exit, {\tt A} contains $\mathbf{A^{-1}}$.
\end{DLtt}
\Method
Modified Cholesky factorization (without square roots). See Ref. 1.
\Accuracy
On computers with IBM 370 architecture, inner products are accumulated
using double precision arithmetic internally for arrays of type
{\tt REAL}.
\Notes
Only those elements $a_{ij}$ of the original matrix {\bf A} for which
$i \geq j$ are required on entry to {\tt tSINV}, {\tt tSEQN}
and {\tt tSFACT}.
\Errorh
If $\mathtt{N<1}$ or $\mathtt{IDIM<N}$ or $\mathtt{K<0}$ {\tt (tSEQN)}
or $\mathtt{K < 1}$ {\tt (tSFEQN)}, a message is printed and program
execution is terminated by calling {\tt ABEND} (Z035).
\Examples
Assume that the $10 \times 10$ matrix {\bf A} and the $10 \times 3$
matrix {\bf B} are stored according to the Fortran convention in
arrays {\tt A} and {\tt B} respectively of a program containing the
declarations
\begin{verbatim}
    REAL A(25,30),B(25,10)
\end{verbatim}
To replace {\bf B} by the $10 \times 3$ solution matrix {\bf X} of the
system of equations $\mathbf{AX=B}$, with a jump to label {\tt 100} if
{\bf A} is not positive definite:
\begin{verbatim}
    CALL RSEQN(10,A,25,IFAIL,3,B)
    IF(IFAIL .NE. 0) GO TO 100
\end{verbatim}
\Refer
\begin{enumerate}
\item J.H. Wilkinson  and C. Reinsch (eds.), Handbook for
automatic computation, Vol.2: Linear algebra (Springer-Verlag,
New York 1971), Chapter 2.
\end{enumerate}
$\bullet$
