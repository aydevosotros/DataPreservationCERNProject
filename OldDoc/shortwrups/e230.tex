%  12 may 1995  ksk
\Version {TL}                                \Routid{E230}
\Keywords{FIT FITTING LINEAR LEAST-SQUARE CONSTRAINED UNCONSTRAINED}
\Author{W. Hart, W. Matt}                    \Library{KERNLIB}
\Submitter{}                                 \Submitted{01.01.1975}
\Language{Fortran}                           \Revised{04.02.1986}
\Cernhead {Constrained and Unconstrained Linear Least Squares Fitting}
The {\tt TL} package finds the least squares solution to a set of
unweighted linear equations, possibly subject to a set of equality
constraints. The solution is found by Householder triangularisation
(see Ref. 1 for details) with parameter elimination if constraints are
present. This write-up ends with a few words on generalised least
squares fitting (unequal weighting) which is a simple application of
the {\tt TL} package.
\par
All matrices are assumed to be stored {\bf row-wise and without gaps},
{\bf contrary} to the Fortran convention, i.e., if the
Fortran statement {\tt DIMENSION  A(NJ,NI)} reserves
memory for the matrix {\bf A} the element $A_{ij}$ is found in word
{\tt A(J,I)}.
\Structure
{\tt SUBROUTINE} subprograms\\
User Entry  Names: \Rdef{TLSC}, \Rdef{TLS}, \Rdef{TLERR}, \Rdef{TLRES}\\
Internal Entry Names: {\tt TLSMSQ}, {\tt TLSWOP}, {\tt TLUK}, {\tt
TLSTEP}, {\tt TLPIV}
\Usage
 {\bf General Description}\\[2mm]
Consider the set of $M$ linear equations
$$ \sum^N_{j=1} A_{ij}x_j = b_i
\mbox{\hspace{1cm}} (i=1,2,\ldots,M \ \mbox{with} \ N\leq M)$$
to be solved such that the Euclidian norm
$\mathbf{||Ax - b||_2} = S^2$ is minimised. Instead of determining
{\bf x} from the Normal Equation $\mathbf{x = (A'A)^{-1} A'b}$
it is found by applying successive Householder transformations
({\bf Q}) which reduce {\bf A} to upper triangular
form without changing the norm of the columns of {\bf A} or the vector
{\bf b}. This is beneficial from the point of view of stability and
flexibility of application. Writing
$$\mathbf{QA = R = \binoms{R_1}{O}}
\begin{array}{@{ \ \} \ }l} N \ \mbox{rows} \\ M-N \ \mbox{rows}
\end{array}
\mbox{\quad and \quad} \mathbf{Qb = y = \binoms{y_1}{y_2}}
\begin{array}{@{ \ \} \ }l} N \ \mbox{rows} \\ M-N \ \mbox{rows}
\end{array} $$
we have that $\mathbf{||Rx - y||_2 = ||Ax - b||_2}$
and the vector {\bf x} is obtained by backward substitution in
$\mathbf{R_1x = y_1}$. As a byproduct, the sum of squares of
residuals is directly calculated as $S^2 = \mathbf{||y_2||_2}$.
\par
Now consider {\bf A} and {\bf b} to be composed of $M_1$ constraints
to be satisfied exactly, followed by $M-M_1$ equations to be minimised.
Writing
$$\mathbf{A = \binoms{A_1}{A_2}}
\begin{array}{@{ \ \} \ }l} M_1 \ \mbox{rows} \\ M-M_1 \ \mbox{rows}
\end{array}
\mbox{\quad , \quad}  \mathbf{b = \binoms{b_1}{b_2}}
\begin{array}{@{ \ \} \ }l} M_1 \ \mbox{rows} \\ M-M_1 \ \mbox{rows}
\end{array} $$
then $\mathbf{||A_2x-b_2||_2} = S^2$  has to be minimized
subject to $\mathbf{A_1x-b_1} = 0$.
\newpage
This problem is solved by eliminating $M_1$ parameters and then
evaluating the reduced set of parameters (see Ref. 2 for details).
\par
An attractive feature of the unitary Householder transformations
is that when each parameter is eliminated ("solved for")
column pivoting allows the selection of that parameter which gives the
maximum reduction in the current value of $S^2$.
Thus it is possible to terminate the calculation whenever $S^2$ or its
current reduction become acceptably small. This can be exploited when
iterating. If there is more than one RHS vector, then {\bf x} and {\bf b}
become $N \times L$ and $M \times L$ matrices with the pivoting strategy
applied to the first column of {\bf b}.
\par
The triangular form of $\mathbf{R_1}$ allows the error matrix, {\bf E},
of the fitted parameters to be derived directly from $\mathbf{R_1}$
itself without inverting. The equation is
$$\mathbf{E=R_1^{-1}(R_1^{-1})'}.$$
Moreover, the vector of fitted residuals is most easily computed by
applying the inverse Householder transformation to $\mathbf{y_2}$, i.e.
$$\mathbf{Ax-b=Q^{-1} \binoms{O}{y_2}}.$$
Note that these residuals do {\it not have to be calculated} to find
the fitted $S^2$ which is output from the fitting routines.
\par
In all routines described below, the dimensionality of the problem
is transmitted via the common block
\begin{verbatim}
    COMMON /TLSDIM/ M1,M,N,L,IER
\end{verbatim}
The parameter {\tt IER} returns the number of parameters solved for,
or else {\tt -1001} if either $\mathtt{M1>N}$,
$\mathtt{N>M}$ or {\bf A} has rank less than {\tt N}. \\[3mm]
{\bf Constrained Least Squares Fitting}
\begin{verbatim}
    CALL TLSC(A,B,AUX,IPIV,EPS,X)
\end{verbatim}
\begin{DLtt}{123456}
\item[A] ({\tt REAL}) The combined constraint / derivative matrix of
dimension $\mathtt{M \times N}$, the upper {\tt M1} rows being the
constraints.
\item[B] ({\tt REAL}) The combined constraint / measurement matrix of
dimension $\mathtt{M \times L}$, the upper {\tt M1} rows being the
constraints.
\item[X] ({\tt REAL}) The matrix of dimension $\mathtt{N \times L}$
returning the {\tt L} least squares solutions.
\item[AUX] ({\tt REAL}) Working array of length
$\mathtt{N+\max(N,L)}$. On output {\tt AUX(J),(J=1,L)}
contain the minimised sum of squares.
\item[IPIV] ({\tt INTEGER}) Working array of length {\tt N} which holds
the exchange information (column pivoting is employed if necessary).
\item[EPS] ({\tt REAL}) Parameter specifying a pivoting criterium.
There is no exchange of columns $I$ and $1$ unless
$\mathtt{EPS*PIVOT(I) > PIVOT(1)}$. Typically ${\tt EPS \simeq 0.1}$.
\end{DLtt}
Subroutines called: {\tt TLSMSQ}, {\tt TLSWOP}, {\tt TLUK},
{\tt TLSTEP}.\\[3mm]
When constraint equations are present, the full pivoting
strategy cannot be adopted and so all parameters are solved for,
i.e., {\tt IER} returns the value {\tt N} or {\tt -1001}. Under these
circumstances {\tt EPS} is used to reduce the amount of pivoting to
those cases where it is felt to be absolutely necessary. \\[3mm]
\newpage
{\bf Unconstrained Least Squares Fitting}
\begin{verbatim}
    CALL TLS(A,B,AUX,IPIV,EPS,X)
\end{verbatim}
\begin{DLtt}{123456}
\item[A] ({\tt REAL}) $\mathtt{M \times N}$ derivative matrix.
\item[B] ({\tt REAL}) $\mathtt{M \times L}$ matrix of measurements.
\item[X] ({\tt REAL}) $\mathtt{N \times L}$ parameter solution matrix.
\item[AUX] ({\tt REAL}) Working array as for {\tt TLSC}.
\item[IPIV] ({\tt INTEGER}) Working array as for {\tt TLSC}.
\item[EPS] ({\tt REAL}) Input parameter used for prematurely
terminating the calculation: \\
$\mathtt{> 0:}$ Termination when r.m.s. residual $\mathtt{<|EPS|}$,\\
$\mathtt{< 0:}$ Termination when the reduction in the residual
$\mathtt{<|EPS|}$, \\
$\mathtt{= 0:}$ Unconditionally solve for all $X_j.$
\end{DLtt}
Subroutines called: {\tt TLSMSQ}, {\tt TLSWOP}, {\tt TLUK},
{\tt TLSTEP}, {\tt TLPIV}.\\[3mm]
As previously indicated, full pivoting is possible
without constraints, hence the allowance for premature exit.\\[3mm]
{\bf  Fitted Error Matrix}
\begin{verbatim}
    CALL TLERR(A,E,AUX,IPIV)
\end{verbatim}
The parameter and subroutine arguments defined previously in
{\tt COMMON /TLSDIM/} require the output values from a call to
{\tt TLS} or {\tt TLSC}. {\tt E} is an $N \times N$ matrix
which, upon return, will contain the unnormalised covariance matrix of
the fitted parameters, $\mathbf{(A'A)^{-1}}$. {\tt A} may be overwritten
by {\tt E} and the routine may be called independently from
{\tt TLS/TLSC} by  setting {\tt IER} to zero. \\[3mm]
Subroutines called: {\tt TLUK}, {\tt TLSTEP}.\\[3mm]
{\bf Fitted Residuals}
\begin{verbatim}
    CALL TLRES(A,B,AUX)
\end{verbatim}
All the arguments and common variables require the output
values from a call to {\tt TLS} or {\tt TLSC}. Upon return,
{\tt B} will give the matrix of residuals, i.e., for each set of
least squares equations the column vector $\mathbf{Ax-b}$.\\[3mm]
Subroutine called: {\tt TLSTEP}.
\Notes
\begin{enumerate}
\item The pivoting and exit criteria of {\tt TLS} are calculated using
the first vector of measurements; therefore it is wise to have
$\mathtt{EPS=0}$ if $\mathtt{L>1}$.
\item {\tt TLERR} and/or {\tt TLRES} may be called in any order after
{\tt TLS} or {\tt TLSC}.
\item {\tt TLS} or {\tt TLSC} may be used for solving simultaneous linear
equations by setting $\mathtt{M=N}$ or $\mathtt{M1=N}$.
\item Useful examples in the application of these routines can
be found in the {\tt HYDRA} Geometry / Kinematics processors.
\end{enumerate}
\newpage
{\bf Generalized Least Squares Fitting} \\[3mm]
The problem is to minimise $\mathbf{(Ax-b)'G(Ax-b)}$ where {\bf G}, the
weight matrix, is the inverse of the error matrix of the
measurement vector {\bf b}. Once again Householder
triangularisation offers an attractive alternative to the Normal
Equation solution $\mathbf{x=(A'GA)^{-1} A'Gb}$.
The first step is to perform the Choleski decomposition of {\bf G},
which is positive semi-definite (see {\tt TR} (F112)), such that
$\mathbf{G=U'U}$, {\bf U} being upper triangular. The
problem is then reduced to minimising $\mathbf{||A^1x-b^1||_2}$,
where $\mathbf{A^1=UA}$ and $\mathbf{b^1=Ub}$,
which is just the unweighted case previously described. This has the
feature that if {\bf A} has already been triangularised then the product
{\bf UA} remains triangular and only back substitution is necessary to
find the weighted least squares solution.
\Refer
\begin{enumerate}
\item G. Golub, Numerical methods for solving linear least squares
problems, Numer. Math. {\bf 7} (1965)  206--216.
\item {\AA}. Bj\"orck and G. Golub, Iterative refinement of linear least
square solutions by Householder transformation, BIT {\bf 7} (1967)
322--337.
\end{enumerate}
$\bullet$
