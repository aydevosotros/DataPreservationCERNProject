\Version{CWERF}                         \Routid{C335}
\Keywords{COMPLEX ERROR FUNCTION}
\Author{K.S. K\"olbig}                  \Library{MATHLIB}
\Submitter{}                             \Submitted{07.12.1970}
\Language{Fortran}                       \Revised{15.03.1993}
\Cernhead{Complex Error Function}
Function subprograms {\tt CWERF} and {\tt WWERF} calculate
the complex error function
$$ w(z) \ = \ \displaystyle e^{-z^2}\left[1+\frac{2i}{\sqrt\pi}
\int^z_0 e^{t^2} dt\right]
\ = \ \displaystyle e^{-z^2}\left[1-\mathrm{erf}\,(-iz)\right]
\ = \ \displaystyle e^{-z^2}\mathrm{erfc}\,(-iz)$$
for complex arguments $z$, where $i = \sqrt{-1}$.
\par
The double-precision version {\tt WWERF} is available only on computers
which support a {\tt COMPLEX*16} Fortran data type.
\Structure
{\tt FUNCTION} subprograms \\
User Entry Names: \Rdef{CWERF}, \Rdef{WWERF}
\Usage
In any arithmetic expression,
\begin{center}
{\tt CWERF(Z)} \quad or \quad {\tt WWERF(Z)} \quad has the value \quad
$w(\mathtt{Z})$,
\end{center}
where {\tt CWERF} is of type {\tt COMPLEX}, {\tt WWERF} is of type
{\tt COMPLEX*16}, and {\tt Z} has the same type as the function name.
\Method
The method is described in Ref. 2.
\Accuracy
{\tt CWERF} (except on CDC and Cray computers)
has full single-precision accuracy.
For most values of the argument {\tt Z}, {\tt WWERF}
(and {\tt CWERF} on CDC and Cray computers) has an accuracy of
approximately two significant digits less than the machine precision.
\Notes
This subprogram is a modified version of the algorithm
presented in Ref. 1.
\Refer
\begin{enumerate}
\item  W. Gautschi, Algorithm 363, Complex Error Function,
Collected Algorithms from CACM (1969).
\item W. Gautschi, Efficient Computation of the Complex Error Function,
SIAM J. Numer. Anal. {\bf 7} (1970) 187--198.
\item K.S. K\"olbig, Certification of Algorithm 363 Complex
Error Function, Comm. ACM {\bf 15} (1972) 465--466.
\end{enumerate}
$\bullet$
