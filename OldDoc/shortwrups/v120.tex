% 20 jan 95   ksk
\Version {RNORML}                       \Routid{V120}
\Keywords{GAUSSIAN DISTRIBUTION NUMBER RANDOM GENERATOR}
\Author{F. James}             \Library{MATHLIB}
\Submitter{}          \Submitted{15.03.1994}
\Language{Fortran}                   %\Revised{}
\Cernhead{Gaussian-distributed Random Numbers}
{\tt RNORML} and {\tt RNORMX} generate (vectors of)
single-precision random numbers in a Gaussian distribution of mean
zero and variance one. {\tt RNORML} uses the uniform generator
{\tt RANMAR} underneath, and {\tt RNORMX} allows the user
to choose the uniform generator to be used underneath.  The code
is portable Fortran, but the results are not guaranteed to be
identical on all platforms because there is branch on a floating-point
compare which may (very rarely) cause the sequence produced on a given
platform to be out of step with that of a different platform.
\Structure
{\tt SUBROUTINE} Subprograms\\
User Entry Names: \Rdef{RNORML}, \Rdef{RNORMX}
\Usage
\begin{verbatim}
    CALL RNORML(RVEC,LEN)
\end{verbatim}
generates a vector {\tt RVEC} of {\tt LEN} Gaussian-distributed
random numbers. {\tt RVEC} is an array of type {\tt REAL} and of
length {\tt LEN} at least.
\par
The uniform generator used is {\tt RANMAR},
so it may be initialized by calling {\tt RMARIN} (V113),
but beware that this also initializes {\tt RANMAR} (V113)!
\par
An alternative subroutine is supplied which allows the user to select
the underlying uniform generator, for example {\tt RANLUX} (V115).
\begin{verbatim}
    EXTERNAL urng
    ...
    CALL RNORMX(RVEC,LEN,urng)
\end{verbatim}
where {\tt urng} is a uniform random number generator of
standard calling sequence: {\tt CALL urng(VEC,LENG)}.
\par
For example,
\begin{verbatim}
    DIMENSION RVEC(10)
    LEN = 10
    EXTERNAL RANLUX
    CALL RLUXGO(4,7675039,0,0)
      DO ...
         CALL RNORMX(RVEC,LEN,RANLUX)
\end{verbatim}
would generate vectors of 10 Gaussian-distributed pseudorandom numbers
of the highest quality.  Note that initialization is now performed by
the initializing entry for {\tt RANLUX}, which is {\tt RLUXGO}.
\Method
  The method used to transform uniform deviates to Gaussian deviates
is that known as the ratio of random deviates, discovered by Kinderman
and Monahan, and improved by Leva (see {\bf References}).
The generation of one
Gaussian random number requires at least two, and on average 2.74 uniform
random numbers, as well as one floating-point division and on average
0.232 logarithm evaluations.
\newpage
\Refer
\begin{enumerate}
\item J.L. Leva, A fast normal random number generator,
ACM Trans. Math. Softw. {\bf 18} (1992) 449--453.
\item J.L. Leva, Algorithm 712. A normal random number generator,
ACM Trans. Math. Softw. {\bf 18} (1992) 454--455.
\end{enumerate}
$\bullet$
