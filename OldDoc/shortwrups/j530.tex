\Version {BINSIZ}                         \Routid{J530}
\Keywords{BINNING HISTOGRAM INTERVAL REASONABLE}
\Author{F. James}                          \Library{KERNLIB}
\Submitter{}                               \Submitted{01.10.1974}
\Language{Fortran}                    %\Revised{}
\Cernhead{Reasonable Intervals for Histogram Binning}
{\tt BINSIZ} determines reasonable lower and upper limits and
bin width for a histogram, given the lower and upper limits of the
data and the desired maximum number of bins. The output bin width
is always an integral power of $10 \times 1, 2, 2.5$ or 5, and the
output lower and upper limits are the nearest multiples of the bin
width containing the specified range. Another option allows the bin
width to be imposed and determines only the new limits.
\Structure
{\tt SUBROUTINE} subprogram \\
User Entry Names: \Rdef{BINSIZ}
\Usage
\begin{verbatim}
    CALL BINSIZ(AL,AH,NA,BL,BH,NB,BWID)
\end{verbatim}
\begin{DLtt}{123456}
\item [AL] ({\tt REAL}) Lower limit of data to be histogrammed.
\item [AH] ({\tt REAL}) Upper limit of data to be histogrammed.
\item [NA] ({\tt INTEGER}) Maximum number of bins desired.
\item [BL] ({\tt REAL}) Lower limit determined by {\tt BINSIZ}
$\mathtt{(BL \leq  AL)}$.
\item [BH] ({\tt REAL}) Upper limit determined by {\tt BINSIZ}
$\mathtt{(BH \geq  AH)}$.
\item [NB] ({\tt INTEGER}) Number of bins determined by {\tt BINSIZ}
$\mathtt{(NA/2 < NB \leq  NA)}$.
\item [BWID] ({\tt REAL}) Bin width $\mathtt{(BH-BL)/NB}$.
\end{DLtt}
If $\mathtt{NA=0}$ or $\mathtt{NA=-1}$, {\tt BINSIZ} always makes
exactly one bin.
\par
If $\mathtt{NA = 1}$, {\tt BINSIZ} takes {\tt BWID} as {\it input}
and determines only {\tt BL}, {\tt BH}, and {\tt NB}. This is
especially useful when it is desired to have the same bin width for
several histograms (or for the two axes of a scatter-plot).
\par
If $\mathtt{AL>AH}$, {\tt BINSIZ} takes {\tt AL} to be the upper limit
and {\tt AH} to be the lower limit, so that in fact {\tt AL} and
{\tt AH} may appear in any order. They are not changed by {\tt BINSIZ}.
If $\mathtt{AL = AH}$, {\tt BINSIZ} takes the lower limit as {\tt AL},
and the upper limit is set to $\mathtt{AL+1}$.
\\ $\bullet$
