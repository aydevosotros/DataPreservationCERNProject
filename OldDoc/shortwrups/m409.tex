%   20 feb 1995    ksk
\Version {UBUNCH}                               \Routid{M409}
\Keywords{BUNCH CHARACTER CONCENTRATE DISPERSE STRING}
\Author{J. Zoll}                                \Library{KERNLIB}
\Submitter{}                                    \Submitted{01.03.1968}
\Language{Fortran or Assembler}             \Revised{09.09.1991}
\Cernhead {Concentrate and Disperse Character Strings}
\begin{center}
\fbox{\parbox{120mm}{\POBSOLETE
Please note that this routine has been partially
obsoleted in CNL 219. Users are
advised not to use the entries refering to Hollerith
any longer and to replace them in older programs.
No maintenance for this part
will take place and it will eventually disappear.
\\[3mm]
Suggested replacement: {\tt CHPACK} (M432)  }}
\end{center}
The concept {\it string of n Hollerith characters} is machine
independent, but its usual representation in {\tt Am} format
(with {\tt m} = character capacity of a machine word:
{\tt A10}, {\tt A8}, {\tt A6}, {\tt A4}) is not.
\par
Supposing any computer to have a character capacity of at
least {\tt A4}, string representations in {\tt A4}, {\tt A3}, {\tt A2} or
{\tt A1} are transportable. Representations {\tt A1} and {\tt A4}
are particularly interesting.
\par
Fortran 77 defines a new data type {\tt CHARACTER}
though most compilers also support Hollerith strings (without a clear
definition of the differences). A set of routines has been added to
this package in its Fortran 77 implementation to convert between
{\tt CHARACTER} strings and Hollerith strings.
\par
The routines {\tt UBLOW}, {\tt UBUNCH} and {\tt UTRANS} work on
Hollerith only {\bf and so should be considered obsolete},
while {\tt UCTOH}, {\tt UCTOH1} and {\tt UHTOC} and
{\tt UH1TOC} copy between {\tt CHARACTER} and Hollerith.
Unpredictable results will be obtained if wrong data types are
passed to these routines. On most machines text strings passed in
quotes are implicitly of type {\tt CHARACTER} while a string preceeded by
{\tt nH} is not.
\par
The routines of this package perform transformations between different
representations.
\Structure
{\tt SUBROUTINE} subprograms \\
User Entry Names: \Rdef{UBUNCH}, \Rdef{UBLOW}, \Rdef {UTRANS},
\Rdef{UCTOH}, \Rdef{UCTOH1}, \Rdef{UHTOC}, \Rdef {UH1TOC} \\
{\tt COMMON} Block Names and Lengths: {\tt /SLATE/ NI,NJ,DUMMY(38)}
\Usage
\begin{verbatim}
    CALL UBLOW(IVM,IV1,NCH)
\end{verbatim}
disperses the string of {\tt NCH} Hollerith characters
from {\tt IVM} into {\tt IV1}.
\begin{DLtt}{123456}
\item [IVM] Input vector, continuous string of {\tt NCH} Hollerith
characters in {\tt Am} form
(i.e. {\tt A10}, {\tt A8} or {\tt A4} depending on the machine).
\item [IV1] Output vector, {\tt NCH} words in {\tt A1} form, i.e.
a single Hollerith character per word with blank-fill.
\item [NCH] Number of Hollerith characters to be copied.
\end{DLtt}
\begin{verbatim}
    CALL UBUNCH(IV1,IVM,NCH)
\end{verbatim}
concentrates the string of {\tt NCH} Hollerith characters from
{\tt IV1} into {\tt IVM}.
\begin{DLtt}{123456}
\item [IV1] Input vector, {\tt NCH} words in {\tt A1} form (one
Hollerith character per word).
\item [IVM] Output vector, continuous string of {\tt NCH} Hollerith
characters in {\tt Am} form
(i.e. {\tt A10}, {\tt A8} or {\tt A4} depending on the machine),
with blank-fill of trailing characters in the last word, if any.
\item [NCH] Number of Hollerith characters to be copied.
\end{DLtt}
\newpage
\begin{verbatim}
    CALL UTRANS(IVI,IVJ,NCH,I,J)
\end{verbatim}
copies the string of {\tt NCH} Hollerith characters from
{\tt IVI} into {\tt IVJ}.
\begin{DLtt}{123456}
\item [IVI] Input vector of {\tt NCH} Hollerith characters
with {\tt I} characters per machine word in Ai form.
The variable {\tt NI} in  {\tt /SLATE/} is set
to the number of machine words used from {\tt IVI}.
\item [IVJ] Output vector of {\tt NCH} Hollerith characters
with {\tt J} characters per machine word in Aj form, with blank-fill.
The variable {\tt NJ} in {\tt /SLATE/} is set to the number
of machine words built in {\tt IVJ}.
\item [NCH]  Number of Hollerith characters to be copied.
\item [I,J] Number of Hollerith characters per word in {\tt IVI}
and {\tt IVJ}.
If either {\tt I} or {\tt J} is greater than the maximum possible
number of characters storable in a machine word
then this maximum is used instead.
\end{DLtt}
\begin{verbatim}
    CALL UCTOH(MCH,IVJ,J,NCH)
\end{verbatim}
copies the {\tt CHARACTER}-type string in {\tt MCH}
into Hollerith characters in {\tt IVJ} in {\tt Aj} form.
\begin{DLtt}{123456}
\item [MCH] Input vector of {\tt NCH} characters,
either of type {\tt CHARACTER} or of type {\tt INTEGER}
holding Hollerith in {\tt Am} form.
\item [IVJ] Output vector of {\tt NCH} Hollerith characters
with {\tt J} characters per machine word in Aj form, with blank-fill.
\item [J] Number of Hollerith characters to put in each word of
{\tt IVJ}. If {\tt J} is larger than the maximum possible number of
Hollerith characters per word this maximum will be used instead.
\item [NCH] Number of characters to copy.
\end{DLtt}
\begin{verbatim}
    CALL UCTOH1(MCH,IV1,NCH)
\end{verbatim}
disperses the {\tt CHARACTER}--type string in {\tt MCH}
into Hollerith characters in {\tt IV1} in {\tt A1} form.
\begin{DLtt}{123456}
\item [MCH] Input vector of {\tt NCH} characters,
either of type {\tt CHARACTER} or of type {\tt INTEGER}
holding Hollerith in {\tt Am} form.
\item [IV1] Output vector, {\tt NCH} words in {\tt A1} form, i.e.
a single Hollerith character per word with blank-fill.
\item [NCH] Total number of characters to copy.
\end{DLtt}
\begin{verbatim}
    CALL UHTOC(IVI,I,CHV,NCH)
\end{verbatim}
copies the Hollerith characters in {\tt IVI} into the
{\tt CHARACTER} variable {\tt CHV}.
\begin{DLtt}{123456}
\item [IVI] Input vector of {\tt NCH} Hollerith characters
with {\tt I} characters stored per word in Ai form.
\item [I] Number of Hollerith characters to take from each word of
{\tt IVI}. If {\tt I} is larger than the maximum possible number of
Hollerith characters per word this maximum will be used instead.
\item [CHV] Output variable of type {\tt CHARACTER} to receive {\tt NCH}
characters.
\item [NCH] Number of characters to copy. If the
{\tt CHARACTER} variable {\tt CHV} is of length greater than {\tt NCH}
trailing characters will not be changed.
\end{DLtt}
\begin{verbatim}
    CALL UH1TOC(IV1,CHV,NCH)
\end{verbatim}
concentrates a Hollerith string in {\tt A1} form
into the {\tt CHARACTER} variable {\tt CHV}.
\begin{DLtt}{123456}
\item [IV1] Input vector of {\tt NCH} words containing one Hollerith
character each in A1 form.
\item [CHV] Output variable of type {\tt CHARACTER} to receive {\tt NCH}
characters.
\item [NCH] Total number of characters to copy. If the
variable {\tt CHV} is of length greater than
{\tt NCH} trailing characters will not be changed.
\end{DLtt}
\Errorh
$\mathtt{NCH \leq 0}$ acts as do-nothing.
\Examples
({\tt b} = blank).
\begin{verbatim}
1)    CALL UBLOW(11HABCDEFGHIJK,V,11)
\end{verbatim}
fills {\tt V}: $\mathtt{V(1) = 1HA, \ldots, V(11) = 1HK}$,
with blank padding of each word.
\begin{verbatim}
2)    CALL UBUNCH(V,X,11)
\end{verbatim}
gives the inverse transformation, thus on the CDC 7600 $({\tt m=10})$:
\begin{verbatim}
      X(1) = 10ABCDEFGHIJ, X(2) = 10HKbbbbbbbb
\end{verbatim}
\begin{verbatim}
3)    CALL UTRANS(X,Y,11,99,4)
\end{verbatim}
copies the continuous {\tt X} string to {\tt A4} representation in
{\tt Y}:
\begin{verbatim}
      Y(1) = 4HABCD, Y(2) = 4HEFGH, Y(3) = 4HIJKb
\end{verbatim}
with blank padding if $\mathtt{m > 4}$.
\begin{verbatim}
4)    CALL UTRANS(Y,X,11,4,99)
\end{verbatim}
gives the inverse of example {\tt 3).}
\begin{verbatim}
5)    CALL UTRANS(V,X,11,1,99)
\end{verbatim}
gives the same result as example {\tt 2)}, but is much slower.
\begin{verbatim}
6)    DIMENSION V(4)
      CHARACTER*14 C/'THIS IS A TEST'/
      CALL UCTOH(C,V,4,14)
\end{verbatim}
copies the {\tt CHARACTER} string in {\tt C} into {\tt V} such that
\begin{verbatim}
      V(1) = 4HTHIS, V(2) = 4HbISb, V(3) = 4HAbTE, V(4) = 4HSTbb
\end{verbatim}
\begin{verbatim}
7)    DIMENSION V(4)
      CHARACTER*14 C
      DATA V /14HTHIS IS A TEST/   or   DATA V /4HTHIS,4H IS ,4HA TE,2HST/
      CALL UHTOC(V,4,C,14)
\end{verbatim}
copies the Hollerith strings in {\tt V} into {\tt C} such that
{\tt C='THIS IS A TEST'}.
\begin{verbatim}
8)    DIMENSION V(4)
      CHARACTER*4 C/'TEST'/
      CALL UCTOH1(C,V,4)
\end{verbatim}
copies the {\tt CHARACTER}--string in {\tt C} into {\tt V} such that
\begin{verbatim}
      V(1) = 4HTbbb, V(2) = 4HEbbb, V(3) = 4HSbbb, V(4) = 4HTbbb
\end{verbatim}
\begin{verbatim}
9)    DIMENSION V(4)
      CHARACTER*4 C
      DATA V/1HT,1HE,1HS,1HT/
      CALL UH1TOC(V,C,4)
\end{verbatim}
copies the Hollerith characters in {\tt V} into the {\tt CHARACTER}
string {\tt C} such that {\tt C='TEST'}.
\\ $\bullet$
