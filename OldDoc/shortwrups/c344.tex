% 20 may 1992  mg
\Version{CBSJA}                    \Routid{C344}
\Keywords{COMPLEX BESSEL FUNCTION NON-INTEGER ORDER}
\Author{K.S. K\"olbig}              \Library{MATHLIB}
\Submitter{}                         \Submitted{24.01.1986}
\Language{Fortran}                   \Revised{15.03.1993}
\Cernhead{Bessel Functions J with Complex Argument and Non-Integer Order}
Subroutine subprograms {\tt CBSJA}, {\tt WBSJA} and {\tt WQBSJA}
calculate a sequence of Bessel functions
$$J_{a+n}(z),$$
for complex arguments $z$, $ 0\leq a<1$, and
$n=0,1,2,\ldots,N$.
\par
The quadruple-precision version {\tt WQBSJA} is available only on
computers which support a {\tt COMPLEX*32} Fortran data type.
\Structure
{\tt SUBROUTINE} subprograms\\
User Entry Names: \Rdef{CBSJA}, \Rdef{WBSJA}, \Rdef{WQBSJA}  \\
Files Referenced: {\tt Unit 6} \\
External References: \Rind{GAMMA}{C302}, \Rind{DGAMMA}{C302}, 
\Rind{QGAMMA}{C302}, \Rind{MTLMTR}{N002}, \Rind{ABEND}{Z035}
\Usage
{\bf Single-precision version:}
\begin{verbatim}
    CALL CBSJA(Z,A,NL,ND,CB)
\end{verbatim}
\begin{DLtt}{1234}
\item[Z] ({\tt COMPLEX}) Argument $z$.
\item[A] ({\tt REAL}) Order $a$ of the first Bessel
function in the computed sequence.
\item[NL] ({\tt INTEGER}) Specifies the order $a+\mathtt{NL}$ of the last
Bessel function in the computed sequence.
\item [ND] ({\tt INTEGER}) Requested number of correct
significant decimal digits.
\item[CB] ({\tt COMPLEX}) One-dimensional array with dimension
(0:\texttt{d}) where $\mathtt{d \ge NL}$. \\
On exit, {\tt CB(n)}, $\mathtt{(n=0,1,2,\ldots,NL)}$,
contains $J_{a+\mathtt{n}}(\mathtt{Z})$.
\end{DLtt}
{\bf Double-precision version:}
\begin{verbatim}
    CALL WBSJA(Z,A,NL,ND,CB)
\end{verbatim}
where {\tt A} is of type {\tt DOUBLE PRECISION}, {\tt Z} and {\tt CB}
are of type {\tt COMPLEX*16}.
\par
On computers whose Fortran compiler does not support {\tt COMPLEX*16}
arithmetic (e.g. CDC and Cray) the storage conventions for
{\tt Z} and {\tt CB} are as follows:
\begin{DLtt}{1234}
\item [Z]({\tt DOUBLE PRECISION}) Array of declared dimension {\tt (2)}
containing Re $\mathtt{Z}$ in {\tt Z(1)} and Im $\mathtt{Z}$ in
{\tt Z(2)}.
\item[CB] ({\tt DOUBLE PRECISION}) Two-dimensional array with dimensions
{\tt (2,0:d)} where $\mathtt{d \ge NL}$. On exit, {\tt CB(1,n)}
contains Re $J_{a+\mathtt{n}}(\mathtt{Z})$ and
{\tt CB(2,n)} contains Im $J_{a+\mathtt{n}}(\mathtt{Z})$,
$\mathtt{(n=0,1,2,\ldots,NL)}$.
\end{DLtt}
{\bf Quadruple-precision version:}
\begin{verbatim}
    CALL WQBSJA(Z,A,NL,ND,CB)
\end{verbatim}
where {\tt A} is of type {\tt REAL*16}, {\tt Z} and {\tt CB}
are of type {\tt COMPLEX*32}.
\Method
The method is an extension to complex arguments of a
variant of Miller's backwards recurrence algorithm (see Ref. 1).
The requested accuracy is obtained, when possible, by a judicious
choice of the initial value of the recursion index, together with at
least one repetition of the recursion with this index increased by 5.
\Restrict
Im $\mathtt{Z \ne 0}$ if Re $\mathtt{Z < 0}$,
\ $\mathtt{0 \leq A < 1}$, \ $\mathtt{0 \leq NL \leq 100}$.
\newpage
\Accuracy
If {\tt Z} does not lie on the real axis, the requested
accuracy is usually obtained. There may be a loss of accuracy
if {\tt A} is close to 0 or 1, and in other special situations.
\Errorh
Error {\tt C344.1}: $\mathtt{Z=X+iY}$ with $\mathtt{X \le 0}$
and $\mathtt{Y=0}$.\\
Error {\tt C344.2}: $\mathtt{A<0}$ or $\mathtt{A \ge 1}$. \\
Error {\tt C344.3}: $\mathtt{NL<0}$ or $\mathtt{NL>100}$. \\
Error {\tt C344.4}: Difficulties of convergence. Try smaller
$\mathtt{|ND|}$. \\
In all cases, a message is written on
{\tt Unit 6}, unless subroutine {\tt MTLSET} (N002) has been called.
If Error 1 to 3 occurs, the initial contents of array {\tt CB}
is left unchanged.
If the requested accuracy cannot be obtained after the initial
recursion index has been increased fifty times (Error 4),
the contents of array {\tt CB} is undefined.
\Source
The subprogram is based on Algol procedures described in Ref. 1.
\Refer
\begin{enumerate}
\item W. Gautschi, Algorithm 236: Bessel functions of the first kind,
Collected Algorithms from CACM (1965)
\end{enumerate}
$\bullet$
