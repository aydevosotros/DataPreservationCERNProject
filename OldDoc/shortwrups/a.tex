\documentclass[]{article}
\usepackage{amssymb,amsmath}
\begin{document}
\section{Complete Elliptic Integrals of First, Second, and Third
Kind}
\begin{eqnarray*}
\mathbf{G}(k',p,a,b) & = & \displaystyle
\int_0^\infty \frac{a+b\xi^2}{(1+p\xi^2)\sqrt{(1+\xi^2)(1+{k'}^2\xi^2)}}
\, d\xi  \\[3mm]
& = & \displaystyle \int_0^{\pi/2} \frac{a \cos^2 \phi + b \sin^2 \phi}
{\cos^2 \phi + p \sin^2 \phi} \frac{d\phi}
{\sqrt{\cos^2 \phi + {k'}^2 \sin^2 \phi}} \qquad ({k'}^2 > 0).
\end{eqnarray*}
For $p < 0$, this integral is defined by its principal value.
See {\bf Notes} for special cases.

{\bf The functions K(k) and E(k):}
\begin{eqnarray*}
\mathrm{K}(k) & = & \displaystyle \int_0^{\pi/2}
\frac{d\psi}{\sqrt{1-k^2\sin^2 \psi}} \qquad (|k| < 1), \\
\mathrm{E}(k) & = & \displaystyle \int_0^{\pi/2}
\sqrt{1-k^2\sin^2 \psi} \, d\psi \qquad (|k| \le 1).
\end{eqnarray*}

Other common definitions of the complete elliptic integrals and their
relations to $\mathbf{F}_1^*$, $\mathbf{F}_2^*$, $\mathbf{F}_3^*$ are
listed here for convenience ($k^2+{k'}^2 = 1$): \\[2mm]
{\bf First kind:}
$$ \begin{array}{rcl}
F(k,\pi/2) & = & \mathrm{K}(k) \quad = \quad \mathbf{F}_1^*(k') \qquad
(|k| < 1), \\[6mm]
\widehat{F}(1,k) & = & \displaystyle \int_0^1
\frac{d\eta}{\sqrt{(1-\eta^2)(1-k^2\eta^2)}} \quad = \quad
\mathbf{F}_1^*(k') \qquad (|k| < 1).
\end{array} $$
{\bf Second kind:}
$$ \begin{array}{rcl}
E(k,\pi/2) & = & \mathrm{E}(k) \quad = \quad \mathbf{F}_2^*(k',1,{k'}^2)
\qquad (|k| \le 1), \\[6mm]
\widehat{E}(1,k) & = & \displaystyle \int_0^1
\sqrt{\frac{1-k^2 \eta^2}{1-\eta^2}} \, d\eta \quad = \quad
\mathbf{F}_2^*(k',1,{k'}^2) \qquad (|k| \le 1).
\end{array} $$
{\bf Third kind:}
$$ \begin{array}{rclcl}
\Pi(\pi/2,h,k) & = & \displaystyle \int_0^{\pi/2}
\frac{d\psi}{(1+h\sin^2 \psi)\sqrt{1-k^2\sin^2 \psi}} & = &
\mathbf{F}_3^*(k',h+1) \qquad (|k| < 1), \\[6mm]
\widehat{\Pi}(1,h,k) & = & \displaystyle \int_0^1
\frac{d\eta}{(1+h\eta^2)\sqrt{(1-\eta^2)(1-k^2\eta^2)}} & = &
\mathbf{F}_3^*(k',h+1) \qquad (|k| < 1).
\end{array} $$

{\tt FUNCTION} subprograms \\
User Entry Names:

The redundant parameter {\tt AK2} in {\tt RELI3C} and {\tt DELI3C}
permits improved accuracy when $k^2$ is small, i.e. $k' \approx 1$. In
this case, $\mathtt{AK2} = k^2$ should be calculated using
higher-precision arithmetic and then truncated before calling the
subprogram.
Special examples are
$$\begin{array}{rcl}
\mathrm{K}(k)                           & = & \mathbf{G}(k',1,1,1), \\[1mm]
\mathrm{E}(k)                           & = & \mathbf{G}(k',1,1,{k'}^2)\\[2mm]
(\mathrm{K}(k)-\mathrm{E}(k))/k^2       & = & \mathbf{G}(k',1,0,1), \\[3mm]
(\mathrm{K}(k)-{k'}^2\mathrm{E}(k))/k^2 & = & \mathbf{G}(k',1,1,0), \\[4mm]
\Pi(h,k)                                & = & \mathbf{G}(k',h+1,1,1),\\[5mm]
(\mathrm{K}(k)-\Pi(h,k))/h              & = & \mathbf{G}(k',h+1,0,1),
\end{array} $$
If $ab \ge 0$ then $\mathbf{G}$ will evaluate any linear
combination of K$(k)$, E$(k)$, $\Pi(h,k)$ without cancellation
(such as would occur, for example, if (K$(k)-$E$(k))/k^2$ were to be
computed from values of K$(k)$ and E$(k)$ which had been computed
separately.

Other functions which can be represented by $\mathbf{G}$ are the Jacobian
Zeta function $\mathbf{Z}(\Phi,k)$ and the Heuman Lambda function
$\Lambda_0(\Phi,k)$ (see Ref. 5):
$$\begin{array}{rcl@{\qquad}l}
\mathbf{Z}(\Phi,k) & = & \displaystyle k^2 \,
\frac{\sin \Phi \cos \Phi}{\mathrm{K}(k)} \, \mathbf{G}(k',q,0,\sqrt{q})
& (q = \cos^2 \Phi + {k'}^2 \sin^2 \Phi) \\[4mm]
\Lambda_0(\Phi,k) & = & \displaystyle \frac{2}{\pi}
\sqrt{q} \sin \Phi \ \mathbf{G}(k',q,1,{k'}^2) &
(q = 1 + k^2 \tan^2 \Phi).
\end{array} $$
{\it (Quoted from Ref. 3, slightly modified)}.\\

The subprograms for $\mathbf{F}_1^*$, $\mathbf{F}_2^*$ are based on the
Algol60 procedures {\it cel1, cel2} in Ref. 1, those for
$\mathbf{F}_3^*$ on {\it cel3} in Ref. 2, and those for $\mathbf{G}$
on {\it cel} in Ref. 3.

\begin{enumerate}
\item R. Bulirsch, Numerical calculation of elliptic integrals and
elliptic functions, Numer. Math. {\bf 7} (1965) 78--90.
\item R. Bulirsch, Numerical calculation of elliptic integrals and
elliptic functions II, Numer. Math. {\bf 7} (1965) 353--354.
\item R. Bulirsch, Numerical calculation of elliptic integrals and
elliptic functions III, Numer. Math. {\bf 13} (1969) 305--315.
\item W.J. Cody, Chebyshev approximations for the complete elliptic
integrals $K$ and $E$, Math. Comp. {\bf 19} (1965) 105--112.
\item P.F. Byrd and M.D. Friedman, Handbook of elliptic integrals
for engineers and scientists, 2nd ed., Springer-Verlag Berlin (1971)
33--37.
\end{enumerate}
\end{document}
