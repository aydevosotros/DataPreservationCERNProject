\Version{FREQ}         \Routid{C301}
\Keywords{ERROR NORMAL FREQUENCY FUNCTION}
\Author{G.A. Erskine}             \Library{MATHLIB}
\Submitter{K.S. K\"olbig}         \Submitted{07.06.1992}
\Language{Fortran}                \Revised{}
\Cernhead{Normal Frequency Function}
Function subprograms {\tt FREQ} and {\tt DFREQ} compute the
normal frequency function
$$ \begin{array}{r@{ \quad = \quad }l}
\mathrm{freq}(x) & \displaystyle \frac{1}{\sqrt{2\pi}}\int_{-\infty}^x
e^{-\frac{1}{2}t^2}\, dt,
\end{array} $$
defined for all values of the real argument $x$.
\par
On CDC and Cray computers, the double-precision version {\tt DFREQ} is
not available.
\Structure
{\tt FUNCTION} subprograms\\
User Entry Names: \Rdef{FREQ}, \Rdef{DFREQ}
\Usage
In any arithmetic expression,
\begin{center}
{\tt FREQ(X)} \quad or \quad {\tt DFREQ(X)}
\quad has the value \quad freq({\tt X}),
\end{center}
where {\tt FREQ} is of type {\tt REAL}, {\tt DFREQ} is of type
{\tt DOUBLE PRECISION}, and {\tt X} has the same type as the function
name.
\Method
Computation by rational Chebyshev approximation for the error function,
using the formula
$$ \mathrm{freq}(x) \ = \ \left\{ \begin{array}{l@{\qquad}l}
\textstyle \frac{1}{2} + \frac{1}{2}\, \mathrm{erf}\,(x/\sqrt{2}) &
(x\geq 0), \\[1mm]
\textstyle \frac{1}{2}\, \mathrm{erfc}\,(|x|/\sqrt{2}) & (x<0).
\end{array} \right. $$
\Accuracy
{\tt FREQ} has full single-precision accuracy (slightly less on CDC and
Cray computers). {\tt DFREQ} has an accuracy  of 15 significant digits.
\Refer
\begin{enumerate}
\item W.J. Cody, Rational Chebyshev approximations for the error
function, Math. Comp. {\bf 22} (1969) 631--637.
\end{enumerate}
$\bullet$
