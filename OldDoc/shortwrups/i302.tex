%  20 feb 95   ksk
\Version {FFREAD}                            \Routid{I302}
\Keywords{FORMAT FREE INPUT}
\Author{See below}                     \Library{PACKLIB}
\Submitter{J.C. Lassalle}             \Submitted{30.01.1980}
\Language{Fortran}                   \Revised{17.12.1991}
\Cernhead {Format-Free Input Processing}
{\bf Authors}: R. Brun, R. Hagelberg, M. Hansroul, I. Ivanchenko,
J.C. Lassalle, G. Misuri, J. Vorbrueggen \\[3mm]
\begin{center}
\fbox{\parbox{120mm}{\OBSOLETE
Please note that this routine has been obsoleted in CNL 219. Users are
advised not to use it any longer and to replace it in older programs.
No maintenance for it will take place and it will eventually disappear.
\\[3mm]
Suggested replacement: {\tt KUIP} (I202) }}
\end{center}
{\tt FFREAD} provides the user with a facility for free-format data
input, providing a suitable tool to transmit and/or modify
variables at run-time without recompilation.
\Structure
{\tt SUBROUTINE} subprograms \\
User Entry Names: \Rdef{FFREAD}, \Rdef{FFINIT}, \Rdef{FFSET},
\Rdef{FFKEY}, \Rdef{FFGO}, \Rdef{FFGET}\\
Internal Entry Names: {\tt FFCARD}, {\tt FFFIND}, {\tt FFGOR},
{\tt FFSKIP}, {\tt FFUPCA}\\
Files Referenced: Input, Output (both default or user defined)\\
External References:
\Rind{UCOPY}{V301}, \Rind{UCTOH}{M409}, \Rind{UHTOC}{M409},
\Rind{FFUSER} (optionally user-supplied)
\Usage
See {\bf Long Write-up}.
\\ $\bullet$
