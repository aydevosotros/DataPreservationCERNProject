\Version {FIO999}                              \Routid{M440}
\Keywords{FORTRAN READ WRITE SIMULATION INTERNAL INPUT OUTPUT}
\Author{J. Ehrman (SLAC)}     \Library{KERNLIB, IBM Fortran 4 only}
\Submitter{W. Simon}                \Submitted{01.08.1976}
\Language{Assembler}  \Revised{01.12.1980}
\Cernhead {Fortran Read/Write Simulation and Internal Input/Output
Buffer}
\begin{center}
\fbox{\parbox{120mm}{
\begin{center}
{\bf OBSOLETE}
\end{center}
Please note that this routine has been obsoleted in CNL 204. Users are
advised not to use it any longer and to replace it in older programs.
No maintenance for it will take place and it will eventually disappear.
}}
\end{center}
{\tt FIO999} allows a Fortran programmer to perform I/O
conversions to and from specified buffer areas in memory without an
accompanying I/O
operation on a physical device. Each call to {\tt FIO999} establishes a
correspondence between the specified unit number and buffer area. All
succeeding {\tt READ} and {\tt WRITE} statements specifying that unit
number transmit to and from the associated buffer rather than from an I/O
device with that number. A call to {\tt FIO99X} causes all previously
established correspondences to be deleted.
\Structure
{\tt SUBROUTINE} subprogram \\
User Entry Names: \Rdef{FIO999}, \Rdef{FIO99X}\\
External References: \Rind{IBCOM\#}, \Rind{FIOCS\#}
(IBM Fortran library)
\Usage
\begin{verbatim}
    CALL FIO999(NUNIT,BUFLOC,IRECLG,IBUFLG,&nnn)
\end{verbatim}
\begin{DLtt}{12345678}
\item [NUNIT] ({\tt INTEGER}) "Unit number" to be associated with a
core-memory buffer area.
\item [BUFLOC] ({\tt array name}) First word of buffer area (must be
located on a full word boundary).
\item [IRECLG]({\tt INTEGER}) Length in fullwords of each record in the
buffer.
\item [IBUFLG]({\tt INTEGER}) Length in fullwords of the entire buffer
(must be an integral multiple of {\tt IRECLG}).
\item [nnn] Statement number to which control will be returned
if errors detected in the other parameters.
\end{DLtt}
\begin{verbatim}
    CALL FIO99X
\end{verbatim}
Cancels the effect of any previous calls to {\tt FIO999}.
\newpage
\Examples
Print two 80-character records on {\tt Unit 6} using {\tt "Unit 88"}
as the internal unit.
\begin{verbatim}
      DIMENSION OUTPUT(40)
      CALL FIO999(88,OUTPUT(1),20,40,&3)
      I=10
      J=15
      WRITE(88,10) I,J
      WRITE(6,20) OUTPUT
   10 FORMAT(5X,'I=',I5/5X,'J='I5)
   20 FORMAT (20A4)
    3 STOP
\end{verbatim}
To do the equivalent of {\tt ENCODE/DECODE:}
assuming that an array {\tt WORDS(4)} contains some Hollerith text
with three embedded integers. {\tt WORDS} might be:
\begin{verbatim}
    NUM1- 23- 20-  7
\end{verbatim}
then to access the three numbers one could code the following:
\begin{verbatim}
      DIMENSION WORDS(4)
      READ(5,10) WORDS
   10 FORMAT(4A4)
C     WHEN THE THREE INTEGERS IN WORDS ARE NEEDED
      CALL FIO999(88,WORDS(1),4,4,&100)
      READ(88,20) I,J,K
      WRITE(6,30) I,J,K
   20 FORMAT (5X,I3,2(1X,I3))
   30 FORMAT ('DECODED INTEGERS WERE=',3I5)
      CALL FIO99X
  100 STOP
\end{verbatim}
$\bullet$
