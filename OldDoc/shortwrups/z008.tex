% 20 may 1992  mg
\Version {TIMAL}                    \Routid{Z008}
\Keywords{IBM JOB TIMAD TIMAL TIMAST TIMAX TIME}
\Author{A. Berglund, R. Matthews}      \Library{KERNLIB, CERN IBM only}
\Submitter{}                            \Submitted{01.12.1981}
\Language{Assembler}                    \Revised{20.06.1985}
\Cernhead {Job Time in IBM Accounting Units}
{\tt TIMAL} interfaces with the system of any CERN IBM-like machine to
obtain the central processor time used by and remaining to the job
and then converts these times to CERN IBM accounting seconds.
\par
Currently this unit is one second of IBM 370/168-3 CPU time.
\Structure
{\tt SUBROUTINE} subprograms \\
User Entry Names: \Rdef{TIMAD}, \Rdef{PTIMAL}, \Rdef{TIMAST},
\Rdef{TIMAX}\\
External References: Machine dependent
\Usage
\begin{verbatim}
    CALL TIMAX(T)
\end{verbatim}
returns execution time in CERN IBM accounting seconds used by the job
so far; {\tt T} is the normalised central processor time in seconds,
a {\tt REAL} number with fractional part.
\begin{verbatim}
    CALL TIMAL(T)
\end{verbatim}
returns execution time remaining in CERN IBM accounting seconds
until time-limit (user time-limits are always specified in accounting
seconds); {\tt T} in  seconds as for {\tt TIMAX}.
\begin{verbatim}
    CALL TIMAD(T)
\end{verbatim}
returns execution time interval in CERN IBM accounting seconds
since last call to {\tt TIMAD}; {\tt T} in seconds as for {\tt TIMAX}.
\begin{verbatim}
    CALL TIMAST(TLIM)
\end{verbatim}
This routine is necessary to initialise the timing operations
in the interactive mode of VM-CMS. In other systems (including VM-CMS
batch) it is a dummy do-nothing routine.
\par
It must be called once (subsequent calls are ignored) before any
calls to {\tt TIMAX} and {\tt TIMAL}. Before this routine is called
{\tt TIMAX} will return zero and {\tt TIMAL} will return
$999.0 \times$ a multiplicative constant. {\tt TLIM} is an input floating
point value which will be used inside {\tt TIMAL} as if it were the job
time-limit in CERN IBM accounting seconds. The first call to {\tt TIMAST}
also establishes the time origin for subsequent calls to {\tt TIMAX} and
{\tt TIMAL}.
\Accuracy
See the remarks in {\tt DATIME} (Z007). A consequence of the 1 second
resolution of {\tt TIMEL} is that {\tt TIMAL} has a resolution of about
8 secs on the IBM 3090.
\newpage
\Notes
\begin{enumerate}
\item The information returned by these routines is obtained by
a system request. On some machines this is expensive in real time, so
one should avoid too many calls, to {\tt TIMAL} in particular.
\item These routines are identical to those in {\tt DATIME} (Z007)
except for a multiplicative constant. These constants are averages over
a range of CERN work and may not be valid for an individual job.
The current values of these constants are: \\[1mm]
\begin{tabular}{@{\hspace*{10mm}}l@{\qquad}l}
Machine       & Constant \\[1mm]
IBM 168-3     & 1.0   \\
Siemens 7-890 & 5.0     \\
IBM 3090      & 8.0    \\
\end{tabular}
\item These routines will not work outside the CERN version of the
IBM operating system and VM-CMS.
\end{enumerate}
$\bullet$
