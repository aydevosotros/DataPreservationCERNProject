% 20 may 1992  mg
\Version {A1MANI}                    \Routid{M410}
\Keywords{MANIPULATE BCD STRING}
\Author{J. Zoll}                     \Library{KERNLIB}
\Submitter{C. Letertre}              \Submitted{04.09.1972}
\Language{FORTRAN or Assembler}      \Revised{15.09.1978}
\Cernhead {Manipulating BCD Strings in A1 Representation}
{\tt A1MANI} manipulates and interprets BCD-strings in A1 representation.
\Structure
{\tt SUBROUTINE} and {\tt FUNCTION} subprograms \\
User Entry Names: \Rdef{ULEFT}, \Rdef{URIGHT}, \Rdef{USET},
\Rdef{IUFNBL}, \Rdef{IULOOK}, \Rdef{IUFORW}, \Rdef{IUBACK},
\Rdef{IUEND} \\
External References: \Rind{IUCOMP} (V304), \Rind{UBUNCH} (M409) \\
{\tt COMMON} Block  Names and Lengths:  {\tt /SLATE/ 40}
\Usage
Let the vector {\tt CH} contain {\tt BCD} characters in {\tt A1} format
with blank padding. {\tt CH(J)} is called the $j$-th character. A section
of this vector is specified by {\tt (JL,JR)} where {\tt CH(JL)} is the
first or left-most, and {\tt CH(JR)} the last or right-most character.
\par
The {\tt COMMON} block {\tt /SLATE/ ND,NE,\ldots} holds certain search
parameter, which are set by some of these routines. These
parameters can also be accessed with $\mathtt{NE=IUEND(ND)}$, but this is
not a recommended procedure.
\begin{verbatim}
    CALL ULEFT(CH,JL,JR)
\end{verbatim}
left-justifies the section {\tt (JL,JR)}.
\begin{DLtt}{12345678}
\item [ND] Number of non-blank characters in the field.
\item [CH(NE)] First blank character after left-justifying,
or $\mathtt{NE=JR+1}$ if there are no blanks.
\end{DLtt}
\begin{verbatim}
    CALL URIGHT(CH,JL,JR)
\end{verbatim}
right-justifies the section {\tt (JL,JR)}.
\begin{DLtt}{12345678}
\item [ND] Number of non-blank characters in the field.
\item [CH(NE)] Last blank character after right-justifying,
or $\mathtt{NE=JL-1}$ if there are no blanks.
\end{DLtt}
\begin{verbatim}
    CALL USET(INT,CH,JL,JR)
\end{verbatim}
writes the positive integer {\tt INT} into the area {\tt (JL,JR)}
right-justified. If this integer is too large, the most significant
characters are lost. Unused positions are not cleared to blank.
\begin{DLtt}{12345678}
\item [ND] Number of digits which have been set.
\item [CH(NE)] Right-most unchanged character.
\end{DLtt}
\begin{verbatim}
    J=IUFNBL(CH,JL,JR)
\end{verbatim}
returns in {\tt J} the address of the first non-blank character in
the area {\tt (JL,JR)}, or {\tt JR+1} if none.
\begin{verbatim}
    INTEXT=IULOOK(N,CH,JL,JR)
\end{verbatim}
takes the first {\tt N} (at most) non-blank characters in the section
{\tt (JL,JR)} and returns them as the function value. {\tt N} must not
exceed the BCD-capacity of one computer word.
\newpage
\begin{verbatim}
    I=IUFORW(CH,JL,JR)
\end{verbatim}
reads the unsigned integer whose BCD-representation starts at
{\tt CH(JL)} and stops on the first non-numeric character or at
{\tt CH(JR)}. Blanks are ignored.
\begin{DLtt}{12345678}
\item [ND] Number of digits read.
\item [CH(NE)] First non-numeric, non-blank character in the field.
$\mathtt{NE=JR+1}$ if pure numeric or blank.
\end{DLtt}
\begin{verbatim}
    I=IUBACK(CH,JL,JR)
\end{verbatim}
reads the integer whose BCD-representation starts just after the last
non-numeric character in the field and stops at {\tt CH(JR)}.
Blanks are ignored.
\begin{DLtt}{12345678}
\item [ND] Number of digits read.
\item [CH(NE)] Last non-numeric, non-blank character in the field.
$\mathtt{NE=JL-1}$ if pure numeric or blank.
\end{DLtt}
\Notes
These routines were written to handle Hollerith with Fortran 4;
they should not be used with new programs.
The routines of {\tt CHPACK, M432} can be used instead.
\\ $\bullet$
