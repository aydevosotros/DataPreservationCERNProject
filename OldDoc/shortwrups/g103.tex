\Version{TKOLMO}                               \Routid{G103}
\Keywords{DISTRIBUTION TEST KOLMOGOROV COMPATIBLE COMPARE}
\Author{F. James}                              \Library{MATHLIB}
\Submitter{}                                   \Submitted{15.02.1991}
\Language{Fortran}                             %\Revised{}
\Cernhead{Kolmogorov Test}
Subroutine subprogram {\tt TKOLMO} tests whether two one-dimensional
sets of points are compatible with coming from the same parent
distribution, using the Kolmogorov test. That is, it is used to compare
two experimental distributions of unbinned data.
\Structure
{\tt SUBROUTINE} subprogram \\
User Entry Name: \Rdef{TKOLMO}            \\
External routine referenced: \Rdef{PROBKL} (G102)
\Usage
\begin{verbatim}
    CALL TKOLMO(A,NA,B,NB,PROB)
\end{verbatim}
\begin{DLtt}{12345}
\item[A,B] ({\tt REAL}) One-dimensional arrays of length {\tt NA},
{\tt NB}, respectively. The elements of {\tt A} and {\tt B} must
be given in ascending order. (This can be accomplished, for example,
by using {\tt FLPSOR} (M103)).
\item[NA,NB] ({\tt INTEGER}) The number of points in {\tt A} and
{\tt B}, respectively.
\item[PROB] ({\tt REAL}) A calculated confidence level which gives a
statistical test for compatibility of {\tt A} and {\tt B}.
\end{DLtt}
Values of {\tt PROB} close to zero are taken as indicating a small
probability of compatibility.  For two point sets drawn randomly
from the same parent distribution, the value of {\tt PROB} should be
uniformly distributed between zero and one.
\Method
The Kolmogorov test is used.  The test statistic is the maximum deviation
between the two integrated distribution functions, multiplied by the
normalizing factor $\sqrt{MN/(M+N)}$, where $M$ and $N$ are the
numbers of points in the two samples.
\Accuracy
Approximately seven digits are correct.
\Notes
Probabilities smaller than $10^{-40}$ are set to zero. However, the
method has a statistical meaning only for "large" $M$ and $N (>10)$.
\Refer
\begin{enumerate}
\item W.T. Eadie, D. Drijard, F.E. James, M. Roos and B. Sadoulet,
Statistical Methods in Experimental Physics, (North-Holland, Amsterdam
1971) 269-271.
\end{enumerate}
$\bullet$
