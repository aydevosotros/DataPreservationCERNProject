\Version{TRACEQ}                         \Routid{N105}
\Keywords{PRINT TRACE OUTPUT TRACE-BACK}
\Author{J. Zoll}                          \Library{KERNLIB}
\Submitter{}                              \Submitted{01.12. 1973}
\Language{Fortran}                         \Revised{15.09.1978}
\Cernhead {Print Trace-Back}
{\tt TRACEQ} prints the Fortran trace-back leading to
{\tt TRACEQ}. The maximum number of trace-back levels is specified as an
argument. Fewer levels may be printed either because the main program
has been reached or because the trace-back linkage is invalid.
\Structure
{\tt SUBROUTINE} subprogram \\
User Entry Names: \Rdef{TRACEQ}\\
Internal Entry Names: {\tt TRAC1Q}, {\tt TRAC2Q}\\
Files Referenced: User defined\\
{\tt COMMON} Block Names and Lengths: {\tt /SLATE/ 40}
\Usage
\begin{verbatim}
    CALL TRACEQ(LUN,N)
\end{verbatim}
\begin{DLtt}{123456}
\item [LUN] Logical unit number of the print file, $\mathtt{LUN=0}$ is
accepted to mean the standard print file.
\item [N] Maximum number of trace-back levels to be printed.
\end{DLtt}
\Notes
The implementation of {\tt TRACEQ} depends on the machine; on some
machines this cannot be done at all and the routine is a dummy. On some
other machines the unit for printing or the number of levels printed is
not under program control.
\\ $\bullet$
