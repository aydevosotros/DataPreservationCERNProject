% 19 may 1995 ksk
\Version {RCHSUM}                            \Routid{E407}
\Keywords{SUM CHEBYSHEV SERIES}
\Author{K.S. K\"olbig}                      \Library{MATHLIB}
\Submitter{}                                \Submitted{24.01.1986}
\Language{Fortran}                     \Revised{15.11.1995}
\Cernhead {Summation of Chebyshev Series}
Function subprograms {\tt RCHSUM} and {\tt DCHSUM} compute, for real
arguments $x$ in the specified intervals, one of the following four sums:
$$ \begin{array}{r@{ \ = \ }l@{\qquad}l@{\mbox{\hspace{20mm}}}l}
S(x) & \displaystyle \sum^N_{n=0}c_nT_n(x) & (-1\leq x\leq 1) & (1)
\\ [4mm]
S(x) & \displaystyle \sum^N_{n=0}c_nT_{2n}(x) & (-1\leq x\leq 1) & (2)
\\ [4mm]
S(x) & \displaystyle \sum^N_{n=0}c_nT_{2n+1}(x) & (-1\leq x\leq 1) & (3)
\\ [4mm]
S(x) & \displaystyle \sum^N_{n=0}c_n T_n^*(x) & (0\leq x\leq 1) & (4)
\end{array} $$
where $T_n(x)$ is the Chebyshev polynomial of degree $n$ and
$T_n^*(x) = T_n(2x - 1)$.
\par
On CDC and Cray computers, the double-precision version {\tt DCHSUM}
is not available.
\Structure
{\tt FUNCTION} subprograms \\
User Entry Names: \Rdef{RCHSUM}, \Rdef{DCHSUM} \\
Obsolete User Entry Names: \Rdef{CHSUM} $\equiv$ \Rdef{RCHSUM}
\Usage
In any arithmetic expression,
\begin{center}
{\tt RCHSUM(MODE,C,N,X)} \qquad or \qquad {\tt DCHSUM(MODE,C,N,X)}
\end{center}
has the value of the sum selected by {\tt MODE}.
{\tt RCHSUM} is of type {\tt REAL}, and {\tt DCHSUM} is of type
{\tt DOUBLE PRECISION}. {\tt C} and {\tt X} have the same type as the
function name. {\tt MODE} and {\tt N} are of type {\tt INTEGER}.
\begin{DLtt}{123456}
\item[MODE] Type of sum to be evaluated $(\mathtt{MODE = 1,2,3,4})$.
\item[C] One-dimensional array with dimension {\tt (0:d)},
$\mathtt{d \geq N}$, containing the coefficients \\
$c_0,c_1,\ldots,c_N$.
\item[N] Limit $N$ of summation.
\item[X] Argument $x$.
\end{DLtt}
\Notes
Note that some authors use a different definition for the
constant term in (1), (2) and (4), i.e. $c_0/2$ instead of $c_0$.
Here, the definition of Ref. 1 is used.
\Refer
\begin{enumerate}
\item Y.L. Luke, Mathematical functions and their approximations,
(Academic Press, New York 1975)
\item C.W. Clenshaw, Chebyshev series for mathematical functions,
Mathematical Tables, Vol.5 (National Physical Laboratory, London, 1962).
\end{enumerate}
$\bullet$
 
