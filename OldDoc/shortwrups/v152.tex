% 13 mar 1996  ksk
\Version {FUNLUX}                    \Routid{V152}
\Keywords{DISTRIBUTION NUMBER RANDOM ARBITRARY}
\Author{F. James}                    \Library{MATHLIB}
\Submitter{}                         \Submitted{22.02.1996}
\Language{Fortran}                     %\Revised{}
\Cernhead {Random Numbers According to Any Function}
{\tt FUNLUX} generates random numbers distributed according to any
(one-dimensional) distribution $f(x)$. The distribution is supplied by
the user in the form of a {\tt FUNCTION} subprogram. If the distribution
is known as a histogram only, {\tt HISRAN} (V150) should be used instead.
\Structure
{\tt SUBROUTINE} subprograms \\
User Entry Names: \Rdef{FUNLUX}, \Rdef{FUNLXP}\\
Internal Entry Names: {\tt FUNPCT, FUNLZ}\\
Files Referenced: Printer\\
External References: \Rind{RADAPT}{D102}, \Rind{RANLUX}{V115},
user-supplied {\tt FUNCTION} subprogram \\
{\tt COMMON} Block Names and Lengths: {\tt /FUNINT/ 1}
\Usage
\begin{tabular}{@{\hspace*{8mm}}l@{\qquad}l}
{\tt CALL FUNLXP(F,FSPACE,XLOW,XHIGH)}  & (once for each function) \\
{\tt CALL FUNLUX(FSPACE,XRAN,LEN)}   & (for each vector of random numbers)
\end{tabular}
\begin{DLtt}{12345678}
\item[F] ({\tt REAL}) A name of a {\tt FUNCTION} subprogram
declared {\tt EXTERNAL} in the calling program. This subprogram must
calculate the (non-negative) density function $f(\mathtt{X})$, for all
{\tt X} in the interval $\mathtt{XLOW \leq X \leq XHIGH}$.
\item[FSPACE] ({\tt REAL}) One-dimensional array of length {\tt 200}.
\item[XLOW] ({\tt REAL}) Lower limit of the requested interval.
\item[XHIGH] ({\tt REAL}) Upper limit of the requested interval.
\item[XRAN] ({\tt REAL}) A vector of random numbers returned by
{\tt FUNRAN}.
\item[LEN]  ({\tt INTEGER}) Length of the vector {\tt XRAN}.
\end{DLtt}
A call to {\tt FUNLXP} calculates the percentiles of {\tt F} between
{\tt XLOW} and {\tt XHIGH} and stores them into the array {\tt FSPACE}.
\Method
In {\tt FUNLXP}, the 100 percentiles of the integral of $f(\mathtt{X})$
are calculated using a combination of
trapezoidal and Gaussian integration to a rather high accuracy,
which is printed out by {\tt FUNLXP}.
Then both the left-hand and right-hand 2 percentiles are expanded to
50 percentiles each in order to cater for functions with long tails.
If the desired accuracy is not obtained, a warning is printed in addition.
\par
Subroutine {\tt FUNLUX} finds the desired random number by calling
{\tt RANLUX} (V115) and doing a 4-point interpolation on {\tt FSPACE}
to transform the uniform random number to the distribution specified.
This method produces quite accurately distributed numbers even when
the function {\tt F} is badly skew or spiked as long as the width
of a spike is not less than 1/1000 of the total range.
\Errorh
An error message is printed
\begin{DLtt}{1234}
\item[] -- if the integral of the user-supplied function {\tt F} is zero
or negative,
\item[] -- if $\mathtt{XLOW \geq XHIGH}$,
\item[] -- if $\mathtt{F(X) < 0}$ somewhere between {\tt XLOW} and
{\tt XHIGH}.
\end{DLtt}
\newpage
\Notes
Some additional information which may be of use is contained in
\begin{verbatim}
    COMMON / FUNINT/ FINT
\end{verbatim}
After a call to {\tt FUNLXP}, {\tt FINT} contains the integral of
{\tt F} from {\tt XLOW} to {\tt XHIGH}.
\par
After a call to {\tt FUNLUX}, {\tt FINT} contains the integral of
{\tt F} from {\tt XLOW} to {\tt XRAN(LEN)}, divided by the total integral
to {\tt XHIGH} (i.e., it will be a number uniformly distributed
between zero and  one).
\\ $\bullet$
