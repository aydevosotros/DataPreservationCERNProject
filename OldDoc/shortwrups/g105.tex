\Version{GAUSIN}               \Routid{G105}
\Keywords{NORMAL FREQUENCY DISTRIBUTION GAUSS GAUSSIAN INVERSE}
\Author{K.S. K\"olbig}             \Library{MATHLIB}
\Submitter{}                \Submitted{01.12.1988}
\Language{Fortran}                  \Revised{15.03.1993}
\Cernhead{Inverse of Normal Frequency Function}
Function subprograms {\tt GAUSIN} and {\tt DGAUSN} calculate the inverse
$X(P)$ of the normal frequency function (Gaussian distribution)
$$ P(X) \ = \ \frac{1}{\sqrt{2\pi}} \
\int_{-\infty}^{X(P)}e^{-\frac{1}{2}t^2}dt $$
for real arguments $P$, where $0 < P < 1$.
\Structure
{\tt FUNCTION} subprogram \\
User Entry Name: \Rdef{GAUSIN}, \Rdef{DGAUSN} \\
Files Referenced: {\tt Unit 6} \\
External References: \Rind{MTLMTR}{N002}, \Rind{ABEND}{Z035}
\Usage
In any arithmetic expression,
\begin{center}
{\tt GAUSIN(P)} \quad has the value \quad $X(\mathtt{P})$,
\end{center}
where {\tt GAUSIN} and {\tt P} are of type {\tt REAL}.
\Method
The method is described in Ref. 1.
\Accuracy
 
\Accuracy
{\tt GAUSIN} (except on CDC and Cray computers)
has an accuracy of about six digits.
For most values of the argument {\tt P}, {\tt DGAUSN}
(and {\tt GAUSIN} on CDC and Cray computers) has an accuracy of
approximately one significant digit less than the machine precision.
\Errorh
Error {\tt G105.1}: $\mathtt{P \le 0}$ or $\mathtt{P \ge 1}$. \\
The function value is set equal to zero, and a message is written on
{\tt Unit 6}, unless subroutine {\tt MTLSET} (N002) has been called.
\Source
This subprogram is based on an Algol60 procedure published in Ref. 1.
\Refer
\begin{enumerate}
\item G.W. Hill and A.W. Davis, Algorithm 442, Normal Deviate,
Collected Algorithms from CACM (1973)
\end{enumerate}
$\bullet$
