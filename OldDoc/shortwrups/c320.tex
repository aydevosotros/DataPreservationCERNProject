%  05 jan 95   ksk
\Version{CELFUN}                       \Routid{C320}
\Keywords{COMPLEX JACOBIAN ELLIPTIC FUNCTION SN CN DN INVERSE INTEGRAL}
\Author{H.-H. Umst\"atter}                  \Library{MATHLIB}
\Submitter{K.S. K\"olbig}               \Submitted{30.01.1980}
\Language{Fortran}                      \Revised{07.06.1992}
\Cernhead{Jacobian Elliptic Functions sn, cn, dn for Complex Argument}
Function subprograms {\tt CELFUN} and {\tt WELFUN} calculate, for complex
argument $z$ and real modulus $k$, the Jacobian elliptic functions
$\mathrm{sn}(z,k)$, $\mathrm{cn}(z,k)$ and $\mathrm{dn}(z,k)$.
The function $\mathrm{sn}(z,k)$ is the inverse of the elliptic
integral of the first kind and is defined implicitly by
$$ z \ = \ \displaystyle \int_0^{\mbox{\small sn(\,{\it z,\,k}\,)}}
\frac{dw}{\sqrt{(1-w^2)(1-k^2w^2)}} \qquad (k^2 \leq 1). $$
The functions $\mathrm{cn}(z,k)$ and $\mathrm{dn}(z,k)$ are defined by
$$ \mathrm{sn}^2(z,k) + \mathrm{cn}^2(z,k) = 1, \quad
k^2 \mathrm{sn}^2(z,k) + \mathrm{dn}^2(z,k) = 1, \quad
\mathrm{cn}(0,k) = \mathrm{dn}(0,k) = 1. $$
For $k = 0$ and $k^2 = 1$ these functions are elementary:
$$ \mathrm{sn}(z,0) = \sin z, \quad
\mathrm{cn}(z,0) = \cos z, \quad \mathrm{dn}(z,0) = 1,$$
$$ \mathrm{sn}(z,\pm 1) = \tanh z, \quad
\mathrm{cn}(z,\pm 1) = \mathrm{dn}(z,\pm 1) = \mathrm{sech}\, z.$$
Note that the Jacobian elliptic functions are doubly-periodic in the
$z$-plane. For details see the relevant treatises or handbooks.
\par
The double-precision version {\tt WELFUN} is available only on computers
which support a {\tt COMPLEX*16} Fortran data type.
\Structure
{\tt SUBROUTINE} subprograms \\
User Entry Names: \Rdef{CELFUN}, \Rdef{WELFUN} \\
External References: \Rind{MTLMTR}{N002}, \Rind{ABEND}{Z035}
\Usage
For $\mathtt{t=C}$ (type {\tt COMPLEX}), $\mathtt{t=W}$ (type
{\tt COMPLEX*16}),
\begin{verbatim}
    CALL tELFUN(Z,AK2,SN,CN,DN)
\end{verbatim}
\begin{DLtt}{1234}
\item[Z] (type according to {\tt t}) The argument $z$.
\item[AK2] ({\tt REAL} for $\mathtt{t=C}$, {\tt DOUBLE PRECISION}
for $\mathtt{t=W}$)
The value of $k^2$ (the square of the modulus).
\item[SN] (type according to {\tt t}) On exit,
$\mathtt{SN}=\mathrm{sn}(\mathtt{Z},k)$.
\item[CN] (type according to {\tt t}) On exit,
$\mathtt{CN}=\mathrm{cn}(\mathtt{Z},k)$.
\item[DN] (type according to {\tt t}) On exit,
$\mathtt{DN}=\mathrm{dn}(\mathtt{Z},k)$.
\end{DLtt}
\Method
The Jacobian elliptic functions with complex argument $z=x+iy$
are computed from their re\-presentations in terms of Jacobian elliptic
functions with real arguments $x$ or $y$ (Ref. 1, formula 125.01).
See also the Short Write-up for {\tt ELFUN} (C318).
\newpage
\Accuracy
{\tt CELFUN} (except on CDC and Cray computers)
has full single-precision accuracy.
For most values of the arguments, {\tt WELFUN}
(and {\tt CELFUN} on CDC and Cray computers) has an accuracy of
approximately two significant digits less than the machine precision.
\Restrict
$|\mathrm{Re}\,z| \le 3 \mathrm{K}(k)$,
$|\mathrm{Im}\,z| \le 3 \mathrm{K}(k')$
where $k'=\sqrt{1-k^2}$ is the complementary modulus, and
$\mathrm{K}(x)$ is the complete elliptic integral of the first kind.
(See entries
{\tt RELIKC} and {\tt DELIKC} in {\tt RELI1C} (C347)).
\Errorh
Error {\tt C320.1:} $\mathtt{|AK2| > 1}$. The function value
is set equal to zero, and a message is written on {\tt Unit 6},
unless subroutine {\tt MTLSET} (N002) has been called.
\Refer
\begin{enumerate}
\item P.F. Byrd and M.D. Friedman, Handbook of elliptic integrals for
engineers and scientists, 2nd ed., Springer-Verlag Berlin (1971).
\end{enumerate}
$\bullet$
