%%%%%%%%%%%%%%%%%%%%%%%%%%%%%%%%%%%%%%%%%%%%%%%%%%%%%%%%%%%%%%%%%%%
%                                                                 %
%   MINUIT User Guide -- LaTeX Source                             %
%                                                                 %
%   Chapter 5                                                     %
%                                                                 %
%   The following external EPS files are referenced:              %
%                                                                 %
%   Editor: Michel Goossens / CN-AS                               %
%   Last Mod.: 17 Mar. 1992 10.20 mg                              %
%                                                                 %
%%%%%%%%%%%%%%%%%%%%%%%%%%%%%%%%%%%%%%%%%%%%%%%%%%%%%%%%%%%%%%%%%%%
 
\chapter{How to get the right answer from Minuit.}

The goal of Minuit --- to be able to minimize and analyze parameter
errors for all possible user functions with any number of variable
parameters --- is of course impossible to realise, even in principle,
in a finite amount of time. In practice, some assumptions must be made
about the behaviour of the function in order to avoid evaluating it
at all possible points.
In this chapter we give some hints on how the user can help Minuit
to make the right assumptions.

\section{Which Minimizer to Use.}

One of the historically interesting advantages of Minuit is that it
was probably the first minimization program to offer the user a choice
of several minimization algorithms.
This could be taken as a reflection of the fact that none of the
algorithms known at that time were good enough to be universal, so
users were encouraged to find the one that worked best for them.
Since then, algorithms have improved considerably, but Minuit still
offers several, mostly so that old users will not feel cheated, but also to
help the occasional user who does manage to defeat the best algorithms.
Minuit currently offers five commands which can be used to find a
smaller function value, in addition to a few others,
like \Cind[MINOs]{MINOS} and \Cind[IMProve]{IMPROVE},
which will retain a smaller function value if they stumble on one
unexpectedly (or, in the case of \Cind[IMProve]{IMPROVE}, hopefully).
The commands which can be used to minimize are:

\subsection{MIGRAD}

This is the best minimizer for nearly all functions. It is a
variable-metric method with inexact line search, a stable
metric updating scheme, and checks for positive-definiteness.
It will run faster if you \Cind[SET STRategy]{SET STRATEGY 0}
and will be more reliable if you \Cind[SET STRategy]{SET STRATEGY 2}
(although the latter option may not help much).
Its main weakness is that it depends heavily on knowledge of the
first derivatives, and fails miserably if they are very inaccurate.
If first derivatives are a problem, they can be
calculated analytically inside \Rind{FCN} (see elsewhere in this writeup) or
if this is not feasible, the user can try to improve the accuracy of
Minuit's numerical approximation by adjusting values using the
\Cind[SET EPSmachine]{SET EPS} and/or \Cind[SET STRategy]{SET STRATEGY} commands
(see Floating Point Precision and \Cind[SET STRategy]{SET STRATEGY}).

\subsection{MINIMIZE}

This is equivalent to \Cind[MIGrad]{MIGRAD}, 
except that if \Cind[MIGrad]{MIGRAD} fails,
it reverts to \Cind[SIMplex]{SIMPLEX} and then calls 
\Cind[MIGrad]{MIGRAD} again.
This is what the old \Cind[MIGrad]{MIGRAD} command used to do, 
but it was removed from
the \Cind[MIGrad]{MIGRAD} command so that users would have a choice, 
and because it is
seldom of any use to call \Cind[SIMplex]{SIMPLEX} when 
\Cind[MIGrad]{MIGRAD} has failed (there are of course exceptions).

\subsection{SCAN}

This is not intended to minimize, and just scans the function, one
parameter at a time.
It does however retain the best value after each scan,
so it does some sort of highly primitive minimization.

\subsection{SEEK}

We have retained this Monte Carlo search mainly for sentimental
reasons, even though the limited experience with it is
less than spectacular.
\index{metropilis algorithm}
\index{Monte Carlo}
The method now incorporates a Metropolis algorithm which always
moves the search region to be centred at a new minimum,
and has probability 
$ e ^{(-F/F_{\mathrm{min}})}$
of moving the search region to a higher point with function value $F$.
This gives it the theoretical ability to jump
through function barriers like a multidimensional
quantum mechanical tunneler in search of isolated minima, but it is
widely believed by at least half of the authors of Minuit that this
is unlikely to work in practice (counterexamples are welcome)
since it seems to depend critically on choosing
the right average step size for the random jumps,
and if you knew that, you wouldn't need Minuit.

\Subsection{4cm}{SIMPLEX}

\index{minimization!multidimensional}
\index{multidimensional minimization}
This genuine multidimensional minimization routine is usually much
slower than \Cind[MIGrad]{MIGRAD}, but it does not use first derivatives,
so it should not be so sensitive to the precision of the \Rind{FCN}
calculations, and is even rather robust with respect to
gross fluctuations in the function value.
However, it gives no reliable information about parameter errors,
no information whatsoever about parameter correlations,
and worst of all cannot be expected to converge accurately
to the minimum in a finite time.
Its estimate of \Lit{EDM} is largely fantasy, so it would not even
know if it did converge.

\Section{4cm}{Floating point Precision}

Minuit figures out at execution time the precision with which it was
compiled, and assumes that \Rind{FCN} provides about the same precision.
That means not just the length of the numbers used and returned
by \Rind{FCN}, but the actual mathematical accuracy of the calculations.
The section on Floating point Precision in Chapter One describes
what to do if this is not the case.

\section{Parameter Limits}

Putting limits (absolute bounds) on the allowed values for
a given parameter, causes Minuit to make a non-linear
transformation of its own internal parameter values to obtain the
(external) parameter values passed to \Rind{FCN}.
To understand the adverse effects of limits, see ``The Transformation
for Parameters with Limits'' in Chapter 1.
Basically, the use of limits should be avoided unless needed to
keep the parameter inside a desired range.
 
If parameter limits are needed, in spite of the effects described
in Chapter One, then the user should be aware of the following
techniques to alleviate problems caused by limits:

\subsection{Getting the Right Minimum with Limits.}

If \Cind[MIGrad]{MIGRAD} converges normally to a point where no parameter is
near one of its limits, then the existence of limits has
probably not prevented Minuit from finding the right minimum.
On the other hand, if one or more parameters is near its limit
at the minimum, this may be because the true minimum is indeed
at a limit, or it may be because the minimizer has become 
``blocked'' at a limit.  
This may normally happen only if the parameter
is so close to a limit (internal value at an odd multiple
of $\pm \frac{\Tstm \pi}{\Tstm 2}$ that Minuit prints a warning to this effect
when it prints the parameter values.

The minimizer can become blocked at a limit, because at a limit
the derivative seen by the minimizer 
$\partial F / \partial P_{\mathrm{int}}$
is zero no matter what the real derivative
$\partial F / \partial P_{\mathrm{ext}}$ is.

$$
\frac{\partial F}{\partial P_{\mathrm{int}}}                =
\frac{\partial F}{\partial P_{\mathrm{ext}}}
\frac{\partial P_{\mathrm{ext}}}{\partial P_{\mathrm{int}}} =
\frac{\partial F}{\partial P_{\mathrm{ext}}}                = 0
$$

For a stepping method (like \Cind[SIMplex]{SIMPLEX}) 
this seldom poses any problem,
but a method based on derivatives (\Cind[MIGrad]{MIGRAD}) may become blocked
at such a value.
If this happens, it may be necessary to move the value of the
parameter in question a significant distance from the
limit (with \Cind[SET PARameter]{SET PARam}) and restart the minimization, perhaps
with that parameter fixed temporarily.
We are investigating ways to induce Minuit to extricate itself from
such situations automatically, but it is not so obvious as it seems,
and for the moment must sometimes be done by hand.

\subsection{Getting the right parameter errors with limits.}
\label{sec:right-errors}

In the best case, where the minimum is far from any limits,
Minuit will correctly transform the error matrix, and the
parameter errors it reports should be accurate and very
close to those you would have got without limits.
In other cases (which should be more common, since
otherwise you wouldn't need limits), the very meaning of
parameter errors becomes problematic.  Mathematically, since
the limit is an absolute constraint on the parameter, a parameter
at its limit has no error, at least in one direction.
The error matrix, which can assign only symmetric errors, then
becomes essentially meaningless.
On the other hand, the \Cind[MINOs]{MINOS} analysis is still meaningful,
at least in principle, as long as \Cind[MIGrad]{MIGRAD} (which is called
internally by \Cind[MINOs]{MINOS}) does not get blocked at a limit.
Unfortunately, the user has no control over this aspect of
the \Cind[MINOs]{MINOS} calculation, although it is possible to get enough
printout from the \Cind[MINOs]{MINOS} command to be able to determine whether
the results are reliable or not.

\section{Fixing and Releasing Parameters}

When Minuit needs to be guided to the ``right'' minimum,
often the best way to do this is with the \Cind{FIX} 
and \Cind[RELease]{RELEASE} commands.
That is, suppose you have a problem with ten free parameters,
and when you minimize with respect to all at once, Minuit goes to
an unphysical solution characterized by an unphysical or unwanted
value of parameter number four.
One way to avoid this is to \Cind{FIX} parameter four at a ``good'' value
(not necessarily the best, since you presumably don't know that yet),
and minimize with respect to the others. 
Then \Cind[RELease]{RELEASE 4} and minimize again. 
If the problem admits a ``good'' physical solution, you will
normally find it this way.  
If it doesn't work,
you may see what is wrong by the following sequence
(where \Lit{xxx} is the expected physical value for parameter four):
\begin{XMP}
SET PARAM 4 xxx
FIX 4
MIGRAD
RELEASE 4
SCAN 4
\end{XMP}
where the \Cind[SCAn]{SCAN} command gives you a picture of \Rind{FCN} as a 
function of parameter four alone,
the others being fixed at their current best values.
If you suspect the difficulty is due to parameter five,
then add the command
\begin{XMP}
CONTOUR  4  5
\end{XMP}
to see a two-dimensional picture.

\section{Interpretation of Parameter Errors}

There are two kinds of problems that can arise:
The {\bf reliability} of Minuit's error estimates, and their
{\bf statistical interpretation}, assuming they are accurate.

\subsection{Statistical Interpretation.}

For discussuion of basic concepts, such as the meaning of the elements
of the error matrix, parabolic versus \Cind[MINOs]{MINOS} errors,
the appropriate value for \Lit{UP} 
(see \Cind[SET ERRordef]{SET ERRdef}), and setting of exact
confidence levels, see (in order of increasing complexity and completeness):

\begin{UL}            
\item {\em ``Interpretation of the Errors on Parameters'',}
      see Part 3 of this write-up.
\item {\em ``Determining the Statistical Significance of Experimental
      Results''}\cite{bib-MIN81}.
\item {\em ``Statistical Methods in Experimental Physics''}\cite{bib-EADIE}.
\end{UL}

\subsection{The Reliability of Minuit Error Estimates.}

Minuit always carries around its own current estimates of the
parameter errors, which it will print out on request, no matter how
accurate they are at any given point in the execution.
For example, at initialization, these estimates are just the starting
step sizes as specified by the user.  
After a \Cind[MIGrad]{MIGRAD} or \Cind[HESse]{HESSE} step,
the errors are usually quite accurate, unless there has been a problem.
Minuit, when it prints out error values,
also gives some indication of how reliable it thinks they are.
For example, those marked \Lit{'CURRENT GUESS ERROR'} are only working values
not to be believed, and \Lit{'APPROXIMATE ERROR'} means that they have been
calculated but there is reason to believe that they may not be accurate.
If no mitigating adjective is given, then at least Minuit believes
the errors are accurate, although there is always a small chance
that Minuit has been fooled.
Some visible signs that Minuit may have been fooled are:

\begin{UL}
\item Warning messages produced during the minimization or error analysis.
\item Failure to find new minimum.
\item Value of \Lit{EDM} too big. For a ``normal'' minimization, 
      after \Cind[MIGrad]{MIGRAD}, the value of \Lit{EDM} is usually more 
      than three orders of magnitude smaller than \Lit{UP} 
      (the \Cind{SET ERRordef}), unless a looser tolerance has been specified.
\item Correlation coefficients exactly equal to zero, unless some parameters
      are known to be uncorrelated with the others.
\item Correlation coefficients very close to one (greater than 0.99).
      This indicates both an exceptionally difficult problem, and one
      which has been badly parametrized so that individual errors are not
      very meaningful because they are so highly correlated.
\item Parameter at limit. This condition, signalled by a Minuit warning
      message, may make both the function minimum and parameter errors
      unreliable. See section \ref{sec:right-errors},
      {\em Getting the right parameter errors with limits.}
\end{UL}

The best way to be absolutely sure of the errors, is to use
``independent'' calculations and compare them, or compare the calculated
errors with a picture of the function if possible.
For example, if there is only one free parameter, the command \Cind[SCAn]{SCAN}
allows the user to verify approximately the function curvature.
Similarly, if there are only two free parameters, use \Cind[CONtour]{CONTOUR}.
To verify a full error matrix, compare the results of \Cind[MIGrad]{MIGRAD}
with those (calculated afterward) by \Cind[HESse]{HESSE}, 
which uses a different method.
And of course the most reliable and most expensive technique,
which must be used if asymmetric errors are required, is \Cind[MINOs]{MINOS}.

\Section{4cm}{Convergence in MIGRAD, and Positive-definiteness.}

\Cind[MIGrad]{MIGRAD} uses its current estimate of the covariance matrix of the
function to determine the current search direction, since this is
the optimal strategy for quadratic functions and ``physical'' functions
should be quadratic in the neighbourhood of the minimum at least.
The search directions determined by \Cind[MIGrad]{MIGRAD} are guaranteed to be
downhill only if the covariance matrix is positive-definite,
so in case this is not true, it makes a positive-definite
approximation by adding an appropriate constant along the diagonal
as determined by the eigenvalues of the matrix.
Theoretically, the covariance matrix for a ``physical'' function must be
positive-definite at the minimum, although it may not be so
for all points far away from the minimum, even for a well-determined
physical problem. 
Therefore, if \Cind[MIGrad]{MIGRAD} reports that it has found a
non-positive-definite covariance matrix, this may be
a sign of one or more of the following:

\begin{UL}
\item {\bf A non-physical region.}
      On its way to the minimum, \Cind[MIGrad]{MIGRAD}
      may have traversed a region which has
      unphysical behaviour, which is of course not a serious problem as long
      as it recovers and leaves such a region.
\item {\bf An underdetermined problem.}
      If the matrix is not positive-definite even at the minimum,
      this may mean that the solution is not well-defined, for example
      that there are more unknowns than there are data points, or that the
      parametrization of the fit contains a linear dependence.
      If this is the case, then Minuit (or any other program) cannot solve
      your problem uniquely, and the error matrix will necessarily be
      largely meaningless, so the user must remove the underdeterminedness
      by reformulating the parametrization. Minuit cannot do this itself,
      but it can provide some hints (contours, global correlation coefficients,
      eigenvalues) which can help the clever user to find out what is wrong.
\item {\bf Numerical inaccuracies.}
      It is possible that the apparent lack of positive-definiteness
      is in fact only due to excessive roundoff errors in numerical
      calculations, either in \Rind{FCN} or in Minuit.
      This is unlikely in general, but becomes more likely if the number of
      free parameters is very large, or if the parameters are badly scaled
      (not all of the same order of magnitude), and correlations are
      also large.
      In any  case, whether the non-positive-definiteness is
      real or only numerical is largely irrelevant, since in both cases the
      error matrix will be unreliable and the minimum suspicious.
\end{UL}

\section{Additional Trouble-shooting}

When Minuit just doesn't work, some of the more common causes are:

\begin{UL}
\item {\bf Precision mismatch.}
      Make sure your \Rind{FCN} has been compiled with the same precision as the
      version of Minuit you are using.
      When using \Lit{DOUBLE PRECISION}, it is safest to use the \Lit{IMPLICIT}
      declaration to make sure that everything is \Lit{DOUBLE PRECISION}, not
      just the arguments of \Rind{FCN} but also the internal variables.
      Note that depending on the computer system used, floating-point
      constants may be passed as single precision in subroutine arguments,
      even if there is an \Lit{IMPLICIT DOUBLE PRECISION} statement
      (which is strictly speaking correct since the \Lit{IMPLICIT} statement
      refers only to variables, not constants).
      Therefore, if constants are used as arguments in subroutine calls,
      they must be explicitly of the right precision (for example, on Apollo,
      even 0. is not equal to \Lit{0.D0}).
       
      If the problem is only one of precision, and not of word length mismatch,
      an appropriate \Cind[SET EPSmachine]{SET EPS} command may fix it.
\item {\bf Trivial bugs in \Rind{FCN}.}
      The possibilities for Fortran bugs are numerous. Probably the most
      common among physicists inexperienced in Fortran is the confusion
      between \Lit{REAL} and \Lit{INTEGER} types, 
      which you can sometimes get away with, but not always.
      [For example, if \Lit{A} and \Lit{B} are \Lit{REAL} variables, 
      the Fortran statement \Lit{A = 2*B}   is not good programming, 
      but happens to do what the user
      probably intended, whereas the statement   \Lit{A = B + 2/3}   almost
      certainly will not do what the user intended.]
      Minuit can spot some trivial bugs itself, and issues
      a warning when it detects an unusual \Rind{FCN} behaviour.  Such a warning
      should be taken seriously.
       
      Minuit also offers some tools (especially \Cind[SCAn]{SCAN}) 
      which can help the user to find trivial bugs.
\item {\bf Overwriting in a user routine.}
      Overwriting most often occurs when setting the values of a local
      array or an array in \Lit{COMMON}, and elements outside the
      dimensions of the array are addressed.
      Most computer systems do not detect this error unless you attempt to
      write into a protected area of memory, and of course Minuit is also
      helpless, especially if Minuit itself is being overwritten.
      The symptoms of user overwriting may be almost anything,
      including unusual behaviour of Minuit itself.
      The effects depend critically on where instructions and data are
      loaded in memory, so they may change completely if the same
      program is recompiled with different compiler options or reloaded
      in a different sequence, even though the compiler and loader
      are not at fault.
\item {\bf Changing the values of input arguments.}
      In subroutine \Rind{FCN}, for example, the arguments \Lit{NPAR} and \Lit{IFLAG},
      as well as the values of the parameters themselves, are only
      input to \Rind{FCN} and their values should not be changed inside \Rind{FCN}.
      Minuit is now protected against this in principle, since
      the user only gets a copy of the value, not the actual address
      of the internal Minuit variable, but still this is a symptom of
      misunderstanding by the user.
       
      If you really want to change the number of variable parameters,
      this must be done with commands like \Cind{FIX} and \Cind[RELease]{RELEASE}, 
      by redefining parameters using command \Cind[PARameters]{PARAMETER}
      or \Cind[CLEar]{CLEAR}.
       
      Similarly, if a parameter takes on an unwanted value, it will do no good
      to change its value inside \Rind{FCN}: In the best case, 
      Minuit won't see your
      improved value, and in the worst case, it will produce unpredictable
      results. To set a parameter to a certain value, use the command
      \Cind[SET PARameter]{SET PARam}, and to keep it within certain bounds, use the command
      \Cind{SET LIMits}.  If the parameter must obey more complicated constraints,
      you must find a trick such as adding a penalty value to \Rind{FCN} outside
      of the physical region, to force it back to where you want it.
\item {\bf An ill-posed problem.}
      For questions of parameter dependence, see the discussion above
      on postive-definiteness.
      Other mathematical problems which can arise are:
      {\bf excessive numerical roundoff} ---
      be especially careful of exponential and factorial functions
      which get big very quickly and lose accuracy;
      {\bf starting too far from the solution} ---
      the function may have unphysical local minima, especially
      at infinity in some variables;
      {\bf incorrect normalization} ---
      in likelihood functions, the probability distributions must
      be normalized or at least have an integral which is independent
      of the values of the variable parameters.
\item {\bf A bug in Minuit.}
      This is extremely unlikely, but it did happen once. If a bug is suspected, and
      all other possible causes can be eliminated, please try to save a copy of the
      input and output files, listing of \Rind{FCN}, and other information that may be
      relevant, and send them to \Lit{JAMES at CERNVM} or \Lit{VXCERN::JAMES} or
      \Lit{JAMES\atsign CERNAPO.CERN.CH}.
\end{UL}
