%%%%%%%%%%%%%%%%%%%%%%%%%%%%%%%%%%%%%%%%%%%%%%%%%%%%%%%%%%%%%%%%%%%
%                                                                 %
%   MINUIT User Guide -- LaTeX Source                             %
%                                                                 %
%   Chapter 4                                                     %
%                                                                 %
%   The following external EPS files are referenced:              %
%                                                                 %
%   Editor: Michel Goossens / CN-AS                               %
%   Last Mod.:  1 Apr. 1992 10:15 mg                              %
%                                                                 %
%%%%%%%%%%%%%%%%%%%%%%%%%%%%%%%%%%%%%%%%%%%%%%%%%%%%%%%%%%%%%%%%%%%
 
\chapter{Minuit Commands}

In data-driven mode, Minuit accepts commands in the following
format:
 
\Sboxni{command}{<arg1>   [arg2]  etc.}
 
\begin{DLtt}{MMMMM}
\item[command] One of the commands listed below,
\item[<argi>]           Numerical values of {\bf required} arguments, if any.
\item[\lsb{}argi\rsb]]  Numerical values of {\bf optional} arguments, if any.
\end{DLtt}

The arguments (if any) are separated from each other and from the
command by one or more blanks or a comma.
Commands may be given in upper or lower case, and may be abbreviated,
usually to three characters. The shortest recognized abbreviations
are indicated by the capitalized part of the commands listed below.
Examples of valid commands are:

\begin{XMP}
SET INPUT  21
migrad
mig  500
SET LIMITS  14  -1.0,1.0
contours  1  2
MINOS  500  1,3,5,21,22                            
\end{XMP}
 
In Fortran-callable mode, all the same commands (with a few
obvious exceptions as indicated) can be executed by passing the
command-string and arguments to Minuit in a \Lit{CALL}~\Rind{MNEXCM} statement.
 
\section*{List of Minuit commands}

\Sbox{CALL}{CALl}{<iflag>}

Instructs Minuit to call subroutine \Rind{FCN} with the value of
\Lit{IFLAG=<iflag>}.
(The actual name of the subroutine called is that given by the user
in his call to Minuit or \Rind{MNEXCM};
the name given in this command is not used.)
If \Lit{<iflag> > 5}, Minuit assumes that a new problem is being
redefined, and it forgets the previous best value of the function,
covariance matrix, etc.
This command can be used to instruct the user function to read new
input data, recalculate constants, or otherwise modify the calculation
of the function.

\Sbox{CLEAR}{CLEar}{ }

Resets all parameter names and values to undefined. Must normally be
followed by a \Cind{PARameters} command or equivalent, in order to define
parameter values.

\Sbox{CONTOUR}{CONtour}{<par1>  <par2>  [devs]  [ngrid]}

Instructs Minuit to trace contour lines of the user function with
respect to the two parameters whose external numbers are \Lit{<par1>}
and \Lit{<par2>}.
Other variable parameters of the function, if any, will have their
values fixed at the current values during the contour tracing.
The optional parameter \Lit{[devs]} (default value 2.) gives the number of
standard deviations in each parameter which should lie entirely within
the plotting area. 
Optional parameter \Lit{[ngrid]} (default value 25 unless
page size is too small) determines the resolution of the plot, i.e.
the number of rows and columns of the grid at which the function
will be evaluated. [See also \Cind{MNContour}.]

\Sbox{END}{END}{ }

Signals the end of a data block (i.e., the end of a fit), and implies that
execution should continue, because another Data Block follows.  
A Data Block is a set of Minuit data
consisting of (1) A Title, (2) One or more Parameter Definitions,
(3) A blank line, and (4) A set of Minuit Commands.
The \Cind{END} command is used when more than one Data Block is to be used
with the same \Rind{FCN} function.
\Lit{CALL}~\Rind{FCN}The \Cind{END} command first causes Minuit to issue a \Lit{CALL}~\Rind{FCN}
with \Lit{IFLAG=3},
in order to allow \Rind{FCN} to perform any calculations associated with
the final fitted parameter values,
unless a \Lit{CALL FCN 3} command has already been executed
at the current \Rind{FCN} value.
The obsolete command \Cind[END RETURN (obsolete)]{END RETurn}
is the same as the \Cind[RETurn]{RETURN} command.

\Sbox{EXIT}{EXIT}{ }

Signals the end of execution.
The \Cind{EXIT} command first causes Minuit to issue a
\Lit{CALL}~\Rind{FCN} with \Lit{IFLAG=3},
in order to allow \Rind{FCN} to perform any calculations associated with
the final fitted parameter values,
unless a \Lit{CALL FCN 3} command has already been executed
at the current \Rind{FCN} value.
Then it executes a Fortran \Lit{STOP}.

\Sbox{FIX}{FIX}{<parno> [parno] ... [parno]}

Causes parameter(s) \Lit{<parno>} to be removed from the list of variable
parameters, and their value(s) will remain constant
during subsequent minimizations, etc., until another command changes
their value(s) or status.

\Sbox{HELP}{HELP}{[SET] [SHOw]}

Causes Minuit to list the available commands. The list of
\Cind{SET} and \Cind{SHOw} commands must be requested separately.

\Sbox{HESSE}{HESse}{[maxcalls]}

Instructs Minuit to calculate, by finite differences, the Hessian or
error matrix. That is, it calculates the full matrix of second
derivatives of the function with respect to the currently variable
parameters, and inverts it, printing out the resulting error matrix.
The optional argument \Lit{[maxcalls]} specifies the (approximate) maximum
number of function calls after which the calculation will be stopped.

\Sbox{IMPROVE}{IMProve}{[maxcalls]}

If a previous minimization has converged, and the current values
of the parameters therefore correspond to a local minimum of the function,
this command requests a search for additional distinct local minima.
The optional argument \Lit{[maxcalls]} specifies the (approximate) maximum
number of function calls after which the calculation will be stopped.

\Sbox{MIGRAD}{MIGrad}{[maxcalls]  [tolerance]}

Causes minimization of the function by the method of Migrad, the most
efficient and complete single method, recommended for general functions
(see also \Cind{MINImize}).
The minimization produces as a by-product the error matrix
of the parameters, which is usually reliable unless warning messages
are produced.
The optional argument \Lit{[maxcalls]} specifies the (approximate) maximum
number of function calls after which the calculation will be stopped
even if it has not yet converged.
The optional argument \Lit{[tolerance]} specifies required tolerance on the
function value at the minimum.  The default tolerance is \Lit{0.1}, and the
minimization will stop when the estimated vertical distance to
the minimum (\Lit{EDM}) is less than \Lit{0.001*[tolerance]*UP} 
(see \Cind[SET ERRordef]{SET ERR}).

\Sbox{MINIMIZE}{MINImize}{[maxcalls] [tolerance]}

Causes minimization of the function by the method of Migrad,
as does the \Cind{MIGrad} command, but switches to the \Cind{SIMplex} method
if Migrad fails to converge. Arguments are as for \Cind{MIGrad}.
Note that command requires four characters to be unambiguous with \Cind{MINOs}.

\Sbox{MINOS}{MINOs}{[maxcalls]  [parno] [parno] ...}

Causes a Minos error analysis to be performed on the parameters whose
numbers \Lit{[parno]} are specified.  If none are specified, Minos errors
are calculated for all variable parameters.
Minos errors may be expensive to calculate, but are very reliable since
they take account of non-linearities in the problem as well as
parameter correlations, and are in general asymmetric.
The optional argument \Lit{[maxcalls]} specifies the (approximate) maximum
number of function calls {\bf per parameter requested},
after which the calculation will be stopped for that parameter.

\Sbox{MNCONTOUR}{MNContour}{<par1> <par2> [npts]}

Calculates one function contour of \Rind{FCN} with respect to parameters
\Lit{par1} and \Lit{par2}, with \Rind{FCN} 
minimized always with respect to all other
\Lit{NPAR-2} variable parameters (if any). 
Minuit will try to find \Lit{npts} points on the contour (default 20).  
If only two parameters are variable
at the time, it is not necessary to specify their numbers. 
To calculate
more than one contour, it is necessary to \Cind[SET ERRordef]{SET ERR} 
to the appropriate
value and issue the \Cind{MNContour} command for each contour desired.

\Sbox{RELEASE}{RELease}{<parno> [parno] ... [parno]}

If \Lit{<parno>} is the number of a previously variable parameter which has
been fixed by a command:
\Cind{FIX}~\Lit{<parno>}, then that parameter will
return to variable status.  Otherwise a warning message is printed
and the command is ignored.
Note that this command operates only on parameters which were at one time
variable and have been \Cind{FIX}ed.
It cannot make constant parameters variable;
that must be done by redefining the parameter with a \Cind{PARameters} command.

\Sbox{RESTORE}{REStore}{[code]}

If no \Lit{[code]} is specified, this command restores all previously 
\Cind{FIX}ed parameters to variable status. 
If \Lit{[code]=1}, then only the last parameter
\Cind{FIX}ed is restored to variable status.  
If code is neither zero nor one, the command is ignored.

\Sbox{RETURN}{RETurn}{ }

Signals the end of a data block, and instructs Minuit to
return to the program which called it.
The \Cind{RETurn} command first causes Minuit to \Lit{CALL FCN} 
with \Lit{IFLAG=3},
in order to allow \Rind{FCN} to perform any calculations associated with
the final fitted parameter values,
unless a \Lit{CALL FCN 3} command has already been executed
at the current \Rind{FCN} value.
Then it executes a Fortran \Lit{RETURN}.

\Sbox{SAVE}{SAVe}{ }

Causes the current parameter values to be saved on a file in such a
format that they can be read in again as Minuit parameter definitions.
If the covariance matrix exists, it is also output in such a format.
The unit number is by default 7, or that specified by the user in
his call to \Rind{MINTIO} or \Rind{MNINIT}.
The user is responsible for opening the file previous to
issuing the \Cind[SAVe]{SAVE} command 
(except where this can be done interactively).

\Sbox{SCAN}{SCAn}{[parno]  [numpts] [from]  [to]}

Scans the value of the user function by varying parameter number
\Lit{[parno]}, leaving all other parameters fixed at the current value.
If \Lit{[parno]} is not specified, all variable parameters are scanned in
sequence. The number of points \Lit{[numpts]} in the scan is 40 by default,
and cannot exceed 100.
The range of the scan is by default 2 standard deviations on each side
of the current best value, but can be specified as from 
\Lit{[from]} to \Lit{[to]}.
After each scan, if a new minimum is found, the best parameter values
are retained as start values for future scans or minimizations.
The curve resulting from each scan is plotted on the output unit
in order to show the approximate behaviour of the function.
This command is not intended for minimization, but is sometimes useful
for debugging the user function or finding a reasonable starting point.

\Sbox{SEEK}{SEEk}{[maxcalls]  [devs]}

Causes a Monte Carlo minimization of the function, by choosing
random values of the variable parameters, chosen uniformly over a
hypercube centered at the current best value.  The region size is by
default 3 standard deviations on each side, but can be changed by
specifying the value of \Lit{[devs]}.

\Sbox{SETBAT}{SET BATch}{ }

Informs Minuit that it is running in batch mode.

\Sbox{SETEPS}{SET EPSmachine}{<accuracy>}

Informs Minuit that the relative floating point arithmetic
precision is \Lit{<accuracy>}.
Minuit determines the nominal precision itself, but the \Cind[SET EPSmachine]{SET EPS} command
can be used to override Minuit's own determination, when the user
knows that the \Rind{FCN} function value is not calculated to the nominal
machine accuracy. Typical values of \Lit{<accuracy>} are between
$10^{-5}$ and $10^{-14}$.

\Sbox{SETERR}{SET ERRordef}{<up>}

Sets the value of \Lit{UP} (default value= 1.), defining parameter errors.
Minuit defines parameter errors as the change in parameter value
required to change the function value by \Lit{UP}.  
Normally, for chisquared
fits \Lit{UP=1}, and for negative log likelihood, \Lit{UP=0.5}.

\Sbox{SETGRA}{SET GRAdient}{[force]}

Informs Minuit that the user function is prepared to calculate
its own first derivatives and return their values in the array
\Lit{GRAD} when \Lit{IFLAG=2} 
(see specification of the function \Rind{FCN}).
If \Lit{[force]} is not specified, Minuit will calculate the \Rind{FCN}
derivatives by finite differences at the current point and
compare with the user's calculation at that point,
accepting the user's values only if they agree.
If \Lit{[force]=1}, Minuit does not do its own derivative calculation,
and uses the derivatives calculated in \Rind{FCN}.

\Sbox{SETINP}{SET INPut}{[unitno]  [filename]}

Causes Minuit, in data-driven mode only, to read subsequent
commands (or parameter definitions or title) from a different input file.
If no \Lit{[unitno]} is specified, reading reverts to the previous input
file, assuming that there was one.  If \Lit{[unitno]} is specified, and that
unit has not been opened, then Minuit attempts to open the file
\Lit{[filename]} if a name is specified. If running in interactive mode and
\Lit{[filename]} is not specified and \Lit{[unitno]} is not opened,
Minuit prompts the user to enter a file name.
If the word \Cind{REWIND} is added to the command (note:
{\bf no blanks} between \Lit{INPUT} and \Lit{REWIND}),
the file is rewound before reading.
{\em Note that this command is implemented in standard Fortran 77
and the results may depend on the operating system; for example,
if a filename is given under VM/CMS, it must be preceeded by a slash.}

\Sbox{SETINT}{SET INTeractive}{ }

Informs Minuit that it is running interactively.

\Sbox{SETLIM}{SET LIMits}{[parno]  [lolim]  [uplim]}

Allows the user to change the limits on one or all parameters.
If no arguments are specified, all limits are removed from all parameters.
If \Lit{[parno]} alone is specified, limits are removed from parameter 
\Lit{[parno]}.
If all arguments are specified, then parameter \Lit{[parno]} will be bounded
between \Lit{[lolim]} and \Lit{[uplim]}. 
Limits can be specified in either order,
Minuit will take the smaller as \Lit{[lolim]} and the larger as \Lit{[uplim]}.
However, if \Lit{[lolim]} is equal to \Lit{[uplim]}, an error condition results.

\Sbox{SETLIN}{SET LINesperpage}{ }

Sets the number of lines that Minuit thinks will fit on
one page of output.
The default value is 24 for interactive mode and 56 for batch.

\Sbox{SETNOG}{SET NOGradient}{ }

The inverse of \Cind{SET GRAdient}, instructs Minuit not to use the
first derivatives calculated by the user in \Rind{FCN}.

\Sbox{SETNOW}{SET NOWarnings}{ }

Supresses Minuit warning messages. \Cind{SET WARnings} is the default.

\Sbox{SETOUT}{SET OUTputfile}{<unitno>}

Instructs Minuit to write further output to unit \Lit{<unitno>}.

\Sbox{SETPAG}{SET PAGethrow}{<integer>}

Sets the carriage control character for ``new page'' to \Lit{<integer>}.
Thus the value 1 produces a new page, and 0 produces a blank line,
on some output devices (see \Cind{TOPofpage} command).

\Sbox{SETPAR}{SET PARameter}{<parno>  <value>}

Sets the value of parameter \Lit{<parno>} to \Lit{<value>}.  
The parameter
in question may be variable, fixed, or constant, but must be defined.

\Sbox{SETPRI}{SET PRIntout}{<level>}

Sets the print level, determining how much output
Minuit will produce.
The allowed values and their meanings are displayed
after a \Cind{SHOw PRInt} command, and are currently \Lit{<level>=}:

\begin{DLtt}{12}
\item[-1]            no output except from \Cind[SHOw]{SHOW} commands
\item[\phantom{-}0]  minimum output (no starting values or intermediate results)
\item[\phantom{-}1]  default value, normal output
\item[\phantom{-}2]  additional output giving intermediate results.
\item[\phantom{-}3]  maximum output, showing progress of minimizations.
\end{DLtt}
 
Note: See also the \Cind{SET WARnings} command.

\Sbox{SETRAM}{SET RANdomgenerator}{<seed>}

Sets the seed of the random number generator used in \Cind{SEEk}.  
This can be any integer between 10~000 and 900~000~000, for example
one which was output from a \Cind{SHOw RANdom} command of a previous run.

\Sbox{SETSTR}{SET STRategy}{<level>}

Sets the strategy to be used in calculating first and second derivatives
and in certain minimization methods. In general, low values of \Lit{<level>}
mean fewer function calls and high values mean more reliable minimization.
Currently allowed values are 0, 1 (default), and 2.

\Sbox{SETTIT}{SET TITle}{ }

Informs Minuit that the next input line is to be considered the (new)
title for this task or sub-task.  This is for the convenience of
the user in reading his output. This command is available only in data-driven
mode; in Fortran-callable mode use \Lit{CALL}~\Rind{MNSETI}.

\Sbox{SETWAR}{SET WARnings}{ }

Instructs Minuit to output warning messages when suspicious conditions
arise which may indicate unreliable results.
This is the default.
\Sbox{SETWID}{SET WIDthpage}{ }

Informs Minuit of the output page width. Default values are 80 for
interactive jobs and 120 for batch.

\Sbox{SHOXXXX}{SHOw XXXX}{ }

All \Cind{SET XXXX} commands have a corresponding \Cind{SHOw XXXX} command.
In addition, the \Lit{SHOw} commands listed starting here have no corresponding
\Lit{SET} command for obvious reasons.  The full list of \Lit{SHOw} commands
is printed in response to the command \Cind{HELP}~\Lit{SHOw}.

\Sbox{SHOCOR}{SHOw CORrelations}{ }

Calculates and prints the parameter correlations from the error matrix.

\Sbox{SHOCOV}{SHOw COVariance}{ }

Prints the (external) covariance (error) matrix.

\Sbox{SHOEIG}{SHOw EIGenvalues}{ }

Calculates and prints the eigenvalues of the covariance matrix.

\Sbox{SHOFCN}{SHOw FCNvalue}{ }

Prints the current value of \Rind{FCN}.

\Sbox{SIM}{SIMplex}{[maxcalls]  [tolerance]}

Performs a function minimization using the simplex method of Nelder and
Mead. Minimization terminates either when the function has been called
(approximately) \Lit{[maxcalls]} times, or when the estimated vertical
distance to minimum (\Lit{EDM}) is less than \Lit{[tolerance]}. 
The default value of \Lit{[tolerance]} is \Lit{0.1*UP} 
(see \Cind[SET ERRordef]{SET ERR}).

\Sbox{STA}{STAndard}{ }

Causes Minuit to execute the Fortran instruction \Lit{CALL STAND}
where \Lit{STAND} is a subroutine supplied by the user.

\Sbox{STOP}{STOP}{ }

Same as \Cind{EXIT}.

\Sbox{TOP}{TOPofpage}{ }

Causes Minuit to write the character specified in a
\Cind{SET PAGethrow} command (default = 1) to column 1 of the
output file, which may or may not position your output medium
to the top of a page depending on the device and system.
This command can be expected to work properly only for printed
output, unfortunately it does not solve the IBM terminal problem.
