%%%%%%%%%%%%%%%%%%%%%%%%%%%%%%%%%%%%%%%%%%%%%%%%%%%%%%%%%%%%%%%%%%%
%                                                                 %
%   COMIS - Reference Manual -- LaTeX Source                      %
%                                                                 %
%   Chapter 5                                                     %
%                                                                 %
%   Files referenced: none                                        %
%                                                                 %
%   Editor: Michel Goossens / CN-AS                               %
%   Last Mod.: 29 Jun 1993  10:35  mg                             %
%                                                                 %
%%%%%%%%%%%%%%%%%%%%%%%%%%%%%%%%%%%%%%%%%%%%%%%%%%%%%%%%%%%%%%%%%%%

\Filename{H1comis-builtin-editor}
\chapter{Built-in editor}
\label{sec:comiseditor}
\index{editor}

\COMIS{} provides a built-in editor which is automatically 
invoked by the interpreter when an error is detected. 
The \COMIS{} built-in editor is similar to the
VAX line-mode EDIT/EDT. 
A line can be referenced by its line number. 
The \Lit{INSERT} and \Lit{DELETE} commands change a the line numbering. 
To operate on a set of lines it is possible to specify 
a line range in the following forms:

\begin{DLtt}{12345}
\item[N1:N2]   specifies the set of lines from \Lit{N1} to \Lit{N2}, where
               \Lit{N1<N2} or \Lit{N1>N2}.
\item[N]       specifies \Lit{N}-th line from the beginning of the file.
\item[+N]      specifies \Lit{+N}-th line from the current pointer position.
\item[-N]      specifies \Lit{-N}-th line from the current pointer position.
\end{DLtt}
 
The character ``\Lit{F}'' is recognized as the first line of the routine and
the character ``\Lit{L}'' is recognized as its last line; ``\Lit{W}'' is recognized
as ``\Lit{F:L}''.

\section{Commands explanation}

\SKUIP{T}{[range]}
 
This command types the lines in the given range. 
If \Lit{range} is omitted the current line is typed.
The leading  \Lit{T} can be omitted and only the range specified.
In response to a \Lit{<CR>} the next line is typed.
 
\SKUIP[S]{S/old/new/}{[range]}
 
This command substitutes the ``\Lit{old}'' text string
by the ``\Lit{new}'' one for all lines in the \Lit{range}.
 
\SKUIP{D}{[range]}
 
This command deletes the set of lines specified
(the line numbers will be changed).
 
\SKUIP{I}{[line\_number]/ line1<CR> line2<CR>...<lineN>/}
 
This command inserts the given text lines after the line
specified by the \Lit{line_number}.
To insert new lines at the very beginning you should
specify \Lit{line_number=0} (the line numbers will be changed ).
 
\SKUIP{EXIT}{}
 
This command cause exits from the editor. The text is processed
by the \COMIS{} compiler automatically.
 
\SKUIP{QUIT}{}
 
This command cause exits from the editor without any compilation.
 
\SKUIP{EDIT}{}
 
\COMIS{} invokes the local editor of the operating system.
After edit session control is returned to the built-in editor.
 
\SKUIP[hHELP]{HELP}{}
 
This command types HELP information about the editor commands.
 
