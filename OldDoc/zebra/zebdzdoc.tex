%%%%%%%%%%%%%%%%%%%%%%%%%%%%%%%%%%%%%%%%%%%%%%%%%%%%%%%%%%%%%%%%%%%
%                                                                 %
%   ZEBRA User Guide -- LaTeX Source                              %
%                                                                 %
%   Chapter DZDOC                                                 %
%                                                                 %
%   The following external EPS files are referenced:              %
%   dzddiv, dzdflow, dzdirz, dzdisp, dzmctree, dzprsc             %
%                                                                 %
%   Editor: Michel Goossens / CN-ASD                              %
%                                                                 %
%   Last Mod.: 17 Feb 1995   17:30 os                             %
%                                                                 %
%%%%%%%%%%%%%%%%%%%%%%%%%%%%%%%%%%%%%%%%%%%%%%%%%%%%%%%%%%%%%%%%%%%
\fboxsep0.5cm
 
\chapter{DZDOC -- Bank Documentation and Display}
\label{sec:dzdocdescription}
 
This chapter describes at set of programs
to generate and maintain documentation of ZEBRA data structures.
Documentation is entered by the user in the form of card image files which
describe the contents and the structure of the data (section~\ref{sec:dzdocformat}).
A template of this file may be generated for an existing data structure
in memory by one subroutine call (section~\ref{sec:dzdocdzdtmp}.
The system puts this information into an RZ~file~\cite{bib-ZEBRARZ}
to allow fast access to the documentation. 
This file can then be used in the following applications:
 
\begin{UL}
\item produce documentation of bank trees in graphical and alphanumeric
      form;
\item make the documentation available in an interactive session;
\item generate Fortran code with calls to \Rind{MZBOOK}/\Rind{MZLIFT}
      to build the documented bank trees and with \break
      \Lit{PARAMETER} statements 
      for link and dataword offsets including the needed type declarations.
\end{UL}
 
Figure~\ref{fig:DZDOCFIG1} shows the data flow through the 
system and where the different parts of the package are needed.
 
\begin{Fighere}
  \begin{center}
     \mbox{\epsfig{figure=dzdflow.eps,height=13cm}}
    \caption{Dataflow within \protect\Pind{DZEDIT} -- \protect\Pind{DZDISP}}
    \label{fig:DZDOCFIG1}
  \end{center}
\end{Fighere}
 
Section~\ref{sec:dzdocdzedit} 
describes the interactive program \Pind{DZEDIT} which is used to
generate and maintain the RZ~file and to produce documentation in
various forms. 
It can also check the completeness and consistency of the documentation.
Fortran code generation is also controlled by this program, details
are found in the description of the command \Cind{MAKECODE}.
 
Section~\ref{sec:dzdocdzdisp} 
explains the application of the package to display ZEBRA bank
structures in an interactive session.
This package can be used independently from the documentation parts 
to display data structures and the contents of data and system words.
 
Section~\ref{sec:dzdocexamples} 
contains examples of bank descriptions and calling sequences.
 
\section{Format of the bank descriptor cards}
\label{sec:dzdocformat}
 
\begin{Note}
All cards start by \Lit{"*B"} (all other cards are ignored), explanations
are put inside braces \Lit{\lcb\ \rcb}.
\end{Note}
 
\subsection{General information about the bank}
\index{bank!general information!documentation}
\label{sec:dzdocgeneral}
 
\begin{XMP}
*B..BKID  Short description of bank with hollerith Id BKID
*B.AU     Author1, Author2
*B.VE     5.12         \lcb version number\rcb 
*B.ST     ISTOR        \lcb mnemonic of the store\rcb 
*B.DV     IRODIV       \lcb mnemonic of the division\rcb 
*B.NI     NUMID        \lcb numerical identifier\rcb 
*B.ND     NDATA        \lcb number of data words\rcb 
*B.IO     IOCHAR       \lcb IO characteristics\rcb 
*B.NZ     NZERO        \lcb number of data words preset to zero\rcb 
*B.NL     NLINKS       \lcb total number of links\rcb 
*B.NS     NSLINKS      \lcb number of structural links\rcb 
*B.NX     NXID         \lcb NXID: Id of a possible next bank\rcb 
*B.UP     UPID  -JB    \lcb UPID: Id of Up-bank,
                        JB is the link offset of bank BKID\rcb 
*B.OR     ORID         \lcb ORID: Id of Origin-bank\rcb 
\end{XMP}
 
\begin{Notes}
\item The text on the title card \Lit{*B..} is displayed by DZDISP
      (see figure \ref{fig:DZDOCFIG2})
      in the big box representing the data part of a bank. Sometimes  a
      meaningful title is contained as data in the bank itself.
      To display this one may use the syntax:
\begin{XMP}
*B..BKID  TITLE@D(N1:N2)
  or
*B..BKID  TITLE@D(N1:N2) FORMZ
\end{XMP}
      where N1:N2 describes the range of datawords with hollerith text.
      If \Lit{FORMZ} is specified the text is first converted by ZITOH
      from ZEBRA internal character to hollerith format. 
      
      Similiarly the fields where the hollerith Id, the numerical Id,
      and the number of data words are displayed may be filled with text
      actually contained in the bank itself using the syntax:
\begin{XMP}
*B..BKID  HID@D(N1:N2)
*B..BKID  NID@D(N1:N2)
*B..BKID  ND@D(N1:N2)
\end{XMP}
 
A picture showing a Monte Carlo decay tree 
in figure \ref{fig:DZDOCFIG7} was produced using this feature.
 
\item The cards: \Lit{*B..} and \Lit{*B.UP} are mandatory, all others are optional. 
\item \Lit{NDATA}, \Lit{NLINKS} or \Lit{NSLINKS} may also be given as integer numbers. 
\item When generating Fortran code,
      defaults are assumed as described in section~\ref{sec:makecode}.
\end{Notes}
 
\subsection{Link part of the bank}
\index{bank!link!documentation}
\index{link!bank documentation}
 
\begin{XMP}
*B.LINK            \lcb indicate start of link description\rcb 
*B.1      D1ID     Short description of D1ID  \lcb D1ID: Id of first down bank\rcb 
*B.2      D2ID     ....
*B/LINK            \lcb indicate end of link description\rcb 
\end{XMP}
 
If a link is not (yet) defined  the mnemonic \Lit{NOTU} should be used
to avoid that DZDOC produces a diagnostic message.
 
\subsubsection*{Reference links of the bank}
\index{bank!reference link!documentation}
\index{link!bank documentation}
\index{reference link!bank documentation}
 
\begin{XMP}
*B.RLINK            \lcb indicate start of Rlink description\rcb 
*B.1      D1ID     Short description of D1ID  \lcb D1ID: Id of referenced bank\rcb 
*B.2      D2ID     ....
*B/RLINK            \lcb indicate end of Rlink description\rcb 
\end{XMP}
 
\subsection{Status word of bank}
\index{bank!status word!documentation}
\index{status word!bank documentation}
 
\begin{XMP}
*B.BI                \lcb indicates start of status word bits description\rcb 
*B.1      NOAUTH     no authorisation to modify directory
*B.2      MODIFIED   directory has been modified
*B.3      LOCKED     file locked by 'RZFILE'
*B.4      ORGREL     ORGANIZATION='RELATIVE' on VAX
*B.5      CACCESS    C file access routine used
*B.11-17  LOGLEVEL   11-17 LOG level (default taken from MZ)
*B/BI                \lcb indicates end of status word bits description\rcb 
\end{XMP}
 
\subsection{Data words of the bank}
\index{bank!data word!documentation}
\index{data word!bank documentation}
 
\begin{XMP}
*B.DATA              \lcb indicates start of data word description\rcb 
*B.1      DATAWOR1   Description of data word 1
*B.REP    5
*B.1      DATAWOR2   Description of data words 2 - 6
*B/REP
*B/DATA              \lcb indicates end of data word description\rcb 
\end{XMP}
 
\Lit{DATAWOR1} .. \Lit{DATAWOR7} are (upto) {\bf 8 characters} mnemonics
which can be selected by the user.
 
The data type and range may optionally be given in the description.
The format is as given in the following examples:
\begin{XMP}
*B.DATA 
*B.1    EVENTTYP    IO:I [1,200] Event type, Integer, range: 1-200
*B.2    ENERGY      IO:F [0.,100.] Energy
*B/DATA 
\end{XMP}
Allowed IO specifiers are \Lit{B, I, F, D, H, R ,U} corresponding to the
ZEBRA IO characteristics, \Lit{R} is the same as \Lit{F},
\Lit{U} the same as \Lit{B}.
 
\subsubsection*{Special mnemonics}
 
The following special mnemonics are used to specify the content of
a data word in more detail.
They are interesting not only for documentation
purposes but even more when using the documentation within the 
interactive bank display: They allow to selectively print lines of
the documentation depending on values of data words or byte portions
of them. They must appear on cards following one with a general 
description whereby the dataword number has to be be repeated 
(i.e. the 3 in \Lit{*B.3}). This is to clearly distinuish them from 
normal continuation cards, for which the sequence number and the 
mnemonic field is left blank. The different types may be mixed for 
one data word.
 
\paragraph{\underline{\bf A choice with wildcard characters}}
 
The mnemonic \Lit{WILDCHAR} or \Lit{MASK} allows to describe a choice
depending on the {\bf printed} value of the dataword allowing
wildcard characters. The mask is taken from the first characters of
the comment field separated by a blank from the description.
An asterix is used to represent one wild character. 
In the following example it assumed that ATTRIBUT describes the
attributes of a graphic object. They may be decimal coded in the
following manner:  ATTRIBUT = 10000*TYPE + 100*SIZE + COLOR
 
\begin{XMPt}{Example of a choice with wildcard characters}
*B.3   ATTRIBUT  Graphics attributes: 
*B.3   MASK        1****  line,
*B.3   MASK        2****  box,
*B.3   MASK       11****  circle,
*B.3   MASK       ****01  black
*B.3   MASK       ****02  red
*B.3   MASK       ****03  green
\end{XMPt}
If the value of ATTRIBUT happens to be 20503 the documented
printout is:
 
\Lit{3  ATTRIBUT 20503  Graphics attributes: box, green}
 
If hexadecimal coding is appropriate one can force hexadecimal 
formatting of the printout by preceeding the mnemonic be \Lit{B:}
like in this example (Note: Only 6 characters are left for the name):
 
\begin{XMPt}{Example with hexadecimal formatting}
*B.4   B:TRTYPE   Trigger type: 
*B.4   MASK       *******1 energy,
*B.4   MASK       *******F multiplicity,
*B.4   MASK       ******1* tracks,
*B.4   MASK       ******2* clusters,
\end{XMPt}
 
\paragraph{\underline{\bf An integer choice value}}
 
The mnemonic \Lit{C1234567}, where \Lit{1234567} can be any 7-digits integer
number, describes a variable with a choice value. 
If a data word is described like this, 
then in an interactive session, where values of
datawords together with their documentation can be shown,
only the line for which the choice applies will be displayed.
Leading zeros may be left out or replaced by underscores. 
Negative values are possible.
 
\begin{XMPt}{Example of an integer choice value}
*B.8      MYCHOICE   The following applies
*B.8      C0000010   Meaning of 8. word, if it has value the 10
*B.8      C____111   Meaning of 8. word, if it has value the 111
*B.8      C-10       Meaning of 8. word, if it has value the -10
*B.8      C12        Meaning of 8. word, if it has value the 12
\end{XMPt}
 
\paragraph{\underline{\bf A bit mask}}
 
The memonic \Lit{BITVAL12}, where \Lit{12} stands for any 2-digits integer
between 0 and 31, describes a data word where single bits are significant. 
In an interactive session only those lines are shown
for which the corresponding bit is set in the dataword.
 
\begin{XMPt}{Example of a bit mask}
*B.9      YOURBITS  Single bits are significant
*B.9      BITVAL01  Significance of bit 1
*B.9      BITVAL15  Significance of bit 15
\end{XMPt}
 
\paragraph{\underline{\bf Description of any bit portion}}
 
The memonic \Lit{BITS0309}, where \Lit{03} and \Lit{09} 
stand for any 2-digits integers between \Lit{0} and \Lit{31}, 
describes the bit portion of a data word.
When datawords are dumped with documentation the value
of the bit portion will be printed.
 
\begin{XMPt}{Example of a bit description}
*B.9      IZCODE     Coded hit value
*B.9      BITS0007   Wire number
*B.9      BITS0815   Charge
*B.9      BITS1623   Theta value
\end{XMPt}
 
\paragraph{\underline{\bf ZEBRA exchange formatted Holleriths,
dates, binaries}}
ZEBRA provides and uses a Hollerith format, which is transportable.
These words should be documented with a mnemonic starting with
\Lit{Z:}. In this case DZDOC will call \Rind{ZITOH} and \Rind{UHTOC} to print
it as a character. A mnemonic starting with \Lit{B:} forces DZDOC
to print the value in hexadecimal. \Lit{D:} assumes that the upper
bits of the word contain a packed date and time which is decoded
by the RZ routine \Rind{RZDATE}.
 
\paragraph{\underline{\bf Pointers, labels and repetition counts}}
\index{bank!documentation!pointer}
\index{bank!documentation!label}
\index{bank!documentation!repetition}
\label{pointer (bank documentation)}
\index{label (bank documentation)}
\label{repetition (bank documentation)}
 
The mnemonic \Lit{P:ABCDEF}, where \Lit{ABCDEF} can be any name,
describes a variable used as a pointer within the current bank. 
If during the display of 
the contents of a bank the word number to which the variable points, 
is reached, the documentation for this section of data is expected
at \Lit{L:ABCDEF} and is tagged in the output with the string \Lit{--Label:}.
The string \Lit{N:ABCDEF} may be used to flag a variable when 
used later as a repetition count (see below). 
The description of the RZ top bank in Section~\ref{sec:dzdocexamples}
contains extended applications of pointers, labels etc.
\begin{XMPt}{Example of the use of a pointer}
*B.4      P:LDIR     Pointer to subdirectories
*B.5      N:NSDIR    Number of subdirectories
     ...
*B.1      L:LSDIR    Subdirectory structure
*B.REP    N:NSDIR    Loop through subdirectories
*B.1      Z:NAME     Name
*B.2      IRECSD     Record number
*B.3      D:DATE     creation date and time
*B/REP
\end{XMPt}
Pointers and repetition counts can also be bitportions of a variable.
Assume that the lower 16 bits of a word contain a pointer, the upper
16 are the repetition count.
In this case the syntax is as given in the following example:
Note that the name of the pointer  is taken from the first 6 characters
of the comment field or until a blank is encountered.
\pagebreak[3]
 
\begin{XMPt}{Pointer, count which is a bitportion}
*B.4      SUBDIR     pointer to, and number of subdirs
*B.4      P:BI0015   LDIR     Pointer to subdirectories
*B.4      N:BI1631   NSDIR    Number of subdirectories
\end{XMPt}
 
{\bf Note:} When generating FORTRAN code the prefixes N: etc. will
            be removed.
 
\subsection{Repetitions}
\index{bank!documentation!repetition}
If the same kind of data words or links are repeated several times
they may be described within a \Lit{*B.REP} .. \Lit{*B/REP} clause. 
Repetitions may include groups of data words or links. 
The repetition count may be fixed, variable
(indefinite) or calculated from a data word or taken from a previous
data word as shown above.
 
\subsubsection*{Fixed number of repetitions}
 
\begin{XMP}
*B.LINK
*B.REP    25
*B.1      DXID     Short description of DXID
*B/REP
*B.26     DYID     ....
*B.27     DZID     ....
*B/LINK
\end{XMP}
 
\subsubsection*{Variable number of repetitions}
 
\begin{XMP}
*B.LINK
*B.REP    NREPS    Comment
*B.1      DXID     Short description of DXID
*B/REP
*B/LINK
\end{XMP}
 
\subsubsection*{Repetition counts in a data word}
\index{bank!data word!documentation}
\index{data word!bank documentation}
\begin{minipage}{\textwidth}
 
Repetition counts given as a simple expression of the current data
word are decoded by routine \Rind{DZDDWD}.
 
\begin{XMPt}{Example of repetition counts for documenting data words}
*B.DATA
*B.REP  NUMH
*B.1    IHEAD      Header word for a hit
*B.1    BITS0007   IWNUM   wire number
*B.1    BITS0819   IT0     start of scanner pulse
*B.1    BITS2031   LENH    number of samples for this scanner pulse
*B.REP  BITS2031+1 / 2     ! =(LENH+1)/2
*B.1    IHIT       Coded FADCs Samples
*B.1    BITS2429   1st sample HV side
*B.1    BITS1621   1st sample signal side
*B.1    BITS0813   2nd sample HV side
*B.1    BITS0005   2nd sample signal side
*B/REP  (LENH+1)/2
*B/REP  NUMH
*B/DATA
\end{XMPt}
\end{minipage}
 
In this case the outer loop runs over a indefinite number \Lit{NUMH} of
hits each of which is preceeded by a header word \Lit{IHEAD}.
The 12 most significant bits of \Lit{IHEAD}
are used to calculate the repetition count of the inner loop describing
the individual samples of one hit.
Note that in the present example 2 samples fit into one 32-bit word).
 
 
For specifying a repetition count the allowed operators are:
\Lit{BITSnnmm} indicating the bit portion, \Lit{+}, \Lit{-}, \Lit{/} and
\Lit{*} with constant parameters.
The evaluation is strictly left to right, no brackets are allowed.
Blanks are not significant, comments are separated by `!'.
The decoding is triggered by the keyword \Lit{BITSnnmm}.
In the simplest case the current word could be directly the repetition
count:
 
\begin{XMPt}{Example of how to specify a repetition count}
*B.1     NSAMPLES
*B.REP   BITS0031      ! current word = repetition count
*B.1     IPULSE1        Pulseheight of Channel 1
*B.2     IPULSE2        Pulseheight of Channel 2
*B/REP
\end{XMPt}
 
\subsection{Banks with identical descriptions}
 
It may happen (especially for detectors which are split into two
sides) that the description of data words and links of two banks are
identical only differing in the linkage 
(I.e. only the Up-bank is different).
In this case the tag \Lit{*B.IDEM} may be used to avoid
duplication of the documentation.
 
\begin{XMPt}{Example of the use of the \Lit{IDEM} keyword}
  *B..DEHR   Header bank of right side
  ...
  *B.LINK
  *B.1    DEDA   Bank containing actual data
  ..
  *B/
  --------------------------------------------------------
  *B..DEHL   Header bank of left side
  ...
  *B.LINK
  *B.1    DEDA   Bank containing actual data
  ..
  *B/
  --------------------------------------------------------
  *B..DEDA  Bank containing actual data
  ..
  *B.UP    DEHR
  ....
  full description of bank DEDA (possibly including links)
  ....
  *B/
  --------------------------------------------------------
  *B..DEDA
  *B.UP      DEHL
  *B.IDEM    DEHR
  *B/
\end{XMPt}
 
The bank which is fully described (in this example \Lit{DEDA} with
Up-bank \Lit{DEHR}) must preceed the description containing the 
\Lit{IDEM} tag.
 
 
\subsection{Terminating the bank description}
 
The documentation for a bank must be terminated by:
\begin{verbatim}
  *B/
\end{verbatim}

\begin{Notes}
\item Continuation cards should start with \Lit{*B. } with the 
      sequence number and mnemonic field
      left blank. (Note however the special mnemonics like 
      \Lit{BITVALnn}, \Lit{BITSnnmm} etc. described above.)
\item The program will try to put as many characters on an output line 
      as will fit, embedded multiple blanks are removed unless the last 
      character is a \Lit{"/"}, in
      which case no formatting is done, except for the removal of 
      {\bf leading blanks}.
\item The pair \Lit{BKID} and \Lit{UPID}, i.e. the identifiers of a 
      bank and its up-bank should be unique in one \RZfile. 
      \index{link!up}%
      The system recognises the structure of the banks by looking for 
      \Lit{BKID}s, \Lit{UPID}s and \Lit{ID}s of the banks described 
      in the links. 
\item \Lit{JB} on the card \Lit{"*B.UP UPID-JB"}
      is not used to recognise the structure but is printed only.
\item Repetitions can be nested (3 levels maximum), 
      each \Lit{"*B.REP"} requiring a \Lit{"*B/REP"}.
\end{Notes}
 
\subsection{An example of a bank description}
 
\begin{XMPt}{Card image File describing the bank \Lit{EV}}
*B..EV  Event header bank.
*B.AU    S.Holmes
*B.VE    402
*B.ND    10
*B.NL    2
*B.NS    2
*B.NX    None
*B.UP    None
*B.OR    None
*B.IO    '3H 3I -F'
*B.LINK
*B.1   VX       Vertex bank
*B/LINK
*B.DATA
*B.1   LABNA     Name of laboratory
*B.2   EXPTNA    Name of experiment
*B.3   DAQNA     Initials of shift crew 
*B.4   IHDAT     Data type
*B.4   C1        Experiment data
*B.4   C3        Test beam data
*B.4   C4        Cosmic ray data
*B.4   C5        Monte-carlo data
*B.5   IIEVT     Trigger number
*B.6   IIFITY    Filter type (Bit string)
*B.6   BITVAL00  Sum E(clus) > 2 GeV
*B.6   BITVAL01  E(EB cls/blk) > 200MeV
*B.6   BITVAL04  Lumi Event
*B.7   IITHRU    Thrust * 10000
*B.8   IICTHR    Cosine of the thrust axis * 10000
*B.9   IIECAL    Total Electromagnetic Energy
*B.              (in units of MeV)
*B.10  IIHCAL    Total Hadronic Energy
*B/DATA
*B/
\end{XMPt}
 
\section{Generation a template for the bank descriptor file}
\label{sec:dzdocdzdtmp}  
 
A subroutine is provided to generate a template of the bank
description for an existing
data structure in memory which is not yet documented.
 
\Shubr{DZDTMP}{(ISTOR, LTOP, LUN, CHOPT)}
 
\Idesc
\begin{DLtt}{123456}
\item[ISTOR] Index of the store containing the datastructure
\item[LTOP]  Link to the top bank
\item[LUN]   Unit number of an (open) file to receive the documentation
\item[CHOPT] CHARACTER variable specifying the desired option:
\begin{DLtt}{123}
\item['T'] Treat the tree below \Lit{LTOP} 
           (default is bank at \Lit{LTOP} only).
\item['L'] Follow next links
\item['A'] Generate Author tag: \Lit{*B.AU nomen nescio}
\item['V'] Generate Version tag: \Lit{*B.VE   1.00}
\item['S'] Generate Store tag:   \Lit{*B.ST   Storename}
\item['D'] Generate division tag: \Lit{*B.DV Divname}
\end{DLtt} 
\end{DLtt} 
 
The routine recognizes when all down banks of a bank have the same
Hollerith identifier and in that case a repetition section is generated
and only the first down link is followed, i.e. it assumes that
the other substructures are identical.
 
This routine can be called interactively from \Rind{DZDISP}.
 
%\finalnewpage

\section[{\tt DZEDIT}, a program to maintain documentation]%
        {\Pind{DZEDIT}, a program to maintain documentation}
\label{sec:dzdocdzedit}  
 
\Pdef{DZEDIT} is an interactive program to manipulate the documentation
database.
The program can handle one \RZfile{} and one input
cards file in the same session. 
All file handling (\Lit{OPEN}, \Lit{CLOSE} and \Lit{FILDEF}) 
are done by the program. 
The \RZfile{} is opened with \Lit{STATUS = `NEW'} 
if it should by created, the other files
(Listing, plotfiles) are opened with \Lit{STATUS='UNKNOWN'},
i.e. they will be overwritten if they exist.
It allows to create the \RZfile,  to update it and
produce printed documentation of all or selected entries.
Documentation including graphical representation of the bank trees
may be produced in various formats:
 
\begin{ULc}
\item Pure PostScript~\cite{Adobe:red2} output on a file \Lit{xxxx.ps} can be
      directly printed on a PostScript laser printer.
      The \Ropt{P} option should be used with the appropriate commands.
\item Pure \LaTeX~\cite{bib-LATEX} formatted output on a file \Lit{xxxx.tex} 
      is obtained with the \Ropt{L} option.
\item The command \Lit{HTML} produces output in HTML format.
      ZEBRA links are translated into hypertext links that allow one
      to step through the data structure with a WWW browser. 
\end{ULc}
 
\Pind{DZEDIT} uses \Pind{KUIP}~\cite{bib-KUIP} as command interface. 
 
\subsection[{\tt DZEDIT} command overview]{\Pind{DZEDIT} command overview}
The program remembers the names of the \RZfile{} and the card image
file in a file called \Lit{dzedit.las}.
 
\SKUIP{CREATEDOC}{chcard chrzf}
 
\begin{verbatim}
chcard     C 'Input card image file'            D=' '
chrzf      C 'Output RZ-file'                   D=' '
\end{verbatim}
 
Create a new \RZfile{} from documentation card image file
 
\SKUIP{UPDATEDOC}{chcard chrzf [ chsubd ]}
 
\begin{verbatim}
chcard     C 'Input card image file'            D=' '
chrzf      C 'Output RZ-file'                   D=' '
chsubd     C 'Subdirectory name (blank=topdir)' D=' '
\end{verbatim}
 
Update an existing \RZfile{} with new documentation, this option
is also used to put the documentation into any subdirectory
of an existing \RZfile.
 
\SKUIP{EXPORTFILE}{chfzf}
 
\begin{verbatim}
chfzf      C 'FZ-file name'                     D=' '
\end{verbatim}
 
Write currently open \RZfile{} as a \FZfile{} in Alpha exchange mode for
transport to a different computer.
 
\SKUIP{IMPORTFILE}{chfzf chrzf}
 
\begin{verbatim}
chfzf      C 'FZ-file name'                     D=' '
chrzf      C 'RZ-file name'
\end{verbatim}
 
Read a \FZfile{} in Alpha exchange mode previously exported via the
command \Cind{EXPORTFILE} on different (or same) computer. 
A new \RZfile{} will be generated.
 
\SKUIP{OPENRZFILE}{chrzf [ choopt ]}
 
\begin{verbatim}
chrzf      C 'Input RZ-file'                    D=' '
choopt     C 'Option for RZFILE (U=Update)'
\end{verbatim}
 
Open an existing \RZfile{} for later use with \Lit{list/draw/export} 
commands
(the file will be \Lit{READONLY} unless the \Lit{'U'=update} option is given)
 
\SKUIP{PURGEKEY}{[ nkeep ]}
 
\begin{verbatim}
nkeep      I 'Number of cycles to be kept'      D=-1
\end{verbatim}
 
Purge cycles of all keys keeping the last \Lit{nkeep}.
If \Lit{nkeep < 0} keep just the highest cycle.
 
\SKUIP{DELETEKEY}{chbsbk chbsup [ icycle chdopt ]}
 
\begin{verbatim}
chbsbk     C 'Hollerith Id of selected bank'
chbsup     C 'Hollerith Id of its up-bank'
icycle     I 'Cycle number'                     D=0
             > highest: delete highest cycle,
             =0:        delete lowest cycle,
             =-1, -2 :  delete highest-1, -2..
chdopt     C 'Delete option'                    D='C'
             'C': delete all cycles
             'K': delete all keys,
             'S': delete all cycles smaller ICYCLE.
                  (See ZEBRA users guide: RZDELK)
\end{verbatim}
 
Delete a key specifying \Lit{BankId} and \Lit{UpBankId} and cycle number.
 
\SKUIP{LISTDIRECTORY}{chrzf}
 
\begin{verbatim}
chrzf      C 'Input RZ-file'                    D=' '
\end{verbatim}
 
List directory of an existing \RZfile.

%\finalnewpage 
\SKUIP{LISTONEBANK}{chbsbk chbsup [ chlist chlopt choyno idatch ]}
 
\begin{verbatim}
chbsbk     C 'Hollerith Id of selected bank'    D=' '
chbsup     C 'Hollerith Id of its up-bank'      D='****'
chlist     C 'File for listing'                 D=' '
chlopt     C 'List option:                      D=' '
             'P' PostScript file (default: simple text file)
             'I' use format of the bank descriptor cards
                 (i.e. Input to CREATEDOC)   
 
choyno     C 'List all cycles'                  D='NO' R='NO,YES'
idatch     I 'List only after date'             D=0
\end{verbatim}
 
List documentation for a selected bank or a group of banks.
An \Lit{'*'} (asterix) may be used as wild card character.
Exactly two times 4 characters are needed to define bank and up-bank.
only the last cycle of one entry is listed regardless of its date.
List of all cycles or selection by date of entering the \RZfile{} may
also be choosen. The date is an integer of the form \Lit{YYMMDD}.
 
\SKUIP{LISTALL}{[ chlist chlopt choyno idatch ]}
 
\begin{verbatim}
chlist     C 'File for listing'                 D=' '
chlopt     C 'List option:                      D=' '
             'P' PostScript file (default: simple text file)
             'I' use format of the bank descriptor cards
                 (i.e. Input to CREATEDOC)   
choyno     C 'List all cycles'                  D='NO' R='NO,YES'
idatch     I 'List only after date'             D=0
\end{verbatim}
 
List documentation for all banks
 
\SKUIP{HTML}{ chbsbk chbsup [ chopt ]}

\begin{verbatim}
chbsbk     C 'Hollerith Id of selected bank'
chbsup     C 'Hollerith Id of its up-bank'
chopt      C 'Option: number of output files written D=' '
              default: a different file is written for each bank
              = 1 all output is written into a single file
\end{verbatim}

Generate documentation in HTML format suitable for a WWW browser. 
One should use a \Lit{chbsbk} equal to \Lit{'****'} 
and \Lit{chbsup} equal to \Lit{'NONE'} to generate a description
for all banks.

\SKUIP{DRAWONETREE}{chbsbk chbsup [chmeta chtext chpost chopt ctitle]}
 
\begin{verbatim}
chbsbk     C 'Hollerith Id of selected bank'
chbsup     C 'Hollerith Id of its up-bank'
chmeta     C 'Name of plotfile'                 D=' '
chtext     C 'Name of text file'                D=' '
chpost     C 'Name of PostScript file'          D=' '
chopt      C 'Option (P=PostScript L=Latex)'    D=' ' R=' ,P,C,L,PC,LC'
              'P' PostScript file
              'L' Latex file
              'Q' do not produce output files
              'C' check consistency
ctitle     C 'Global title'                     D='ZEBRA-Datastructures'
\end{verbatim}
 
Draw tree below a selected bank. 
All banks belonging to the tree will be
actually lifted in memory each with three data words. 
Word 1, 2, 3 are the number of data words, links and structural links
as described in the documentation, a $-1$ indicates a variable number.
A global title may be given which appears on the front page of
thge document.
 
\SKUIP{DRAWALL}{[ chmeta chtext chpost chopt ctitle ]}
 
\begin{verbatim}
chmeta     C 'Name of plotfile'                 D=' '
chsgml     C 'Name of text file'                D=' '
chpost     C 'Name of PostScript file'          D=' '
chopt      C 'Output:                           D='S'
             'P' PostScript file
             'L' Latex file
             'N' do not check consistency
             'Q' do not produce output files (check only)
ctitle     C 'Global title'                     D='ZEBRA-Datastructures'
\end{verbatim}
 
Draw tree below all top banks (i.e. banks having \Lit{"None"} as
up-bank. This also checks the consistency and completeness
of the documentation if option \Copt{N} is not given.
 
\SKUIP{DZDISP}{[ chopt ]}
\begin{verbatim}
chopt      C 'Option:                           D=' ' 
             'C' use color
\end{verbatim}
 
Display the last generated tree if graphics is available. 
Note that the number of data words is three for each bank. 
Their
contents indicates the documented number of data words and links.
 
\SKUIP{DZDDIV}{[ chopt ]}
\begin{verbatim}
chopt      C 'Option:                           D=' ' 
             'C' use color
\end{verbatim}
 
Call subroutine \Rind{DZDDIV}.
 
\SKUIP{DZDIRZ}{[ chopt ]}
\begin{verbatim}
chopt      C 'Option:                           D=' ' 
             'C' use color
\end{verbatim}
 
Call subroutine \Rind{DZDIRZ}, display RZ-directory tree
 
\subsection{Preparing Fortran code}
\label{sec:makecode}
\enlargethispage{\baselineskip}
 
\Pdef{MAKECODE} generates \Pind{PATCHY} \Lit{KEEP} sequences 
containing Fortran code.  
With option \Copt{S} only the selected bank is treated, 
with option \Copt{T} all banks in the selected tree are treated. 
The \Lit{KEEP} sequences are named
with the hollerith Identifiers of the bank(trees) prefixed by mnemonics
like \Lit{BOOK}, \Lit{LKOFF}, \Lit{DAOFF} etc.
 
\SKUIP{DATAOFFSETS}{chbsbk chbsup [ chlist chpfix  choptd ]}
 
\begin{verbatim}
chbsbk     C 'Hollerith Id of selected bank'
chbsup     C 'Hollerith Id of its up-bank'
chlist     C 'File for listing'                 D=' '
chpfix     C 'Prefix to mnemonic'               D=' '
choptd     C 'Option'                           D='T'
             'S' One selected bank
             'T' Complete tree
\end{verbatim}
Generate sequences of data word offsets for a selected bank
or a complete tree. The mnemonics given in the documentation can be
prefixed by a (upto 8 char) string like: \Lit{DO_}
 
\begin{XMPt}{Example of data word documentation}
         *B.4 IIDAT4     This is the 4th data word
 {\rm would produce:}
         INTEGER IIDAT4
         PARAMETER (IIDAT4=4)
\end{XMPt}
 
\SKUIP{LINKOFFSETS}{chbsbk chbsup [ chlist chpfix choptl ]}
 
\begin{verbatim}
chbsbk     C 'Hollerith Id of selected bank'
chbsup     C 'Hollerith Id of its up-bank'
chlist     C 'File for listing'                 D=' '
chpfix     C 'Prefix to BankId'                 D='LO'
choptl     C 'Option'                           D='T'
             'S' One selected bank
             'T' Complete tree
\end{verbatim}
 
Generate sequences of link offsets for a selected bank or tree.
Exactly two times 4 characters are needed to define bank and up-bank.
The variable generated is named \Lit{chpfix} concatenated with the
bank Identifier.
 
%\finalnewpage

\begin{XMPt}{Example of the description of a link}
         *B.LINK
         ...
         *B.7 IDBK  Bank containing anything
 {\rm would produce:}
         INTEGER LOIDBK
         PARAMETER (LOIDBK=7)
\end{XMPt}
 
\SKUIP{LINKASSIGNMENT}{chbsbk chbsup [ chlist chpfix choptl ]}
 
\begin{verbatim}
chbsbk     C 'Hollerith Id of selected bank'
chbsup     C 'Hollerith Id of its up-bank'
chlist     C 'File for listing'                 D=' '
chpfix     C 'Prefix to BankId'                 D='LO'
choptl     C 'Option'                           D='T'
             'S' One selected bank
             'T' Complete tree
\end{verbatim}
 
Generate sequences of link assignment statements for selected
bank or tree.
Exactly two times four characters are needed to define bank and up-bank.
The assumed linkoffset is named \Lit{chpfix} concatenated with the
bank identifier.
Example: if a 
\par
\begin{XMPt}{Example of link to bank \Lit{IDBK} with Up-bank \Lit{IDUP}}
         *B.UP IDUP
         *B.LINK
         ...
         *B.7  IDBK  Bank containing anything
 {\rm would produce:}
         INTEGER LOIDBK
         LOIDBK=LQ(LOIDUP-7)
\end{XMPt}
 
\SKUIP{BOOK}{chbsbk chbsup [ chlist choptb ]}
 
\begin{verbatim}
chbsbk     C 'Hollerith Id of selected bank'
chbsup     C 'Hollerith Id of its up-bank'
chlist     C 'File for listing'                 D=' '
choptb     C 'Option'                           D='TB'
             'S' One selected bank
             'T' Complete tree
             'B' Generate call to MZBOOK
             'L' Generate call to MZLIFT
\end{verbatim}
 
Generate code to book a bank or bank tree with \Rind{MZBOOK}
or \Rind{MZLIFT}.
For \Rind{MZLIFT} the bank parameters are put into arrays
\Lit{MMIDBK}, which go into their own \Lit{KEEP} sequence.
The link to the bank \Lit{"BANK"} is named \Lit{LBANK}, 
the uplink is assumed \Lit{LUPBK} if the up-bank is called \Lit{"UPBK"}.
For \Lit{LBANK} the declaration \Lit{"INTEGER LBANK"} is generated, 
for \Lit{LUPBK} it is assumed to be done already. 
A call to \Rind{MZFORM} is generated if not all data
words are of the same type. The IO-char may be given expicitely for
each data word or in line like: \Lit{*B.IO  3I 4F}.
The link offset (\Lit{JBIAS}) is taken from the card \Lit{*B.UP UPBK -JBIAS}, 
if \Lit{JBIAS} is given. 
If it is not given it is searched for in the documentation of the Up-bank. 
If the string specified for \Lit{UPBK} is \Lit{"NONE"},
it is set to \Lit{+1}.
 
\begin{XMPt}{Defaults for non explicitly documented parameters}
  IXDIV:  0 (i.e. div 2 in store 0)
  JBIAS:  no default
  NL:     0
  NS:     0
  ND:     0
  IOChar: no default, if not given with the data word description
  NZERO:  0
\end{XMPt}
 
\SKUIP{LIFT}{chbsbk chbsup [ chlist choptb ]}
 
\begin{verbatim}
chbsbk     C 'Hollerith Id of selected bank'
chbsup     C 'Hollerith Id of its up-bank'
chlist     C 'File for listing'                 D=' '
choptb     C 'Option'                           D='TL'
             'S' One selected bank
             'T' Complete tree
             'B' Generate call to MZBOOK
             'L' Generate call to MZLIFT
\end{verbatim}
 
Generate code to lift a bank or bank tree.
See the description of \Cind{BOOK} above.
 
\subsection*{Remarks on \Pind{DZEDIT}}
 
\begin{itemize}
\item In all \Pind{DZEDIT} commands \Lit{CHBSBK} and \Lit{CHBSUP} can contain 
      the wildcard character \Lit{*}, e.g. \Lit{"CV**"} for \Lit{CHBSBK} 
      and \Lit{"****"} for \Lit{CHBSUP} would select all banks starting
      with \Lit{"CV"} (\Lit{CV}, \Lit{CVRA}, ...). 
\item The command \Cind{DRAWALL} will scan the \RZfile{}
      for all banks having as Up-bank \Lit{"None"} (or \Lit{"NONE"}) 
      and generate the graphical documentation. 
      At the same time it checks whether all banks in the file are connected 
      in a tree and are not used more than once
      (i.e., the pair \Lit{BankId}  and \Lit{UpBankId} is not unique).
\end{itemize}
 
\subsection*{Remarks on updating}
 
When the update option is specified, the input Cards file is
first read and new keys (or a new cycle of a key if it is already present)
are produced. 
It then checks for the cases where a new cycle has been added, if the data
in the new cycle are identical to the one in the previous cycle.
If so, it deletes the newly created cycle. 
So the update can be done
with a Cards file containing documentation on all banks or just
those which should be added or modified. 
Thus old cycles of the updated banks are kept and can be listed on request.
(See option ``list all cycles'')
 
%\finalnewpage

\section{The interactive bank display tool}
\label{sec:dzdocdzdisp} 
 
The following routines are provided to visualise the bank structure,
the layout of banks in divisions or the contents of RZ~directories 
in an interactive session. 
The routine \Rind{DZDISP}
is used to step through bank trees, display the contents of system
words, links and datawords together with the documentation contained
in an \RZfile{} if available.
Since ZEBRA provides forward and backward linkage of banks,
the routine can reach each bank of a datastructure if it is provided
with the link to any bank of the structure. 
Any picture may be written to a plotfile 
(GKS, PostScript, LaTeX) to get a hardcopy of it.
 
Routine \Rind{DZDDIV} is used to graphically display the bank layout
in a division. 
Banks are marked with their hollerith Identifier.
One can interactivley select any division of any store and
zoom out regions of divisions. 
By clicking on a bank in the division
one can directly enter \Rind{DZDISP} to display the corresponding bank tree.
Both routines assume that the graphics package (\cite{bib-GKS1} or
\cite{bib-HIGZ}) and a workstation
have been opened by the appropriate calls.
 
\Rind{DZDIRZ} is used to display and step through RZ~directories.

\finalnewpage

\subsection[{\tt DZDISP} -- Display bank trees in an interactive session]%
           {\Rind{DZDISP} -- Display bank trees in an interactive session}
\begin{Fighere}
\begin{center}
    \fbox{\epsfig{figure=dzdisp.eps,height=12cm}}
\caption{Example of output generated by DZDISP}
\label{fig:DZDOCFIG2}
\end{center}
\end{Fighere}
 
\Shubr{DZDISP}{(ISTOR,*LTREE*,RZPATH,CHOPT,IWDISP,IWMETA,ILOCNR,IWKTYP)}
 
\Idesc
\begin{DLtt}{123456}
\item[ISTOR]  Index of the store with the bank tree to be analyzed
\item[LTREE]  Entry address into the bank tree structure
\item[RZPATH] Character variable specifying the path name of a RZ-directory
              containing the DZDOC bank descriptions.
              If no RZ-file is available one sets \Lit{RZPATH = ' '}.
\item[CHOPT]  Character variable specifying the desired options:
  \begin{DLtt}{123}
      \item['N'] Do not activate and deactivate workstation,
                 assumed to be done by caller
      \item['M'] write on plotfile also the menu boxes
      \item['L'] return link of selected bank to caller in \Lit{LTREE}
      \item['R'] handle ``Modify data'' as ``Quit''
  \end{DLtt}
\item[IWDISP] Workstation identifier for the display
\item[IWMETA] Workstation identifier for a possible plotfile, 
              a value of \Lit{0} means no plotfile.
\item[ILOCNR] Locator number (normally = 1)
\item[IWKTYP] The workstation type (display screen)
\end{DLtt}
\Odesc 
\begin{DLtt}{123456}
\item[LTREE]  Address of the selected  bank, if \Ropt{'L'} option is given.
\end{DLtt}
 
\subsubsection*{Interaction with \Rind{DZDISP}}
 
Routine \Rind{DZDISP} activates the workstation (\Lit{IWDISP}) if
the \Ropt{`N'} option is not given, clears
the screen and draws the bank tree. 
Then it adds menu boxes on the picture allowing the user 
to choose one of the following options
\begin{DLtt}{1234567890}
\item[Quit]       Return to caller
\item[Cont]       Continue to draw down banks which did not fit on page
\item[=>plotfile] Output onto plotfile (if \Lit{IWMETA > 0})
\item[Drop bank(tree) IDBK\ ]
                  Drop the currently selected top bank with hollerith 
                  Identifier \Lit{IDBK} and its dependents. 
                  A possible next bank is not dropped.
\item[Modify data in bank IDBK\ ]
                  Modify data words in the currently selected top bank. 
                  It asks for the first and last word to be modified. 
                  If the first and last word are the same only one word 
                  is affected, otherwise all selected words will be set
                  to the same value. 
                  System words may be modified by choosing offsets less
                  or equal zero.
                  Only INTEGER values are allowed in this case.
                  A \Lit{$} in front of an INTEGER allows hexadecimal input.
\item[Help]       Print instructions
\item[=>LaTeX]    Print current picture in \LaTeX{} format to \Lit{UNIT} (see below)
\item[CHOPT]      Character options for \Rind{DZSHOW} with the
                  following extensions: 
                  \Lit{'W'} bitwise dump of data words with
                  modifiers \Lit{'1'}, \Lit{'2'}, \Lit{'3'} giving the field width, 
                  \Lit{'0'} forces also the bits with zero value to be shown as \Lit{0}, 
                  the default is blank. 
                  The option \Lit{'C'} forces only the data content to be shown 
                  without any other text like sequence numbers etc., 
                  this is useful if the dump should be read by another
                  program (e.g. \Pind{PAW}).
\item[FIRST]      First word to show (with \Rind{DZSHOW})
\item[LAST]       Last word to show (with \Rind{DZSHOW})
\item[UNIT]       Unit for printed output (\Rind{DZSHOW}, \LaTeX, etc.,  (6 = terminal).
                  If a number $\neq 6$ is specified, then a file
                  \Lit{forxxx}, where \Lit{xxx} is set equal to \Lit{UNIT} 
                  will be opened and used. 
                  Whenever \Lit{UNIT} changes the old file is closed.
\end{DLtt}
 
It then requests activation of the locator (mouse) and takes the
following actions  according to its location:
 
\begin{enumerate}
\item If it is inside the (shaded) bank center or in the
      arrow indicating a possible next bank or in the box at top of
      the picture indicating the Up-bank it calls \Rind{DZDRAW}
      for the selected bank, i.e. it draws the tree for the selected
      bank.
\item If it is inside the data box it displays the selected datawords
      by a call to \Rind{DZSHOW} for this bank.
\item If it just above the data box it lists the selected datawords
      together with the documentation (if it is provided in the \RZfile).
      (See also routine: \Rind{DZDDWD})
\item If it is inside the link region (or just left of the bank
      center if there are no links) it outputs the system words
      and links of the bank to the selected unit.
\item If the cursor is just above the shaded bank center the documentation
      for the bank as contained in the \RZfile{} is displayed. 
      If it does not exist a template of the documentation is written to the selected
      output stream (see \Lit{UNIT} above).
\end{enumerate}
 
\begin{Note}
The link \Lit{LTREE} pointing to the data structure should be in a
link area.
On some graphics terminals (e.g. PERICOM GRAPH) the alpha screen
is used for the printed output. 
It needs input of any character
followed by RETURN to switch back to the graphics screen.
\end{Note}
 
\subsection[{\tt DZDDWD} -- display contents of data words with documentation]%
           {\Rind{DZDDWD} -- display contents of data words with documentation} 
 
\Shubr{DZDDWD}{(ISTOR, LBANK, CHOPT, IFIRST, LAST, LUNPRT)}
 
\Idesc
 
\begin{DLtt}{123456}
\item[ISTOR]  Index of the store with the bank tree to be analyzed
\item[LBANK]  Address of the bank about which information has to be displayed
\item[CHOPT]  Character variable specifying the following options:
  \begin{DLtt}{123}
     \item['C'] short display, show contents of data words only,
                useful if data should subsequently read by a program again.
     \item['U'] do not display if mnemonic is \Lit{UNDEFIND}.
     \item['Z'] force hexadecimal
     \item['T'] force one line of output/4 characters in case of
                hollerith text. By default text is assembled in lines
                of 60 characters, no formatting is attempted.
     \item['E'] Check if values are within allowed range (Examine), if
                a range is defined in documentation.
  \end{DLtt}
\item[IFIRST] First word to show
\item[LAST]   Last word to show
\item[LUNPRT] Unit number for output
\end{DLtt}
 
\finalnewpage

\begin{Notes}
\item \Lit{IFIRST} and \Lit{LAST} may be 0, in this case all datawords
      of a bank will be listed. 
\item If no documentation for the bank exists, only the contents of the 
      datawords will be shown (\Rind{DZSHOW}).
\item Only two levels of repetitions are supported, the inner level
      must be given explicitly except for cases where the repetition count
      is a simple expression of the current data word.
\end{Notes}
 
\subsection[Examples of {\tt DZDISP} output]{Examples of \Rind{DZDISP} output}
 
\begin{XMPt}{Documentation for a bank (click above bank center)}
------------------------------------------------------------------------------
| EV   | Event header bank.
 ---------------------------------------------- entered file at 19-Oct-92  9: 3
 Bank IDH  EV       Event header bank.
 Author             S.Holmes
 Version            402
 NL               2
 NS               2
 ND              10
 Next      None
 Up        None
 Origin    None
 IO-Charac          3H 3I -F
              ---------- Description of the links  ----------
 1         VX       Vertex bank
              ---------- Description of the data words   ----------
 1         LABNA    Name of laboratory
 2         EXPTNA   Name of experiment
 3         DAQNA    Initials of shift crew
 4         IHDAT    Data type
 4         C1       Experiment data
 4         C3       Test beam data
 4         C4       Cosmic ray data
 4         C5       Monte-carlo data
 5         IIEVT    Trigger number
 6         IIFITY   Filter type (Bit string)
 6         BITVAL00 Sum E(clus) > 2 GeV
 6         BITVAL01 E(EB cls/blk) > 200MeV
 6         BITVAL04 Lumi Event
 7         IITHRU   Thrust * 10000
 8         IICTHR   Cosine of the thrust axis * 10000
 9         IIECAL   Total Electromagnetic Energy
 10        IIHCAL   Total Hadronic Energy
\end{XMPt}
 
\finalnewpage
%\bigskip
 
\begin{XMPt}{Display of Undocumented data (click in data box)}
 DZSHOW ---  DZSHOW called from DZDISP
 
EV  .     1     8781(QDIV2   ) SY/US/IO    1/    0/2153 
NL/NS/ND    7/    7/      10 N/U/O/@O       0/       0/       1/
   8781
 --------  DATA part of bank  --------
 
       1 /       "Cern       "Mars       "McKi           5      123456
                         3   9945.       8678.       .9120E+05   1300.
\end{XMPt}
 
\bigskip
 
\begin{XMPt}{Display of documented data (\Rind{DZDDWD})(click above data box)}
 -------- Data of Bank/UpBank: EV  /NONE Doc Version: 402 ----------
     1    1  LABNA          "Cern  Name of laboratory
     2    2  EXPTNA         "Mars  Name of experiment
     3    3  DAQNA          "McKi  Initials of shift crew
     4    4  IHDAT              5  Data type
     4    4                        Monte-carlo data
     5    5  IIEVT         123456  Trigger number
     6    6  IIFITY             3  Filter type (Bit string)
     6    6                        Sum E(clus) > 2 GeV
     6    6                        E(EB cls/blk) > 200MeV
     7    7  IITHRU     9945.      Thrust * 10000
     8    8  IICTHR     8678.      Cosine of the thrust axis * 10000
     9    9  IIECAL     .9120E+05  Total Electromagnetic Energy
    10   10  IIHCAL     1300.      Total Hadronic Energy
\end{XMPt}
 
\bigskip
 
\begin{XMPt}{List system words (click in link area)}
 System words + links
 Offset to Bank-Centre              20
 I/O characteristic          3H 3I -F
 Bank status word (HEX)          40000
 Link to the bank                 8781
 Hollerith ID                     EV
 Numerical ID                        1
 Total number of links               7
 Number of structural links          7
 Number of data words               10
 Next link                           0
 Up - link                           0
 Origin link                         1
     1. down link                 8595 VX
     2. down link                    0 
     3. down link                    0
     4. down link                    0
     5. down link                    0
     6. down link                    0
     7. down link                    0
\end{XMPt}
 
\finalnewpage
 
\subsection[{\tt DZDDIV} -- Display the layout of stores and divisions]%
           {\protect\Rind{DZDDIV} -- Display the layout of stores and divisions}
 
\Rind{DZDDIV} does a sequential search through the specified division and
displays the location of the banks graphically. 
Hence it displays all banks (also dropped ones) regardless of their
linkage. 
In this respect it is principally different from \Rind{DZDISP}.
Active banks are marked by one horizontal line in the box
representing the bank, dropped banks are shaded. 
The program tries to fit the hollerith identifier of the bank into the box.
Characters which do not fit in a box are truncated from the end of the string.
The whole division is represented by about 30 horizontal boxes.
To the left of each box the \Lit{LQ-}address in the store is shown.
The length of the box is adjusted to represent an integer multiple
of powers of ten (e.g. 200 or 3000) words.
In addition to the selected division the layout of the store
containing the division is displayed at top of the picture
showing the locations and printing names of all divisions in the store.
\Rind{DZDISP} may be directly entered by clicking on a bank in the display.
 
\begin{minipage}{\textwidth}
\begin{Fighere}
  \begin{center}
    \fbox{\epsfig{figure=dzddiv.eps,height=132mm}}
    \caption{Example of a display of \protect\Rind{DZDDIV}}
  \end{center}
\end{Fighere}
\end{minipage}
 
%\finalclearpage
 
\Shubr{DZDDIV}{(IXDIV,L,RZPATH,CHOPT,IWDISP,IWMETA,ILOCNR,IWKTYP)}
 
\Idesc
 
\begin{DLtt}{123456}
\item[IXDIV]  Index of the division to be displayed, if just the store index
              is given division 2 is taken as default.
\item[L]      not used at present.
\item[RZPATH] CHARACTER variable specifying the pathname of the \RZfile{} containing
              the bank documentation, if no \RZfile{} is available one sets 
              \Lit{RZPATH= ' '}.
\item[CHOPT]  CHARACTER variable specifying the desired option:
  \begin{DLtt}{123}
     \item['I']  Interactive (i.e. request cursor input after display)
     \item['N']  Do not activate and deactivate workstation, assumed to be done
                 by caller
     \item['P']  print addresses of banks to terminal (or file)
  \end{DLtt}
\item[IWDISP] workstation Identifier for the display, 
              \Lit{IWDISP} must be open, activation and deaction is done by the routine
              if the \Lit{'N'} option is not given.
\item[IWMETA] Workstation Id for a possible Meta file (0 means no plotfile)
\item[ILOCNR] locator number (normally = 1)
\item[IWKTYP] the workstation type (display screen)
\end{DLtt}
 
\subsubsection*{Interaction with \Rind{DZDDIV}}
 
If the interactive option has been choosen additional menu
boxes are drawn and the program expects input
of the locator with the following choices.
 
\begin{DLtt}{123456789012345}
\item[QUIT]       Return to caller
\item[=>Plotfile] Write current picture on GKS--plotfile
\item[ZOOM]       Show enlarged part of division.
                  The program expects then 2 inputs by the locator
                  indicating the area to be enlarged. 
                  It chooses the start of the bank with the lower address and the end
                  of the bank with the higher address.
\item[In division list] Allows selection of a division.
\item[In store list]    Allows selection of a store.
\end{DLtt}
 
If the locator is activated on a division in the box at the top
representing the store then  the picture is erased and redrawn
with the selected division expanded.
 
If the locator is activated on a bank in the division display
\Rind{DZDISP} is called with the corresponding bank selected.
This allows to acces and display any bank tree in a store
without prior knowledge of its entry address.
 
\finalnewpage
 
\subsection[{\tt DZDIRZ} display RZ directory trees in an interactive session]%
           {\Rind{DZDIRZ} display RZ directory trees in an interactive session}
 
This routine displays RZ--directories and the keys in a similiar manner as
\Rind{DZDISP} does it for bank trees. 
A bank structure reflecting the \RZfile{}
structure is lifted and displayed by  the routines of \Lit{DZDISP}.
To distinguish the output from bank trees the boxes are
drawn like diamonds rather then rectangular. 
The keys are indicated with rectangular boxes. 
By default only the 10 most recent keys in a directory are displayed. 
This might be overwritten by specifying a parameter \Ropt{NKnnn}
in the \Rarg{CHOPT} argument where \Lit{nnn} represents an integer number.
Additional keys (of the selected working directory) 
may be shown using the menu option \Lit{"more keys"}.
The first 4 keywords are shown in the box. 
To the left of the box the sequence
and cycle number of the key are displayed.
If there are more they may be shown by clicking above the box (see below).
The data of the key can be brought into memory by clicking into the
box representing the key and subsequently examined by calls to \Rind{DZDISP}.
Optionally the contents of the key vector is returned in a next bank (Option \Ropt{'K'}).
At the upper left corner the full name of the current working directory
is displayed which remains selected when returning to the caller.
So this routine may be used to select interactively a working directory.
The routine needs about 20 words/directory or key depending on
the length of the name or number of keywords in division 2 of the
selected store. 
On return from \Rind{DZDIRZ} the datastructure is dropped
by default, if the \Ropt{'S'}-option is given it is kept and must be dropped
explicitly (\Lit{"drop+quit"}). 
This is useful for very big \RZfile s to save time building the datastructure. 
Note however if a directory tree which is
not part of the current one in memory should be shown
one needs to do a \Lit{"drop+quit"}.
before the change of \Rarg{RZPATH} containing the directory
to be displayed has an effect.
 
A simple implementation of a documentation scheme for RZ-directories is
supported. 
The documentation is contained in a formatted text file,
where a line starting with \Lit{'/'} gives the directory name followed by
any number of comment lines not starting with \Lit{'/'} in the first column. 
The  directory name should be unique, i.e. leading directories may be 
omitted. 
The first  comment line is put into the box representing the
directory on the display. 
Clicking above this box shows all lines.
The file must be open when \Rind{DZDIRZ} is called, the logical unit is passed
in the \Rarg{CHOPT} argument as \Ropt{LUnn} where \Lit{nn} is the unit number. 
 
 
\subsubsection*{Interaction with \Rind{DZDIRZ}}
 
The following actions can be triggered by clicking the mouse at the
following places:
 
\begin{enumerate}
\item If it is inside the (shaded) center or in the
      box at top of the picture indicating the mother directory
      it changes the current working directory to the selected one
      and calls \Rind{DZDIRZ} again, i.e. one steps down or up the directory tree.
\item If it is inside the big box of a directory, which shows the number of
      keys and its quota, it lists the contents of this directory (\Rind{RZLDIR}).
\item If it is above a rectangular box representing a key all keywords will be shown.
\item If it is inside a rectangular
      box representing a key, this key is read into memory (\Rind{RZIN}) at link
      \Rarg{LDATA} with offset \Rarg{JBIAS} (or as top bank if \Lit{JBIAS=1}).
\end{enumerate}
 
\finalnewpage
 
\begin{Fighere}
\begin{center}
    \fbox{\epsfig{figure=dzdirz.eps,height=14cm}}
\end{center} 
\caption[Example of output generated by {\tt DZDIRZ}]{Example of output generated by \Rind{DZDIRZ}}
\end{Fighere}
 
\Shubr{DZDIRZ}{(ISTOR, LD*, JB, RZPATH, CHOPT, IWDISP, IWMETA, ILOCNR)}
 
\Idesc
 
\begin{DLtt}{123456}
\item[ISTOR]  Index of the store to be used
%\item[LD]     Link where to put read in data structure
\item[JB]     The link offset (or 1 if top link)
\item[RZPATH] Character variable specifying the path name of a RZ-directory
\item[CHOPT]  Character variable specifying the options desired:
  \begin{DLtt}{12345}
    \item['N']    Do not activate and deactivate workstation,
                  assumed to be done by caller
    \item['A']    Show all keys (default 10)
    \item['S']    Save (time) i.e. do not drop banks containing
                  the RZ-structure on return.
    \item['K']    Return the keyvector in a next bank if the data for
                  a key a read into a bank (\Rind{RZIN}). 
                  In this case control is immediatly returned to the caller.
    \item['DUnn'] \Lit{nn} is the logical unit number of an open file
                  containing documentation. This (an the following 2)
                  must follow the previous options (\Lit{NASK}).
    \item['LUnn'] \Lit{nn} is the logical unit number of an open file
                  receiving printed output.
    \item['NKnn'] Maximum number of keys to show on display.
  \end{DLtt}
\item[IWDISP] Workstation identifier for the display
\item[IWMETA] Workstation identifier for a possible plotfile,
              a value of 0 means no plotfile
\item[ILOCNR] Locator number (normally = 1)
\end{DLtt}
 
\Odesc
 
\begin{DLtt}{123456}
\item[LD]  Link to a possibly read in data structure or bank containing 
           a vector
\end{DLtt}
 
\section{DZDOC with the Motif interface of KUIP}
Large parts of the DZDOC and DZDISP package described in the previous
sections have been implemented using the Motif % \cite{bib-Motif} 
interface of the new KUIP \cite{bib-KUIP}. One single subroutine 
call
 
\Shubr{ZBRDEF}{ }
 
gives access to the features briefly described in the following two
sections which are taken from the HELP available interactivley.
 
\subsection{The Zebra-Browser}
 
Selecting "Zebra" from the KUIP object browser will display
an icon for each Zebra store, for each open Fzfile and each
open Rzfile. A store is named with its number plus the name
given to MZSTOR with slashes (/) replaced by underscores (\_).
A Fzfile is named with the logical unit number, a Rzfile with
the top directory given to it by RZFILE.
 
Double click in the store symbol will display icons for the
divisions in this store, clicking the divisions will display
the banks im this divisions, clicking a bank will show the
tree for this bank in the graphics window (see DZDISP).
Popup menus (use right mouse key) are provided which allow to
print information on stores, divisions and banks (DZSTOR,
DZSNAP, DZSURV). The layout of banks in divisions can be
shown graphically by the command 'Display\_division'.
 
Double click in the FZfile symbol will read the next data
structure from the file into division 1 of store 0 and
display the bank tree for the top bank in the graphics
window. A popup menu allows to read the User header only,
to display information (see FZINFO) on the file or close the
file. Trying to read beyond End of Data will force the
file to be rewound to avoid exit via ZFATAL.
 
Double click in the RZfile symbol will display directories
and/or keys contained in the file. Double click on a key
symbol will read the data structure for this key into
division 1 of store 0 and display the bank tree for the
top bank in ths graphics window. Show status (RZSTAT) and
close are accessible by a popup menu.
 
Several parameters used by the browser may be set by commands
found in the menu \Lit{DZDOC/SET_PAR}. These allow to choose
if workstations should be actived, if a new window should be opened
to receive printed output, if colors should be used in the 
bankdisplay etc.
 
\subsection{Embedded DZDOC}
 
The Zebra browser popup menus give access to parts of the
DZDOC package. DZDOC allows to describe Zebra banks (linkage
and data words) with a defined format.
It puts this description into a Rzfile for direct access
and makes it available to DZDISP which can then display the
data words of a bank together with their meaning.
 
Opening, updating and using of the Rzfile can be done in the
same session. The file is opened via the menu item
'Open\_bank\_doc\_Rzfile', one has to choose if it should be
opened 'New' for 'Update' or 'Readonly'.
Information can put into the file using the item
'Put\_doc\_into\_Rzfile', for more information see help for this
command. This menu is invoked pressing the right mouse key on 
the item ZEBRA in the list of browsables.
 
An alternative way is to use the item 'Edit\_documentation' in
the popup menu associated with each bank (in the browser or
graphics window). In this case the editor is called with the
documentation contained in an (open) Rzfile if it is there.
If not a template with the documentation generated from the
selected bank in memory is presented in the editor.
If the file gets modified the documentation will be put into
the Rzfile if it is open for update. Note that the file
naming convention uses the hollerith Id of a bank
concatinated with the Id of its up-bank ('none' for a top
bank) with the extension '.dzdoc'.
 
\subsection{Interactive Ntuple filling}
 
A simple facility is provided to mark words in a data structure
read from an FZ-file for filling into a Ntuple. The entries may
be scalar variables (i.e. one value of a bank) or arrays. 
Marking of data words may be done explicitly giving the absolute
offset in a bank (a range in case of an array) or symbolically
if a (correct) documentation for the bank is available. In the
second case all values of a variable (max 100) in a bank are 
extracted and filled into the Ntuple. 
The marking is done from the popup menu associated to banks in
the graphics display, the system finds and remembers the links
to the bank.
Filling is controlled from the pop menu for the FZ-file icon.
For further details consult the Help items for these commands.
 
\begin{figure}[p]
  \begin{center}
     \mbox{\epsfig{figure=dzprsc.eps,height=21.5cm}}
     \vspace{-1.5cm}
    \caption{Example of screen layout for the Motif interface}
    \label{fig:DZDOCFIG6}
  \end{center}
\end{figure}
\clearpage 
\section{Examples}
\label{sec:dzdocexamples} 
 
\subsection[Example session with {\tt DZEDIT}]{Example session with \Rind{DZEDIT}}
 
The following example shows how to get the (graphical) bank
documentation for a subtree of banks. 
It assumes that the \RZfile{} previously created by \Rind{DZEDIT} 
is available and the command 
\Ucom{dzedit} (\Lit{DZEDIT EXEC} on IBMVM, 
\Lit{RUN DZEDIT} on VAXVMS or \Lit{dzedit} on Unix) is defined. 
 
Comment in the example below are put between /* .. */.
Help is available by typing \Ucom{help commandname}.
 
\begin{XMPt}{Example session with \Rind{DZEDIT}}
\Ucom{dzedit}                                         /* Invoke DZEDIT                         */
Workstation type (?=HELP) <CR>=0 :             /* Answer 0 if not a graphics terminal   */
\Ucom{OPEN dztest.rzbank}                             /* Open RZ-file                          */
\Ucom{DRAWONE EV NONE}                                /* Draw bank tree below CT (UpBank CD)   */
   Draw tree below: EV                         
   Total # of banks in tree:       3       
   # of documented banks           3       
 3 banks documented with  1 pictures on  2 pages
 PostScript output on: ev.ps                  
 
                                               /* The ZEBRA bank tree corresponding     */
                                               /* to the document is still in memory    */
                                               /* It is thus possible to visualize it   */
                                               /* on a workstation or graphics terminal */
\Ucom{DZDISP}                                         /* Display bank tree                     */
                                               /* All banks have ND=3 at this moment    */
\Ucom{DZDDIV}                                         /* Draw physical layout of Division      */
\Ucom{DZDIRZ}                                         /* Show structure of RZ-directory of     */
                                               /* documentation file.                   */
\Ucom{QUIT}                                           /* end of DZEDIT session                 */
\end{XMPt}
 
\subsection[Example of code calling {\tt DZDDIV}]{Example of code calling \Rind{DZDDIV}}
 
The following code fragment assumes that  ZEBRA is initialized and
some banks are booked (or read in via \Rind{FZIN}). 
Routines \Rind{DZDDIV} (or \Rind{DZDISP}) can be called without providing 
a bank descriptor file. 
For details on the graphics package HIGZ please consult the user guide \cite{bib-HIGZ}.
 
\begin{XMPt}{Using routine \Rind{DZDDIV}}
      INTEGER NWPAW,NWHIGZ
      PARAMETER (NWPAW=100000,NWHIGZ=30000)
      REAL PAW(NWPAW)
      COMMON/PAWC/PAW(NWPAW)
      CHARACTER*9 CHRZDD
      INTEGER LUNRZD,ISTAT,IXDIV,LDUMMY,IWMETA,IWKTYP
      INTEGER IERFIL,LUNMET,ILOCNR,IWKID
      PARAMETER (IERFIL=6, LUNMET=10,ILOCNR=1,IWKID=1)
 
C--   Open HIGZ and workstations
 
      CALL MZPAW(NWPAW,' ')
      CALL IGINIT(NWHIGZ)
 
C--   query workstation type
 
      CALL IGWKTY(IWKTYP)
      IF(IWKTYP .GT. 0)THEN
          CALL IGSSE(IERFIL,IWKTYP)
      ELSE
          WRITE(*,*)'Illegal workstation type'
          GOTO 99
      ENDIF
 
C--   open a plot file (if needed) otherwise put: IWMETA=0
 
      CALL KUOPEN(LUNMET,'myplot.ps','UNKNOWN',ISTAT)
      IF(ISTAT.NE.0)THEN
         WRITE(*,*)' Error opening plotfile',ISTAT
         IWMETA=0
      ELSE
 
C--      initialize a PostScript plotfile
 
         CALL IGMETA(LUNMET,-111)
         IWMETA=2
 
C--      deactivate output to plot file at start, it will be
C        activated by DZDDIV if requested
 
         CALL IGMETA(0,-111)
      ENDIF
 
C     Open the RZ file for bank descriptors.
C     If you do not have a descriptor file (yet) just put:
C     CHRZDD=' ' to notify it to DZDDIV
 
      CALL RZOPEN(LUNRZD,CHRZDD,'mydoc.rzdoc',' ',0,ISTAT)
 
*     specify directory name to be used or signal error
 
      IF(ISTAT.EQ.0)THEN
         CHRZDD = '//BANKDOC'
         CALL RZFILE(LUNRZD, CHRZDD(3:),' ')
      ELSE
         CHRZDD = ' '
         WRITE(*,*)' Error opening descriptor file',ISTAT
      ENDIF
 
C--   now display division layout
 
      CALL DZDDIV(IXDIV, LDUMMY, CHRZDD,'I',
     &            IWKID, IWMETA, ILOCNR, IWKTYP)
99    CONTINUE
      .....
\end{XMPt}
 
%%%% \begin{landscapebody}
 
%%%% \begin{minipage}[t]{.49\textwidth}
 
\finalnewpage

\subsection{Description of the banks used in previous examples}
\begin{XMPt}{Description of \Lit{EV} bank (see chapter \ref{sec:h1dzexamples})}
*B..EV  Event header bank.
*B.AU    S.Holmes
*B.VE    402
*B.ND    10
*B.NL    2
*B.NS    2
*B.NX    None
*B.UP    None
*B.OR    None
*B.IO    '3H 3I -F'
*B.LINK
*B.1   VX       Vertex bank
*B/LINK
*B.DATA
*B.1   LABNA     Name of laboratory
*B.2   EXPTNA    Name of experiment
*B.3   DAQNA     Initials of shift crew 
*B.4   IHDAT     Data type
*B.4   C1        Experiment data
*B.4   C3        Test beam data
*B.4   C4        Cosmic ray data
*B.4   C5        Monte-carlo data
*B.5   IIEVT     Trigger number
*B.6   IIFITY    Filter type (Bit string)
*B.6   BITVAL00  Sum E(clus) > 2 GeV
*B.6   BITVAL01  E(EB cls/blk) > 200MeV
*B.6   BITVAL04  Lumi Event
*B.7   IITHRU    Thrust * 10000
*B.8   IICTHR    Cosine of the thrust axis * 10000
*B.9   IIECAL    Total Electromagnetic Energy
*B.10  IIHCAL    Total Hadronic Energy
*B/DATA
*B/
\end{XMPt}
%%%% \end{minipage}
%%%% \hfill
%%%% \begin{minipage}[t]{.49\textwidth}
\begin{XMPt}{Description of \Lit{VX} bank (see chapter \ref{sec:h1dzexamples})}
*B..VX     VerteX bank (V0)
*B.AUTH    S.Holmes 
*B.VERS    403
*B.ND      12
*B.NL      1
*B.NS      1
*B.UP      EV
*B.OR      EV
*B.NX      VX
*B.IO      '3I -F'
*B.LINK
*B.1   TK       Track bank
*B/LINK
*B.DATA
*B.1   JVTYPE   Vertex type
*B.2   JVALGO   Algorithm
*B.3   JVNCTR   Number of tracks associated with this vertex
*B.4   JVCSRP   Cos(r.p) in the x-y plane
*B.5   JVD0     Abs(d0) w.r.t. event vertex            [cm]
*B.6   JVFLEN   Flight length w.r.t. event vertex      [cm]
*B.7   JVSD0    Sum of abs(d0) w.r.t. event vertex     [cm]
*B.8   JVCSVP   Cosine of opening angle at vertex point
*B.9   JVDXY    Separation in x-y at vertex point      [cm]
*B.10  JVDZ     Separation in z at vertex point        [cm]
*B.11  JVMPP    Invariant mass assuming pi+pi-        [GeV]
*B.12  JVMPPR   Invariant mass assuming pi+ p-bar     [GeV]
*B/DATA
*B/
\end{XMPt}
%%%% \end{minipage}
%%%% \newpage
%%%% \begin{minipage}[t]{.49\textwidth}
\begin{XMPt}{Description of \Lit{TK} bank (see chapter \ref{sec:h1dzexamples})}
*B..TK     Track bank
*B.AUTH    Watson
*B.VERS    3.13
*B.ND      12
*B.NL      0
*B.NS      0
*B.UP      VX
*B.OR      VX
*B.NX      TK
*B.IO      F
*B.DATA
*B.1      JCPX       best value for px at primary vertex    [GeV]
*B.2      JCPY       best value for py at primary vertex    [GeV]
*B.3      JCPZ       best value for pz at primary vertex    [GeV]
*B.4      JCQ        charge                                 [e]
*B.5      JCKAPP     Kappa=1/2rho                           [1/cm]
*B.6      JCPHI0     phi0                                   [rad]
*B.7      JCD0       d0 = phi x d.z                         [cm]
*B.8      JCTLAM     TAN(lambda) = COT(theta) track polar angle
*B.9      JCZ0       z at p.c.a of track                    [cm]
*B.10     JCCHIR     Chisq/DGF in r-phi
*B.11     JCCHI3     Chisq/DGF in s-z
*B.12     JCEM11     <kappa0 kappa0>                    [1/cm**2]
*B/DATA
*B/
\end{XMPt}
%%%% \end{minipage}
%%%% \hfill
%%%% \begin{minipage}[t]{.49\textwidth}

\finalnewpage

\subsection{Example using pointers, repetition counts etc.}
\begin{XMPt}{Description of RZ top bank (for output see page \pageref{xmp:rztop})}
*B..RZ  RZ system top bank
*B.AU   R.Brun
*B.ST   0
*B.DV   SYSTEM
*B.UP   NONE
*B.NX   RZ0
*B.ND   *
*B.NL   10
*B.NS   9
*B.IO   I
*B.LINK
*B.1    LSDIR      Pointer to first subdirectory
*B.2    LFREE      Pointer to list of free records
*B.3    LUSED      pointer to list of used records
*B.4    LFROM      Pointer to copied directory
*B.5    LPURG      Pointer to list of purged records
*B.6    LROUT      Pointer to output buffer
*B.7    LRIN       Pointer to input buffer
*B.8    LCORD      Pointer to ordered cycles (RZCOPY)
*B.9    LSNUSED    Free 
*B/LINK
*B.RLINK
*B.1    LRNUSED    Free reference link                     *
*B/RLINK
*B.DATA
*B.1     Z:IDNAME1     Directory name  (up to 16 characters)    
*B.2     Z:IDNAME2                 "                              
*B.3     Z:IDNAME3                 "                              
*B.4     Z:IDNAME4                 "                              
*B.5     RECPT1        Record number of the mother directory,   
*B.6     RECPT2        or C file pointer (words 5 and 6)        
*B.7     B:IWPW1       Write password (1st part)                
*B.8     B:IWPW2                      (2nd part)                
*B.9     NCHDRW        No. of char. DIR(1:5),WPW(6:10), 
*B.                    and  bit 12 eX mode   
*B.10    D:IDATEC      Creation date/time                       
*B.11    D:IDATEM      Last mod date/time                       
*B.12    NQUOTA        Maximum number of records QUOTA          
*B.13    N:NRUSED      Number of used records                   
*B.14    NWUSED        Number of words used MOD 1000000         
*B.15    NMEGA         Number of megawords used                 
*B.16    RESERVED      Reserved                        
*B.17    IRIN          Record number currently in LRIN          
*B.18    IROUT         Record number currently in LROUT         
*B.19    IRLOUT        Number of the last record written        
*B.20    IP1           Pointer to first free word in IRLOUT     
*B.21    ICONT         Record number continuation               
*B.22    NFREE         Number of words free in F                
*B.23    N:NSD         Number of subdirectories                 
*B.24    P:LD          Pointer to directory records             
*B.25    P:LB          Pointer to file descriptor (only for TOP)
*B.26    P:LS          Pointer to first subdirectory S          
*B.27    P:LK          Pointer to first KEY   K                 
*B.28    P:LF          Pointer to free space  F                 
*B.29    LC            Pointer to last cycle  C                 
*B.30    LE            Pointer to end of directory              
*B.31    N:NKEYS       Number of keys in that directory         
*B.32    N:NWKEY       Number of elements in one key 
*B.REP                 N:NWKEY + 9 / 10           
*B.1     B:KDES        KEYS descriptor (3 bits per el. ,
*B.                    10 keys per word) 
*B/REP    
*B.REP               N:NWKEY           
*B.1     Z:TAG1       First part of CHTAG(1) 4 characters      
*B.2     Z:TAG2       Second part                              
*B/REP    
*B.1     L:LD          Directory records structure
*B.1     N:NRD         Number of records to describe this dir.  
*B.REP               N:NRD           
*B.1     IREC(I)       Record number I of directory             
*B/REP
*B.1     L:LB          file descriptor (only for TOP)
*B.1     N:NWREC       Number of words for bitmap descriptor      
*B.2     LREC          Physical record length (in words)          
*B.3     D:IDATE       Creation date of the file                  
*B.REP               N:NWREC            
*B.1     B:BITMAP      1 bit per record on the file                       
*B/REP
*B.1     L:LS          Subdirectory  structure
*B.REP               N:NSD            
*B.1     Z:NAM1        Name of subdirectory                   
*B.2     Z:NAM2        "                                          
*B.3     Z:NAM3        "                                          
*B.4     Z:NAM4        "                                          
*B.5     NCHSD         Number of characters in subdirectory name  
*B.6     IRECSD        Record number of this subdirectory         
*B.7     D:IDTIME      Date and Time of creation of subdirectory  
*B/REP
*B.1     L:LK          KEYS structure
*B.REP               N:NKEYS        ! I=1,NKEYS   
*B.1     P:LCYC        Pointer to highest cycle in C for key I
*B.REP               N:NWKEYS       ! K=1,NWKEYS 
*B.1     KEYS(I,K)     Element K of key I                     
*B/REP
*B/REP
*B.1     L:LF          Start of free space
*B.REP   NFREE
*B.1     EMPTY         Free space
*B/REP
*B.1     L:LCYC        Cycles structure
*B.1     PTOCYCLE       
*B.1     P:BI0015      LCYC Pointer to prev cycle of KEY (0 if no)
*B.1     P:BI1631      SECREC Second record (if there)
*B.2     D:CREATD      Creation date/time relative to 1986
*B.2     BITVAL04      RZKEEP
*B.2     BITVAL03      Append mode
*B.2     BITS0002      Vector format (if RZVOUT)
*B.3     PTODATA       Pointer to the data
*B.3     BITS1631      Record number where data str. starts
*B.3     BITS0015      Offset in record
*B.4     CYWORD4       
*B.4     BITS0019      Number of words in data structure
*B.4     BITS2031      Cycle number
*B/
\end{XMPt} 
%%%% \end{minipage}
 
\subsection{Example where titles are contained in the bank itself}
\begin{XMPt}{Fragment of the bank description}
*B..EXTR HID@D(23:24) TITLE@D(25:30) NID@D(31:34) ND@D(35:39)
*B.ND    40
*B.NX    None
*B.UP    EXTR -1
*B.OR    EXTR -1
*B.IO    10I 12F -H
\end{XMPt} 
 
\begin{XMPt}{Data values of the actual bank}
    11  TRPX          4.940      Px
    12  TRPY          26.85      Py
    13  TRPZ         -36.53      Pz
    14  TRE           45.64      Energy
    15  TRM           1.784      Mass
    16  TRCH         -1.000      Charge
    17  TRSX          .5641E-01  Start x
    18  TRSY          .5541E-01  Start y
    19  TRSZ          .3131      Start z
    20  TREX          .8039E-01  End x
    21  TREY          .1857      End y
    22  TREZ          .1357      End z
    23  NAME              "[t^-
    24  NAME              "
    25  START             "Prim
    26  START             " Par
    27  START             "t ->
    28  END               " Dca
    29  END               "yed_
    30  END               "in_G
    31  P                 "p= 4
    32  P                 "5.60
    33  P                 "5 Ge
    34  P                 "V
    35  RANGE             "R:
    36  RANGE             " .1-
    37  RANGE             ">
    38  RANGE             ".2 c
    39  RANGE             "m
\end{XMPt} 
 
\finalnewpage

\begin{figure}[thb]
  \begin{center}
     \mbox{\epsfig{figure=dzmctree.eps,width=\textwidth}}
     \vspace{-1.5cm}
    \caption{Picture produced using the above indicated documentation
    and data values. Note: The escape characters defined in HIGZ to
    produce greek characters are recognized.}
    \label{fig:DZDOCFIG7}
  \end{center}
\end{figure}
 
%%%% \end{landscapebody} 
    
