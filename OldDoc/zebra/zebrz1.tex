%%%%%%%%%%%%%%%%%%%%%%%%%%%%%%%%%%%%%%%%%%%%%%%%%%%%%%%%%%%%%%%%%%%
%                                                                 %
%   ZEBRA RZ - Reference Manual -- LaTeX Source                   %
%                                                                 %
%   Overview                                                      %
%                                                                 %
%   Original: Michel Goossens (from SGML source)                  %
%   Additions: Jamie Shiers                                       %
%                                                                 %
%   Last Mod.: 30 Sep 1993 21:50 mg                               %
%                                                                 %
%%%%%%%%%%%%%%%%%%%%%%%%%%%%%%%%%%%%%%%%%%%%%%%%%%%%%%%%%%%%%%%%%%%

\Filename{H1rzover-direct-access-input-output}

\chapter{Direct access input-output}

\Filename{H2rzover-main-goals}
\section{Main goals}

\subsection{General}

The ZEBRA RZ package permits the storage and retrieval of 
ZEBRA data structures or Fortran vectors 
in random access files. Files may reside on standard
direct access devices such as magnetic disk, or be
mapped to virtual memory. 
RZ files can be accessed by several users simultaneously,
even across networks.
Remote file access and transfer is provided for RZ files
using standard tools, such as NFS and ftp. In the heterogeneous
environment, the tools provided in the CSPACK~\cite{bib-CSPACK} 
package may be used.

The RZ package is not a relational database management system,
but organises data in a hierarchical manner which is suitable
for many applications in High Energy Physics, and probably outside.

\subsection{Pathnames}

The basic unit of information addressed in an RZ file
is a ZEBRA data structure, in the simplest case a single ZEBRA bank.
We call this an RZ
{\bf data object}.
Each data object is referred to by a unique object name.
Object names are composed of a
{\bf pathname}, and one or more identifiers known as {\bf keys}.
\index{pathname}
\index{key}

The pathnames used by the RZ package were inspired by
\index{Unix}
the Unix file naming syntax and hence they typically 
carry mnemonic meanings and show the relationships
between different objects.
Unlike Unix, however, RZ pathnames are {\bf not} case sensitive, i.e.
upper and lower case are both treated as upper case.

As in the Unix file system, one may have directories and subdirectories
seperated by slash characters ``{\tt/}''.
An interrelated set of objects may be grouped together in a directory.

When an RZ file is opened, a user specified name is associated with it.
This name is known as the {\bf top directory} and is not
part of the file itself. This allows the user to have simultaneous
access to multiple files with the same RZ directory structure.

At the very highest level in the RZ tree is the root, referred
to by a double slash, ``{\tt //}''.

The directory above a given subdirectory is known as the
{\bf parent directory} and may be referred to by a backslash
character ``\bs'' .

Two other concepts are also provided, namely the {\bf current working directory},
or {\tt CWD} and the {\bf naming directory}. Objects are retrieved
and stored relative to the current working directory. The naming directory
is a mechanism for referring to a frequently used directory. 
It is initially set to the top directory, but may be reset at any time.
The naming directory may be referred to by the symbol ``{\tt\~{}}''.

The following Fortran program provides examples of the above
terms. For simplicity, the code to initialise the ZEBRA system
and open the RZ files (via the routine \Rind{RZOPEN}) has
been omitted.

\newpage
\begin{XMPt}{Example of RZ pathnames and terms}
*
*     Initialise RZ files on Fortran units LUN1, LUN2
*     with top directory names TOP1 and TOP2
*
      CALL RZFILE(LUN1,'TOP1',' ')
      CALL RZFILE(LUN2,'TOP2',' ')
*
*     Print the current naming directory
*     (It will have been set to TOP2 by RZFILE)
*
      CALL RZNDIR(' ','P')
*
*     Set the current working directory
*
      CALL RZCDIR('//TOP1/SUB1/SUB2/SUB3',' ')
*
*     Set the naming directory
*
      CALL RZNDIR('//TOP1/SUB1/SUB2/SUB3',' ')
*
*     Change directory relative to current working directory
*     (to parent directory in this case)
*
      CALL RZCDIR('\bs ',' ')
*
*     Change directory to naming directory
*
      CALL RZCDIR('\~{}',' ')
\end{XMPt}
\index{object}
\index{naming!tree}

\subsection{Keys and Cycles}
\index{key}
\index{cycle}

Data objects are identified beyond the pathname by {\bf keys},
which may be a single word of information
(integer, bit string or Hollerith)
or a vector of such words. The keys are not part of the pathname itself.

For example, in the case of HBOOK histograms a single integer
key, the histogram ID, may be used. Histograms relating to different
information could be stored in separate subdirectories and referred
to in a unique and clear manner by the associated pathname and
key, e.g. {\tt//HISTOS/CUT1}, keys (or IDs) 1--10.

Successive versions of objects with identical
pathname/key combination may exist simultaneously.
They are distinguished by a {\bf cycle number},
which is incremented automatically upon creation of successive data
objects. Cycles may be referred to explicitly,
the usual default is the highest cycle number.
The concept of cycles for successive versions of data objects with
identical names was taken from the VAX/VMS file system.
\index{VAX/VMS}

\newpage
\Filename{H2rzover-practical-examples}
\section{Practical examples of usage of the RZ package}
\subsection{HBOOK}
\index{HBOOK}

The RZ package is probably most widely used to store HBOOK 
histograms and ntuples, e.g. for subsequent analysis
with PAW. 
In such cases, shared write access is not normally
required. The file is typically created by a single user
or job and subsequently read a small number of times.

\begin{XMPt}{Example of storing HBOOK histograms in an RZ file}
PAW > \Ucom{ldir}

 ************** Directory ===> //LUN1 <===

                  Created 911030/1215  Modified 911030/1215

 ===> List of objects 
     HBOOK-ID  CYCLE   DATE/TIME   NDATA   OFFSET    REC1    REC2     
          1       1   911030/1215    103       1       3    
          2       1   911030/1215    104     104       3    
          3       1   911030/1215    107     208       3    
          4       1   911030/1215    106     315       3    
          5       1   911030/1215    106     421       3    
          6       1   911030/1215     56     527       3    

  Number of records =    2  Number of megawords =  0 +  1606 words
  Per cent of directory quota used =    .050
  Per cent of file used            =    .050
  Blocking factor                  =  28.418
 PAW >
\end{XMPt}

The above output from the PAW command LDIR shows the contents
of an RZ file which has no subdirectories and a few histograms.
The objects are accessed using the top directory name {\tt //LUN1}
and the histogram ID. 

One could of course have used a more complex directory structure,
but the number of directories and objects in such a file is typically
rather small.

\newpage
\subsection{CMZ}
\index{CMZ}

The CMZ code management system provides a good example of the use
of the cycle facility of the RZ package. In a CMZ file, code is
stored in the familiar two level structure of \PATCHY, namely
{\bf patches} and {\bf decks}. Each patch is a directory immediately
below the top level directory of the file. Each deck is a Fortran
vector in the directory corresponding to the appropriate
patch, as is shown in the following example.

\begin{XMPt}{Example of the directory structure of a CMZ file}
fatmen [0] \Ucom{ls}
 Current Working Directory = //FATMEN
 Following subdirectories :
  HISTORY           FATFLAGS          FATDOC            *FATCAT         
  *DSYIBM           *GSIIBM           *SHIFT            *CERNVM         
  *CERNVMB          *FRCPN11          *LEPICS           *APOL3          
  *FAT2SQL          *FATSQL           *FATUSER          *FATO2Z         
  *FATO2F           *FATNEW           *FATSRV           *FATSEND        
  *FMCDF            *FMKUIP           *FATLIB           *SQL            
  *FODEL            *FOGET            *FOPUT            FFATMEN         
  FATHEAD           FATCDES           FATBODY           FATUTIL         
  FMTMS             FATUSER           FATSRV            FMUTIL          
  FMINT             FATUOUS           FATASM            L3UTIL          
  SQLCOM            FMLOGI            FODEL             FOGET           
  FOPUT             FMZTOR            FATO2F            FMOTOZ          
  FATNEW            FMKUIP            FMCDF             FATSQL          
  FMORAC            FMH               FMC               FATSTAT         
  TAPELOAD          NAMES             REXX              FATTEST         
  UNREF             DCL               SCRIPT            FATULOK         
  FATCAT            EXAMPLE           SQLINT            JCL             
  FAT2SQL           SQL               FATSEND           FMVAX
Following DECKS :
 TITLE;22    TITLE;21    
 Number of DECKS =   1 Number of CYCLES =  2
 fatmen [1] \Ucom{cd fmtms}
 fatmen/fmtms [2] \Ucom{ls}
 Current Working Directory = //FATMEN/FMTMS
 00_PATCH;1  FMALLO;1    FMGTMS;1    FMLOCK;1    FMPOOL;2    FMPOOL;1    
 FMQTMS;1    FMSREQ;1    FMULOK;1    FMPREF;1    FMXVID;1    FMTAGS;1    
 FMPROT;1    FMUTMS;1    FMUALL;1    FMQVOL;2    FMQVOL;1    FMUVOL;1    
 FMEDIA;1    
 Number of DECKS =  17 Number of CYCLES =  19
 fatmen/fmtms [3]
\end{XMPt}

A listing of a given directory in 'ZEBRA' format shows that each deck
is identified by a single Hollerith key, namely the deckname.
RZ cycles are used to identify different versions of a deck. Each
time it is editted and changed, a new cycle is automatically 
created.

\finalnewpage

\begin{XMPt}{Example of the keys and cycles structure in a CMZ file}
 fatmen/fmtms [5] \Ucom{ldir}

 ************** Directory ===> //FATMEN/FMTMS <===

                  Created 910923/1423  Modified 911028/1628

 ===> List of objects 
     DECKNAME      CYCLE   DATE/TIME   NDATA OFFSET   REC1    REC2     

     00_PATCH         1   910923/1423     19      1    471    
     FMALLO           1   910923/1423   1145     20    471     472 ==> 480   
     FMGTMS           1   910923/1423    441     13    480     481 ==> 483   
     FMLOCK           1   910923/1423    455     70    483     484 ==> 487   
     FMPOOL           2   911021/1503    906     19    541    5669 ==> 5675   
     FMPOOL           1   910923/1423    905     13    487     488 ==> 494
 ...
\end{XMPt}

\subsection{FATMEN}
\index{FATMEN}

The FATMEN system uses the ZEBRA RZ package in a more complex manner.
In this case the RZ files are read by many jobs simultaneously,
often over the network. Much more complex object names are used,
with pathnames such as the following example from the DELPHI 
collaboration. 

\begin{XMPt}{Example of an RZ pathname in FATMEN}
FM> \Ucom{pwd}
Current Working Directory = //CERN/DELPHI/P01_ALLD/CDST/PHYS/Y90V03/E093.3/L0312
FM>
\end{XMPt}

A single RZ file that is used by FATMEN may well
contain in excess of one hundred thousand 
entries in several thousand directories.
In addition, these RZ files are constantly updated and must
retain consistancy to long running batch jobs.

These goals are met by ensuring that only a single process ever
has write access to a FATMEN RZ file. All updates are performed
by dedicated servers.

% Local Variables: 
% mode: latex
% TeX-master: "zebramain"
% End: 
