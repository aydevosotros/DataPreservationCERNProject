\Filename{H1-MZ-Internal-details}
\chapter{Internal technical details for ZEBRA MZ}
\label{sec:H1-MZ-Internal-technical-details}

\Filename{H2-MZ-Master-Tables}
\section{Master Tables}

\subsection*{Master parameters of the process in \FCdef{/MZCA/}}

\begin{verbatim}
NQSTOR       number of the latest store initialized
             (NQSTOR = 0, 1, ..., 15)    set to -1 by MZEBRA

NQOFFT(J+1)  table off-set KQT for store J, J=0,1,...,NQSTOR

NQOFFS(J+1)  store off-set KQS for store J,  = LQSTOR - LQASTO

NQALLO(J+1)  allocation status of store J:
               = 0:  normal, this store belongs to the running
                      process for exclusive use
                -1:  store disconnected
                -2:  read-only store attached by MZATTM
                -3:  store attached by MZATTM

NQIAM        process identifier, currently zero and unused

LQATAB       = LOCF(IQTABV(0))
             absolute adress-1 of the store control table
             for the primary store

LQASTO       = LOCF(LQ(0))
             absolute adress-1 of the general store base

LQBTIS       = LQATAB - LQASTO
             off-set for the primary table to the store base

LQWKTB       adr of the Zebra working area IQWKTB rel. to /ZEBQ/,
                    ie. LQ(LQWKTB) is word IQWKTB(1)
NQWKTB       size of IQWKTB available to MQ, NORMALLY = 2560
                    may be reduced temporarily by FQ
NQWKTT       total size of IQWKTB, = 2560
LQWKFZ       adr of the FZ work area in IQWKTB rel. to /ZEBQ/

MQKEYS(3)    check on MZEBRA initialization already done
NQINIT       initialization status

NQTSYS       size of each system table
             = LOCF(IQDN2(21)) - LOCF(IQTABV(1))

NQM99        unused
NQPERM       = zero: normal, error exit to ZFATAL
                = 1: permit error return
                     (used at the moment only in NZNEED for MZGAR1)

NQFATA       number of words to be printed from IQUEST on fatal
             error termination; zero during normal operation

NQCASE       fatal error case; zero during normal operation

NQTRAC       number of active trace words
MQTRAC(44)   ZEBRA internal trace words
\end{verbatim} 

\subsection*{Current store / division: in \FCdef{/MZCB/}}

When a particular store is selected by the user, and if it changes,
these numbers are collected into here for easy access.
Also, the parameters relevant for the division
whose free reserve space has been calculated last,
are kept in here.

\begin{verbatim}
JQSTOR       number of the current store
KQT          table off-set for the current store
KQS          store off-set for the current store

JQDIVI       number of the current division

JQDIVR       division whose free space has last been calculated
JQKIND       = JBYT (IQKIND(JQDIVR),21,4)
JQMODE       = JBIT (IQMODE(JQDIVR),1)
JQDIVN       neighbour division of division JQDIVR at the moving end
JQSHAR       = JQDIVN if the two divisions share the same reserve area
             = zero otherwise
JQSHR1       = forward division sharing the reserve area with JQDIVI
JQSHR2       = reverse division sharing the reserve area with JQDIVI
NQRESV       free space available to division JQDIVR

IQCUR(1->12) this vector receives a copy of IQTABV(1->12),
             see just below
\end{verbatim} 

\subsection*{\Lit{IQTABV}/\Lit{IQCUR}  vector}

\begin{verbatim}
     primary   current  store
     IQTABV    IQCUR

 (1) LQPSTO    LQSTOR    = LOCF(LQ(@0)),  absolute adr-1 of store
 (2) NQPFEN    NQFEND    number of fence words preceding the store

 (3) NQPSTR    NQSTRU    number of structural permanent links
 (4) NQPREF    NQREF     number of permanent links
 (5) NQPLK     NQLINK    number of permanent + working space links

 (6) NQPMIN    NQMINR    minimum size of the reserve area
 (7) LQP2E     LQ2END    minimum value of LQEND(2)

 (8) JQPDVL    JQDVLL    last low division
 (9) JQPDVS    JQDVSY    system division, also: first high division

(10) NQPLOG    NQLOGL    log level for this store

(11) NQPNAM(2) NQSNAM(2) printing name of the store

(13) IQTABV(13)  lowest  possible origin-adr for the store
(14) IQTABV(14)  highest possible origin-adr for the store
(15) IQTABV(15)  initial allocation to last division created
(16) IQTABV(16)  table-offset J of the table, if contained in store
                 ie. LQ(@J+1) is the first word of the table, unless
                 J = 0: the table is not contained in the store
\end{verbatim} 

\subsection*{Store control table}

This table exists for each store in the 'system table'
at the end of the space
provided for the store by the user in the call to \Rind{MZSTOR},
except for the primary store when it is held in \FCdef{/MZCC/}.

For a particular store \Lit{JQSTOR}, acces to a particular number,
say \Lit{IQTABV(J)}, is via
\begin{verbatim}
            KQT  = NQOFFT(JQSTOR+1)
            IVAL = IQTABV(KQT+J)
\end{verbatim} 
ie. by off-setting to the store control table for the primary store.
In the following we give simply the structure of this primary
store control table; exactly the same structure is repeated in every
store control table.

\begin{verbatim}
IQTABV(16)   store parameters, see the separate IQTABV vector
             description on the previous page

LQSYSS(  )   store system links, see the separate LQSYS vector
LQSYSR(  )   descriptions on the next page

LQSTA(21)    start adr of division J, J=1,2,...20
             LQSTA(21) is end-of-store+1
LQEND(20)    end+1 adr of division J
NQDMAX(20)   maximum size of division J

IQMODE(20)   mode of the division J
                bit 1 = 0 / 1  division forward / reverse

IQKIND(20)   kind of the division J
             JBYT(IQKIND(J),21,4)= 1  user short-range division
                                   2  user long-range division
                                   4  package division
                                   8  system division
             JBIT(IQKIND,J) = 1, other bits 1 to 20 are zero

IQRCU(20)    division bits: 'references to' are active, globals unfolded
IQRTO(20)    division bits: 'references to' are active
IQRNO(20)    division bits: 'references to' are permitted (mask)
NQDINI(20)   initial size of division J
NQDWIP(20)   number of times 'wipe' executed
NQDGAU(20)   number of times user garbage collection
NQDGAF(20)   number of times forced garbage collection for no space
NQDPSH(20)   number of times 'push' executed with relocation pass
NQDRED(20)   number of times 'reduced'
NQDSIZ(20)   maximum size of the division seen
IQDN1(20)    chars  1:4
IQND2(20)           5:8 of the printing name of the division

Primary store only:
KQFT         such that LQFSTA(1) is LQSTA(KQFT+1)  (=342 at present)
LQFSTA(21)   used in MZCOPY
\end{verbatim} 

\subsection*{\Lit{LQSYS} system link vectors: in \FCind{/MZCC/}}

The vectors LQSYSS and LQSYSR are the system link areas
attached to each store.
Their words are reserved as follows:


\subsection*{structural: \Lit{LQSYSS}}

\begin{verbatim}
   1       :  system bank holding the link area descriptions
   2 LQMST :  working link for MQ
   3 LQT   :  TZ package support link
   4 LQFS  :  FZ package support link
   5 LQFORM:  MZFORM format d/s support link
   7 LQRS  :  RZ package support link
  10 LQZEP :  ZEP support link
\end{verbatim} 

\subsection*{reference: \Lit{LQSYSR}}

\begin{verbatim}
   1 + 2 :  2 working links for MZ
   4 LQFF:  FZ, last control-bank retrieved
   5 LQFI:  FZ, control-bank for current input unit
   6 LQFX:  FZ, control-bank for current output unit
   7 LQRRF:  RZ package working link
\end{verbatim} 

\subsection*{Table of Link Areas}

This table, one for each store,
is held in a bank in the system division of the store;
its address is  \Lit{L = LQSYSS(KQT+1)}.

The table contains one entry of 5 words for each link area
as follows:

\begin{verbatim}
   word 1:  LLA   for  LQ(@LLA)    being the first word,
        2:  LLAE  for  LQ(@LLAE-1) being the  last word of the area

        3:  bits 1-15:  number of structural links
             bit  31      = 0  if permanent link area
                          = 1  if temporary link area

        4:  characters 1:4
        5:             5:8  of the printing name of the area
\end{verbatim} 

The first entry of the table describes the permanent links
plus the working space links of the store;
this area may be of length zero.
The second entry describes the system links contained in
the store control table for this store.
Further entries appear in the order in which they have been
created by either \Rind{MZLINK} or \Rind{MZLINT}.

The table as a whole in the bank at \Lit{L} is as follows:

\begin{verbatim}
         IQ(L+1)   NWU  active size of the table of n entries,
                        NWU = 5*n + 1
   entry 1
             +2)   LLA = 1
             +3)   LLE = 1 + NQLINK
             +4)   number of permanent structural links
             +5)   4HQWSP
             +6)   blank
   entry 2
             +7)   LLA = LOCF(LQSYSS(KQT+1)) - LQSTOR
             +8)   LLE = LLA + number of system links
             +9)   number of structural system links
            +10)   4HQLAS
            +11)   4HYST
   entry 3
            +12)   LLA = LOCF(first word) - LQSTOR
             ...   ...
   entry i

      IQ(L+5i-3)   LLA                    for entry i
             ...   ...
\end{verbatim} 

\Filename{H2-MZ-relocation-tables}
\section{Relocation Tables}

\subsection*{Memory Occupation Table}

The table has one entry of 8 words for each memory region
taking part in global memory operations.
Its bounds are given by \Lit{LQMTA} and \Lit{LQMTE}:

\begin{verbatim}
      LQMTA  start adr of the first entry
      LQMTE  end+1 adr of the last  entry
\end{verbatim} 

A particular entry has this format:

\begin{verbatim}
   LQ(L+0)   division number
             if zero: pseudo division

   LQ(L+1)   activity flag:

                -1  division is empty
                 0  inactive
                 1  relocation only
                 2  active, eg. global shift, or MZPUSH, or MZREPL
                 3  garbage collection
                 4  wipe

   LQ(L+2)   NWSH, global shift of this division by NWSH words

               -ve  shift to low
                 0  no shift
               +ve  shift to high

   LQ(L+3)   start adr of the first bank
   LQ(L+4)   end+1 adr of the last  bank

   LQ(L+5)   rel. adr of the first associated relocation entry
             ie. LF = LQRTA + LQ(L+5), LQ(LF) is the first entry
             in the link relocation table for this memory region

   LQ(L+6)   rel. adr of the last+1 entry, (only if LQ(L+1)=3)
             = -3 if division with garbage collection reset because
                  of 'table full'

   LQ(L+7)   NFREE, number of words collected or wiped
\end{verbatim} 

\subsection*{Link Relocation Table}

This table contains the prescription of how any link is to be
updated.

The table covers the 'total relocation interval'
\Lit{(LFIXLO,LFIXHI)},
links pointing outside this interval are not changed.

The area covered by the relocation interval is considered as a
series of alternating 'live' and 'dead' regions,
described by the link relocation table.
One entry in this table specifies a particular live region
\textbf{and}
the dead region just behind it.
Each region represents an integral number of live or dead banks,
or alternatively a region of non-occupied store (reserve area).
For the live region the table entry specifies the relocation
constant, for the dead region it specifies whether a structural link
pointing into this region should be bridged or not.

The table is constructed in the largest gap of the ZEBRA stores,
with the following structure:

\begin{verbatim}
  it starts at     LQ(LQTA-1)  holding LFIXLO
  continues        LQ(LQTA)    first entry
                   LQ(LQTA+4)  second entry
           . . .
                   LQ(LQTE-4)  last entry
  and ends at      LQ(LQTE)    holding LFIXHI.

\end{verbatim} 
The format for entry i at  \Lit{L = LQTA + 4*(i-1)} is:

\begin{verbatim}
  LQ(L+0)    LAi:    start adr of the live area
      +1)    LDi:    start adr of the adjacent dead area
      +2)    NRELi:  relocation constant for the live area
      +3)    IFLi:   bridging flag for the dead area
                      zero: no, 1: yes, -1: link to remain unchanged
\end{verbatim} 

The store from \Lit{LQ(@LAi)} to \Lit{LQ(@LDi-1)} is occupied
by live banks.
The store from \Lit{LQ(@LDi)} to \Lit{LQ(@LAj)-1} with \Lit{j=i+1}
is a dead region,
containing dead banks to be bridged only if  \Lit{IFLi=1}.

The store from \Lit{LQ(@LFIXLO)} to \Lit{LQ(@LA1)}
is a non-bridging dead region.

It is in the nature of this table that the link relocation table
represents at the same time the detailed instructions for
the memory move, if any.

\subsection*{relocation parameters in \FCdef{/MZCT/}}

The parameters in \FCind{/MZCT/} control the relocation process.

\begin{verbatim}
   IQFLIO      flag I/O, init to zero by MZTABM
               flags special treatment for IO relocation
               non-zero: relocation called from FZIN (also MZCOPY)
               triggers actions in MZRELB:
               1) links pointing outside <LQ(LQTA),LQ(LQTE> are
                  simply reset to zero
               2) do not go to ZFATAL for bank-chaining clobbered
                  but return IQFLIO = -7 to signal bad input structure

   IQGAP(5,4)  returns the parameters of the largest gaps found by MZFGAP
               Entry 'J' in the table contains:
                 IQGAP(1,J) = number of words available
                 IQGAP(2,J) = abs. zebra adr of first word in the gap
                 IQGAP(3,J) = division number
                 IQGAP(4,J) = store number
                 IQGAP(5,J)   user flag
               Gaps J=1 and 2 are outside the moving region,
                      3 and 4 could be inside the moving region

   IQPART      partial garbage collection, init to zero by MZTABM
               handle not enough space for the relocation tables
               if = zero: no problem
               set to =7  by MZTABH if not enough table space
               set to =-7 by MZTABH if table moved into the forbidden gap
               set to =1  by MZTABC if not enough space
               set to =1  by MZTABR if not enough space

   IQTBIT      status-bit for table building, init to IQDROP by MZTABM
   IQTVAL      value of status-bit,           init to   zero by MZTABM
               construct table for banks having IQTVAL in bit IQTBIT

   IQTNMV      init to zero by MZTABM if JQSTMV < 0 on entry
                set to zero by MZGAR1, MZDIV
                set to   -7 by MZCOPY
               used by MZTABH on first entry:
               = 0: alright to move Mem.occ.table to forbidden gap
                     return IQPART = 7 if no gap at all
               < 0: alright to move Mem.occ.table to forbidden gap
                     return IQPART = -7 if so done
               > 0: Mem.occ.table may not go into a forbidden gap
                     return IQPART = 7 if no allowed gap

   JQGAPM      gap for Memory occupation table, init to zero by MZTABM
               if non-zero: the Memory occupation table has been moved
               to gap JQGAPM described by IQGAP(1/5,JQGAPM)

   JQGAPR      gap for the Relocation table, init to zero by MZTABM
               if non-zero: the Link Relocation table has been moved
               to gap JQGAPR

   JQSTMV      moving store, -1 if none, controlling MZFGAP
   JQDVM1      first moving division
   JQDVM2      last  moving division
   NQDVMV      move divisions JQDVM1/2 by that many words, -ve: left
               JQDVM1, JQDVM2, NQDVMV init to zero
               by MZTABM if JQSTMV < 0 on entry

      Original meaning of JQSTMV,JQDVM1,JQDVM2,NQDVMV (cf. MZTABS):
      unless JQSTMV<0, the divisions JQDVM1 to JQDVM2 inclusive
      of the store JQSTMV may have to be shifted by NQDVMV words,
      to the left if -ve, to the right if +ve.
      NQDVMV=0 means to the left by a yet unknown amount.
      It follows that the gaps after divisions JDIV are not available
      for the relocation table because they may be over-written
      by the memory move, with:
         NQDVMV  0, -ve:  JDIV = <JQDVM1-1,JQDVM2-1>
                    +ve:  JDIV = <JQDVM1,  JQDVM2  >

      MZFGAP is also used for I/O to find a gap where to put the
      relocation table on input. In this case there is just one
      gap unavailable, namely the one which will receive the data.
      This is the gap before a reverse division (JQMODE=1) or after
      a forward division (JQMODE=0). To block this gap one can give
      JQDVM1=JQDVM2=JDIV and NQDVMV=1-2*JQMODE, setting JQSTMV also.

   LQMTE       end+1 adr of the Memory occupation table,
                     init to LQWKTB+NQWKTB-1 by MZTABM
   LQMTA       start adr of the Memory occupation table,
                     init to LQMTE-160 by MZTABM
   LQMTB       curr. adr of the Memory occupation table,
                     init to LQMTA by MZTABM

   LQMTLU      adr in the Mem.occ.table of the last division used
               init by MZTABM to point to division 20
               reset by MZTABX

   LQMTBR      init to zero by MZTABM
               set by MZTABR to point to the current entry into
               the Memory occupation table when it runs out of
               table space.
               The code handling this is a remenant from an
               earlier approach to handle "table space full"
               and has been left in MZTABR for safety.
               In fact, now MZGAR1 recalls MZTABM after the first
               partial garbage collection.
               (The code handling LQ(LQMTB+6) = -3 is also part
               of this old approach)

   LQRTA       start adr of the memory available to the Relocation table
               init to LQWKTB by MZTABM
   LQRTE       end+1 adr of the memory available to the Relocation table
               init to LQMTA-10 by MZTABM
               both are updated by MZTABH if the tables are moved

   LQTA        start adr of the Link Relocation table,
               init to LQRTA+1 by MZTABM
   LQTE        end+1 adr of the Link relocation table,
               init to LQRTE    by MZTABM

               parameters driving MZTABC, set by MZTABR
   LQMTC1      start adr of first bank
   LQMTC2      end+1 adr of  last bank
   LQTC1       first table word available, init to LQRTA+1 by MZTABM
   LQTC2       last  table word available, init to LQRTE   by MZTABM

   MQDVGA      flag word indicating the divisions with garbage collection
               bit j, j=1,2,...,20 for garbage collection in division j

   MQDVWI      flag word indicating the divisions to be wiped
               bit j, j=1,2,...,20 for wiping division j

   MQDVAC      flag word indicating the active divisions
               constructed from MQDVGA + MQDVWI by MZTABM
               and updated by MZTABS and MZTABR
               seems to be used only for printing

   NQDVMV      move divisions JQDVM1/2 by that many words, -ve: left
               init to zero by MZTABM if JQSTMV < 0 on entry

   NQFREE      number of words to become free, init to zero by MZTABM
   NQFRTC      number of words found free by MZTABC (per call)

   NQGAPN      number of normal gaps available in IQGAP
   NQGAP       number of all gaps, including forbidden, in IQGAP

   NQLIVE      number of live banks found by MZTABC (per call)

   NQNOOP      no-operation flag constructed by MZTABF
               = zero normally
               = -7 really no operation
               = +7 only a memory shift with no garbage collection
                    and no links pointing into the shifted region
                    for example: left-shift empty division 2
               usage in MZMOVE: if non-zero update the division tables
                                 without execution of MOVE
\end{verbatim} 

\Filename{H2-MZFORM-data-structure}
\section{MZFORM data structure}

\begin{verbatim}

  IQ(LID+1)  n = number of ID's stored
         2   ji  of the last characteristic retrieved
         3   IDH of the last characteristic retrieved
         4   IDH  #1
             ...
       n+3   IDH  #n


  IQ(LIX+1)  ji in QIOD for IDH #1
         2   ji         for IDH #2
             ...


 IQ(LIOD+1)  n = number of words occupied;          j1 = 1
      j1+1   start of 1st characteristic, n1 words, j2 = j1 + n1
      j2+1            2nd                 n2 words, j3 = j2 + n2
             ...

    characteristic i is stored at ji in QIOD as NW+1 words:

         word 1:   16 bits: control-byte |
                    | 5 bits: NW | 5 bits: NW+1 | 6 bits: 1
         word 2:   first extra I/O word (if any)
                    ...
      word NW+1:   last  extra I/O word


                                 LID

                                  |
                                 / \
                ______________         ______________
               |              |       |              |
LQFORM ---->   |     QID      |       |     QID      |  list of
               | for +ve ID's |       | for -ve ID's |  identifiers
            -- |______________|       |______________|
           |
           |      |                      |
           | -2   | -1                   | -1
           |    ______________         ______________
           |   |              |       |              |
           |   |     QIOX     |       |     QIOX     |  adr in QIOD
           |   |     +ve      |       |     -ve      |  corresponding
           |   |______________|       |______________|
           |
           |                     \ /
           |                      |
           |
           |                     LIX
          __________
         |          |
LIOD ->  |   QIOD   |   I/O characteristics
         |__________|   stored
\end{verbatim} 

% Local Variables: 
% mode: latex
% TeX-master: "zebramain"
% End: 
