\Filename{H1-FZPrinciples}
\chapter{Principles}

\Rind{FZOUT} transfers a \textbf{data-structure}
from the dynamic store to a sequential file, i.e. to disk or tape,
or to flat memory, or via a user routine to the outside world.
\Rind{FZIN} does the inverse transfer.
In high-energy physics most FZ files contain data-structures
representing events; but the concepts are completely general.

Two kinds of user information can be associated to a
data-structure on the file.
The ``user header vector'' may contain identifying information,
essentially numeric, to allow rapid selection of data-structures.
The ``text vector'' may contain zero,
one, or several lines of Hollerith text,
giving context information about the data-structure in one of the
natural languages
(the implementation of user access to text vectors is pending).

On the file, a data-structure is represented by a straight
copy of the relevant sections of the dynamic store,
preceded by its ``pilot information'' which describes
the data-structure to follow.
It predicts the memory requirements of the d/s,
and it carries the relocation table necessary to recalculate all
links in the data-structure for the new position in memory.
It further contains the associated user header vector
and the text vector.
Files to be read back on the computer of creation can be written in
``\textbf{native mode}''.
In this mode, writing and reading is done with standard Fortran
unformatted WRITE and READ statements,
the representation of numbers and Hollerith is defined by the
internal representation of the machine,
blocking of logical records into physical records is defined by the
Fortran I/O system of the machine.
The data are written directly from the dynamic store to the file
without going through a Zebra buffer;
a data-structure is hence represented in general by several
logical records.

Files to be transported from one computer to another one of a
different make must be written in
``\textbf{exchange mode}''.
Exchange mode implies two separate features:
\begin{itemize}
\item The Zebra Exchange \textbf{File Format}
      provides for reading and writing machine independent files,
      it supports logical records blocked
      onto a string of fixed-length physical records
      without any system control-words.
      The data flow through a Zebra buffer,
      and each data-structure written by \Rind{FZOUT} 
      is one single logical record.
      For transport of the data via a network which does not allow
      transmission of binary files, 
      the ``ALFA exchange format'' is available:
      this maps each physical record onto a string of 
      80 column card-images
      containing only alpha-numeric characters (and a few others).
      Such a file can be sent as ordinary text over any network;
      it is even possible to include some test events onto the
      PAM file of some program.
\item The Zebra Exchange \textbf{Data Format}
      provides a machine independent interface for the representation
      of the data.
      On each machine Zebra is capable to convert between the
      machine internal and the exchange representation;
      this relies on the I/O characteristic carried by each bank describing
      the nature of its contents (integer, floating, etc.).
      On most modern 32-bit machines the ``native'' data format is identical
      to the exchange representation;
      thus on these machines no conversion is needed.
\item It is possible to combine the exchange file format
      with the native data format;
      this permits to parcel the data into fixed length records,
      without also translating to or from the exchange data
      representation.
\end{itemize}

If the file medium is \textbf{Disk} or \textbf{Tape}
the records representing a data-structure are transfered
between the Zebra store and the medium by WRITE or READ statements
(or equivalent).
If the ``file'' medium is \textbf{Memory}
the records are tranfered by copying to or from a region
of the user's memory
(possibly involving packing or byte inversion operations).
If the file medium is \textbf{Channel}
the records are handed to, or obtained from,
a user routine one at a time.
This routine is supposed to be an interface to a computer link,
permitting to shuffle the records of the data-structure
from one machine to another one.
For the media Memory and Channel \Rind{FZIN}/\Rind{FZOUT} always operate
with the Exchange File Format,
ie. with a string of fixed length records,
but the Data Format may be either ``exchange'' or ``native''.

A disk file with Exchange File Format can be read with
\textbf{Direct Access}
permitting a random access to the data, since it is a
simple string of fixed-length records.
For each data-structure written (or read) \Rind{FZOUT} or \Rind{FZIN} deliver
its D/A address which the user can compile into a Direct Access Table
together with other relevant information about each d/s.
If the table is put into a bank it can be stored
at the end of the file itself by calling \Rind{FZODAT},
to be recovered by \Rind{FZIDAT} when one comes back to read the file.
Handing a D/A address to \Rind{FZINXT} will reset the current read point
of the file, and the next call to \Rind{FZIN} will deliver the wanted d/s.

The Fortran implementations on some machines running Unix
cannot handle sequential access of fixed-length records,
they require system control words with each record.
One can get around this by using the Direct Access mechanisme:
Zebra operates sequentially, also for a file which has been
initialized for direct access, until the user tells it otherwise
by calling \Rind{FZINXT}.
However, this is not a satisfactory solution for handling tapes.
Therefore, yet another mode, reading and writing through calls
to the
\textbf{C Library}, is programmed into Zebra, both for sequential
and for direct access, but this only on some machines running
the Unix operating system.

Direct access with FZ is only for input;
creation of files is a strictly sequential process.
It serves well in a particular situation which is logically simple,
but not at all for managing a data-base with key-words
and up-dates.
For this one should use the RZ package of Zebra.

It was one of the design aims of the FZ package to provide
a representation of the data on external media in exchange mode
which fits both off-line data-processing and on-line data-taking
requirements.
Speed is important in both cases, but it is more critical in the
on-line context.
For this the \ZEBRA{} Exchange File Format has been defined to allow
dumping large areas of memory to a file,
without the need  to insert control information  for physical
records during the dumping process;
all control information needed is grouped at the very beginning
of the data.

The program running on a data-taking on-line computer is
likely not to be a \ZEBRA{} program,
in which case one will have to write an ad hoc output routine to
produce a file readable by \Rind{FZIN} in exchange mode.
This point has been kept in mind when designing the exchange format,
to make sure that it is simple enough
to produce files in this format.

Although the unit of information for \ZEBRA{} is a data-structure
in full generality,
in the on-line application the
``data-structure'' will most likely consist of just one maybe
very large bank,
or at most of a few banks.
For such simple data-structures the interconnection
by links does not need to exist,
and the on-line program can disregard this aspect of
data-structures.
When a ``link-less'' data-structure of several banks is read
all its banks will be connected by \Rind{FZIN} into
a linear structure, to permit the reading program to access
the banks.
