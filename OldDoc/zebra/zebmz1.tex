% 13 November 1993 mg
\Filename{H1-MZ-memory-management}
\chapter{Memory management}
\label{sec:H1-MZ-memory-management}

The memory management package MZ of ZEBRA is fundamental
to all other ZEBRA packages.

The ZEBRA package MZ derives from its predecessors:

\begin{itemize}
\item the ZEBRA data structure derives from the HYDRA data structure~\cite{bib-HYDRAMZ};
\item the multi-store strategy derives from ZBOOK~\cite{bib-ZBOOK}
\end{itemize}

\Filename{H2-MZ-ZEBRA-bank-format}
\section{ZEBRA bank format}

\begin{verbatim}
    LQ(L-NL-NIO-1) IOcb/NOFF    IOcb: I/O control byte (16 bits)
                                NOFF = NIO + NL + 12 (16 bits)

    LQ(L-NL-NIO)   IOW (1)      first extra I/O descriptor word (if any)
                   ...
    LQ(L-NL-1)     IOW (NIO)    last  extra I/O descriptor word

    LQ(L-NL)       link NL      last reference link
                   ...
    LQ(L-NS-1)     link NS+1    first reference link
    LQ(L-NS)       link NS      last structural link
                   ...
    LQ(L-1)        link 1       first structural link

    LQ(L)          next         adr of the next bank in the linear str.
    LQ(L+1)        up           adr of the supporting bank
    LQ(L+2)        origin       adr of the supporting link

    IQ(L-5)        IDN          numeric   bank ID
    IQ(L-4)        IDH          Hollerith bank ID  (4 characters)
    IQ(L-3)        NL           number of links
    IQ(L-2)        NS           number of structural links
    IQ(L-1)        ND           number of data words

    IQ(L)          status       bits  1-18 user
                                     19-32 system   25: drop bit
                                     19-22 NIO      26: mark bit
    IQ(L+1)        first data word
                   ...
    IQ(L+ND)       last  data word

            This layout requires:

                  DIMENSION    LQ(999), IQ(999), Q(999)
                  EQUIVALENCE (LQ(9),IQ(1),Q(1))


 ---- Format of a short dead region ----

    word 1:  bits  1-16: NW, # of words, with  0 < NW < 12
                   17-24: NW again for redundancy
                      25: drop-bit set
                  26-end: zero
     2->NW:  dead words, content irrelevant
\end{verbatim} 

\Filename{H2-MZEBRA}
\section{MZEBRA - initialize the ZEBRA system}

To initialize the ZEBRA system Common blocks the user must call
\Rind{MZEBRA}, before any other request to ZEBRA.

In particular, the following COMMON variables of interest to
the user are initialized:

\begin{verbatim}
logical unit numbers:
      COMMON /ZUNIT/ IQREAD,IQPRNT,IQPR2,IQLOG,IQPNCH,IQTTIN,IQTYPE

default logging level:
      NQLOGD in /ZSTATE/

where
      IQREAD   lun for standard system input ('card reader')

      IQPRNT   lun for standard  user print output
      IQPR2    lun for secondary user print output
               initialization: IQPRNT=IQPR2=IQLOG

      IQLOG    lun for standard system output ('line printer')
               this is used for all ZEBRA system printing

      IQPNCH   standard system coded output to be read back by
               program ('card punch')

      IQTTIN   standard on-line terminal input  (zero if none)
      IQTYPE   standard on-line terminal output (zero if none)

      NQLOGD   system-wide default logging level, see next para.
               standard default initialization to zero
\end{verbatim} 

On any particular machine ZEBRA knows the correct values for
the logical unit numbers, for example it knows that 6 and 7
are the right values for \Lit{IQREAD} and \Lit{IQLOG} on the IBM,
similarly  \Lit{L"INPUT"} and \Lit{L"OUTPUT"}  on the CDC Cybers.
It is of advantage to the user to direct all his print output
through the unit numbers provided by ZEBRA in \FCind{/ZUNIT/}.
After the call to \Rind{MZEBRA} he is free to change any particular
value in \FCind{/ZUNIT/}, and with the call to \Rind{MZEBRA} he may also
modify the initialization, in particular he may re-direct
the standard print output to the on-line terminal.

The parameter in the call to \Rind{MZEBRA} may select initialization
options either with an integer constant or with a list:

\Shubr{MZEBRA}{(LIST)}

\begin{verbatim}
 --- short form of options:

   with  LIST = 0:  standard defaults
               -1:  preset IQLOG = IQTYPE, ie. output to terminal
               -2:  preset NQLOGD= -2,     ie. only error messages
               -3:  preset NQLOGD= -2 and IQLOG=IQTYPE

\end{verbatim}

\begin{verbatim}
 --- options specified by a list:

   with  LIST(1)   N elements in the list to follow (excluding itself)
                   if zero or -ve: standard options

             (2)   -> NQLOGD, the system-wide default log level
                              (see MZLOGL on the next page)

                   for example: LIST(2)= -3: suppress all messages
                   standard default: NQLOGD = 0

             (3)   -> IQLOG,  lun for standard log printing
                              unless absent or zero
                              if = -1: set IQLOG = IQTYPE

             (4)   -> IQPRNT, lun for standard output printing
                              if absent/zero: set IQPRNT = IQLOG
                              if = -1:        set IQPRNT = IQTYPE
\end{verbatim} 

\Shubr{MZVERS}{}

 prints at logging level -1 or above an initialisation message
on unit \Rarg{IQLOG}, showing amongst other things,
the version number of the current ZEBRA system.
In \Lit{\IQUEST(1)} it returns as an integer
the Zebra MZ version number multiplied by 10000;
thus version 3.66 /7 would give 36607.

\Examples

\begin{verbatim}
   set normal logging, set printer output to terminal:

      CALL MZEBRA (-1)
      CALL MZVERS

   set monitor logging, set unit 4 as print file:

      DIMENSION    LIST(3)
      DATA  LIST   / 2,  2, 4 /
      CALL MZEBRA (LIST)
      CALL MZVERS
\end{verbatim} 

\Rind{MZEBRA} only initializes the general ZEBRA system commons,
it does not initialize the dynamic store.
Before any request to the ZEBRA system involving the store,
the user must initialize it by calling \Rind{MZSTOR}.

\Shubr{MZEND}{}

Should be called at the end of execution by the user
to obtain the statistics of usage of the dynamic store.

\Filename{H2-MZLOGL}
\section{MZLOGL - change the MZ logging level}

Various parts of the ZEBRA system write log messages
to the standard system output,
and occasionally also to the on-line terminal, if any.
Examples of messages provided for are:

\begin{verbatim}
   a) messages for recoverable errors:
      read errors, data errors

   b) intialization messages for:
      stores, divisions, link areas, files

   c) termination messages giving statistics of usage of
      various facilities like memory, files

   d) operation messages:
      change in program phase, end-of-file

   e) watch messages for hopefully rare, expensive events:
      garbage collection, MZPUSH with relocation

   f) monitor messages to help the user debug his program
\end{verbatim} 

To control the amount of information thus provided to the user,
a log level is defined and can be set and reset by the user
at execution time.
The default log level zero enables the messages which one would
usually like to see for record in a production run.
The user may reduce the log level to cut out most or all messages;
he may increase the level to watch the running of his program,
or even to debug his data or his input files.

Separate ZEBRA entities, such as dynamic stores or files,
each have their own attached log level,
which may be changed by the user at any time.
By default they inherit the global system-wide default log-level
set by \Rind{MZEBRA}, whose own default is zero.

A somewhat similar system has been used
for the debugging of the ZEBRA system itself;
the corresponding WRITE statements are still present in the code
on the PAM files, although not on the object libraries,
and could be activated after re-compilation
by setting a special log level.

(The code for generating the logging messages is conditional and can
be de-selected at generation time of the ZEBRA binary library.
This is controlled by \PATCHY{} conditionals:
\begin{verbatim}
   +USE, QPRINT, T=INH.  deselects all messages
   +USE, QDEBUG, T=INH.  deselects all messages at or above level 2
   +USE, QDEVZE.         selects the messages for debugging Zebra. )
\end{verbatim} 

The log level attached to a particular dynamic store is initialized
by \Rind{MZSTOR}, normally to the global default log level.
The user may change and re-change it at any time with:

\Shubr{MZLOGL}{(IXSTOR,LOGL)}

with
\begin{verbatim}
      IXSTOR  index of the store,
              zero for the primary store

        LOGL  the desired log level, as shown in the following
              table, which also shows which MZ routines print
              at this (or higher level):

     level  -3:  no log messages at all

            -2:  error messages        ZFATAL, ZPHASE

            -1:  terse logging         MZEBRA, MZSTOR, ZPHASE

             0:  normal logging        MZDIV, MZLINK

            +1:  log to watch          MZLINT, MZGARB, MZPUSH

            +2:  log to monitor calls to ZEBRA
                  MZLINT, MZWORK, MZBOOK, MZLIFT, MZDROP,
                  MZPUSH, MZREPL, MZGARB, MZLOGL

   ( Messages to debug the ZEBRA system itself:

      giving  LOGL = 100+n  sets the log-level to MIN(n,2)
                            and the debug print level to "n"

                   this is not normally available !          )
\end{verbatim} 

\Filename{H2-MZSTOR}
\section{MZSTOR - initialize a dynamic store}

A call to \Rind{MZSTOR} is required to initialize the dynamic store before any
operation using this store.

ZEBRA can handle up to 16 different dynamic stores. Each such store
must reside in a Common block, not in a local vector.
Each store must be intialized by calling \Rind{MZSTOR} once,
and once only for this store.
The first store initialized is the primary store,
its store-index \Rarg{IXSTOR} will be zero.
Further secondary stores may be initialized,
their \Rarg{IXSTOR} will be allocated non-zero values by \Rind{MZSTOR}.

In the call to \Rind{MZSTOR} the user specifies the first and the last
word of his dynamic store, he indicates the number and kinds
of permanent links contained at the beginning of the store,
and he communicates the Fortran name of the common block he
is using for this store for printing purposes.
A "fence" region of at least one word must be reserved preceding
the store to allow catching errors due to using L=0.

\Rind{MZSTOR} initializes the store with 3 divisions:
the forward "working space" division~1
followed by the reverse division 2, which is the default division
in many instances, and the reverse "system" division
at the end of the store.

For each secondary store, the system needs an area of
about 400 words to hold the system tables for this store.
This area is allocated by \Rind{MZSTOR} on the last words
of the dynamic store itself.

The use of several dynamic stores introduces an execution time
overhead proportional to the number of times that ZEBRA has to
operate in a store other than the "current" one.
All ZEBRA routines check on the current store being the right one;
if not, a call to \Rind{MZSDIV} changes the "environment".
A normal application uses only one store, the primary store.
Apart from allowing to point from any data to any other,
this also saves having to carry the store index,
which is simply zero.

A given dynamic store is initialized by

\Shubr{MZSTOR}{(IXSTOR*, chNAME, chOPT, FENCE, LQ(1), LQ(LR), LQ(LW), LQ(LIM2), LQ(LAST))}

with
\begin{verbatim}
IXSTOR* returns the index of the store, to be used when specifying this store in
subsequent calls to the ZEBRA system.  The indices for the divisions 1 + 2
created by MZSTOR are: for division 1: IXDIV = IXSTOR + 1 2: IXDIV = IXSTOR + 2

              IXSTOR will be set to zero for the primary store.

      chNAME  name of the store for printing purposes,
              8 characters maximum

      chOPT  character string of options:

            log:      set log level to the default set up by MZEBRA
                   Q  quiet, set log level to -2

       FENCE  safety area immediately in front of the store to
              protect against illegal use of LQ(0), LQ(-1), etc.
              The fence extends from FENCE(1) to LQ(0).

       LQ(1)  first word of the dynamic store

      LQ(LR)  first permanent reference link, if any

      LQ(LW)  first word in the store after the permanent links,
              this is the first word available to the working space,
              or to division 1. The following words are allocated
              as permanent user links, if any:

                LQ(1)  to LQ(LR-1)  structural links, if any
                LQ(LR) to LQ(LW-1)   reference links, if any

    LQ(LIM2)  lowest position of the end of division 2,
              to protect divisions 1 and 2 from being squeezed out
              of existence by other divisions created later.

    LQ(LAST)  last word of the dynamic store.

Required:

  -  the fence must have one word at least, but at most 1000 words.
  -  the data region of the store (ie. the total store minus
     the permanent links) must not be less than 2000 words.
\end{verbatim} 

The store is allocated by \Rind{MZSTOR} as follows:

\begin{verbatim}
 _________________________________________________________________ _ _ _ _ 
|       |        permanent       | div |         | div | division |
| fence | structural | reference | 1   | reserve | 2   |  system  | 
|       |          links         | --> | area    | <-- |    <---- | [table]
|_______|____________|___________|_____|_________|_____|__________|_ _ _ _ 
|       |(1)         |(LR)       |(LW) |         |     |    (LAST)| 
 FENCE   LQ                                                       [or LAST]
\end{verbatim} 

The fence region is preset to contain IQNIL in every word, this must never be
changed; the debug aids will check for over-writing.  The permanent links are
cleared to zero; the rest of the store is not touched.

The use of division 1 is somewhat special as explained
in section~\ref{sec:MZWORK} for \Rind{MZWORK}.

\Examples

\begin{verbatim}
for a normal primary store:

      PARAMETER   (LIM2Q=40000)
      PARAMETER   (NNQ=120000)
      DIMENSION    LQ(NNQ), IQ(NNQ), Q(NNQ)
      EQUIVALENCE (LSAFE,LQ(1)), (Q(1),IQ(1),LQ(9))

      COMMON //    IXDIV1, ...            division indices
     +,            FENCE(16),LSAFE(10)    ten unused links for safety
     +,            LMAIN, ...             more structural links
     +,            L1, ...                more reference  links
     +,            DIV12(99)

      CALL MZSTOR (0,'//','.',FENCE,LQ,L1,DIV12,Q(LIM2Q),Q(NNQ))

for a secondary store without permanent links:

      DIMENSION    LZ(40000), IZ(40000), Z(40000)
      EQUIVALENCE (Z(1),IZ(1),LZ(9))

      COMMON /ZDYN/IXSTZ,IXDV1,IXDV2,IXHIT
     +,            FENDZ(16),LZ,LASTZ

      CALL MZSTOR (IXSTZ,'/ZDYN/','.',FENDZ,LZ,LZ,LZ,LZ(30000),LASTZ)
      IXDV1 = IXSTZ + 1
      IXDV2 = IXSTZ + 2
\end{verbatim} 

\subsubsection*{Note:} 
To ease the use of double-precision variables in the working space,
the number of words in the fence, more generally: the number of words in the
Common block preceding the dynamic store, as well as the number of permanent
links, should be \textem{even}.

\Rind{MZSTOR} prints at log level -1 or above
an initialization message on unit \Lit{IQLOG},
showing the whereabouts and the sizes of the store.

\Filename{H2-MZDIV}
\section{MZDIV  - create a new division}

\Rind{MZDIV} may trigger relocation.

A dynamic store is physically subdivided into 'divisions'.
Up to 20 divisions are allowed, which permits 17 divisions
created by the user with \Rind{MZDIV}, beyond the 3 divisions
created initially by \Rind{MZSTOR}.

Dividing the store into divisions is a device for keeping
different data-structures in physically separate parts
of the dynamic store.
In principle the user does not need to care where and in what
order his banks are kept physically in the store,
since all logical relations are described by links,
not by arrangement.
The reason for using divisions in spite of this general principle
is exclusively the possible gain in efficiency,
when deleting a whole data-structure, for example,
or with the output of a data-structure to tape or disk.
This may be seen by comparing the operations necessary to output
either a data-structure whose banks co-exist intermixed with
the banks of other structures, or a data-structure which has
the exclusive use of a division:
the first case requires a logical walk through the data-structure
to identify all the banks which belong to it,
plus the construction of a table to indicate their where-abouts,
and the write-out of the memory according to that table;
in the second case a contiguous chunk of memory
can be written out directly.

Mode of a division:  depending on whether a division grows
at its higher or at its lower end,
a division is said to be of mode 'forward' or 'reverse'
(\Lit{MODE = 0} or \Lit{1}).
Reverse mode is selected by the \Ropt{R} option in the call to \Rind{MZDIV},
else forward mode is assumed.
By arranging a forward division to be followed by a reverse
division (of the same kind),
the 'reserve areas', ie. the reserved space for these divisions
not currently allocated to banks,
of the two divisions is contiguous and is hence available
to either division,
thereby reducing the total memory requirement in general.

Kind of a division:  Depending on its usage,
we distinguish three kinds of divisions (apart from the
system division, which is used by the ZEBRA system itself):

\begin{itemize}
\item a 'user short-range' division is a division which is
      wiped clean after every event;
\item a 'user long-range' division is a division which carries
      data living beyond several events,
      or many events, or until the termination phase;
\item a 'package' division is a division used by some service
      packages,
      and whose contents are not normally of any concern to the user.
\end{itemize}

Links, including structural links,
may in general point from one division to any other division
of the same store,
except that the link 'next-of-same' may not,
ie. a linear stucture must be contained within one division.
To reduce the load on relocation,
the user has the possibility to indicate which division points
to which other divisions,
as explained in section~\ref{sec:MZXREF} for \Rind{MZXREF}.
By default ZEBRA assumes that all user divisions may point
to all other user divisions.

User short-range divisions are allocated one after the other
from left to right, starting with division 1;
long-range and package divisions
are allocated one before the other from right to left,
starting with division 20, pushing the system division downwards.

A given division in a particular store is identified by its 'index';
thus if a bank is to be created in this division,
its index has to be specified to \Rind{MZLIFT}.
The division index carries the store-number and the division-number
encoded onto a word of 32 bits;
when a new division is created by a call to \Rind{MZDIV} the next free
division-number is allocated to it,
and the encoded division index is returned to the user as
an output parameter.
The store index, constructed by \Rind{MZSTOR}, is formally a division index
for the (non-existing) division zero.
The format of the division index permits the simultaneous selection
of several divisions by a 'compound index',
see section~\ref{sec:MZIXCO} for the specifications.

A division is created with:

\Shubr{MZDIV}{(IXSTOR,IXDIV*,chNAME,NW,NWMAX,chOPT)}

with
\begin{XMP}
      IXSTOR  index of the store for creation,
              zero for the primary store

      IXDIV*  returns the index of the division created,
              to be used when specifying this division
              in subsequent calls to the ZEBRA system.

      chNAME  name of the division for printing purposes,
              8 characters maximum

          NW  number of words to be allocated to the division initially,
              during execution later the division may grow, but not beyond
       NWMAX  maximum size of the division

       chOPT  character string of options:

             mode:      forward mode is default    (gives MODE = 0
                     R  reverse division                  MODE = 1
                     M  match the neighbour division      MODE = 0 or 1

             kind:      user short-range is default       KIND = 1
                     L  user long-range division          KIND = 2
                     P  package division                  KIND = 3 )
                        (P implies C, over-rules L)

             xref:      by default all user divisions point to all other
                                   user divisions (see section~\ref{sec:MZXREF})
                     C  division contained, ie. no links point outside

Required:     NW at least 100 words,  NWMAX at least NW words

\end{XMP}

\Examples

\begin{verbatim}
      CALL MZDIV (0,IXHITS,'HITS',10000,20000,'.')

           creates a user short-range division HITS in forward mode.

      CALL MZDIV (0,IXCALI,'CALIB',8000,8000,'RL')

           creates a user long-range division CALIB in reverse mode.
\end{verbatim} 

If the user creates several divisions it is economic to create
pairs of forward/reverse divisions.
With Zebra version 3.67 the mode option M has been introduced
to request automatic pairing with the divisions already existing.

\Example

\begin{verbatim}
      CALL MZSTOR (0, '//', ...
      CALL MZDIV  (0, IXABC, 'ABC', NW3,  NWMAX3, 'M')
      CALL MZDIV  (0, IXDEF, 'DEF', NW4,  NWMAX4, 'M')
      CALL MZDIV  (0, IXXYZ, 'XYZ', NW20, NWMX20, 'LM')
      CALL MZDIV  (0, IXUVW, 'UVW', NW19, NWMX19, 'LM')
\end{verbatim} 

This will give a primary store with this lay-out:

\begin{verbatim}
   | div 1      div 2 | div 3      div 4 |   div    | div 19      div 20 |
   |                  |                  |  system  |                    |
|  | ----->    <----- | ----->    <----- |     <--  | ------>    <------ |
|__|__________________|__________________|__________|____________________|
\end{verbatim} 

The forward divisions 3 and 19 are followed by the reverse
divisions 4 and 20.
The divisions of each pair share the same memory region,
originally of NW3+NW4 and NW19+NW20 words.
Thus the occupied space of one division can be large
(even larger than its own declared maximum) at a particular
moment during execution,
if the space occupied by the other division of the pair is small
enough to keep the sum below the maximum.

\subsubsection*{Remember:}
the higher logical entity above the bank
is the 'data-structure' and \textem{not} the division;
the division is a physical concept entirely different
from the logical concept of the data-structure,
and the two must not be confused.

\Rind{MZDIV} prints at log level 0 or above
an initialization message on unit \Lit{IQLOG}.

\Filename{H2-MZLINK-MZLINT}
\section{MZLINK / MZLINT - initialize a link area}

\Rind{MZLINK} and \Rind{MZLINT} may trigger garbage collection.

A link area is a vector of links outside any dynamic store,
with all its links pointing to one particular store,
consisting of \Lit{NS} structural links followed by 
\Lit{NR} reference links.
Either \Lit{NS} or \Lit{NR} may be zero.

We distinguish two kinds of link areas:

\subsection*{Permanent link area}

A permanent link area is initialized once at the steering level
and stays alive for the whole program;
it consists of just the vector of links.

\subsection*{Temporary link area}

A temporary link area is requested and de-activated
by lower-level code any number of times.
Each such area has two words pre-fixed to the link-vector
for efficiency:

\begin{DLtt}{123456}
\item[word 1] is a flag indicating whether the area is active or not;
              if this word is zero, the contents of the area will
              not be updated in case of a memory move.
              The user may reset this word to zero to de-activate the area.
\item[word 2] is a key allowing the system to easily find the
              where-abouts of this area in the system tables without searching.
              This word must never be touched by the user.
\end{DLtt}

A link area must be in COMMON storage;
if it is in local storage there is a danger that Fortran
optimization causes a register to preserve the old value of a link
across a relocation operation,
for garbage collection,
but also for simple updating with \Rind{MZDROP}, \Rind{ZSHUNT}, etc.

As for links in banks, a structural link may only contain zero or the
valid address of a bank; it must never contain garbage.

To initialize a permanent link area, one calls once,
and once only for this area:

\Shubr{MZLINK}{(IXSTOR,chNAME,!LAREA,!LREF,!LREFL)}

with
\begin{verbatim}
      IXSTOR  index of the store into which the links will point,
              (IXDIV of any division in this store allowed)
              zero for the primary store

      chNAME  name of the Fortran Common in which the link area resides,
              for printing purposes, 8 characters maximum

      !LAREA  first word of the link area,
              being also the first link of this area

       !LREF  first reference link, if any;
              last structural link, if no reference links

      !LREFL  last reference link, if any,
              if none: give LAREA in this parameter position
\end{verbatim} 

\Rind{MZLINK} will clear the links to zero.

\Examples

\begin{verbatim}
mixed link area:

      COMMON /LAMIX/ LS1,...,LSN, LR1,...,LRN

      CALL MZLINK (0,'/LAMIX/',LS1,LR1,LRN)

structural link area:

      COMMON /LASTR/ LS1, ..., LSN

      CALL MZLINK (0,'/LASTR/',LS1,LSN,LS1)

reference link area:

      COMMON /LAREF/ LR1, ..., LRN

      CALL MZLINK (0,'/LAREF/',LR1,LR1,LRN)
\end{verbatim} 

Note that in a permanent link area with exactly one link
this link cannot be a reference link.

\Rind{MZLINK} prints at log level zero (or above) an initialization
message on unit \Lit{IQLOG}.

To activate a temporary link area, one calls with:

\Shubr{MZLINT}{(IXSTOR,chNAME,LAREA,!LREF,!LREFL)}

with
\begin{verbatim}
      IXSTOR  index of the store, as for MZLINK

      chNAME  name of the link area, as for MZLINK

       LAREA  first word of the link area,
              with: LAREA(1) the user flag word
                     LAREA(2) system word
                     LAREA(3) the first link of the area

       !LREF  first reference link, if any, as for MZLINK

      !LREFL  last reference link, if any, as for MZLINK
\end{verbatim} 

\Rind{MZLINT} will clear the links to zero,
set the flag-word \Lit{LAREA(1)} to be non-zero,
and set the system-word \Lit{LAREA(2)} on first contact.

To de-activate a temporary link-area the user sets \Lit{LAREA(1)=0}.
From then on the links in this area are no longer relocated,
and hence will be meaningless.
To re-activate the area the user could set \Lit{LAREA(1)=1},
but he must then clear the contents of the links himself;
it is safer to call \Rind{MZLINT}, which will do the necessary.

\Examples

\begin{verbatim}
mixed link area:

      COMMON /LAMIX/ LAMIX(2), LS1,...,LSN, LR1,...,LRN

      CALL MZLINT (0,'/LAMIX/',LAMIX,LR1,LRN)

structural link area:

      COMMON /LASTR/ LASTR(2), LS1, ..., LSN

      CALL MZLINT (0,'/LASTR/',LASTR,LSN,LASTR)

reference link area:

      COMMON /LAREF/ LAREF(2), LR1, ..., LRN

      CALL MZLINT (0,'/LAREF/',LAREF,LR1,LRN)
\end{verbatim} 

\Rind{MZLINT} prints a log message at level 1 for initialization
and at level 2 for re-initialization.

\Filename{H2-MZWORK}
\section{MZWORK - allocate working space}
\label{sec:MZWORK}

\Rind{MZWORK} may wipe division~1 and it may trigger garbage collection.

The region at the beginning of the dynamic store just after
the permanent links may be used as working space,
consisting of a set of reference links followed by
a set of data words.

The user requests reservation of the working space with

\Shubr{MZWORK}{(IXSTOR,DFIRST,DLAST,IFLAG)}

\begin{verbatim}
with  IXSTOR  index of the store or of any of its divisions,
              zero for the primary store

      DFIRST  first data word of the working space,
              the preceding words are taken as reference links;
              this parameter is ignored if IFLAG is =2 or =-1

       DLAST  last data word,
              this parameter is ignored if IFLAG is =-1 or =3 or =4

       IFLAG  = 0  define a new working space,
                   previous contents are not to be retained

              = 1  vary the limits of the working space,
                   keeping intact the links which are common to
                   both the old and the new working space.

              = 2  vary only the DLAST limit of the working space,
                   keeping alive all links and the data words
                   which are common to the old and the new working space.

              = 3  keep the DLAST limit unchanged, keep division 1,
                   re-define the DFIRST limit for new links,    
                   ie. clear the links to zero.             

              = 4  keep the DLAST limit unchanged, keep division 1,
                   re-define the DFIRST limit keeping intact the links
                   which are common to the old and the new working space.

              =-1  reset the working space to null,
                   ie. to zero links, zero data words
\end{verbatim} 
Staring with Zebra version 3.67, the links of the working space
are cleared to zero by \\
\Rind{MZWORK}; if \Lit{IFLAG = 1} or 4 only the new links are cleared.


A call to \Rind{MZWORK} with \Lit{IFLAG < 3} destroys division 1 of the store
\textem{without} a relocation pass to reset links pointing into division 1.
This is done in this way for efficiency,
hence normally only very temporary data should be kept in division 1,
and only working space links should point to them.
To say it differently: division 1 is logically part of the
working space,
its time of existence is the same as that of the working space,
and it is good practice to maintain links into division 1
only in the working space.
If however it is necessary to hold such links elsewhere,
one should either reset them explicitly or wipe the division with
\Lit{CALL \Rind{MZWIPE}(1)} before calling \Rind{MZWORK}
[ or with \Lit{CALL \Rind{MZWIPE}(IXSTOR+1)} for a secondary store ].

\Example

\begin{verbatim}
+CDE, Q.
     +,         LR1, ...     working space reference links
     +,         DFIRST, ...  working space data words
     +,         DLAST

      CALL MZWORK (0,DFIRST,DLAST,0)
\end{verbatim} 

will give this layout of the store:

\begin{verbatim}
    _____________________________________________________________
   |           |       |             |               |           |
   | permanent |  wsp  |   w. space  | div 1   div 2 | other     |
   |   links   | links |     data    | --->     <--- | divisions |
   |___________|_______|_____________|_______________|___________|
   |LQ         |LR1    |DFIRST  DLAST|                           |
\end{verbatim} 

\Rind{MZWORK} prints a monitor log message at level 2.

\Filename{H2-MZLIFT-MZBOOK}
\section{MZLIFT / MZBOOK - bank creation}

\Rind{MZBOOK} and \Rind{MZLIFT} may cause garbage collection.

A bank may be lifted with either

\Shubr{MZBOOK}{(IXDIV,!L*,!LSUP,JB, chID,NL,NS,ND,IOD,NZERO)}

or

\Shubr{MZLIFT}{(IXDIV,!L*,!LSUP,JB, NAME, NZERO)}

with
\begin{verbatim}
        IXDIV   index of the division in which the bank is to be lifted
                 = 0       use default in   primary store
                [= IXSTOR  use default in secondary store ]
                see below for details

          !L*   returns the address of the bank created

        !LSUP   if JB < 0:  address of the supporting up bank
                if JB = 0:  address of the supporting previous bank
                if JB = 1:  supporting link*
                if JB = 2:  LSUP not used

           JB   if JB < 1:  link bias in the supporting bank
                             (must be a structural link)
                if JB = 1:  create top-level bank
                if JB = 2:  create stand-alone bank

         NAME   specifies the properties of the bank:

 NAME(1) = ID:  4-character bank identifier, alphanumeric,
                  MZBOOK: variable of type CHARACTER
                  MZLIFT: Hollerith string of the form 4Hxxxx
     (2) = NL:  total number of links, NL < 64k
     (3) = NS:  number of structural links, NS <= NL
                  (not counting the 3 links next, up, origin)
     (4) = ND:  number of data words, ND < 1 Mwords

     (5) = IOD:  I/O descriptor for the data words, which may be:
                 - result from MZFORM, 1 word, I/O index
                 - result from MZIOBK, 1 or more words, I/O characteristic
                 - immediate, the whole bank is of the same type:
                    IOD = 0  undefined, bank cannot be transported
                          1  32-bit unsigned integer
                          2  signed integer
                          3  floating         4  double-precision
                          5  Hollerith        7  self-descriptive
                 - requests:
                    IOD = 9  retrieve I/O characteristic by IDH,
                             stored in the system by MZFORM
                         11  take the I/O characteristic from any
                             bank in the target linear structure,
                             if this is empty act as for 9
\end{verbatim}

\begin{verbatim}
        NZERO   controls whether and how much of the data-part
                of the bank is preset to zero:
                   N = -1:  no presetting
                   N = 0:  the whole bank is cleared
                   N > 0:  the first N words are cleared
                The links of the bank are always cleared to zero.
\end{verbatim} 
In what follows we will need the parameter \Rarg{LNEXT},
which is the address of the successor bank to the bank to be lifted;
its address will be stored into link 0 of the new bank,
ie. into its 'next' link.
\Rarg{LNEXT} is obtained as follows:
\begin{verbatim}
        if  JB < 1:   LNEXT = LQ(LSUP+JB)
            JB = 1:   LNEXT = LSUP
            JB = 2:   LNEXT = 0
\end{verbatim} 
The \textem{division} [ and the store ] into which the bank
is to be lifted is selected by the parameter \Rarg{IXDIV}.
If a definite division is specified, this is used,
but it must be compatible with \Lit{LNEXT};
a linear structure must be contained within one and the same
division !

If 'default division' is specified,
by giving zero [ or \Rarg{IXSTOR} ] for \Rarg{IXDIV},
the bank will be created in the division holding the logically
nearest bank of the data structure into which the new bank
will be inserted, in this order of priority:

\begin{itemize}
\item if the bank is added to an existing, non-empty, linear structure,
      i.e. if \Lit{LNEXT} is non-zero, or if \Lit{JB=0},
      use the division holding this linear structure;
\item if \Lit{JB < 0} use the division holding the supporting up bank at \Rarg{LSUP};
\item use division 2.
\end{itemize}

The \textem{numeric bank identifier} \Lit{IDN} serves to readily identify
a particular bank within a linear structure for interactive work.
Normally the Hollerith identifiers \Lit{IDH} in a linear structure are
all identical;
it is desirable that the \Lit{IDN}'s are all different.
In principle the user can set any positive integer into \Lit{IQ(L-5)} after
the bank has been created, but \Rind{MZLIFT}/\Rind{MZBOOK} will store a default
value in this order of priority:

\begin{verbatim}
  -  if LNEXT is non-zero: use IDN+1 of the bank at LNEXT;

  -  if JB = 0: use IDN+1 of the bank at LSUP;

  -  if JB < 0: use the value IDN = -JB;

  -  IDN = 1 will be set.
\end{verbatim} 

The bank being lifted is \textem{inserted}
into an existing data-structure according to \Rarg{LSUP}
under control of the parameter \Rarg{JB}:

\begin{verbatim}
--- case 1: JB < 0       insert as dependent of the bank at LSUP
                          ex.:  CALL MZLIFT (0,L,LSUP,-4,NAME,0)
            _______
           |       |
         __|  SUP  |  <-------------------------.------ ...
        |  |       |  <---------.               |
  link  |  |_______|            |               |
JB= -4  |                       |               |
        |                       |               |
        |                    up |            up |
        |                  ________        ________
        |                 |        |      |        |
        `---------------> |  lift  | ---> |  NEXT  | --->  ...
                          |________|      |________|

  SUP   is the supporting bank whose address is LSUP
  lift  is the new bank being lifted
  NEXT  is the first bank of the linear structure in front of which
        the bank 'lift' is inserted, ie. the initial content of
        link JB in bank SUP was LNEXT = LQ(LSUP+JB).

The up-link and the origin-link in the bank lifted are set as

                  up-link  <--  LSUP
              origin-link  <--  LSUP+JB


--- case 2: JB = 0       insert inside a linear structure
                          ex.:  CALL MZLIFT (0,L,LSUP,0,NAME,0)
     ______
    |      |
  __|  UP  |  <------------------------------------------.---- ...
 |  |      |  <------------.                             |
 |  |______|               |                             |
 |                         |<-------------.              |
 |                         |              |              |
 |                      up |           up |           up |
 |      ____          ________       ________       ________
 |     |    |        |        |     |        |     |        |
 `---> |    | -...-> |  SUP   | --> |  lift  | --> |  NEXT  | --> ...
       |____|        |________|     |________|     |________|


The up-link and the origin-link in the bank lifted are set as

                  up-link  <--  copy of the up-link in bank SUP
              origin-link  <--  LSUP
\end{verbatim} 
\Lit{SUP}, \Lit{lift}, \Lit{NEXT} have the same significance as in case 1.
\Lit{UP} is the bank pointed to by the up-link in bank \Lit{SUP},
which link might be zero.
If bank \Lit{NEXT} does not exist, ie. if \Lit{LNEXT = LQ(LSUP)} is zero,
the bank \Lit{'lift'} is added at the end of the linear structure.

\begin{verbatim}
--- case 3: JB = +1      create new top-level bank
                          ex.:  CALL MZLIFT (0,L,LSUP,1,NAME,0)

              usually: zero  <------------------.
                                                 |
                                                 |
                                  -------------->|
                                 |               |
                              up |            up |
                            ________        ________
                           |        |      |        |
 result  LSUP and L  ----> |  lift  | ---> |  NEXT  | ---> ...
                           |________| .--> |________|
                                      |
  input  LSUP  -----------------------^
\end{verbatim} 
For  \Lit{JB=+1}  the origin-link and the up-link are taken from
the bank \Lit{NEXT} pointed to by \Lit{LSUP} on input;
the first time round, when \Lit{NEXT} does not exist,
ie. when \Lit{LSUP=0},
the up-link is set to zero and the origin-link is made to point
to the location containing \Lit{LSUP}.

The location pointed to by the origin-link is filled with the address
of the bank created, unless the origin-link is zero.
\begin{verbatim}

--- case 4: JB = +2      no inserting at all
                          ex.:  CALL MZLIFT (0,L,0,2,NAME,0)
\end{verbatim} 
For  \Lit{JB=2}  the bank is lifted without being linked into a structure;
both the origin-link and the up-link in the bank are set to zero. 

\subsubsection*{Note:} 

because structural links are used to connect banks as
illustrated by these examples, and are hence manipulated by the system,
they must never contain anything but valid bank addresses, or zero.

\Rind{MZLIFT} and \Rind{MZBOOK} print a monitor log message at level 2.

\Filename{H2-MZPUSH}
\section{MZPUSH - alter the size of a bank}

\Rind{MZPUSH} (increasing) may cause garbage collection.

To increase or decrease the size of a bank, one may

\Shubr{MZPUSH}{(IXDIV,*!L*,INCNL,INCND,chOPT)}

with
\begin{verbatim}
       IXDIV  index of the division; zero [or IXSTOR] allowed,
              in which case MZPUSH will find the division

        *!L*  address of the bank to be pushed,
              on return L contains the new address of the bank
              (cannot be a link in a bank !)

       INCNL  number of additional links,
              zero for no change, negative for decrease.
              Additional links will be given type 'reference',
              unless the original bank has only structural links.

       INCND  number of additional data words,
              zero for no change, negative for decrease.

       chOPT  character string of options:

                 default: any link may point to the bank at L,

              R  but not into the abandoned bank region
                        (in case of bank reduction)

              I  isolated: only the inherent structural links
                           point to this bank
\end{verbatim} 
For 'isolated' the user is sure that no link other than the
supporting structural link, and the link passed in the parameter \Rarg{L},
and the reverse links in the first level dependents,
point to this bank.

The I/O descriptor of the old bank must also be valid for the
new bank; if this would not be the case one should use 
\Rind{MZREPL} instead of \Rind{MZPUSH}.

New link words and new data words acquired by the bank
are cleared to zero.

Except for special cases,
increasing the size of a bank is an expensive operation,
and should be avoided.
The correct solutions to the problems of indefinite number of objects
or of indefinite size are either the use of a linear structure
or the lifting of a bank of maximum size,
which is reduced to the final size once the contents are complete.

The increase of a bank is handled in general as follows:
a new bank is lifted with the new dimensions,
the contents of the original bank are copied over,
and the old bank is marked as dropped.
Any link pointing to the original bank must be re-routed
to the replacement bank by \Rind{MZPUSH}.
In full generality this can only be done by a relocation pass
over all memory (ie. all link areas and all divisions pointing
to the division in which the bank resides.)

The expensive part in this operation is the relocation pass,
and this can be saved under special circumstances:

 a) Increasing the data part of the last bank in a forward division,
or the link part in a reverse division,
can be done in situ without the lifting of a new bank.
Hence any link pointing to the bank remains valid.
(However note: this is not a good reason for creating an extra
division, because the lifting of a maximum size bank is
a better solution.)

 b) If there are no links pointing to the bank (except the
standard structural support links)
there is no point to scan all links,
because the relocation pass would have no effect.
However, ZEBRA could not detect this situation without actually
performing the relocation pass,
but the user may know the situation a priori,
in which case he can signal it to \Rind{MZPUSH} with the \Ropt{I} option.

Reducing the size of a bank is less onerous (if
the \Ropt{R} option is given).
The parameters of the original bank are adjusted to the
new dimension in situ and the abandoned memory is marked as dead.
Links which point into the abandoned region, if any,
can only be reference links and must be reset to zero.
To have links pointing into a bank, rather than to the status-word,
is relatively rare.
Again, the relocation pass can be saved,
and for reducing a bank this is the normal situation,
if the user knows the situation a priori and signals it to
\Rind{MZPUSH} with the \Ropt{R} option (or \Ropt{I} which is stronger).

\Rind{MZPUSH} prints a monitor log message at level 2,
but if the operation involves a relocation pass a log message
will be given at level 1.

\Rind{MZPUSH} accumulates statistics of the number of times that an
expensive operation was performed;
this is printed by \Rind{MZEND}.

\Filename{H2-MZREPL}
\section{MZREPL - connect replacement banks}

If one wants to replace a given 'old' bank in a data-structure
by a 'new' bank,
and if there are links elsewhere in the data-structure(s)
or in link-areas pointing to the old bank,
this is a non-trivial operation because these links
have to be relocated to point to the new bank.
Normally this can be done with \Rind{MZPUSH},
but not if the I/O descriptor has to change.
For these cases \Rind{MZREPL} is provided.

The user is supposed to lift and fill the 'new' bank and also a
tiny 'index' bank with 2 (structural) links and 1 data word,
recording the address of the old bank in link 1 and the address
of the new bank in link 2.
This index bank is handed to \Rind{MZREPL} which will then execute
the necessary relocation, using the data word as working storage.

If several such operations are needed one should not loop
over calls to \Rind{MZREPL}, but construct a linear structure of
index banks for \Rind{MZREPL}, which will then do all the replacements
with one single relocation pass.
Because of this feature \Rind{MZREPL} can be more economical than
\Rind{MZPUSH} for multiple replacements.

Both the old and the new banks are required to be in one single
division, but the index banks may be in a different division.

\Shubr{MZREPL}{(IXDIV, *!LIX*, chOPT)}

with
\begin{verbatim}
       IXDIV  index of the division; zero [or IXSTOR] allowed,
              in which case MZREPL will find the division

      *!LIX*  the address of the linear structure of the index banks.

       chOPT  character string of options:

                 default: drop the old and the index banks

              K  keep the old and the index banks

              I  isolated, as for MZPUSH
\end{verbatim} 

If one needs to keep the old banks beyond the call to \Rind{MZREPL},
note this: really all links pointing to the 'old' banks are
relocated, and therefore one would loose access to the old banks.
To overcome this the K option selects a special processing:
before going into the relocation the link pointing to the
old bank is saved into the data-word of the index bank
(this works because there is no garbage collection involved);
just before returning to the user all index banks are changed
to have only 1 structural link restored to support the old bank,
and link 2 continues to point to the corresponding new bank,
but now as a reference link.

Each old bank is changed to have zero structural links,
the 'next' link is reset to zero, and the bank is made to
depend from its index bank.

When one has finished with the old banks one can drop the
structure of index banks, thereby also dropping the old banks.

The linear structure of index banks is re-ordered by \Rind{MZREPL} for
increasing address of the old banks;
\Rarg{LIX} returns the address of the new first bank.


\Example

\begin{verbatim}

   Replace all the banks of a linear structure.

   LIX, LOLD, LNEW  are temporary links,
   either in the working space or in a link area.

   IXDIVA is the division index of the banks;
   IXDIVX is the index of some short range division to hold the
          index banks, division 1 by preference.

      LIX = 0    clear LIX to start (reference link allowed)

      LOLD = get the adr of the first old bank

   24 CALL MZBOOK (IXDIVX, L, LIX,1, 'IXIX',2,2,1,0, -1)
      LQ(LIX-1) = LOLD

      CALL MZLIFT (IXDIVA, LNEW, LIX,-2, ...

          fill the new bank at LNEW
               ie. links LQ(LNEW-NL) to LQ(LNEW-1)
                    data IQ(LNEW+1)  to IQ(LNEW+ND)
               but not 'next', 'up', 'origin'

      LOLD = LQ(LOLD)
      IF (LOLD.NE.0)         GO TO 24

      CALL MZREPL (IXDIVA, LIX, '.')
\end{verbatim} 

\Rind{MZREPL} prints a monitor log message at level 2.

\Filename{H2-MZDROP}
\section{MZDROP - drop a bank and its dependents}

With \Rind{MZDROP} one may either drop the bank at \Lit{L} or,
if the \Ropt{L} option is given, the linear structure at \Lit{L}.
Dropping a bank implies dropping also the whole partial
data-structure which depends on this bank downwards.

Dropped banks stay in memory with the \Lit{IQDROP} status-bit set,
links pointing to them continue to do so,
except for the immediate structural link indicated via the
origin-link of the bank at \Lit{L}, which is bridged or reset to zero,
whichever is appropriate.

To drop one gives

\Shubr{MZDROP}{(IXSTOR, !L, chOPT)}

with
\begin{verbatim}
      IXSTOR  index of the store or of any division in this store,
              zero for the primary store

          !L  the address of the bank or linear structure to be dropped

       chOPT  character string of options:

                  default: drop the bank at L only,
                  ie. the 'next' link of this bank is not followed

               L  drop the linear structure pointed to by L
                  ie. the 'next' link of the bank at L is followed

               V  drop only the partial data-structure
                  dependent vertically downwards from the bank at L,
                  but not the bank itself.
\end{verbatim} 

\Rind{MZDROP} prints a monitor log message at level 2.

The parameter \Rarg{L} in the call to \Rind{MZDROP} is not as such changed
on return,
but if it is the structural link supporting the d/s it will
in fact be up-dated for the removal of the first bank
in this structure.

Suppose this artifical sequence of operations (with \Lit{LS=0} initially):
\begin{verbatim}
      CALL MZLIFT (0,L,LS,1,...)          create bank '3'
      CALL MZLIFT (0,L,LS,1,...)          create bank '2'
                                  LS now points to bank 2

      CALL MZLIFT (0,L,LS,1,...)          create bank '1'
                                  LS now points to bank 1

      CALL MZDROP (0,LS,'.')              drop bank 1
                                  LS now points again to bank 2

if however the dropping is done with

      CALL MZDROP (0,L,'.')               drop bank 1

         then LS will be made to point to bank 2,
         but L will continue to point to the (dead) bank 1.
\end{verbatim}
Since the parameter \Rarg{L} in the call to \Rind{MZDROP} is not changed,
selective dropping of banks in a linear structure can be done
with a loop like this:
\begin{verbatim}
      L = start adr of the linear structure
   12 IF (L.NE.0)  THEN
          IF (bank not wanted)  CALL MZDROP (0,L,'.')
          L = LQ(L)
          GO TO 12
        ENDIF
\end{verbatim}

\Filename{H2-MZWIPE}
\section{MZWIPE - reset complete divisions to empty}

\Rind{MZWIPE} causes relocation normally.

By calling \Rind{MZWIPE} the user can delete the contents of
whole divisions:

\Shubr{MZWIPE}{(IXWIPE)}

with the parameter \Rarg{IXWIPE} indicating the divisions
of a particular store to be wiped out.
\Rarg{IXWIPE} may be any of the three possible forms of a division index:

\begin{verbatim}
  a) specific division index, as returned by MZDIV

  b) generic division index, [ IXSTOR + ] n, where

        n = 21:  all user short range divisions
            22:  all user long  range divisions
            23:  all package divisions

  c) compound division index, as created by MZIXCO, see there for details.
\end{verbatim} 

\Lit{IXWIPE = 0}  is taken to mean \Lit{IXWIPE = 21},
ie. all short range divisions in the primary store;
similarily \Lit{IXSTOR+21} can be used.

Wiping divisions resets the divisions to be empty,
but without reducing the space reserved for them,
followed by a relocation pass to reset to zero all links
pointing into the wiped divisions.
Included into this pass are the links of all link areas,
of the working space and of all divisions which are declared
to point to the divisions in question
(all of this for one particular store only, of course).

If several divisions are to be wiped out,
this must be done by \textem{one} call to \Rind{MZWIPE},
and not by several calls in succession,
to save the time of multiple relocation passes,
each of which would take longer than the single pass.

\Examples

for the primary store one has:

\begin{verbatim}
  1) Wipe the last 'event':      CALL MZWIPE (0)

  2) Wipe division 1:            CALL MZWIPE (1)

  3) Wipe all user's divisions:  IX = MZIXCO (21,22,0,0)
                                  CALL MZWIPE (IX)

  4) Wipe divisions IX1, IX2, IX3, IX4, and IX5:

                   IX = MZIXCO (IX1,IX2,IX3,IX4)
                   IX = MZIXCO (IX,IX5,0,0)
                   CALL MZWIPE (IX)
\end{verbatim} 

\Rind{MZWIPE} operates by calling \Rind{MZGARB}, which will print a monitor
log message at level 2.

\Filename{H2-MZGARB}
\section{MZGARB - garbage collection requested by the user}

Garbage collection is triggered by the system if there is not
enough space to satisfy an \Rind{MZWORK} or \Rind{MZLIFT} request
(hopefully this does not normally occur).
Thus the user does not have to worry about initiating
garbage collection to win space in the dynamic store.
To remove the last event from the store,
the user calls \Rind{MZWIPE} which is much more efficient
than dropping the banks of the event followed by garbage collection.

He may however occasionally want to force a garbage collection
to tidy up his data,
particularly during the initialization phase of his program.
Again, as in \Rind{MZWIPE}, if there are several divisions to be tidied up,
this must be done by \textem{one} call to \Rind{MZGARB}.
Also,
if one or several divisions are to be wiped out at the same moment,
this should be included into the call to \Rind{MZGARB};
one and the same relocation pass can handle wiping
and garbage collection simultaneously.

The calling sequence is:

\Shubr{MZGARB}{(IXGARB,IXWIPE)}

with
\begin{verbatim}
      IXGARB  index of the divisions where garbage
              is to be collected (none if =0)

      IXWIPE  index of the divisions to be wiped out
              = 0:  no divisions to be wiped
\end{verbatim} 

Both \Rarg{IXGARB} or \Rarg{IXWIPE} may be any of the three possible forms
of a division index:
\begin{verbatim}
  a) specific division index, as returned by MZDIV

  b) generic division index, [ IXSTOR + ] n, where

        n = 21:  all user short range divisions
            22:  all user long  range divisions
            23:  all package divisions

  c) compound division index, as created by MZIXCO, see there for details.
\end{verbatim} 

\Rind{MZGARB} prints a monitor log message at level 2,
but if the operation involves a garbage collection
a message is given at level 1.

\Filename{H2-MZXREF}
\section{MZXREF - set cross-reference between divisions}
\label{sec:MZXREF}

To save time when wiping a given division (or divisions),
and also on garbage collection,
ZEBRA will relocate the links of only those divisions
which reference the division(s) being changed.
To know which division may have links pointing
to which other division,
ZEBRA keeps internally a cross-reference matrix;
the entry for a given division is intialized by \Rind{MZDIV}
and this may be modified by the user calling \Rind{MZXREF}:

\Shubr{MZXREF}{(IXFROM,IXTO,chOPT)}

with
\begin{verbatim}
      IXFROM  index of the division which contains links pointing
              to the divisions indicated by IXTO;
              this must be the index of one particular division.

        IXTO  index of the division(s) which are referenced

       chOPT  character string of options:

            none  set reference(s), ie. overwrite the previous
                  content of the matrix entry
               A  add reference(s), ie. add to the matrix entry,
                  keeping what was there before
               R  remove reference(s), ie. take the references
                  out from the matrix entry, but keep the others
                  (R over-rules A)
               C  contained division, ie. clear the matrix entry
                  (C over-rules A and R)

      IXTO may be any of the three possible forms of a division index:

        a) specific division index, as returned by MZDIV

        b) generic division index, [ IXSTOR + ] n, where

              n = 21:  all user short range divisions
                  22:  all user long  range divisions
                  23:  all package divisions
                  24:  the system division

        c) compound division index as created by MZIXCO.
\end{verbatim} 

\Rind{MZDIV} creates a division with its matrix row initialized
(unless \Ropt{C} option) as follows:
\begin{verbatim}
      user division:  references all other user divisions
      package div.:  no references at all
\end{verbatim} 

Note: if division \Lit{FR} contains a bank which supports structurally
banks in division \Lit{TO},
then the forward links point from division \Lit{FR} to division \Lit{TO},
but there are also the reverse links in the supported banks
which point from division \Lit{TO} to division \Lit{FR}.
As a result one would need both
\begin{verbatim}
          CALL MZXREF (IXFR,IXTO,'A')
     and  CALL MZXREF (IXTO,IXFR,'A')
\end{verbatim} 

\Examples

for the primary store one has:

\begin{verbatim}
  1) User's division IXTHIS may reference all other user divisions:

         nothing needs to be done, this is the default assumption

  2) Division IXTHIS references only banks in division 2:

                   CALL MZXREF (IXTHIS, 2, '.')

  3) Division IXTHIS references only, but maybe all,
     the short-range divisions:

                   CALL MZXREF (IXTHIS, 21, '.')

  4) Division IXTHIS references all short-range divisions
     and also the long-range division IXLONG:

                   CALL MZXREF (IXTHIS, 21, '.')
                   CALL MZXREF (IXTHIS, IXLONG, 'A')

  5) Division IXTHIS to reference all short-range divisions
     except the division IXSH:

                   CALL MZXREF (IXTHIS, 21, '.')
                   CALL MZXREF (IXTHIS, IXSH, 'R')
\end{verbatim} 

To print the matrix entry for one particular division,
or the complete cross-reference matrix of a store, one may

\Shubr{MZXRPR}{(IXDIV)}

with
\begin{verbatim}
       IXDIV  1) index of the division to be shown
              2) = IXSTOR, index of the store to be shown
\end{verbatim} 

\Filename{H2-MZIXCO}
\section{MZIXCO - create compound division index}
\label{sec:MZIXCO}

A compound division index permits to indicate several divisions
of the same store in one single word,
as used with \Rind{MZWIPE} for example.

\Rind{MZIXCO} joins up to four division indices into a compound:

\Sfunc{MZIXCO}{IXCO = MZIXCO (IX1,IX2,IX3,IX4)}

If less than 4 indices are to be joined trailing zeros
should be given.
If more than 4 indices are to be joined this is done by
repeated calls.

Any input parameter \Lit{IXn} may take any one of the three possible
forms of a division index:
\begin{verbatim}
  a) specific division index, as returned by MZDIV

  b) generic division index, [ IXSTOR + ] n, where

        n = 21:  all user short range divisions
            22:  all user long  range divisions
            23:  all package divisions
            24:  the system division

  c) compound division index, created by a previous call to MZIXCO.

\end{verbatim} 

\Examples

for the primary store one has:

\begin{verbatim}
  1) Compound to give divisions 1 and 2:

                   IXCO = MZIXCO (1,2,0,0)

  2) Compound to give divisions 1, 2, and IXHITS:

                   IXCO = MZIXCO (1,2,IXHITS,0)

  3) Compound to give all short-range divisions and IXCUMU:

                   IXCO = MZIXCO (21,IXCUMU,0,0)

[ 4) Compound to give all user divisions of the store
     whose index is IXSTOR:

                   IXCO = MZIXCO (IXSTOR+21,IXSTOR+22,0,0) ]
\end{verbatim} 

\subsection*{Format of the division index}

\begin{verbatim}

      bits 25, 26, 32    non-zero is illegal

      bit 31  zero:     simple index
              one:     compound index

      bits 27-30 :     the number of the store,
                         zero for the primary store

Division JDIV is selected if

     simple index:      the value of bits 1-24 is JDIV
   compound index:      bit JDIV is present

JDIV may be 1 to 20 for that particular division,
         or it may be generic with:

          JDIV = 21:  user's short range divisions
                 22:  user's  long range divisions
                 23:  package divisions
                 24:  system division
\end{verbatim} 

\Filename{H2-MZFORM}
\section{MZFORM {\it et al.} - handle the I/O characteristic}

\Rind{MZFORM} may cause garbage collection.

The nature of the contents of any bank which is to be transported
from one computer to another one has to be indicated to ZEBRA,
such that it can do the necessary tranformations.
In the simplest case that all the data words of a bank are
of the same type, this is easily indicated in the parameters
to \Rind{MZLIFT} or \Rind{MZBOOK}.
For anything more complicated the user specifies the "format"
of the bank by calling \Rind{MZIOBK} or \Rind{MZFORM} which encode the format
into a variable number of words to be included into each bank
in the system part as the "I/O characteristic".

Thus the content description is carried by each bank;
this avoids complicated logistics of finding bank descriptors
elsewhere than in the bank itself.
Complex bank contents require a relatively large number
of extra system words.
This could represent a substantial overhead on memory
or file space occupation,
which the user can avoid in the design of his bank contents.
Anyway, the number of these extra descriptor words is limited to 16,
and any descriptor which would need more is refused.
Thus ZEBRA will not handle any arbitrary bank contents
via this basic procedure,
but by using the concept of the "self-describing" sector (see below)
the user can indeed store any kind of information mix,
decided at execution time, into a bank and have it travel
from one computer to another one.

\subsection*{Sectors}

The basic element for setting up an I/O characteristic is the sector,
which is a number of words consecutive in the bank which are
all of the same type.
A sector is described in the \Rarg{format} parameter to \Rind{MZFORM} {\it et al.}
as a combination of its word-count \Lit{"c"} and 
its type \Lit{"t"} as  \Lit{"ct"}.
For example, \Lit{24F} is a sector of 24 single-precision floating-point numbers,
\Lit{24D} is a sector of 24 words holding 12 double-precision numbers,
and  \Lit{1I} is a sector of one integer word.

The possible values for \Lit{"t"} are:
\begin{verbatim}
      t =  B   bit string of 32 bits, right justified
           I   integer
           F   floating-point
           D   double precision
           H   4-character Hollerith
           S   self-describing sector (see below)
\end{verbatim} 
A ``static'' sector is a sector of a fixed number of words,
such as the \Lit{24F} of the example above.

An ``indefinite-length'' sector is a sector whose end is defined
by the end of the bank.
This is written as  \Lit{-t}, for example  \Lit{-F}  signals that the rest
of the bank is all floating-point words.

A ``dynamic'' sector is a sector which is preceded in the bank
by a single positive integer word indicating the sector length;
if this number is zero this means that the rest of
the bank is currently unused.
This is written as  \Lit{*t}, for example  \Lit{*F}  indicates
a dynamic sector of type floating.


Thus the word-count \Lit{"c"} in the sector specification is written as:
\begin{verbatim}
      c =  n   numeric, n words:       static length sector
           -   all remaining words:    indefinite length sector
           *   dynamic length sector
\end{verbatim} 
A ``self-describing'' sector is a dynamic sector whose type
is also encoded into the one word preceding the sector as
\begin{verbatim}
   word 0  =  16*NW + IT

      with  NW = length of the sector
            IT = numeric representation of the type
                 =  1  B bit string    2  I integer
                    3  F floating      4  D double precision
                    5  H Hollerith
                    6  (reserve)       7  (special)
\end{verbatim} 
The form \Lit{"nS"} is meaningless;
the form \Lit{"*S"} indicates one particular sector;
the form \Lit{"-S"} is \textem{special} in that it indicates
that the rest of the bank is filled by self-describing sectors,
as many as there may be.
(Thus the forms, for example, \Lit{'4I 5F / *S'} and \Lit{'4I 5F --S'} are
equivalent, but the second form is more economic;
the user may give either, the internal result will be the
second form.)

\subsection*{Complete Characteristics}

Looking now at the bank as a whole,
we divide it into a "leading part" and a "trailing part",
either of which may be empty.

The leading part consists of one region of maybe several sectors,
occurring once at the beginning of the bank.
This leading region may end with an indefinite-length sector,
in which case the trailing part is empty.

The trailing part of the bank may be empty or it may consist
of an indefinite number of regions which
\textem{all have the same structure},
such that the same format description is valid for all of them.

The symbol \Lit{"/"} marks the break between the leading region
and the trailing regions in the format parameter to \Rind{MZFORM} {\it et al.}

\Examples

\begin{verbatim}
trailing part empty:

      '-F'         the whole bank is floating
      '3I -F'      the first 3 words are integer, the rest is F
      '*I -F'      the first word n=IQ(L+1) is a positive integer,
                   words 2 to n+1 are integers, the rest is F
      '3B *I -F'   the first sector consists of 3 words bit-string,
                   the second sector is dynamic of type integer,
                   the rest of the bank is floating
      '3I *F'      the first 3 words are integer, followed by a
                   dynamic sector of type F, the rest (if any) of
                   the bank is currently unused
\end{verbatim} 

\begin{verbatim}

both parts present

      '3B 7I / 2I 4F 16D'  the leading region has 3 B and 7 I words,
                           each trailing region consists of 2 integer
                           words, followed by 4 F words, followed
                           by 16 D words, ie. 8 double-precision numbers
      '4I / *H'    the bank starts with 4 integer words,
                   the rest is filled with dynamic Hollerith sectors
      '*I / 2I *F' the leading region is one dynamic I sector,
                   each trailing region consists of 2 integers
                   followed by a dynamic F sector
                   (ie. 3 integers plus a number of floating words
                    this number being indicated by the 3rd integer)

leading part empty

      '/ *H'       the bank is filled with dynamic Hollerith sectors
      '/ 4I 29F'   4 integers and 29 floating numbers alternate
\end{verbatim} 

\subsection*{Economic formats}

It is in the interest of the user to design his bank contents
such that the I/O characteristic is as simple as possible,
because the number of system words in any bank increases
with the complexity of the lay-out.
"Simple" means: as few sectors to be described as possible.
\begin{verbatim}
For example:  '2B 2I 2B 2I 2B 2I -F' is much less simple then
               '6B 6I -F'.
\end{verbatim} 
Moreover, if the integers described by this format are sure to
be positive integers, then one can use the even simpler form
\Lit{'12B -F'}.

In the following we give an exhaustive list of the most
economic bank formats, those requiring zero or one extra system
word in the banks.

\subsection*{Zero extra I/O words}

These bank formats can be described by the 16 bits of the
I/O control-byte alone:
\begin{verbatim}
   (0)  '-t' or '*t'
        'ct -t'           if c < 64   (c=* is represented as c=0,
          or 'ct *t'                   hence  '*t -t'  is a sub-case)
   (1)  '*t *t -t'
   (2)  '*t *t *t'
   (3)  'ct / *t'         if c < 64
        '/ ct *t'         if c < 64
        '/ ct'            this is useful only if c=*
                          else the form '-t' is used
   (4)  '*t / *t *t'
        '*t *t / *t'
   (5)  '/ *t *t *t'
\end{verbatim} 
\subsection*{One extra I/O word}

These bank formats can be described by the 16 bits of the
I/O control-byte plus the 32 bits of one extra I/O word:
\begin{verbatim}
   (1)  'ct -t'
        'ct ct -t'        if c < 65536
        'ct ct ct -t'     if c < 1024
   (2)  'ct *t'
        'ct ct *t'        if c < 65536
        'ct ct ct *t'     if c < 1024
   (4)  'ct / ct'         if c < 65536
        'ct / ct ct'      if c < 1024
        'ct / ct ct ct'   if c < 256
        'ct ct / ct'      if c < 1024
        'ct ct / ct ct'   if c < 256
   (5)  '/ ct ct'         if c < 65536
        '/ ct ct ct'      if c < 1024        (remember:
        '/ ct ct ct ct'   if c < 256          c=0 means c=*)
\end{verbatim} 


Three routines are provided to mediate between the user
specifying the bank format in a readable form
and the highly encoded I/O characteristic
to be included into any bank at creation time.

\subsection*{\Rind{MZIOCH}}

analyses the format \Rarg{chFORM} to convert and pack it into
the output vector \Rarg{IOWDS}.
This is the basic routine,
but it is usually called by the user only to specify formats
of objects other than banks, like the user header vector for \Rind{FZOUT}.

To specify bank formats the following two routines serve more
conveniently:

\subsection*{\Rind{MZIOBK}}

is provided for the context of \Rind{MZLIFT};
like \Rind{MZIOCH} it analyses the format \Rarg{chFORM},
but it stores the result as part of the bank-description vector
\Rarg{NAME} for \Rind{MZLIFT}.

\subsection*{\Rind{MZFORM}}

again analyses the format \Rarg{chFORM},
but it does not return the result to the user.
Instead,
it remembers the I/O characteristic in a system data-structure,
returning to the user only the index to the characteristic
in the system.
The user may then either pass this index to \Rind{MZBOOK} (or \Rind{MZLIFT})
at bank creation time,
or alternatively he may request \Rind{MZBOOK} (or \Rind{MZLIFT}) to search
the system data-structure for the I/O characteristic
associated to the Hollerith identifier \Rarg{IDH} of the bank to be
created.

The first word of the I/O characteristic delivered by \Rind{MZIOCH} 
or \Rind{MZIOBK} has the following redundant format:
\begin{verbatim}
          |       16       |    5    |     5     | 6 bits|
          |----------------------------------------------|
          |  control-byte  |  NWXIO  |  NWXIO+1  |     1 |
          |----------------|---------|-----------|-------|

\end{verbatim}
The I/O index delivered by \Rind{MZFORM} has the following format:
\begin{verbatim}
          |       16       |    5    |     5     | 6 bits|
          |----------------------------------------------|
          |      index     |    0    |  NWXIO+1  |     2 |
          |----------------|---------|-----------|-------|
\end{verbatim} 
where NWXIO is the number of extra I/O words,
ie. the total length of the characteristic is NWXIO+1.

\subsection*{Typing rules for chFORM}

The format should be typed giving the "ct" for each sector,
in the order in which they occur in the bank, as shown in the
examples.
Leading, interspersed, and trailing blanks (also comma or dot)
for aeration are allowed and ignored.

Single-word sectors must be typed as '1t', 't' alone is illegal.

The c for double-precision sectors gives the number of words,
thus 14D specifies 7 double-precision numbers; 7D is illegal.

\Shubr{MZIOCH}{(IOWDS*,NWIO,chFORM)}

with
\begin{verbatim}
           IOWDS*  the I/O words to receive the result,
                   a vector dimensioned to NWIO

             NWIO  the maximum size of IOWDS, < 17

           chFORM  the format as a CHARACTER string

\end{verbatim} 

\Shubr{MZIOBK}{(NAME*,NWMM,chFORM)}

with
\begin{verbatim}
            NAME*  the bank description vector for MZLIFT,
                   the resulting characteristic will be stored
                   into the I/O words starting at NAME(5),
                   the IDH contained in NAME(1) will be used
                   if diagnostics are necessary,
                   a vector dimensioned to NWMM

             NWMM  the maximum size of NAME, < 21

           chFORM  the format as a CHARACTER string

\end{verbatim} 
\Shubr{MZFORM}{(chIDH,chFORM,IXIO*)}

with
\begin{verbatim}
            chIDH  the Hollerith IDH of the bank, type CHARACTER

           chFORM  the format as a CHARACTER string

            IXIO*  returns the index to the characteristic stored
                   in a system data-structure,
                   this can be passed to MZBOOK/MZLIFT,
                   in which case it must not be modified
\end{verbatim} 

\Examples

\begin{verbatim}
      DIMENSION    IOHEAD(4), MMLIFT(8)

      CALL MZIOCH (IOHEAD,4, '8I -F')             for an FZIN user header
      CALL MZIOBK (MMLIFT,8, '2I / 2I 8F')        for MZLIFT
      CALL MZFORM ('RCBC', '2I/2I 8F'), IORCBC)   for reference by index
\end{verbatim} 

People creating data outside Zebra,
but destined to be read by FZ of Zebra,
will have to know the representation of the I/O characteristic
stored into any bank:

The physically first word of any bank contains:

\begin{verbatim}
   right half:  NOFF = NIO + NL + 12
   (bits 1-16)   where  NIO: the number of extra
                              I/O descriptor words for the bank
                        NL: the number of links in the bank


   left half:   the I/O control byte, which controls the
   (bits 17-32)  interpretation of the I/O characteristic
\end{verbatim} 

In the simplest cases the I/O control byte alone specifies
the nature of the data in the bank, without needing extra
descriptor words (in which case NIO is zero).
We give here the translation of some of these cases:

\begin{verbatim}
       -B:  0001             *B:  0009
       -I:  0002             *I:  000A
       -F:  0003             *F:  000B
       -D:  0004             *D:  000C
       -S:  0007
\end{verbatim} 

For example: suppose one were to prepare a bank with
two links and 4000 data words which are all un-signed
32-bit integer (type B), a bank which is to travel
in link-less mode such that all standard links are zero:

\begin{verbatim}
   word 1   0001 000E     -B | NOFF = 14
        2   zero          link 2
        3   zero               1
        4   zero          link next
        5   zero               up
        6   zero               origin
        7   IDN           numeric ID
        8   IDH           Hollerith ID
        9   2             number of links
       10   1, say        number of structural links
       11   4000          number of data words
       12   zero          status word
                          bits 19-22 give NIO, here zero
       13   data word 1
            ...
     4012   data word 4000
\end{verbatim} 

Note that the status word contains NIO on bits 19--22
to allow Zebra to reach the start-of-bank.

It is impraticable to tabulate the translation of more
complicated formats. There is a little program \Lit{DIOCHAR}
to interactively take a format \Rarg{chFORM}, translate it
and display the result in hexadecimal.
This is not yet properly installed on the CERN machines,
but on the Apollo people at CERN can run it by giving
the command \Ucom{/user/zoll/uty/diochar}

The subroutine \Rind{MZIOTC} is provided to convert an encoded
IO characteristic back into printable form.
One may hand to this routine the address of a bank and
receive its IO characteristic in a {\tt CHARACTER} variable.
Alternatively one may pass to it an integer array as delivered
by \Rind{MZIOCH} for back-conversion to {\tt CHARACTER},
for example the IO characteristic of a user-header vector
read with \Rind{FZIN}.

\Shubr{MZIOTC}{(IXST, !L, NCHTR*, chIOTR*)}

or

\Shubrz{MZIOTC}{(IOWDS, 0, NCHTR*, chIOTR*)}

with
\begin{verbatim}
             IXST  the index of the store holding the bank,
                          or of any of its divisions
               !L  the address of the bank

            IOWDS  the integer array with the encoded characteristic
                   (L must be zero in this case)

           NCHTR*  number of useful characters stored into chIOTR
                   = 0 if trouble

          chIOTR*  the CHARACTER variable to receive the characteristic
\end{verbatim}

The routine returns zero in \Rarg{NCHTR} if L is non-zero and not
a valid bank address,
or if \Rarg{chIOTR} is not long enough.
