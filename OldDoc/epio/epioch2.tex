%%%%%%%%%%%%%%%%%%%%%%%%%%%%%%%%%%%%%%%%%%%%%%%%%%%%%%%%%%%%%%%%%%%
%                                                                 %
%   EPIO User Guide -- LaTeX Source                               %
%                                                                 %
%   Chapter 2                                                     %
%                                                                 %
%   The following external EPS files are referenced:              %
%                                                                 %
%   Editor: Michel Goossens / CN-AS                               %
%   Last Mod.: 25 Nov 1993 23:00 mg                               %
%                                                                 %
%%%%%%%%%%%%%%%%%%%%%%%%%%%%%%%%%%%%%%%%%%%%%%%%%%%%%%%%%%%%%%%%%%%
 
\Filename{H1EPIO-Examples}
\chapter{Examples}
\label{sec:H1Examples}

\Filename{H2EPIO-Examples-32-bit-computer}
\section{Write EPIO file on a 32-bit computer}
\label{sec:write32bit}

Floating point numbers are generated and written to unit~20
in the standard EP format using default values.
Note the dimension of \Lit{IBUF(900)} to accept the default blocksize
of 1800 16-bit units. 
The user's data is stored in the array \Lit{A}.

\begin{XMP}
      PROGRAM EPEXA1
      DIMENSION IBUF(900),A(300)
      CALL EPINIT
      DO 10  I=1,20
      CALL USER(A,NW)
 C
 C   ROUTINE -USER- HAS STORED NW VALUES IN A TO BE WRITTEN OUT
 C
      CALL EPOUTS(20,3,NW,A,IBUF,IERR)
      IF(IERR.NE.0)STOP 1
   10 CONTINUE
 C **********************************************************
 C
 C                THE FOLLOWING CALL IS ESSENTIAL
 C
 C **********************************************************
      CALL EPCLOS(20,IBUF,IERR)
 C
 C   FOLLOWING CALL ONLY NECESSARY ON SOME COMPUTERS (UNIVAC,IBM)
 C
      CALL EPEND(20,IBUF,IERR)
      IF(IERR.NE.0)STOP 2
      STOP
      END
\end{XMP}

\newpage

\Filename{H2EPIO-Examples-write-60-bit-computer}
\section{Write on a 60-bit computer}

Essentially similar to the previous example, a file is
created using all defaults. 
Note the dimension of \Lit{IBUF(480)}.

\begin{XMP}
      PROGRAM EPEXA2
      DIMENSION IBUF(480),IA(300)
      CALL EPINIT
      DO 10  I=1,30
      NW=10*I
      CALL USER(IA,NW)
 C
 C   ROUTINE -USER- HAS STORED NW VALUES IN A TO BE WRITTEN OUT.
 C   .....
      CALL EPOUTS(20,3,NW,IA,IBUF,IERR)
      IF(IERR.NE.0)STOP 1
   10 CONTINUE
 C **********************************************************
 C
 C                 THE FOLLOWING CALL IS ESSENTIAL
 C
 C **********************************************************
      CALL EPCLOS(20,IBUF,IERR)
 C
 C    FOLLOWING CALL ONLY NECESSARY ON SOME COMPUTERS (IBM, UNIVAC)
 C
      CALL EPEND(20,IBUF,IERR)
      IF(IERR.NE.0)STOP 2
      STOP
      END
      SUBROUTINE USER(IA,NW)
      DIMENSION IA(NW)
      DO 5 J=1,NW
      IA(J)=J
  5   CONTINUE
      RETURN
      END
\end{XMP}

\newpage

\Filename{H2EPIO-Examples-read-long-block-tape-format}
\section{Read Long Block Tape Format}
This example treats data written with "long" blocks
with a maximum size of 14400 bytes. 
Note the call to \Rind{EPSETW} to specify
the buffer size correctly for a 32-bit computer.

\begin{XMP}
      PROGRAM EPEXA3
      DIMENSION IA(300),BUF(3600)
      CALL EPINIT
      CALL EPSETW(10,1,7200,IERR)
C --- LOOP UNTIL EOF, OR ANY OTHER ERROR
    1 CONTINUE
      CALL EPREAD(10,3,NW,IA,BUF,IERR)
      IF(IERR.NE.0)  GOTO 20
C --- PROCESS CONTENTS OF A
C
      CALL USER(IA,NW)
C
C --- LOOP
      GOTO 1
   20 STOP
      END
      SUBROUTINE USER(IA,NW)
      DIMENSION IA(NW)
      PRINT*,NW,(IA(I),I=1,5)
      RETURN
      END
\end{XMP}

\Filename{H2EPIO-Examples-Default-Format}
\section{Read a File in Default Format.}
Read the file of section~\ref{sec:write32bit}.
Note the the conversion routine \Rind{CFRIBM}.

\begin{XMP}
      PROGRAM EPEXA4
      DIMENSION A(300),BUF(480)
      CALL EPINIT
 C--- LOOP UNTIL EOF, OR ANY OTHER ERROR
     1 CONTINUE
      CALL EPREAD(10,3,NW,A,BUF,IERR)
      IF(IERR.NE.0)  GOTO 20
      CALL CFRIBM(A,NW,3)
 C--- PROCESS CONTENTS OF A
 C
      CALL USER(A,NW)
 C
 C--- LOOP
      GOTO 1
    20 STOP
      END
\end{XMP}

\newpage

\Filename{H2EPIO-Examples-write-long-blocks-32bits}

\section{Write a file with long blocks on a 32-bit computer}

Do the same as in the example in section~\ref{sec:write32bit},
but this time with a bigger buffer 
and convert data to IBM format for portability (\Rind{CTOIBM}).

\begin{XMP}
      PROGRAM EPEXA5
      DIMENSION IBUF(4500),A(3000)
      CALL EPINIT
 C--- INCREASE OUTPUT BUFFER = PHYSICAL BLOCK LENGTH TO 9000 16-BIT
 C    WORDS (18000 BYTES), CORRESPONDING TO 4500 32-bit words
      CALL EPSETW(20,1,9000,IERR)
      IF(IERR.NE.0)  STOP 1
      DO 10  I=1,20
      CALL USER(A,NW)
 C
 C   ROUTINE -USER- HAS STORED NW VALUES IN A TO BE WRITTEN OUT.
 C   .....
 C--- NOW CONVERT TO IBM FLOATING POINT FORMAT + WRITE
      CALL CTOIBM(A,NW,3)
      CALL EPOUTS(20,3,NW,A,IBUF,IERR)
      IF(IERR.NE.0)STOP 2
   10 CONTINUE
 C **********************************************************
 C
 C                THE FOLLOWING CALL IS ESSENTIAL
 C
 C **********************************************************
      CALL EPCLOS(20,IBUF,IERR)
      IF(IERR.NE.0)STOP 3
      STOP
      END
\end{XMP}

\newpage

\Filename{H2EPIO-Examples-read-back-long-blocks-32bits}

\section{Reading back the file from the previous example}

Read the file written in the previous example (\Lit{EPEXA5})
on a 32-bit computer, using \Rind{CFRIBM} to convert the real numbers
to the local representation.

\begin{XMP}
      PROGRAM EPEXA6
      DIMENSION A(3000),BUF(4500)
      CALL EPINIT
 C--- INCREASE INPUT BUFFER = PHYSICAL BLOCK LENGTH TO 9000 16-BIT
 C    WORDS (18000 BYTES), CORRESPONDING TO 4500 32-bit words
      CALL EPSETW(10,1,9000,IERR)
      IF(IERR.NE.0)  STOP 1
 C--- LOOP UNTIL EOF, OR ANY OTHER ERROR
     1 CONTINUE
      CALL EPREAD(10,3,NW,A,BUF,IERR)
      IF(IERR.NE.0)  GOTO 20
 C--- THE FOLLOWING CALL HAS NO EFFECT ON IBM (MAY BE LEFT
 C    IN FOR CODE COMPATIBILITY)
      CALL CFRIBM(A,NW,3)
 C--- PROCESS CONTENTS OF A
 C
      CALL USER(A,NW)
 C
 C--- LOOP
      GOTO 1
    20 STOP
      END
\end{XMP}

\newpage

\Filename{H2EPIO-Examples-16bit-Input-32bit}

\section{16-bit Input and 32-bit Output}

Read a tape on unit 10, with a maximum event size of 1000 words,
unpacked as 16-bit words. 
After some reduction results are stored
in array \Lit{IRECO} with a limit of 300 words. 
These results  are  then output to unit 20 packed 32-bits/word. 
The block size for both
units is the EP package default (1800 16-bit words). 
Note the
mandatory call to \Rind{EPCLOS} for the output unit at the end. 

\begin{XMP}
      PROGRAM EPEXA7
      DIMENSION IBUF(900),IBUFO(900),IRECI(1000),IRECO(300)
      CALL EPINIT
 C--   SET INPUT STATUS WORDS \ SETTING WORD 2 PREVENTS PROGRAM
 C--   OVERWRITING SHOULD YOU HAVE BAD DATA
      CALL EPSETW(10,2,1000,IERR)
      IF(IERR.NE.0)STOP 1
 C        THE FOLLOWING CALL IS OPTIONAL; USEFUL TO VERIFY OPTIONS
      CALL EPSTAT
 C--   READ A RECORD, UNPACK AS 16-BIT WORDS
  50  CALL EPREAD(10,2,NW,IRECI,IBUF,IERR)
      IF(IERR.NE.0)GOTO 500
      PRINT*,NW
 C--   DATA REDUCTION PART \ NWO WORDS OF DATA ARE STORED IN IRECO
 C--   NWO CAN VARY BUT THE MAXIMUM IS 300, DEFINED BY EPSETW CALL
 C--    NWO WORDS
 C--
      CALL USER(NWO,IRECO)
 C--    THE OUTPUT IS PACKED 32-BITS PER WORD
      CALL EPOUTS(20,3,NWO,IRECO,IBUFO,IERR)
      IF(IERR.NE.0)GOTO 600
      GOTO 50
 C--   READ ERRORS
 C--     PARITY AND OTHER
  500 IF(IERR.NE.1)GOTO 50
 C--     EOF
      CALL EPCLOS(20,IBUFO,IERR)
      CALL EPEND(20,IBUFO,IERR)
      STOP
 C--   WRITE ERRORS
  600 STOP 600
      END
\end{XMP}

\newpage

\Filename{H2EPIO-Examples-Read-and-Write-ASCII-Hollerith-Strings}

\section{Read and Write ASCII Hollerith Strings}
\index{ASCII string}
\index{Hollerith string}
\index{character string}

The package provides utility routines to convert to and from ASCII 8-bit
character code, which is the way we recommend to write text.
 
Since the package uses a 16-bit word as basic unit,
the user should read and write Hollerith strings as bit
strings (\Lit{MODE=1}), and pad with a
blank to an even number of characters if necessary.
 
Of the computers for which this package exists, only IBM and CDC do not
use the ASCII code internally. 
IBM uses 8-bit EBCDIC, CDC uses 6-bit
DISPLAY code. 
However, when reading and writing on VAXes, the bytes in
successive pairs of bytes have to be swapped in order to come out
correctly, since they get swapped again when reading or writing, to make
16-bit words look correct.
 
Consequently, there exist two routines, \Rind{SFRASC} and 
\Rind{STOASC} on IBM, CDC,
and VAX which perform the necessary conversion or swapping. The
following examples show how they are used.
 
The previous routines \Rind{CFRASC} and \Rind{CTOASC}, working on blown-up
strings will continue to exist.

The following example shows how to write Hollerith strings
using both the old and new routines.

\label{sec:exahollerith}

\begin{XMP}
      PROGRAM EPEXA9
      DIMENSION IBIG(1000),ISMALL(1000),IBUF(1000),STRING(22)
      DATA STRING/
      1 'ABCD','EFGH','IJKL','MNOP','QRST','UVWX','YZab','cdef',
      2 'ghij','klmn','opqr','stuv','wxyz','0123','4567','89+-',
      3 '*/()','$=,.','#[]:','"_!&',4H'?<>,'@---'/
 C
 C   NW IS THE NUMBER OF CHARACTERS
 C
      NW= 88
      CALL EPINIT
 C
 C   USE PREVIOUS ROUTINES ( BLOW, CONVERT, BUNCH )
 C
      CALL BLO8W(STRING,1,IBIG,1,NW)
      CALL CTOASC(IBIG,NW)
      CALL BUN8W(IBIG,1,ISMALL,1,NW)
 C
 C   WRITE AS-BIT STRING (NOTE NUMBER OF 16-BIT WORDS ! )
 C
      CALL EPOUTS(11,1,(NW-1)/2+1,ISMALL,IBUF,IERR)
 C
 C--- String Conversion Routines  add same string behind first, write again
 C
      CALL STOASC(STRING,1,ISMALL,NW+1,NW)
      CALL EPOUTS(11,1,NW,ISMALL,IBUF,IERR)
      CALL EPCLOS(11,IBIG,IERR)
      CALL EPEND(11,IBIG,IERR)
      STOP
      END
\end{XMP}

\newpage

\Filename{H2EPIO-Read-convert-text-strings}

\section{Read and convert text strings}

This example shows the use of \Rind{SFRASC} to read and convert the 
character strings from the file created in the example in 
section~\ref{sec:exahollerith}.

\begin{XMP}
      PROGRAM TEST
      DIMENSION IBUF(1000),IASC(1000)
      CALL EPINIT
   10 CONTINUE
      CALL EPREAD(11,1,NW,IASC,IBUF,IERR)
      IF(IERR.NE.0)  STOP
      CALL SFRASC(IASC,1,IASC,1,2*NW)
      NW=NW/2
      PRINT 2001,NW,IERR
 2001 FORMAT(/" NO. OF WORDS =",I5,"  ERROR =",I3)
      PRINT 2002,(IASC(I),I=1,NP)
 2002 FORMAT(/1X,25A4)
      GOTO 10
      END
\end{XMP}
\index{character string}\index{string!of characters}
\newpage

\Filename{H2EPIO-Examples-Read-old-format-tapes}
\section{Read old Format Tapes}

This example shows typical values to use for OLD format HP or NORD
tapes read on the CDC. Note \Lit{IBUF(512)} and the corresponding number
of 16-bit words is 1920, and the call to \Rind{EPREAD} with \Lit{MODE=2}. 
Long events up to 5000 16-bit words are expected. 
Most data acquisition systems were converted
to the new format in the early eighties.

\begin{XMP}
      PROGRAM EPTEST(TAPE10,OUTPUT)
      DIMENSION IBUF(512),IRECI(5000)
      CALL EPINIT
 C       SET STATUS WORDS FOR INPUT
      CALL EPSETW(10,1,1920,IERR)
      IF(IERR.NE.0)STOP 1
      CALL EPSETW(10,2,5000,IERR)
      IF(IERR.NE.0)STOP 2
 C    SUPPRESS BYTE SWAPPING
      CALL EPSETW(10,27,0,IERR)
      IF(IERR.NE.0)STOP 3
 C        THE FOLLOWING CALL IS OPTIONAL; USEFUL TO VERIFY OPTIONS
      CALL EPSTAT
 C       READ A RECORD, UNPACK AS 16-BIT WORDS
  50  CALL EPREAD(10,2,NW,IRECI,IBUF,IERR)
      IF(IERR.NE.0)GOTO 500
      JJ=NW-5
      PRINT*,NW,(IRECI(I),I=1,5),(IRECI(I),I=JJ,NW)
      GOTO 50
 C       READ ERRORS
 C         EOF
  500 IF(IERR.EQ.1)STOP
 C         PARITY AND OTHER
      GOTO 50
      END
\end{XMP}

\newpage

\Filename{H2EPIO-Examples-Data-Selection-User-Defined-Header}
\section{Data Selection and User Defined Header.}

This example starts by skipping to a selected block on the input
unit \Lit{10} using \Lit{MODE=30} in \Rind{EPREAD}. 
Then a further selection is
made on the logical record header (looking for \Lit{ID=0}) 
using \Lit{MODE=20}.
The data for these events are then unpacked as 16-bit words, and
only the first 100 words are written to the output unit (\Lit{20})
and packed in 32-bits. In addition a user defined header (\Lit{IH}) is
used where \Lit{IH(1-4)} are the default package header, \Lit{IH(5)} is the
original number of data words and \Lit{IH(6)} the original record
sequence number.

\begin{XMP}
      PROGRAM EPTEST
      DIMENSION IBUF(900),IBUFO(900),IRECI(1000),IH(6)
      CALL EPINIT
 C       SET STATUS WORDS FOR INPUT
      CALL EPSETW(10,2,1000,IERR)
      IF(IERR.NE.0)STOP 1
 C--    SKIP TO BLOCK 15
    5 CALL EPREAD(10,30,NW,IRECI,IBUF,IERR)
      IF(IERR.NE.0)STOP 5
      IF(IRECI(3).LT.15)GOTO 5
 C-- READ THE RECORD HEADER AND SELECT RECORDS WITH TYPE ID=0
   50 CALL EPREAD(10,20,NW,IRECI,IBUF,IERR)
      IF(IERR.NE.0)GOTO 500
      IF(IRECI(2).NE.0)GOTO 50
      IH(6)=IRECI(4)
 C-- READ DATA OF CURRENT RECORD, UNPACK AS 16-BIT WORDS
      CALL EPREAD(10,12,NW,IRECI,IBUF,IERR)
      IF(IERR.NE.0)GOTO 500
      JJ=NW-4
      PRINT*,NW,(IRECI(I),I=1,5),(IRECI(I),I=JJ,NW)
      IH(5)=NW
 C-- WRITE AS 32-BIT WORDS
      CALL EPOUTL(20,3,6,IH,100,IRECI,IBUFO,IERR)
      IF(IERR.NE.0)GOTO 600
      GOTO 50
 C       READ ERRORS
 C         PARITY AND OTHER
  500 IF(IERR.NE.1)GOTO 50
 C         EOF
      CALL EPCLOS(20,IBUFO,IERR)
      STOP
 C       WRITE ERRORS
  600 STOP 600
      END
\end{XMP}

\newpage

\Filename{H2EPIO-Examples-Writing-32bit-physical-Headers}
\section{Writing 32 bit physical Headers, logical Headers and Data.}

This example also shows the control language to run in VM/CMS.

\begin{XMP}
/* Test 32 bit headers */
'exec cernlib'
'vfort out32 (noprint '
'filedef iofile20 disk out32 data a4 ( recfm u lrecl 32400 block 32400'
'load out32 ( start '
'erase load map '
'erase out32 text a1'
/*BEGIN OUT32 FORTRAN A3 */
       DIMENSION IBUF(8100),A(50000)
       CALL EPINIT
       CALL EPSETW(20, 1,16200,IERR)
       CALL EPSETW(20, 2,100000,IERR)
       CALL EPSETW(20, 3,  32 ,IERR)
       CALL EPSETW(20,29,   1 ,IERR)
C   OUTPUT 4 long LOGICAL RECORDS
C
       DO 10  I=1,4
        NW=10000 * (I+1)
        CALL USER(A,NW,I)
C  .................................................
C     ROUTINE -USER- HAS STORED NW VALUES IN A TO BE WRITTEN OUT.
C   .........................................
        CALL EPOUTS(20,3,NW,A,IBUF,IERR)
C       IF(IERR.NE.0)STOP 1
   10  CONTINUE
C++++++++++++++++++++++++++++++++++++++++++++++++++++++++++
C
C                THE FOLLOWING CALL IS ESSENTIAL
C
C++++++++++++++++++++++++++++++++++++++++++++++++++++++++++
       CALL EPCLOS(20,IBUF,IERR)
C  .........................................................
C   FOLLOWING CALL ONLY NECESSARY ON SOME COMPUTERS (UNIVAC,IBM)
C    ON VM (AND MVS?) WE THINK THIS SHOULD BE EPRWND(20,IBUF,IERR)
C    WITH APPROPRIATE FILEDEF (IODD IN MVS)
C  .........................................................
C      CALL EPEND(20,IBUF,IERR)
       CALL EPRWND(20,IBUF,IERR)
       IF(IERR.NE.0)STOP 2
       STOP
       END
       SUBROUTINE USER(A,NW,I)
       INTEGER A(1)
       DO 10 J=1,NW
       A(J)=I
   10  CONTINUE
       RETURN
       END
\end{XMP}
