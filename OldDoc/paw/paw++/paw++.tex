\documentstyle[ifthen,calc,dingbat,optarg,11pt,epsfig,rotating]{cernman}

%%%%%%%%% Some commands for including PAW++ EPS screen dumps %%%%%%%%%%%%%%%%%%%

\newenvironment{ICON}[1]{\begin{minipage}{.1\textwidth}
                          \epsfig{file=#1.eps,width=\textwidth}
                          \end{minipage}\hfill
                          \begin{minipage}{.85\textwidth}}%
                          {\end{minipage}}
\newlength{\Mylen}
\newenvironment{PAWf}[2][.3]
               {\setlength{\Mylen}{.95\textwidth-\textwidth*\real{#1}}%
               \begin{minipage}{#1\textwidth}
               \epsfig{file=#2.eps,width=\textwidth}
               \end{minipage}\hfill
               \begin{minipage}{\Mylen}}%
               {\end{minipage}}
%\newenvironment{PAWf}[1]{\begin{minipage}{.3\textwidth}
%                          \epsfig{file=#1.eps,width=\textwidth}
%                          \end{minipage}\hfill
%                          \begin{minipage}{.65\textwidth}}%
%                          {\end{minipage}}
\newenvironment{PAWfR}[1]{\begin{minipage}{.3\textwidth}
                          \begin{turn}{-90}%
                          \mbox{\epsfig{file=#1.eps,height=\textwidth}}
                          \end{turn}
                          \end{minipage}\hfill
                          \begin{minipage}{.65\textwidth}}%
                          {\end{minipage}}
\newcommand{\MENU}[1]{\begin{center}
                      \mbox{\epsfig{file=#1.eps}}%
                      \end{center}}
\newcommand{\PAWF}[1]{\begin{center}
                      \mbox{\epsfig{file=#1.eps,width=\textwidth}}%
                      \end{center}}
\newcommand{\PAWFR}[1]{\begin{turn}{-90}%
                       \mbox{\epsfig{file=#1.eps,height=\textwidth}}%
                       \end{turn}}

\sansfont{helvetica}% Use Helvetica for sans serif font
\romanfont{times}% Use Times for roman serif font

%%%%%%%%%%%%%%%%%% Zapf dingbats enumerate environments %%%%%%%%%%%%%%%%%%%%%%%%

\newcounter{Mycount}
\newcommand{\Denslist}
{\itemsep=0pt\parsep=0pt\partopsep=0pt\topsep=\baselineskip\parskip=0pt}
\newenvironment{EnumZW}{\renewcommand{\labelenumi}
                       {\setcounter{Mycount}{191+\value{enumi}}%
                       \raisebox{-2pt}[0cm][0cm]
                       {\Large\ding{\value{Mycount}}}}%
                       \enumerate\Denslist}%
                       {\endlist}
\newenvironment{EnumZB}{\renewcommand{\labelenumi}
                      {\setcounter{Mycount}{201+\value{enumi}}%
                      \raisebox{-2pt}[0cm][0cm]{\Large\ding{\value{Mycount}}}}%
                        \enumerate\Denslist}%
                       {\endlist}

%%%%%%% Description lists using sans serif font for term %%%%%%%%%%%%%%%%%%%%%%%

\newenvironment{DLsf}[1]% The parameter is the width of the term
                        {\def\DLH{\sf}\begin{DLgen}{#1}}{\end{DLgen}}
\newenvironment{DLsfc}[1]% The parameter is the width of the term
                        {\def\DLH{\sf}\begin{DLgenc}{#1}}{\end{DLgenc}}

%%%%%%%%% Specific commands for tagging PAW++ elements %%%%%%%%%%%%%%%%%%%%%%%%%

\newcommand{\NbDW}[1]{\setcounter{Mycount}{191+#1}\ding{\value{Mycount}}}%
\newcommand{\NbDB}[1]{\setcounter{Mycount}{201+#1}\ding{\value{Mycount}}}%
\newcommand{\Button}[1]{\psboxit{rectcartouche}{\spbox{\footnotesize\sf#1}}}
\newcommand{\Field}[1]{\psboxit{box 0.9 setgray fill}
                      {\spbox{\footnotesize\sf#1}}}

%%%%%%%%%%%%%%%%%%%%%%%%%%%%%%%%%%%%%%%%%%%%%%%%%%%%%%%%%%%%%%%%%%%%%%%%%%%%%%%%
%
%      psboxes.sty
%
% This package enables to put a PostScript drawing behind a TeX box.
% The drawing is parametrized by the position and the size of the
% TeX box. To put a gray [rounded] box behind a word use
%
%       \PScommands % Once at the begining
%
%       ... text text \psboxit{25 cartouche}{THE WORD} text text
%       \psboxit{box 0.5 setgray fill}{\spbox{ANOTHER WORD}}
%       text text ...
%
% WARNINGS : * This was written for dvips translator. You may want to
% change ``ps::'' to ``pstext='' to adapt it to others.
%
%            * If your boxes are ill sized try to change 16384 to
% something else (original code used 65536).
% % 65536 is the internal unit of TeX (scaled point, TeXBook page 57)
% Those macros were adapted from Tom Sheffler (CMU)'s psframe.sty style. In
% particular, the spacebox macro was just copied from his style.
%
%       Je'ro^me MAILLOT, INRIA
%       maillot@bora.inria.fr
%       August 1991
%
%%
%% PSBOXIT
%%
%% \psboxit{PS program}{TeX stuff}
%%
%% The bounding box of the TeX stuff is pushed on the PostScript stack
%% and then the program in the first argument is called
%%
%% EXAMPLE: set some text on a gray background, Use the SPBOX macro to
%% give some space around the text.
%%
%%      \psboxit{box 0.5 setgray fill}{\spbox{Some Text}}
%%
%% See \PScommands for the \box definition
%%

\long\def\psboxit#1#2{%
\begingroup\setbox0=\hbox{#2}%
\dimen0=\ht0 \advance\dimen0 by \dp0%
    % Write out the PS code to set the current path using HEIGHT,
    % WIDTH , DEPTH of box0.
    \hbox{%
    \special{ps: gsave currentpoint translate
        0
        \number\dp0 \space 15800 div    % hand tuned for dvips
        \number\wd0 \space 15800 div    % hand tuned for dvips
        \number\ht0 \space -15800 div   % hand tuned for dvips
%        \number\dp0 \space 16384 div
%        \number\wd0 \space 16384 div
%        \number\ht0 \space -16384 div   % Bounding box
%        \number\dp0 \space 65536 div
%        \number\wd0 \space 65536 div
%        \number\ht0 \space -65536 div   % Bounding box
        #1 grestore}%
    \copy0%
}%HBOX
\endgroup%
}%

% SPACEBOX
%
% This macro simply takes some TeX stuff, and puts FOUR sides on it
% so that the box is the same size as the thing you'd get with
% an \fbox{} command.  (All I did was modify the code for \fbox{}
% so that all rules were replaced with struts).
%
% USAGE: \spbox{text} is just like \fbox{text} but makes no rules
%
% REASON: so that if using \pspath{...}{\fbox{stuff}}
%         there is a way to get another box the same size:
%         \pspath{...}{\spbox{stuff}}
%
\long\def\spbox#1{\begingroup\fboxsep=1pt\leavevmode\setbox1\hbox{#1}%
    \dimen0\fboxrule \advance\dimen0 \fboxsep%
    \advance\dimen0 \dp1%
    \hbox{\lower \dimen0\hbox%
    {\vbox{\hrule height \fboxrule width 0pt%
          \hbox{\vrule width \fboxrule height 0pt \hskip\fboxsep%
          \vbox{\vskip\fboxsep \box1\vskip\fboxsep}\hskip%
                 \fboxsep\vrule width \fboxrule height 0pt}%
                 \hrule height \fboxrule width 0pt}}}\endgroup}%
%
% A Few PostScript definitions to use with \psboxit
% Call \PScommands once at the begining of your program, this will
% define : box roundedbox rectcartouche and cartouche. They are 4
% PostScript programs. Change the values before setlinewidth end
% setgray to customize your boxes
%
%     Ex : \psboxit{25 cartouche}{blah blah}
%          \psboxit{rectcartouche}{blah blah}
%
\long\def\PScommands{\special{! TeXDict begin
/box{%                  Processes the path of a rectangle.
%                       Needs : x0 y0 x1 y1.
newpath 2 copy moveto 3 copy pop exch lineto 4 copy pop pop
lineto exch pop exch pop lineto closepath } bind def
%
/roundedbox{%           Processes the path of a rounded rectangle.
%                       Needs : x0 y0 x1 y1 radius.
%       The bounding box is augmented by +/- radius to allow easily to
%       frame several rounded boxes around the same Texture box. Ex:
%  \psboxit{4 copy 15 roundedbox 25 roundedbox} {\spbox{Some Text}}
%       draws two scaled boxes arond the same word. Delete the `radius
%       sub' and `radius add' commands to suppress that enlargement.
%
/radius exch store
3 2 roll %              x0 x1 y1 y0
2 copy min radius sub /miny exch store max radius add /maxy exch store
2 copy min radius sub /minx exch store max radius add /maxx exch store
newpath
minx radius add miny moveto
maxx miny maxx maxy radius arcto
maxx maxy minx maxy radius arcto
minx maxy minx miny radius arcto
minx miny maxx miny radius arcto 16 {pop} repeat
closepath
}bind def
%
/rectcartouche{%        Draws a filled and framed box
%                       Needs : x0 y0 x1 y1
4 copy .9 setgray 3 setlinewidth box fill .5 setgray box stroke
}bind def
%
/cartouche{%            Draws a filled and framed rounded box
%                       Needs : x0 y0 x1 y1 radius
5 copy .9 setgray 5 setlinewidth roundedbox fill .95 setgray roundedbox stroke
}bind def
%
end }%                  Closes dictionnary
}%
%
\PScommands
%
%%%%%%%%%%%%%%%%%%%%%%%%%%%%%%%%%%%%%%%%%%%%%%%%%%%%%%%%%%%%%%%%%%%%%%%%%%%%%%%%
%                                                                              %
%   HIGZ  User Guide -- LaTeX Source                                           %
%                                                                              %
%   Common tag definitions and index entries                                   %
%                                                                              %
%   Last Mod.: 2 July 1993 oc                                                  %
%                                                                              %
%%%%%%%%%%%%%%%%%%%%%%%%%%%%%%%%%%%%%%%%%%%%%%%%%%%%%%%%%%%%%%%%%%%%%%%%%%%%%%%%

\newcommand{\PATCHY}{{\sf PATCHY}\index{PATCHY}}
\newcommand{\GEANT}{{\sf GEANT}\index{GEANT}}
\newcommand{\ZEBRA}{{\sf ZEBRA}\index{ZEBRA}}
\newcommand{\RZ}{{\sf RZ}\index{RZ}}
\newcommand{\PHIGS}{{\sf PHIGS}\index{PHIGS}}
\newcommand{\GPHIGS}{{\sf GPHIGS}\index{GPHIGS}}
\newcommand{\HPLOT}{{\sf HPLOT}\index{HPLOT}}
\newcommand{\HBOOK}{{\sf HBOOK}\index{HBOOK}}
\newcommand{\COMIS}{{\sf COMIS}\index{COMIS}}
\newcommand{\KUIP}{{\sf KUIP}\index{KUIP}}
\newcommand{\HIGZ}{{\sf HIGZ}\index{HIGZ}}
\newcommand{\XPAW}{{\sf PAW}\index{PAW (Physics Analysis Workstation)}}
\newcommand{\PAWPP}{{\sf PAW++}\index{PAW++ (Physics Analysis Workstation)}}
\newcommand{\CERNLIB}{{\sf CERN Program Library}\index{CERN Program Library}}
\newcommand{\PS}{{\sf Post\-Script}\index{PostScript}}
\newcommand{\EPS}{{\sf En\-cap\-su\-la\-ted Post\-Script}\index{PostScript!Encapsulated}}
\newcommand{\GKS3D}{{\sf GKS-3D}\index{GKS!GKS-3D}}
\newcommand{\GKSGRAL}{{\sf GKS-GRAL}\index{GKS!GKS-GRAL}}
\newcommand{\DECGKS}{{\sf DEC-GKS}\index{GKS!DEC-GKS}}
\newcommand{\SUNGKS}{{\sf SUN-GKS}\index{GKS!SUN-GKS}}
\newcommand{\ATCGKS}{{\sf ATC-GKS}\index{GKS!ATC-GKS}}
\newcommand{\NOVAGKS}{{\sf NOVA-GKS}\index{GKS!NOVA-GKS}}
\newcommand{\UNIGKS}{{\sf UNI-GKS}\index{GKS!UNI-GKS}}
\newcommand{\MGKS}{{\sf MGKS}\index{GKS!MGKS}}
\newcommand{\GKS2000}{{\sf GKS2000}\index{GKS!GKS2000}}
\newcommand{\PLOT10GKS}{{\sf PLOT10-GKS}\index{GKS!PLOT10-GKS}}
\newcommand{\GKS}{{\sf GKS}\index{GKS}}
\newcommand{\CORE}{{\sf CORE}\index{CORE}}
\newcommand{\FALCO}{{\sf FALCO}\index{FALCO}}
\newcommand{\MSDOS}{{\sf MSDOS}\index{MSDOS}}
\newcommand{\MAC}{{\sf MacIntosh}\index{MacIntosh}}
\newcommand{\GDDM}{{\sf GDDM}\index{GDDM}}
\newcommand{\DI3000}{{\sf DI3000}\index{DI3000}}
\newcommand{\XW}{{\sf X Window System}\index{X Window System}}
\newcommand{\X11}{{\sf X11}\index{X11}}
\newcommand{\GL}{{\sf GL}\index{GL}}
\newcommand{\GPR}{{\sf GPR}\index{GPR}}
\newcommand{\GMR}{{\sf GMR}\index{GMR}}
\newcommand{\MOTIF}{{\sf Motif}\index{Motif}}
\newcommand{\XLIB}{{\sf Xlib}\index{Xlib}}

\newcommand{\NDC}{normalized device coordinates\index{coordinates!normalized device}}
\newcommand{\DC}{device coordinates\index{coordinates!device}}
\newcommand{\WC}{world coordinates\index{coordinates!world}}

\newcommand{\ndc}{{\sf NDC}\index{coordinates!normalized device}}
\newcommand{\dc}{{\sf DC}\index{coordinates!device}}
\newcommand{\wc}{{\sf WC}\index{coordinates!world}}

\newcommand{\FORTRAN}{{\sf Fortran}\index{Fortran}}
\newcommand{\CHARACTER}{{\tt CHARACTER}}

\newcommand{\UGP}{underlying graphics package\index{underlying graphics package}}
\newcommand{\UGPs}{underlying graphics packages\index{underlying graphics package}}

\newcommand{\TELNETG}{{\sf Telnetg}\index{Telnetg}}

\newcommand{\HW}{{\tt higz\_windows.dat}\index{higzwindows.dat}}

\newcommand{\nt}{{\sf NT}\index{normalization transformation}}
\newcommand{\wt}{{\sf WT}\index{workstation transformation}}

\newcommand{\NT}{normalization transformation\index{normalization transformation}}
\newcommand{\NTs}{normalization transformations\index{normalization transformation}}
\newcommand{\WT}{workstation transformation\index{workstation transformation}}
\newcommand{\WTs}{workstation transformations\index{workstation transformation}}
\newcommand{\ASCII}{{\tt ASCII}\index{ASCII}}

\newcommand{\UNIX}{{\sf UNIX}\index{UNIX}}
\newcommand{\VMS}{{\sf VAX/VMS}\index{VAX/VMS}}

\newcommand{\MB}{{\bf Main Browser}\index{Main Browser}}
\newcommand{\EW}{{\bf Executive Window}\index{Executive Window}}
\newcommand{\TP}{{\bf Transcript Pad}\index{Transcript Pad}}
\newcommand{\IP}{{\bf Input Pad}\index{Input Pad}}
\newcommand{\NV}{{\bf Ntuple Viewer}\index{Ntuple Viewer}}
\newcommand{\HSP}{{\bf Histogram Style Panel}\index{Histogram Style Panel}}
\newcommand{\PL}{{\bf PAW++ Locate}\index{PAW++ Locate}}
\newcommand{\GW}{{\bf Graphics Window}\index{Graphics Window}}
\newcommand{\CE}{{\bf Cut Editor}\index{Cut Editor}}


%%%%%%%%%%%%%%%%%%%%%%%%%%%%%%%%%%%%%%%%%%%%%%%%%%%%%%%%%%%%%%%%%%%%%%%%%%%%%%%%
\begin{document}

\chapter{\PAW++: A guided tour}

\PAW++{} is a new and powerful OSF/\MOTIF{} based Graphical User Interface to
the popular Physics Analysis Workstation \XPAW.  The graphical user interface
makes the full and rich command set of \XPAW{} available to even the naive
user. Simple point and click operations are enough to execute commands that
were previously accessable only to expert users.

At present it is released on Unix workstations and VAX/VMS.

\PAW++{} has, in addition to the conventional command line and macro types of
interface, the following dialogue modes:

\begin{DL}{Histogram style panel}
\item[Pull Down menus] They are useful to understand the command structure of
      the \XPAW{} system.
\item[Command panels] They give a ``panel representation'' of the commands.
\item[Object Browser] This is in many ways similar to the well-known browsers
      in the PC/MAC utilities or the visual tools on some workstations.
\item[Direct graphics] One can click in the graphics area and identify
      automatically which object has been selected. A pop-up menu appears
      with a list of possible actions on this object. For example, by clicking
      with the right mouse button on a histogram, one can make directly a
      gaussian fit, a smoothing etc.
      Pop-up menus are available by clicking on the graphics window to
      automatically produce PostScript, Encapsulated PostScript, \LaTeX{} files
      or print the picture on your local printer.
\item[\HSP] Buttons are available to change
      histogram attributes, colours, line styles, fonts, and
      axes representation.
      2-D histograms can be rotated interactively. Zooming and rebinning can
      be performed interactivaly in real time.
\item[\NV] Just click on the Ntuple column name to histogram
      the column.
\end{DL}

The new system is largely self-explanatory. Only a subset of \XPAW{} has been
converted to this new user interface, but work is currently in progress to
offer many new facilities in future releases.

On all system on which the \CERNLIB{} is installed, it is enough to
type \Lit{paw++} to enter the system.

\PAW++{} starts up with three windows on the screen:

\begin{DL}{The ``\PAW++{} \EW''}
\item[The ``\PAW++{} \EW'']
   Which is compose with a menu bar, a \TP, a current working
   directory indicator and an \IP.

\item[The ``\PAW++{} Graphics 1'']
   window displays the graphics output from \HIGZ/\X11.
   Objects, e.g. histograms, displayed in the graphics window can be
   manipulated by pointing at them, pressing the right mouse button and
   selecting an operation from the popup menu. Pointing at the edge of the
   graphics window (between displayed object and window border) brings up a
   general popup menu. Up to 4 additional graphics windows can be opened by
   selecting ``Open New Window'' from this menu.

\item[The ``\PAW++{} \MB'']
   displays all browsable classes and connected
   hbook files. Up to 4 additional browsers can be opened via the ``View'' menu
   of the ``\PAW++{} \EW'' or via the ``Clone'' button on the
   browsers. For more information on the browsers see the ``Help'' menus.
\end{DL}

\section{Overview}

\PAWFR{cnlpaw1}

\begin{UL}
\item The upper left corner is the \PAW++{} \EW, with its \IP{}
      at the bottom and the \TP{} at the top.
\item The \PAW++{} Browser, where the various entities (pictures, 1-D and
      2-D histograms and Ntuples) are all defined with their own symbol,
      is shown bottom left.
      A ``pop-up'' menu has been activated for the chosen 1-D
      histogram and the \Lit{Plot} and \Lit{Smooth} buttons have been activated.
\item The graphics window is seen top right.
      Two 2-D views (a lego-plot and a grey scale plot) and a 1-D
      view of the data points are shown.
      For the 1-D case, the results of
      a ``smoothing'' type of fit to the data points is also drawn.
      Information about the data and the fit can be found
      in the inserted window.
\item The \HSP{} at the lower right allows graphics
      attributes of the histogram to be controlled.
\end{UL}

\PAWFR{cnlpaw2}

\begin{UL}
\item The upper left corner shows the \NV.
      The left window shows the name of the various variables, characterizing
      the selected Ntuple. Other windows and press-buttons specify which
      combinations of the various variables and which events
      have to be treated (plotted, scanned, \ldots).
\item The lower left contains the \PAW++{} Browser, with this time an Ntuple
      selected.
      A ``pop-up'' menu has been activated and the \Lit{Open Ntuple Viewer}
      button activated (see above).
\item The graphics window is seen top right and shows a 3-D view
      of the combination \Lit{PX1,PY1,PZ1} of the variables, whose cuts are
      specified with the cut editor (see below).
      A three-dimensional view as well as two one-dimensional projections
      of the selected parameters are displayed.
\item The ``Cut Editor'' panel, shown at the lower right, allows
      various combinations of cuts to be specified and applied.
\end{UL}

\clearpage

\section{Executive Window}

This window allows to type commands on the keyboard like in the normal
\XPAW{} system. In fact this window is the ``kxterm'' program provide with
the \KUIP{} package.

This terminal emulator combines the best features from
the (now defunct) Apollo DM pads (like: \IP and \TP,
automatic file backup of \TP, string
search in pads, etc.) and the Korn shell emacs-style
command line editing and command line recall mechanism.

Commands are typed in the \IP{} \NbDB{1} behind the application prompt.
Via the toggle buttons \Button{H} \NbDB{4} the \IP{} and/or \TP{}
can be placed in hold mode. In hold mode one can paste or type
a number of commands into the \IP{} and edit them without sending
the commands to the application. Releasing the hold button
will causes Kxterm to submit all lines, upto the line containing the
cursor, to the application. To submit the lines below the cursor,
just move the cursor down. In this way one can still edit the
lines just before they are being submitted to the application.

Commands can be edited in the \IP{} using emacs-like key
sequences.

\PAWF{executive}

\subsection{Current working directory indicator}
Every time the current directory is changed, the {\bf Current working directory
indicator} \NbDB{3} is updated. The current working directory can be changed via
the command \Lit{CD}, by clicking on a item in the {\bf PATH window} of the
\MB{} or by clicking on a icon directory in the \MB{} itself.

\subsection{Transcript Pad}
The \TP{} \NbDB{2} shows the executed commands and command
output. When in hold mode \NbDB{4} the transcript pad does not scroll to
make the new text visible.

Mouse operations like ``Copy Paste'' are allowed in the transcript pad.
It is also possible to search a character string (see the menu bardescription).

\subsection{Input Pad}
In the \IP{} one can type, retrieve and edit command line
with the help of a Korn shell emacs-style command line editing mode.
The complete list of the editing keys is given below:

\begin{verbatim}
"C-b" means holding down the Control key and pressing the b key.
"M-" stands for the Meta key and "A-" for the Alt key.

C-b:              backward character
A-b:              backward word
M-b:              backward word
Shift A-b:        backward word, extend selection
Shift M-b:        backward word, extend selection
A-[:              backward paragraph
M-[:              backward paragraph
Shift A-[:        backward paragraph, extend selection
Shift M-[:        backward paragraph, extend selection
A-<:              beginning of file
M-<:              beginning of file
C-a:              beginning of line
Shift C-a:        beginning of line, extend selection
C-osfInsert:      copy to clipboard
Shift osfDelete:  cut to clipboard
Shift osfInsert:  paste from clipboard
Alt->:            end of file
M->:              end of file
C-e:              end of line
Shift C-e:        end of line, extend selection
C-f:              forward character
A-]:              forward paragraph
M-]:              forward paragraph
Shift A-]:        forward paragraph, extend selection
Shift M-]:        forward paragraph, extend selection
C-A-f:            forward word
C-M-f:            forward word
C-d:              kill next character
A-BS:             kill previous word
M-BS:             kill previous word
C-w:              kill region
C-y:              yank back last thing killed
C-k:              kill to end of line
C-u:              kill line
A-DEL:            kill to start of line
M-DEL:            kill to start of line
C-o:              newline and backup
C-j:              newline and indent
C-n:              get next command, in hold mode: next line
C-osfLeft:        page left
C-osfRight:       page right
C-p:              get previous command, in hold mode: previous line
C-g:              process cancel
C-l:              redraw display
C-osfDown:        next page
C-osfUp:          previous page
C-SPC:            set mark here
C-c:              send kill signal to application
C-h:              toggle hold button of pad containing input focus
F8:               re-execute last executed command
Shift F8:         put last executed command in input pad
Shift-TAB:        change input focus
\end{verbatim}

\subsection{Menu Bar}

\PAWFR{menu1}

\subsubsection{File}

\begin{PAWf}{file1}
\begin{DLsf}{Save Transcript As ...}
\item[About Kxterm...]
         Displays version information about Kxterm.
\item[About <Application>...]
         Displays version information about the application
         Kxterm is servicing.
\item[Save Transcript]
         Write the contents of the transcript pad to the current
         file. If there is no current file a file selection box
         will appear.
\item[Save Transcript As...]
         Write the contents of the transcript pad to a user-specified
         file.
\item[Print...]
         Print the contents of the transcript pad (not yet implemented).
\item[Kill]
         Send a SIGINT signal to the application to cause it to
         core dump. This is useful when the application is hanging or
         blocked. Use only in emergency situations.
\item[Exit]
         Exit Kxterm and the application.
\end{DLsf}
\end{PAWf}

\subsubsection{Edit}
\begin{PAWf}{edit1}
\begin{DLsf}{Search ...}
\item[Cut]
         Remove the selected text. The selected text is written to the
         Cut and Paste buffer. Using the ``Paste'' function, it can be
         written to any \X11 program. In the transcript pad ``Cut''
         defaults to the ``Copy'' function.
\item[Copy]
         Copy the selected text. The selected text is written to the
         Cut and Paste buffer. Using the ``Paste'' function, it can be
         written to any \X11 program.
\item[Paste]
         Insert text from the Cut and Paste buffer at the cursor location
         into the \IP.
\item[Search...]
         Search for a text string in the transcript pad.
\end{DLsf}
\end{PAWf}

\subsubsection{View}

\begin{PAWf}{view}
\begin{DLsf}{Command Panel}
\item[Show Input]
         Show in a window all commands entered via the \IP.
\item[Command Panel]
\item[Browser]
\item[Style Panel]
\end{DLsf}
\end{PAWf}

\subsubsection{Options}

\begin{PAWf}{option1}
\begin{DLsf}{Clear Transcript Pad}
\item[Clear Transcript Pad]
         Clear all text off of the top of the transcript pad.
\item[Echo Command]
         Echo executed commands in transcript pad.
\item[Timing]
         Report command execution time (real and CPU time).
\item[Iconify]
         Iconify Kxterm and all windows of the application.

\end{DLsf}
\end{PAWf}

\subsubsection{Help}
\begin{DLsf}{On Edit Keys}
\item[On Kxterm]
         The help you are currently reading.
\item[On Edit Keys]
         Help on the emacs-style edit key sequences.
\end{DLsf}

\newpage

\section{Main Browser}

The \KUIP/\MOTIF{} Browser interface is a general tool to display and
manipulate a tree structure of objects which are defined either by \KUIP{}
itself (commands, files, macros, etc.) or by the application.

The ``Clone'' button at the bottom creates a new independent browser window.
The ``Exit'' button destroys the browser window. The \MB{} cannot be
destroyed (only iconized).

The middle part of the browser is divided into two windows:

\begin{enumerate}
\item The left hand ``class window'' shows the list of all currently connected
   classes of objects.  Some classes, e.g. the command tree and the file
   system, are predefined.  Other classes allow to attach new files using the
   commands in the ``File'' menu.  Clicking with the left mouse button on
   one of
   the items in the class window displays its content in the other window.
   Pressing the right mouse button inside the class window shows a popup menu
   of possible operations, e.g. creating a new object in the current
   directory.

\item The right hand ``object window'' shows the content of the currently
   selected class directory.  The ``View'' menu allows the change the way
   objects are displayed, i.e. to choose the icon size and the amount of
   information shown for each object.  Objects are selected by clicking on
   them with the left mouse button.  Pressing the right mouse button pops up a
   menu of possible operations depending on the object type.
\end{enumerate}

   An item in a popup menu is selected by pointing at the corresponding line
   and releasing the right mouse button.  Double clicking with the left mouse
   button is equivalent to selecting the first menu item.

   Each menu item executes a command sequence where the name of the selected
   object is filled into the appropriate place.  By default the command is
   executed immediately whenever possible. The commands executed can be seen
   by selecting ``Echo Commands'' in the ``Options'' menu of the \EW.
   In case some mandatory parameters are missing a panel is displayed
   where the remaining arguments have to be filled in.  The command is
   executed then by pressing the ``OK'' or ``Execute'' button in that panel.
  (If it is not the last one in the sequence of commands bound to the menu item
 the application is blocked until the ``OK'' or ``Cancel'' button is pressed.)

   The immediate command execution can be inhibited by holding down the
   CTRL-key BEFORE pressing the right mouse button.  Some popup menus also
   contain different menu item for immediate and delayed execution, e.g.
   ``Execute'' and ``Execute...'' for class ``Commands''

   The path of the currently selected directory is always displayed below the
   menu bar.  The directory can be changed by pointing at the tail of the
   wanted subpath and clicking the left mouse button.  Clicking a second time
   on the same path segment performs the directory change and updates the
   object window.  To go downwards in the directory hierarchy double click on
   the subdirectory displayed in the object window.

\PAWF{browser}

\begin{EnumZB}
\item Current PATH (``PATH window'').
\item Class window.
\item Name of file currently selected in the class window.
\item Name of the object currently selected in the object window.
\item Number and type of object currenlty in the the object window.
\item Object window.
\end{EnumZB}

\begin{EnumZW}
\item File menu.
\item View menu.
\item Options menu.
\item Commands menu.
\item Help menu.
\item Clone button.
\item Exit button.
\end{EnumZW}

\subsection{The objects in the ``object window''}

\subsubsection{\HBOOK{} files}
\begin{ICON}{hbookicon}
Double click with the left mouse button on this icon, open the corresponfing
\HBOOK{} file with the command \Lit{HISTOGRAM/FILE}.
\end{ICON}

Select a {\bf \HBOOK{} files} icon with the left mouse button and press
the right mouse button to obtain the following menu:

\MENU{hbookmenu}

\begin{DLsf}{MMMMMMMMMMMMMMMMMM}
\item[Open]              Open the highlighted \HBOOK{} file in read-only mode.
\item[Open Update Mode]  Open the highlighted \HBOOK{} file in update mode.
\end{DLsf}

Note that the \HBOOK{} file name is displayed in the menu title.


\subsubsection{1D histograms}
\begin{ICON}{1dicon}
Double click with the left mouse button on this icon, produce the plot of the
corresponding histogram with the command  \Lit{HISTOGRAM/PLOT}. The histogram
becomes the current histogram for the \HSP.
\end{ICON}

Select a {\bf 1D histograms} icon with the left mouse button and press
the right mouse button to obtain the following menu:

\MENU{1dmenu}

\begin{DLsf}{MMMMMMMMMMMMMMMMMM}
\item[Plot]         Plot the corresponding histogram (default action). The
                    histogram becomes the current histogram for the \HSP.
\item[Fit...]       Perform the command \Lit{Histo/Fit} on the corresponding
                    histogram. The command panel is automatically displayed
\item[Fit Gauss]    Perfom a gaussian fit on the corresponding histogram.
\item[Fit Exp]      Perform an exponential fit on the corresponding histogram.
\item[Fit Const]    Perform a \Lit{P0} fit on the corresponding histogram.
\item[Fit Linear]   Perform a \Lit{P1} fit on the corresponding histogram.
\item[Smooth]       Smooth the corresponding histogram.
\item[Smooth...]    Perform the command \Lit{Smooth} on the corresponding
                    histogram. The command panel is automatically invoked.
\item[Copy ]        Copy corresponding histogram onto an other histogram.
                    The command panel is automatically invoked.
\item[Reset ]       Reset the corresponding histogram.
\item[Delete]       Delete the corresponding histogram.
\end{DLsf}

Note that the histogram identifier is displayed in the menu title.


\subsubsection{2D histograms}
\begin{ICON}{2dicon}
Double click with the left mouse button on this icon, produce the plot of the
corresponding histogram with the command  \Lit{HISTOGRAM/PLOT}. The histogram
becomes the current histogram for the \HSP.
\end{ICON}

Select a {\bf 2D histograms} icon with the left mouse button and press
the right mouse button to obtain the following menu:

\MENU{2dmenu}

\begin{DLsf}{MMMMMMMMMMMMMMMMMM}
\item[Plot]         Plot the corresponding histogram (default action). The 
                    histogram becomes the current histogram for the \HSP.
\item[Project X]    Generate the \Lit{X} projection, perform the projection
                    and plot the result (commands \Lit{ProX}, \Lit{Hi/Proj},
                    and \Lit{Hi/Plot}).
\item[Project Y]    Generate the \Lit{Y} projection, perform the projection
                    and plot the result (commands \Lit{ProY}, \Lit{Hi/Proj},
                    and \Lit{Hi/Plot}).
\item[Slice X]      Generate the \Lit{X} slices, perform the projection
                    and plot the first slice (commands \Lit{SliX},
                    \Lit{Hi/Proj}, and \Lit{Hi/Plot}).
\item[Slice Y]      Generate the \Lit{Y} slices, perform the projection
                    and plot the first slice (commands \Lit{SliY},
                    \Lit{Hi/Proj}, and \Lit{Hi/Plot}).
\item[Band X]       Generate the \Lit{X} bands, perform the projection
                    and plot the first band (commands \Lit{BanX},
                    \Lit{Hi/Proj}, and \Lit{Hi/Plot}).
\item[Band Y]       Generate the \Lit{Y} bands, perform the projection
                    and plot the first band (commands \Lit{BanY},
                    \Lit{Hi/Proj}, and \Lit{Hi/Plot}).
\item[Smooth]       Smooth the corresponding histogram.
\item[Smooth...]    Perform the command \Lit{Smooth} on the corresponding
                    histogram. The command panel is automatically invoked.
\item[Copy ]        Copy corresponding histogram onto an other histogram.
                    The command panel is automatically invoked.
\item[Reset ]       Reset the corresponding histogram.
\item[Delete]       Delete the corresponding histogram.
\end{DLsf}

Note that the histogram identifier is displayed in the menu title.


\subsubsection{Ntuples}
\begin{ICON}{nticon}
Double click with the left mouse button on this icon, open the \NV{} on the
corresponding Ntuple.
\end{ICON}

Select a {\bf Ntuples} icon with the left mouse button and press
the right mouse button to obtain the following menu:

\MENU{ntmenu}

\begin{DLsf}{MMMMMMMMMMMMMMMMMM}
\item[Open Ntuple Viewer]    Open Ntuple Viewer on the highlighted Ntuple.
\item[Project...]            Project the highlighted Ntuple. The Command
                             panel \Lit{Ntuple/Proj} is automatically invoked.
\item[Print]                 Print the highlighted Ntuple (Command
                             \Lit{Ntuple/Print)}.
\end{DLsf}

Note that the ntuple identifier is displayed in the menu title.


\subsubsection{\XPAW{} commands}
\begin{ICON}{cmdicon}
Double click with the left mouse button on this icon, execute the corresponding
\XPAW{} command.
\end{ICON}

Select a {\bf \XPAW{} commands} icon with the left mouse button and press
the right mouse button to obtain the following menu:

\MENU{cmdmenu}

\begin{DLsf}{MMMMMMMMMMMMMMMMMM}
\item[Execute]      Execute the command with the default parameters. If
                    a mandatory parameter is missing, the command panel
                    is automatically invoked.
\item[Execute...]   Display the command panel.
\item[Help]         Display the help on the command.
\item[Usage]        Display the command usge in the \TP{} of the \EW.
\item[Manual]       Equivalent to \Lit{HELP}.
\item[Set Command]  This command becomes the one executed when a directive
                    typed on the keyboard is not an existing \XPAW{} command.
\item[Deactivate]   The command is deactivated.
\end{DLsf}

Note that the command name is displayed in the menu title.


\subsubsection{Deactivated \XPAW{} commands}
\begin{ICON}{dcmdicon}
Double click with the left mouse button on this icon, execute the help on
corresponding \XPAW{} command.
\end{ICON}

Select a {\bf Deactivated \XPAW{} commands} icon with the left mouse button
and press the right mouse button to obtain the following menu:

\MENU{dcmdmenu}

\begin{DLsf}{MMMMMMMMMMMMMMMMMM}
\item[Help]         Display the help on the command.
\item[Activate]     The command is activated.
\end{DLsf}

Note that the deactivated command name is displayed in the menu title.


\subsubsection{Up}
\begin{ICON}{upicon}
Double click with the left mouse button on this icon, allow to go one level up
in the directory tree. This icon is alway the first one of the {\bf content
window}.
\end{ICON}

Select a {\bf Up} icon with the left mouse button and press
the right mouse button to obtain the following menu:

\MENU{upmenu}

\begin{DLsf}{MMMMMMMMMMMMMMMMMM}
\item[List]         Allow to go one level up in the directory tree.
\end{DLsf}


\subsubsection{Directory}
\begin{ICON}{diricon}
Double click with the left mouse button on this icon, change the current
working directory.
\end{ICON}

Select a {\bf Directory} icon with the left mouse button and press
the right mouse button to obtain the following menu:

\MENU{dirmenu}

\begin{DLsf}{MMMMMMMMMMMMMMMMMM}
\item[List] Change the current working directory.
\end{DLsf}


\subsubsection{Read-Write files}
\begin{ICON}{rwicon}
Double click with the left mouse button on this icon, invoke the editor on the
corresponding file.
\end{ICON}

Select a {\bf Read-Write files} icon with the left mouse button and press
the right mouse button to obtain the following menu:

\MENU{rwmenu}

\begin{DLsf}{MMMMMMMMMMMMMMMMMM}
\item[Edit]         Edit the file.
\item[View]         Read the file.
\item[Delete]       Delete the file.
\end{DLsf}

Note that the file name is displayed in the menu title.


\subsubsection{Read-only files}
\begin{ICON}{roicon}
Double click with the left mouse button on this icon, invoke the editor in view
mode on the corresponding file.
\end{ICON}

Select a {\bf Read-only files} icon with the left mouse button and press
the right mouse button to obtain the following menu:

\MENU{romenu}

\begin{DLsf}{MMMMMMMMMMMMMMMMMM}
\item[View]         Read the file.
\item[Delete]       Delete the file.
\end{DLsf}

Note that the file name is displayed in the menu title.


\subsubsection{No-access files}
\begin{ICON}{naicon}
Double click with the left mouse button on this icon, invoke the shell command
\Lit{chmod} on the corresponding file.
\end{ICON}

Select a {\bf No-access files} icon with the left mouse button and press
the right mouse button to obtain the following menu:

\MENU{namenu}

\begin{DLsf}{MMMMMMMMMMMMMMMMMM}
\item[Chmod]         Try to change the permissions of the file.
\end{DLsf}

Note that the file name is displayed in the menu title.


\subsubsection{Executable files}
\begin{ICON}{xicon}
Double click with the left mouse button on this icon, invoke the command
\Lit{SHELL} on the corresponding file.
\end{ICON}

Select a {\bf Executable files} icon with the left mouse button and press
the right mouse button to obtain the following menu:

\MENU{xmenu}

\begin{DLsf}{MMMMMMMMMMMMMMMMMM}
\item[Execute]         Invoke the command \Lit{SHELL} on the file.
\item[Execute...]      Open the command panel \Lit{SHELL} with the file name.
\item[Edit]            Edit the file.
\item[View]            Read the file.
\item[Delete]          Delete the file.
\end{DLsf}

Note that the file name is displayed in the menu title.


\subsubsection{\XPAW{} Macros}
\begin{ICON}{macroicon}
Double click with the left mouse button on this icon, execute the
corresponding macro.
\end{ICON}

Select a {\bf \XPAW{} Macros} icon with the left mouse button and press
the right mouse button to obtain the following menu:

\MENU{macromenu}

\begin{DLsf}{MMMMMMMMMMMMMMMMMM}
\item[Exec]            Execute the macro.
\item[Exec...]         Open the command panel \Lit{EXEC} with the macro name.
                       It is useful to give parameters to the macro.
\item[Edit]            Edit the macro.
\item[View]            Read the macro.
\item[Delete]          Delete the macro.
\end{DLsf}

Note that the macro name is displayed in the menu title.


\subsubsection{Pictures}
\begin{ICON}{picticon}
Double click with the left mouse button on this icon, plot the corresponding
picture.
\end{ICON}

Select a {\bf Pictures} icon with the left mouse button and press
the right mouse button to obtain the following menu:

\MENU{pictmenu}

\begin{DLsf}{MMMMMMMMMMMMMMMMMM}
\item[Plot]             Plot the highlighted picture.
\item[Do PostScript]    Produce the \PS{} file \Lit{PNAME.ps}, where
                        \Lit{PNAME} is the name of the highlighted picture.
\item[Create]           Create a new picture. The command panel
                        \Lit{Picture/Create} is automatically invoked.
\item[Rename]           Rename the highlighted picture. The command panel
                        \Lit{Picture/Rename} is automatically invoked.
\item[Delete]           Rename the highlighted picture.
\end{DLsf}


\subsubsection{Chains}
\begin{ICON}{chainicon}
Double click with the left mouse button on this icon, allow to go one level
deeper in the chain tree.
\end{ICON}

Select a {\bf Chains} icon with the left mouse button and press
the right mouse button to obtain the following menu:

\MENU{chainmenu}

\begin{DLsf}{MMMMMMMMMMMMMMMMMM}
\item[List]                      ...................
\item[Show Tree]                 ...................
\item[Delete Chain]              ...................
\end{DLsf}


\subsubsection{Last chain level}
\begin{ICON}{chainlicon}
Last chain element.
\end{ICON}

Select a {\bf Last chain level} icon with the left mouse button and press
the right mouse button to obtain the following menu:

\MENU{chainlmenu}

\begin{DLsf}{MMMMMMMMMMMMMMMMMM}
\item[List]                      ...................
\item[Delete Chain Entry]        ...................
\end{DLsf}


\subsubsection{\ZEBRA{} Stores}
\begin{ICON}{storeicon}
Double click with the left mouse button on this icon, allow to go inside the
corresponding \ZEBRA{} store.
\end{ICON}

Select a {\bf \ZEBRA{} Stores} icon with the left mouse button and press
the right mouse button to obtain the following menu:

\MENU{storemenu}

\begin{DLsf}{MMMMMMMMMMMMMMMMMM}
\item[List]                      ...................
\item[Show store DZSTOR]         ...................
\end{DLsf}


\subsubsection{\ZEBRA{} Divisions}
\begin{ICON}{divicon}
Double click with the left mouse button on this icon, allow to go inside the
corresponding \ZEBRA{} division.
\end{ICON}

Select a {\bf \ZEBRA{} Divisions} icon with the left mouse button and press
the right mouse button to obtain the following menu:

\MENU{divmenu}

\begin{DLsf}{MMMMMMMMMMMMMMMMMM}
\item[List]                      ...................
\item[Display division]          ...................
\item[Snap division]             ...................
\item[Verify division]           ...................
\item[Collect garbage]           ...................
\item[Set filter for banks]      ...................
\end{DLsf}



\subsubsection{\ZEBRA{} Banks}
\begin{ICON}{bankicon}
Double click with the left mouse button on this icon, draw the bank tree from
the corresponding \ZEBRA{} bank.
\end{ICON}

Select a {\bf \ZEBRA{} Banks} icon with the left mouse button and press
the right mouse button to obtain the following menu:

\MENU{bankmenu}

\begin{DLsf}{MMMMMMMMMMMMMMMMMM}
\item[Display bank tree]         ...................
\item[Show cont documentd]       ...................
\item[DZ Show contents]          ...................
\item[Show system words]         ...................
\item[Survey bank tree]          ...................
\item[Put into vector]           ...................
\item[Show documentation]        ...................
\item[Edit documentation]        ...................
\item[Modify data words]         ...................
\item[Drop bank (tree)]          ...................
\end{DLsf}


\subsubsection{\RZ{} Files}
\begin{ICON}{rzicon}
Double click with the left mouse button on this icon, allow to go inside the
corresponding \ZEBRA/\RZ{} file.
\end{ICON}

Select a {\bf \RZ{} Files} icon with the left mouse button and press
the right mouse button to obtain the following menu:

\MENU{rzmenu}

\begin{DLsf}{MMMMMMMMMMMMMMMMMM}
\item[List]                      ...................
\item[Close Rzfile]              ...................
\item[Show status]               ...................
\end{DLsf}


\subsubsection{\RZ{} Directories}
\begin{ICON}{rzdiricon}
Double click with the left mouse button on this icon, allow to go inside the
corresponding \ZEBRA/\RZ{} directory.
\end{ICON}

Select a {\bf \RZ{} Directories} icon with the left mouse button and press
the right mouse button to obtain the following menu:

\MENU{rzdirmenu}

\begin{DLsf}{MMMMMMMMMMMMMMMMMM}
\item[List]                      ...................
\item[List directory (RZLDIR)]   ...................
\item[Show key definition]       ...................
\item[Set filter on keys]        ...................
\end{DLsf}


\subsubsection{\RZ{} Keys}
\begin{ICON}{keyicon}
Double click with the left mouse button on this icon, allow to read into memory
the corresponding \ZEBRA/\RZ{} key.
\end{ICON}

Select a {\bf \RZ{} Keys} icon with the left mouse button and press
the right mouse button to obtain the following menu:

\MENU{keymenu}

\begin{DLsf}{MMMMMMMMMMMMMMMMMM}
\item[Read key into memory]      ...................
\item[Show key definition]       ...................
\item[Show key words]            ...................
\item[Set filter on keys]        ...................
\end{DLsf}


\subsection{Menu Bar}

\PAWF{browsermenubar}

\subsubsection{File}

\begin{PAWf}{browsermenubarfile}
\end{PAWf}

\subsubsection{View}

\begin{PAWf}{browsermenubarview}
\end{PAWf}

\subsubsection{Options}

\begin{PAWf}{browsermenubaroptions}
\end{PAWf}

\subsubsection{Commands}

\begin{PAWf}{browsermenubarcommands}
\end{PAWf}

\subsubsection{Help}

\subsection{Information Windows}

\subsubsection{Top}

\PAWF{browserinfotop}

\subsubsection{Bottom}

\PAWF{browserinfobottom}

\subsection{Class Window}

\subsection{Content Window}

\subsubsection{Commands}

\PAWF{contentcommands}

\subsubsection{Files}

\PAWF{contentfiles}

\subsubsection{Macro}

\PAWF{contentmacro}

\subsubsection{Zebra}

\PAWF{contentzebra}

\subsubsection{Hbook}

\PAWF{contenthbook}

\subsubsection{Chains}

\subsubsection{PAWC}

\PAWF{contentpawc}

\subsubsection{Hbook Files (//LUNn)}

\PAWF{contentlun}

\newpage

\section{Graphics}

\PAWF{graphics}
\newpage

\newpage

\subsection{Graphics Window}

\begin{PAWf}[.4]{graphicswindow}
\end{PAWf}


\subsection{Ntuple}

\begin{PAWf}{graphicsntuple}
\end{PAWf}

\newpage

\subsection{1d-Histogram}

\begin{PAWf}{graphics1d}
\end{PAWf}

\newpage

\subsection{2d-Histogram}

\begin{PAWf}[.35]{graphics2d}
\end{PAWf}

\clearpage

\subsection{X Axis}

\begin{PAWf}[.4]{graphicsxaxis}
\end{PAWf}


\subsection{Y Axis}

\begin{PAWf}[.4]{graphicsyaxis}
\end{PAWf}

\clearpage

\subsection{Locate on Histograms}
To retrieve interactively on the graphics window an histogram identifier
a bin number, a \Lit{(X,Y)} position etc... , place the mouse cursor on the
graphics area and click with the left mouse button on the interesting region.

\PAWF{locate}

\clearpage

\subsection{Locate on Ntuples}

\PAWF{ntuplelocate}

\clearpage

\subsection{Integrate Histograms}
To integrate interactively an histogram, place the mouse cursor on the
bin from which the integration will start, and drag the cursor with the
left pressed to the last bin. The result will appears in real time in a
separated window called {\sl PAW++ Locate} \NbDB{2}.

\PAWF{integral}
\begin{EnumZB}
\item Integrated area.
\item Output window. It is possible to copy (via the mouse) the
      text inside this window.
\item To release the Output window.
\end{EnumZB}
\begin{EnumZW}
\item Histogram identifier.
\item First bin for the integration.
\item Last bin for the integration.
\item Value of the integral.
\item Normalized integral.
\item ``Mathematical'' integral. Each bin contribution is
      multiply by the bin witdh.
\end{EnumZW}

\clearpage

\section{\HSP}
The \HSP{} allows to manipulate and present histograms. It works on 
one histogram only: the ``Current histogram''. To set the current histogram
it is enough to plot it for the \MB, via a double click on the icon.
\PAWF{histo}
\begin{EnumZB}
\item Plot the current histogram.
\item Add informations on the plots.
\item Define the graphical option used to plot the current histogram.
\item Reset the default attributes.
\item Define the coordinate system used to draw lego and surface plots.
\item Define attibutes used to draw the current histogram.
\item Close the \HSP.
\end{EnumZB}
\begin{EnumZW}
\item File menu.
\item Options menu.
\item Current style name.
\item Current histogram name and type.
\end{EnumZW}

\newpage

\subsection{Menu Bar}
\PAWF{histomenubar}


\subsection{File}

\begin{PAWf}{histofile}
\end{PAWf}


\subsection{Options}

\begin{PAWf}{histooptions}
\end{PAWf}


\subsection{Plot Info}
This set of toggle buttons allow to add some usefull information on the
curren plot. If the Automatic refresh mode is on, the plot is automatically
refresh.

\begin{PAWf}{info}
\begin{DLsf}{Statistics...}
\item[Statistics...]  Allow to draw (or not) the statistics on the plot
                      (\XPAW{} command \Lit{OPTION STAT}). When the toggle
                      button is set on, a panel is displayed in order to
                      specify with parameters will be visible.
\item[Fits...]        Allow to draw (or not) the fit parameters on the plot
                      (\XPAW{} command \Lit{OPTION FIT}). When the toggle
                      button is set on, a panel is displayed in order to
                      specify with parameters will be visible.
\item[File Name...]   Allow to draw (or not) the file name on the plot
                      (\XPAW{} command \Lit{OPTION FILE}).When the toggle
                      button is set on, a panel is displayed in order to
                      specify the file name position.
\item[Date...]        Allow to draw (or not) the date on the plot
                      (\XPAW{} command \Lit{OPTION DATE}).When the toggle
                      button is set on, a panel is displayed in order to
                      specify the date position
\end{DLsf}
\end{PAWf}

\newpage

\subsubsection{Statistics ...}
This panel in the equivalent of the \XPAW{} command \Lit{SET STAT}. It
allows to specify which statistics informations  are displayed on the plot.

\begin{PAWf}{statistics}
\begin{DLsf}{Histogram ID}
\item[Histogram ID] The histogram identifier is displayed.
\item[Entries]      The number of entries is displayed.
\item[Mean value]   The mean value is displayed.
\item[R.M.S.]       The R.M.S. is displayed.
\item[Underflows]   The underflows are displayed.
\item[Overflows]    The overflows are displayed.
\item[All channels] The content of the total number of channel is displayed.
\end{DLsf}
\end{PAWf}


\subsubsection{Fits ...}
This panel in the equivalent of the \XPAW{} command \Lit{SET FIT}. It
allows to specify which fit parameters are displayed on the plot.

\begin{PAWf}{fits}
\begin{DLsf}{Chi Square}
\item[Chi Square]  The chi square is displayed.
\item[Errors]      The errors are displayed.
\item[Parameters]  The fit parameters are displayed.
\end{DLsf}
\end{PAWf}

\newpage

\subsubsection{File Name ...}
This panel in the equivalent of the \XPAW{} command \Lit{SET FILE}. It
allows to specify the file name position on the plot.

\begin{PAWf}{file}
\begin{DLsf}{Bottom Right}
\item[Top Left]     The file name is drawn on the top left of the plot
                    (default).
\item[Top Right]    The file name is drawn on the top right of the plot
\item[Bottom Left]  The file name is drawn on the bottom left of the plot
\item[Bottom Right] The file name is drawn on the bottom left of the plot
\end{DLsf}
\end{PAWf}


\subsubsection{Date ...}
This panel in the equivalent of the \XPAW{} command \Lit{SET DATE}. It
allows to specify the date position on the plot.

\begin{PAWf}{date}
\begin{DLsf}{Bottom Right}
\item[Top Left]     The date is drawn on the top left of the plot
\item[Top Right]    The name is drawn on the top right of the plot
                    (default).
\item[Bottom Left]  The date is drawn on the bottom left of the plot
\item[Bottom Right] The date is drawn on the bottom left of the plot
\end{DLsf}
\end{PAWf}

\newpage

\subsection{Style}

\PAWF{style}
\begin{DLsf}{General Attributes...}
\item[Object Attributes...]  Invoke the ``Object Attributes'' panel.
\item[Viewing Angles...]     Invoke the ``Viewing Angles'' panel.
\item[Axis Scaling...]       Invoke the ``Axis Scaling'' panel.
\item[General Attributes...] Invoke the ``General Attributes'' panel.
\item[Geometry...]           Invoke the ``Geometry'' panel.
\item[Axis Settings...]      Invoke the ``Axis Settings'' panel.
\item[Zones...]              Invoke the ``Zones'' panel.
\item[Font...]               Invoke the ``Font'' panel.
\end{DLsf}


\subsubsection{General Attributes}

\PAWF{generalattributes}


\subsubsection{Object Attributes}

\begin{PAWf}[.5]{objectattributes}
\end{PAWf}


\subsubsection{Viewing Angles}

\begin{PAWf}{angle}
\end{PAWf}


\subsubsection{Geometry}

\begin{PAWf}[.5]{geometry}
\end{PAWf}


\subsubsection{Zones}

\begin{PAWf}{zone}
\end{PAWf}


\subsubsection{Axis Scaling}
\PAWF{scaling}


\subsubsection{Axis Settings}


\subsubsection{Font}


\subsection{Plot Options}

\begin{PAWf}{plotoptions1d}
\end{PAWf}

\begin{PAWf}{plotoptions2d}
\end{PAWf}

\newpage

\section{Ntuple Viewer}
\PAWF{ntuple}

\newpage

\begin{EnumZW}
\item Field showing the current directory and the name of the Ntuple.
\item The names of the variables defined for the Ntuple.
      If you double click on one of the variable names
      a histogram showing the values of the variable will be plotted.
\item The \Field{X}, \Field{Y} and \Field{Z} fields allow
      you to define which variables will be used
      by the \Button{Plot} and \Button{Scan} buttons.
      These fields can be filled in two ways:
      firstly by typing the name or an expression of a variable;
      secondly by double-clicking in one of the \Field{X}, \Field{Y} or
      \Field{Z} fields. In the latter
      case the field pointed at is filled with the variable highligted in
      the list of variables.
\item Defines the first row used in the Ntuple when the \Button{Plot}
      or \Button{Project} buttons are pressed.
\item Defines the number of rows used (starting from \Field{First Row}) when the
      \Button{Plot} or \Button{Project} buttons are pressed.
\item Defines the histogram identifier used when the \Button{Plot}
      or \Button{Project} buttons are pressed.
\item Fields showing the number of rows and columns in the Ntuple.
\end{EnumZW}

\begin{EnumZB}
\item Close the \NV.
\item Invoke the cut editor (see ...).
\item Produce a plot using all the indications specified on the \NV{} panel.
\item Call the Ntuple Scanner (see ...).
\item A toggle button allowing you to enable/disable the cuts defined with the
      cut editor.
\item A toggle button, which, when pressed will produce the next plot
      on top of an already existing one, i.e. without clearing the graphics
      window.
\item Project the selected variables in the histogram specify in \NbDW{6}.
\item Help on the \NV.
\end{EnumZB}

\newpage

\subsection{Cut Editor}

\PAWF{cut}

\subsubsection{File}

\begin{PAWf}{cutfile}
\end{PAWf}

\subsubsection{Edit}

\begin{PAWf}{cutedit}
\end{PAWf}

\subsubsection{Options}

\begin{PAWf}{cutoptions}
\end{PAWf}

\subsection{Ntuple Scanner}

\PAWF{scanner}


\section{\KUIP/\MOTIF{} Panel Interface}

   The PANEL Interface allows to define command sequences which are executed
   when the corresponding button is pressed (like \Lit{STYLE GP} in \XPAW/\X11).
   The command sequence

\begin{verbatim}
      PANEL 0
      PANEL 4.06 'some string'
      PANEL 0 D 'This is my first panel' 500x300+500+600
\end{verbatim}

   creates a panel with 4 rows and 6 columns of buttons.  The text 'some
   string' should be long enough to fit the longest command Sequence which
   should be put onto one of the buttons.  The 'PANEL 0 D' command defines the
   title and the window size and coordinates in the form WxH+X+Y.

   The panels can be edited interactively:

   - Clicking with the right mouse button on an empty panel button the user
   will be asked to give a definition to this button.

   - Clicking with the left mouse button on a panel button removes its
   definition.

   The PANEL commands needed to recreate a panel can be saved into a macro
file by pressing the ``Save Panel'' button.  Panels can be reloaded either by
executing the command 'PANEL 0 D' or by pressing the ``Command Panel'' button
   in the ``View'' menu of the \EW{} and entering the corresponding
   file name.



\appendix
\chapter{X Window resources}
\section{X resources for \PAW++}

   This is a list of the X resources available to \PAW++.  Resources control
   the appearance and behavior of an application.

   Users can specify their own values for these resources in the standard
   \X11/\MOTIF{} way (via their own .Xdefaults file or the system wide
   /usr/lib/X11/app-defaults/Paw++ file).

   Any default values specified by \PAW++{} are given behind the resource name.

\begin{verbatim}
    Paw++*background:
\end{verbatim}

   Specify the background color for all windows.

\begin{verbatim}
    Paw++*foreground:
\end{verbatim}

   Specify the foreground color for all windows.

\begin{verbatim}
    Paw++*kxtermGeometry:                 550x550+5+10
\end{verbatim}

Geometry of Kxterm, the \KUIP{} terminal emulator (\PAW++{} \EW).

\begin{verbatim}
    Paw++*kuipGraphics_shell.geometry:    550x550+585+10
\end{verbatim}

   Geometry of the Graphics Window(s) (if any).

\begin{verbatim}
    Paw++*kuipBrowser_shell.geometry:     495x511+161+481
\end{verbatim}

   Geometry of the Browser(s).

\begin{verbatim}
    Paw++*histoStyle_shell.geometry:      599x360+668+631
\end{verbatim}

   Geometry of the Style Panel.

\begin{verbatim}
    Paw++*ntupleBrowser_shell.geometry:
\end{verbatim}

   Geometry of the \NV.

\begin{verbatim}
    Paw++*XmText*fontList:           *-prestige-medium-r-normal-*-120-*
    Paw++*XmTextField*fontList:      *-prestige-medium-r-normal-*-120-*
\end{verbatim}

   Font used by all text areas.

\begin{verbatim}
    Paw++*kxtermFont:                *-prestige-medium-r-normal-*-120-*
\end{verbatim}

   Font used by Kxterm (\PAW++{} \EW)

\begin{verbatim}
    Paw++*dirlist*fontList:          *-courier-bold-r-normal*-120-*
\end{verbatim}

   Font used for the icon labels in the browser.

\begin{verbatim}
    Paw++*matrix.fontList:           *-courier-medium-r-normal*-120-*
\end{verbatim}

   Font used for the Ntuple/Scan matrix (accessible via the \NV).

\begin{verbatim}
    Paw++*helpFont:                  *-courier-bold-r-normal*-120-*
\end{verbatim}

   Font used for help windows.

\begin{verbatim}
    Paw++*fontList:                  *-swiss*742-bold-r-normal-*-120-*
\end{verbatim}

   Font for the menus, messages and boxes.

\begin{verbatim}
    Paw++*keyboardFocusPolicy:       pointer
\end{verbatim}

   If ``explicit'' focus is determined by a mouse or keyboard command.  If
   ``pointer'' (default), focus is determined by the mouse pointer position.

\begin{verbatim}
    Paw++*doubleClickInterval:       400
\end{verbatim}

   The time span (in milliseconds) within which two button clicks must occur
   to be considered a double click rather than two single clicks.

\begin{verbatim}
    Paw++*dirlist*background:
\end{verbatim}

   Specify the background color for the iconbox part of the browser.

\begin{verbatim}
    Paw++*dirlist*<object>*iconForeground:
\end{verbatim}

   Specify the foreground color for the icons of type <object>.

\begin{verbatim}
    Paw++*dirlist*<object>*iconBackground:
\end{verbatim}

   Specify the background color for the icons of type <object>.

\begin{verbatim}
    Paw++*dirlist*<object>*iconLabelForeground:     black
\end{verbatim}

   Specify the foreground color for the labels of the icons of type <object>.

\begin{verbatim}
    Paw++*dirlist*<object>*iconLabelBackground:     white
\end{verbatim}

   Specify the background color for the labels of the icons of type <object>.
   Currently the following different <object>'s are defined:

\begin{verbatim}
    dir     -- directory
    1d      -- 1d histograms
    2d      -- 2d histograms
    ntuple  -- Ntuples
    pict    -- Higz pictures
    chain   -- Ntuple chains
    entry   -- Ntuple chain entries
    hbook   -- Hbook files
\end{verbatim}

   The default iconForeground and iconBackground colors for these objects are:

\begin{verbatim}
    Paw++*dirlist*dir*iconForeground:      blue
    Paw++*dirlist*1d*iconForeground:       DarkGoldenrod3
    Paw++*dirlist*2d*iconForeground:       DeepPink3
    Paw++*dirlist*ntuple*iconForeground:   SteelBlue3
    Paw++*dirlist*pict*iconForeground:     green4
    Paw++*dirlist*chain*iconForeground:    blue
    Paw++*dirlist*entry*iconForeground:    OrangeRed
\end{verbatim}

   When using a black and white X Server use the following resource settings
   to make the icons visible:

\begin{verbatim}
    Paw++*dirlist*<object>*iconForeground:          black
    Paw++*dirlist*<object>*iconBackground:          white
    Paw++*dirlist*<object>*iconLabelBackground:     black
    Paw++*dirlist*<object>*iconLabelForeground:     white
\end{verbatim}

\section{X resources for for \KUIP/\MOTIF}

   This is a list of the X resources available to any \KUIP/\MOTIF{} based
   application (e.g. \PAW++).  Resources control the appearance and behavior of
   an application.

   Users can specify their own values for these resources in the standard
   \X11/\MOTIF{} way (via the .Xdefaults file or a file in the
   /usr/lib/X11/app-defaults directory). One just has to prefix the desired
   resource by the class name of the application.

   To customize \PAW++, for instance, all the resources have to be prefixed
   with \Lit{Paw++} or they should be stored in the file
   /usr/lib/X11/app-defaults/Paw++.

   Any default values specified by \KUIP{} are given behind the resource name.

\begin{verbatim}
    *background:
\end{verbatim}

   Specify the background color for all windows.

\begin{verbatim}
    *foreground:
\end{verbatim}

   Specify the foreground color for all windows.

\begin{verbatim}
    *kxtermGeometry:                 550x550+5+10
\end{verbatim}

   Geometry of Kxterm, the \KUIP{} terminal emulator (\EW).

\begin{verbatim}
    *kuipGraphics_shell.geometry:    550x550+585+10
\end{verbatim}

   Geometry of the graphics window(s) (if any).

\begin{verbatim}
    *kuipBrowser_shell.geometry:     580x450
\end{verbatim}

   Geometry of the browser(s).

\begin{verbatim}
    *XmText*fontList:                *-helvetica-bold-r-normal*-120-*
    *XmTextField*fontList:           *-helvetica-bold-r-normal*-120-*
\end{verbatim}

   Font used by all text areas.

\begin{verbatim}
    *kxtermFont:
\end{verbatim}

   Font used by Kxterm (\PAW++{} \EW)

\begin{verbatim}
    *dirlist*fontList:
\end{verbatim}

   Font used for the icon labels in the browser.

\begin{verbatim}
    *helpFont:                       *-courier-bold-r-normal*-120-*
\end{verbatim}

   Font used for help windows.

\begin{verbatim}
    *fontList:                       *-helvetica-bold-r-normal*-120-*
\end{verbatim}

   Font for the menus, messages and boxes.

\begin{verbatim}
    *keyboardFocusPolicy:            explicit
\end{verbatim}

   If ``explicit'' (default), focus is determined by a mouse or keyboard
   command. If ``pointer'' focus is determined by the mouse pointer position.

\begin{verbatim}
    *doubleClickInterval:            250
\end{verbatim}

   The time span (in milliseconds) within which two button clicks must occur
   to be considered a double click rather than two single clicks.

\begin{verbatim}
    *dirlist*background:
\end{verbatim}

   Specify the background color for the iconbox part of the browser.

\begin{verbatim}
    *dirlist*<object>*iconForeground:          black
\end{verbatim}

   Specify the foreground color for the icons of type <object>.

\begin{verbatim}
    *dirlist*<object>*iconBackground:          white
\end{verbatim}

   Specify the background color for the icons of type <object>.

\begin{verbatim}
    *dirlist*<object>*iconLabelForeground:     black
\end{verbatim}

   Specify the foreground color for the labels of the icons of type <object>.

\begin{verbatim}
    *dirlist*<object>*iconLabelBackground:     white
\end{verbatim}

   Specify the background color for the labels of the icons of type <object>.

\begin{verbatim}
    *zoomEffect:                     True
\end{verbatim}

   Turn zoom effect on or off when going up and down directories in the
   browser.

\begin{verbatim}
    *zoomSpeed:                      10
\end{verbatim}

   Specify speed of zoom effect in the browser.

   Currently the following different <object>'s are defined:

\begin{verbatim}
    Cmd        -- Command
    InvCmd     -- Deactivated command
    Menu       -- Menu tree
    MacFile    -- Macro File
    RwFile     -- Read-write file
    RoFile     -- Readonly file
    NoFile     -- No access file
    ExFile     -- Executable file
    DirFile    -- Directory
    DirUpFile  -- Up directory (..)
\end{verbatim}

   When using a black and white X Server use the following resource settings
   to make the icons visible:

\begin{verbatim}
    *dirlist*<object>*iconForeground:          black
    *dirlist*<object>*iconBackground:          white
    *dirlist*<object>*iconLabelBackground:     black
    *dirlist*<object>*iconLabelForeground:     white
\end{verbatim}

\end{document}
