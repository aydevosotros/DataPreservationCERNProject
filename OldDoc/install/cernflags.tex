\chapter{Setting the PLINAME variable}
\label{sect-PLINAME}

The CERNLIB cradles contain a {\bf USE} statement for {\bf \$PLINAME}.
This is replaced by the value of the environmental variable {\bf PLINAME}
by the utility {\bf YEXPAND}.

The following list describes the meanings of the various values that
can be assigned to {\bf PLINAME}.

\PLINAME{} may be set to more than one value. For example, on VXCERN
\PLINAME{} is set as follows:

\begin{XMPt}{Example of setting PLINAME}
$ dnetarea=f$getsyi("node_area")
$ _la=(dnetarea.eq.22.or.dnetarea.eq.23)
$ _axp=f$getsyi("arch_name").eqs."Alpha"
$ if f$type(pliname) .eqs. ""
$ then
$   if _la
$     then
$       setenv PLINAME "VAX,VAXVMS,CERN"      ! at CERN
$     else
$       setenv PLINAME "VAX,VAXVMS"           ! elsewhere
$     endif
$   if _axp then setenv PLINAME 'PLINAME',QMALPH
$ endif
\end{XMPt}

Thus, PLINAME will be set to "VAX,VAXVMS,CERN" for systems
in DECnet areas 22 or 23 and "VAX,VAXVMS" elsewhere. If
the system is an Alpha, then ",QMALPH" will be added to this string.


PLINAME should contain one or more of the following strings, as 
appropriate. 

\begin{DLtt}{12345678}
\item[IBMVM]IBM mainframes running VM/CMS
\item[IBMMVS]IBM mainframes running MVS
\item[CONVEX]By itself, implies 64 bit version of the libraries for Convex.
To get the 32 bit version, use also {\bf SINGLE}.
\item[IBM]IBM mainframe - selects features generic to both MVS and VM/CMS
\item[SLACBATCH]Activates code specific to the SLAC Batch system for VM/CMS
systems.
\item[ALLIANT]Indicates Alliant computer. If used in conjunction with
{\bf QMINTEL}, implies Alliant with Intel processor.
\item[AMIGAUX]Amiga Unix
\item[APOLLO]Apollo workstation with the {\bf ftn} compiler. If used in 
conjunction with {\bf APOF77}, then the version appropriate for the {\bf f77}
compiler will be generated.
\item[CDC]Control Data systems
\item[CRAY]Cray computers
\item[DGE]Data General computers
\item[MSDOS]PCs running MSDOS
\item[DECS]DECstations running Ultrix. If used in conjunction with
{\bf QMVAOS}, implies Alpha workstations running {\bf OSF}
\item[DECOSF]{\it New flag for Alphas running OSF}
\item[HPUX]HP workstations running HP/UX.
\item[IBMAIX]IBM mainframes running AIX.
\item[IBMRT]IBM Risc processors (RT, RS) running AIX.
\item[CERN]Select CERN specific features
\item[LINUX]PCs running the LINUX operating system.
\item[MACMPW]Macintosh computers.
\item[NORD500]The NORD 500 series of computers.
\item[NECSX]NEC SX computer.
\item[NEXT]NeXT workstations.
\item[SGI]Silicon Graphics workstations.
\item[SHIFT]Activate code specific to systems running the CORE/SHIFT software.
\item[SUN]Sun workstations running SunOS 4 (or less). If used in conjunction
with {\bf SOLARIS}, implies
Sun workstations running SunOS 5 (=Solaris).

\item[TMO]Transputer with Meiko compiler.

\item[VAXVMS]VMS systems. If used with {\bf QMALPH}, means Alpha processors.

\item[WINNT]Windows/NT systems.
\end{DLtt}

\chapter{PATCHY/CMZ flags and their meanings}
The following information was extracted from the {\bf KERNFOR} 
pam file.

\label{sect-FLAGS}

\section{Flags for different computer types}

\begin{DLtt}{12345678}
\item[QMNNB32] for an unknown 32-bit machine\\

\item[QMALPH]  Alpha eXtended processor (AXP)\\

\item[QMALT]   Alliant
\item[QMAMX]   Amiga Unix\\

\item[QMAPO]   Apollo
\item[QMAPO10] Apollo DPS 10000\\

\item[QMCDCV]  CDC 6000/7000/Cyber  with Fortran 5
\item[QMCDC]   CDC 6000/7000/Cyber  with Fortran 4\\

\item[QMDOS]   MS-DOS with F2C + G cc compilers
\item[QMNDP]   MS-DOS with NDP Fortran\\

\item[QMCRY]   CRAY systems COS or UNICOS
\item[QMCRU]   CRAY system UNICOS only\\

\item[QMCVX]   General Convex flag
\item[QMCV64]  Convex 64-bit mode 
\item[QMCV32]  Convex 32-bit mode\\

\item[QMDGE]   Data General, ECLIPSE\\

\item[QMHPX]   Hewlett Packard HP Unix\\

\item[QMIBM]   IBM 360 / 370
\item[QMIBMVF] IBM Vector facility
\item[QMIBMXA] IBM Xtended Adressing
\item[QMIBMFSI]Fortran Siemens {\bf OBSOLETE - use QMFIBMSI}
\item[QMIBMFVS]Fortran VS {\bf OBSOLETE - use QMFIBMVS}\\

\item[QMIBX]   IBM 3090 with system AIX
\item[QMIRT]   IBM / RT and 6000 with xlf compiler\\

\item[QMND3]   NORD 500
\item[QMNXT]   Next
\item[QMPDP]   DEC PDP 10\\

\item[QMSGI]   Silicon Graphics
\item[QMSUN]   SUN\\

\item[QMTMO]   Transputer with Meiko compiler
\item[QMUNI]   UNIVAC 1100  with earlier compilers
\item[QMUNO]   UNIVAC 1100  with FTN compiler\\

\item[QMVAX]   Digital VAX
\item[QMVMI]   DECstation (MIPS processor unless QMALPH is also specified, then Alpha processor).
\item[QMVAO]   Digital Alpha with OSF, + S for 32-bit
                                      (+ L for 64-bit later)\\

\item[TYPE]    Force strong typing, i.e. {\bf IMPLICIT NONE}
\end{DLtt}

\section{Flags to indicate word capacity}

\begin{DLtt}{12345678}
\item[B32]     32 bits in one computer word
\item[B36]     36 bits in one computer word
\item[B48]     48 bits in one computer word
\item[B60]     60 bits in one computer word
\item[B64]     64 bits in one computer word\\

\item[B36M]    36 bits  or  more per word
\item[B48M]    48 bits  or  more per word
\item[B60M]    60 bits  or  more per word\\

\item[A4]      4 characters in 1 computer word
\item[A5]      5 characters in 1 computer word
\item[A6]      6 characters in 1 computer word
\item[A8]      8 characters in 1 computer word
\item[A10]     10 characters in 1 computer word\\

\item[A5M]     5 characters  or  more per word
\item[A6M]     6 characters  or  more per word
\item[A8M]     8 characters  or  more per word\\
\end{DLtt}

\section{Flags for other computer or C/Fortran features}

\begin{DLtt}{12345678}
\item[QCDEC]     Compiled using DEC C
\item[QCVAX]     Compiled using VAX C\\

\item[QF2C]      Compiled using F2C and gcc
\item[QFDEC]     Compiled using DEC Fortran
\item[QFNDP]     Compiled using NDP Fortran
\item[QMFIBMSI]  Fortran Siemens 
\item[QMFIBMVS]  Fortran VS 
\item[QFMSOFT]   Compiled using Microsoft Fortran\\

\item[QASCII]    Character set is ASCII
\item[QEBCDIC]   Character set is EBCDIC\\

\item[QIEEE]     Floating point representation is IEEE\\

\item[QISASTD]   ISA standard intrinsic functions available :
                 IAND, IOR, NOT, ISHFT
\item[QMILSTD]   MIL standard intrinsic functions available :
                 IBITS, MVBITS, ISHFTC\\

\item[QHOLL]     Hollerith constants exist
\item[EQUHOLCH]  EQUIVALENCE Hollerith/Character ok
\item[QORTHOLL]  orthodox Hollerith storage left to right in word\\

\item[QSYSBSD]   Unix system BSD (system 5 otherwise)
\item[QSIGJMP]   Posix sigsetjmp/siglongjmp for setjmp/longjmp
\item[QENVBSD]   BSD setenv is available
\item[QGETCWD]   BSD getwd is not available, but getcwd is available
\item[QSIGBSD]   signal handling with BSD   sigvec
\item[QSIGPOSIX] signal handling with Posix sigaction\\

\item[QX\_SC]     external names are lower case with underscore
\item[QXNO\_SC]   external names are lower case without underscore
\item[QXCAPT]    external names are capital\\

\item[QCCINDAD]  routine entry addresses are passed double indirect
                 in Fortran calls (needed in C routines)\\

\item[INTDOUBL]  use double precision for some internal calculations
                 (used at present only in the TR routines)\\

\item[NOSHIFT]   left/right shift is not available,
                 sequence Q\$SHIFT cannot be defined\\

\item[HEX]       dumps must be done in hexadecimal representation
                 else: dumps are in octal\\

\item[ENTRET]    multiple entry functions must return by entry name
                 else: return by function name works ok\\

\item[ENTRCDC]   CDC Fortran 4 syntax for ENTRY statement
                 else: ENTRY statement contains argument list
\end{DLtt}

\section{Flags inherited from KERNNUM  -  only used in P=TCNUM}
\begin{DLtt}{12345678}
\item[NUMLOPRE]floating point precision for 32-bit machines\\

\item[NUMHIPRE]=-NUMLOPRE\\

\item[NUME293] maximum exponent = 10**293\\
  
\item[NUME75]  maximum exponent = 10**75\\
  
\item[NUME38]  maximum exponent = 10**38\\

\end{DLtt}
