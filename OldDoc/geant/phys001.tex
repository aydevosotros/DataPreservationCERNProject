%%%%%%%%%%%%%%%%%%%%%%%%%%%%%%%%%%%%%%%%%%%%%%%%%%%%%%%%%%%%%%%%%%%
%                                                                 %
%  GEANT manual in LaTeX form                              %
%                                                                 %
%  Michel Goossens (for translation into LaTeX)                   %
%  Version 1.00                                                   %
%  Last Mod. Jan 24 1991  1300   MG + IB                          %
%                                                                 %
%%%%%%%%%%%%%%%%%%%%%%%%%%%%%%%%%%%%%%%%%%%%%%%%%%%%%%%%%%%%%%%%%%%
\Documentation{M. Maire, F.Carminati}
\Submitted{24.09.84}\Revised{26.07.93}
\Version{Geant 3.16}\Routid{PHYS001}
\Makehead{Introduction to the section PHYS}
\section{Summary of the physics processes}
 
The computer simulation of particles traversing an
experimental setup has to take into account the interactions of those
particles with the material of the detector.
{\tt GEANT} is able to simulate the dominant processes which can occur in the
energy range {\bf from 10 keV to 10 TeV} for electromagnetic interactions.
As far as hadronic interactions are concerned, the range of validity
is the one of the hadronic packages used, which usually does not extend
below a few tens of MeV. For more information the user is invited to 
consult the relevant 
documentation~\cite{bib-GHEI,bib-FLUK,bib-FLU1,bib-FLU2,bib-FLU3,bib-FLU4}.
 
Simulating a given process means:
\begin{UL}
\item
Evaluating the probability of occurrence of the process, by sampling the
{\bf total cross-section} of the process.
\item
Generating the final state after interaction, by sampling the
{\bf differential cross-section} of the process.
\item
In case of (quasi-)continuous processes, e.g. CSDA
(Continuous Slowing Down Approximation), energy losses or multiple 
scattering, computing the mean values of some characteristic quantities.
\end{UL}
In Table~\ref{phys001-1} below we summarise all the processes currently 
implemented in {\tt GEANT}, with a reference to the corresponding sections.
\begin{table}[hbt]
\begin{center}
\begin{tabular}{|l|l|l|}
\hline
& \parbox{5cm}{Computation of total cross-section or energy losses}
& \parbox{5cm}{Generation of the final state particles} \\
\hline
                                            &              &           \\
{\bf  Processes involving the photon }      &              &           \\
($e^+,e^-$ ) pair conversion                & PHYS 210     & PHYS  211 \\
Compton collision                           & PHYS 220     & PHYS  221 \\
Photoelectric effect                        & PHYS 230     & PHYS  231 \\
Photo fission of heavy elements             & PHYS 240     & PHYS  240 \\
Rayleigh effect                             & PHYS 250     & PHYS  251 \\
                                            &              &           \\
{\bf Processes involving $e^-/e^+ $ }       &              &           \\
Multiple scattering                         &              & PHYS 320
or 325 or 328\\
Ionisation and $\delta$-rays production     & PHYS 330     & PHYS 331
or 332 \\
Bremsstrahlung                              & PHYS 340      & PHYS 341 \\
Annihilation of positron                    & PHYS 350      & PHYS 351 \\
Generation of \v{C}erenkov light             & PHYS 260      & PHYS 260 \\
Synchrotron radiation                       & PHYS 360      &         \\
                                            &              &           \\
{\bf Processes involving}$\mu^-/ \mu^+$     &               &          \\
Decay in flight                             & CONS 310      & PHYS 400 \\
Multiple scattering                         &               & PHYS 320
or 325 \\
Ionisation and $\delta$-rays production     & PHYS 430      & PHYS 331
or 332 \\
Ionisation by heavy ions                    & PHYS 431      &         \\
Bremsstrahlung                              & PHYS 440      & PHYS 441 \\
Direct ($ e^+,e^-$) pair production         & PHYS 450      & PHYS 451 \\
Nuclear interaction                         & PHYS 460      & PHYS 460 \\
Generation of \v{C}erenkov light             & PHYS 260      & PHYS 260 \\
                                            &              &           \\
{\bf Processes involving hadrons}           &               &          \\
Decay in flight                             & CONS 310      & PHYS 400 \\
Multiple scattering                         &               & PHYS 320
or 325 \\
Ionisation and $\delta$-rays production     & PHYS 430      & PHYS 331
or 332 \\
Hadronic interactions                       & PHYS 500 or 510&
PHYS 500 or 510  \\
Generation of \v{C}erenkov light             & PHYS 260      & PHYS 260 \\
\hline
\end{tabular}
\end{center}
\label{phys001-1}
\caption{Processes currently implemented in {\tt GEANT}}
\end{table}
 
\subsection{ Simulated Processes }
 
\subsubsection{Hadronic Interactions}

To simulate the interactions of hadrons with the nuclei of the matter 
traversed, two alternatives are provided:
\begin{enumerate}
\item The generator of the \Rind{FLUKA}~\cite{bib-FLUK,bib-FLU1}
hadron shower MonteCarlo and the
interface routines to {\tt GEANT}. See {\tt [PHYS520]} for more information.
\item The generator of the \Rind{GHEISHA}~\cite{bib-GHEI}
hadron shower MonteCarlo and the
interface routines to {\tt GEANT}. See {\tt [PHYS510]} for more information.
\end{enumerate}
 
The code both of the \Rind{GHEISHA} and of the \Rind{FLUKA} generators is
contained in the {\tt GEANT} library. Users should be aware that the
routines of these packages do not follow the {\tt GEANT} naming conventions
and therefore they can clash with the names of user procedures.
 
\subsubsection{Electromagnetic Processes}
By means of systematic fits to the existing
data, the cross-sections of the electromagnetic processes are well reproduced
(within a few percent) from 10 keV up to 100 GeV, both for light (low Z) and
for heavy materials.
 
This feature, together with the use of the interface with one of the
hadronic shower generators available,
makes {\tt GEANT} useful also for shower simulation in gases.
 
\subsubsection{Muonic interactions}
Muonic interactions are simulated up to 10 TeV, making
{\tt GEANT} useful for cosmic rays studies.
 
\subsubsection{Ionisation by charged particles}
The following alternatives are provided to simulate this process:
\begin{itemize}
\item Sampling from the appropriate distribution around the mean value 
of the energy loss ({\tt [PHYS332]}).
\item
Explicit generation of $\delta$-rays (see {\tt [PHYS330/331/430]}) and 
restricted fluctuations below the energy threshold for the production
of $\delta$-rays.
\item
Sampling the contribution of single collisions from statistical 
distributions. This can be used as an alternative to the first one when
simulating energy losses 
in very thin layers (small value of g  cm$^{-2}$) (see {\tt [PHYS334]}).
\end{itemize}
Full Landau fluctuations and generation of $\delta$-rays cannot be used
together in order
to avoid double counting of the fluctuations. An automatic protection
has been introduced in {\tt GEANT} to this effect.
See {\tt [PHYS333/332]} and {\tt [BASE040]} for further information.
 
\subsubsection{Multiple Scattering}
Two methods are provided:
\begin{enumerate}
\item Moli\`ere distribution or plural scattering ({\tt [PHYS325], [PHYS328]}).
\item Gaussian approximation ({\tt [PHYS320]}).
\end{enumerate}
 
\subsubsection{The JMATE data structure}
In order to save time during the transport of the particles, relevant 
energy-dependent quantities are tabulated at the beginning of the run, for 
all materials, as functions of the kinetic energy of the particle.
In particular, the inverse of the total cross-sections 
of all processes involving photons, electrons and muons and the $dE/dx$ and 
range tables for electrons, muons and protons are calculated.
The actual value of, say, the interaction length for a given process (i.e. the
inverse of the macroscopic cross section) is then obtained via a linear
interpolation in the tables. The data structure which contains all this
information in memory is supported by the link {\tt JMATE} in the 
\FCind{/GCLINK/} common block.
See {\tt [PHYS100]} and {\tt [CONS199]} for a more information on these tables.
 
\subsubsection{Probability of Interaction}
The total cross-section of each process is used at tracking time to evaluate 
the probability of occurrence of the process. See {\tt [PHYS010]} for an 
explanation of the method used.
 
{\bf Note}: The section {\tt PHYS} is closely related to the section {\tt CONS}.
Users wishing to have a complete overview of the physics processes
included in {\tt GEANT} should read both sections.

\subsection{ Control of the physical processes}
 
For most of the individual processes the default option (indicated
below) can be changed via data records {\tt [BASE040]}.
The processes are controlled via a control variable which is in the
common \FCind{/GCKING/}. 
If not otherwise noted, 
the meaning of the control variable is the following:
\begin{DLtt}{MMM}
\item[= 0] The process is completely ignored. 
\item[= 1] The process is considered and possible secondary particles generating
from the interaction are put into the \FCind{/GCKING/} common. If the
interacting particle disappears in the interaction, then it is stopped
with {\tt ISTOP=1} (common \FCind{/GCTRAK/})
\item[= 2] The process is considered. If secondary particles result from the
interaction, they are not generated and their energy is simply added in the
variable {\tt DESTEP} (common \FCind{/GCTRAK/}. If the interacting particle
disappears in the interaction, the variable {\tt ISTOP} is set
to $2$.
\end{DLtt}
Below are listed the data record keywords, the flag names and values, and
the resulting action:

\begin{DLtt}{MMMMMMMM}
\item[Keyword] Related process 
\item[DCAY] Decay in flight. The decaying particles stops.
The variable {\tt IDCAY} controls this process.
\begin{DLtt}{MMMMMMMM}
\item[IDCAY =0] No decay in flight.
\item[~~~~~~=1] ({\bf D}) Decay in flight with generation of secondaries.
\item[~~~~~~=2] Decay in flight without generation of secondaries.
\end{DLtt}
\item[MULS] Multiple scattering. The variable {\tt IMULS} controls this process.
\begin{DLtt}{MMMMMMMM}
\item[IMULS =0] No multiple scattering.
\item[~~~~~~=1] ({\bf D}) Multiple scattering according to Moli\`ere theory.
\item[~~~~~~=2] Same as {\tt 1}. Kept for backward compatibility.
\item[~~~~~~=3] Pure Gaussian scattering according to the Rossi formula.
\end{DLtt}
\item[PFIS] Nuclear fission induced by a photon. The photon stops.
The variable {\tt IPFIS} controls this process.
\begin{DLtt}{MMMMMMMM}
\item[IPFIS =0] ({\bf D}) No photo-fission.
\item[~~~~~~=1] Photo-fission with generation of secondaries.
\item[~~~~~~=2] Photo-fission without generation of secondaries.
\end{DLtt}
\item[MUNU] Muon-nucleus interactions. The muon is not stopped.
The variable {\tt IMUNU} controls this process.
\begin{DLtt}{MMMMMMMM}
\item[IMUNU =0] No muon-nucleus interactions.
\item[~~~~~~=1] ({\bf D}) Muon-nucleus interactions with generation of 
secondaries.
\item[~~~~~~=2] Muon-nucleus interactions without generation of secondaries.
\end{DLtt}
\item[LOSS] Continuous energy loss. The variable {\tt ILOSS} controls this 
process.
\begin{DLtt}{MMMMMMMM}
\item[ILOSS =0] No continuous energy loss,IDRAY is forced to 0.
\item[~~~~~~=1] Continuous energy loss with generation of $\delta$-rays 
above {\tt DCUTE} (common \FCind{/GCUTS/}) and
restricted Landau fluctuations below {\tt DCUTE}.
\item[~~~~~~=2] ({\bf D}) Continuous energy loss without generation of 
$\delta$-rays
and full Landau-Vavilov-Gauss fluctuations. In this case the variable {\tt IDRAY}
is forced to $0$ to avoid double counting of fluctuations.
\item[~~~~~~=3] Same as $1$, kept for backward compatibility.
\item[~~~~~~=4] Energy loss without fluctuation. The value obtained from the
tables is used directly.
\end{DLtt}
\item[PHOT] Photoelectric effect. The interacting photon is stopped.
The variable {\tt IPHOT} controls this process.
\begin{DLtt}{MMMMMMMM}
\item[IPHOT =0] No photo-electric effect.
\item[~~~~~~=1] ({\bf D}) Photo-electric effect with generation of the electron.
\item[~~~~~~=2] Photo-electric effect without generation of the electron.
\end{DLtt}
\item[COMP] Compton scattering. The variable {\tt ICOMP} controls this process.
\begin{DLtt}{MMMMMMMM}
\item[ICOMP =0] No Compton scattering.
\item[~~~~~~=1] ({\bf D}) Compton scattering with generation of \Pem.
\item[~~~~~~=2] Compton scattering without generation of \Pem.
\end{DLtt}
\item[PAIR] Pair production. The interacting $\gamma$ is stopped.
The variable {\tt IPAIR} controls this process.
\begin{DLtt}{MMMMMMMM}
\item[IPAIR =0] No pair production.
\item[~~~~~~=1] ({\bf D}) Pair production with generation of \Pem/\Pep.
\item[~~~~~~=2] Pair production without generation of \Pem/\Pep.
\end{DLtt}
\item[BREM] bremsstrahlung. The interacting particle (\Pem, \Pep, $\mu^{+}$,
$\mu^{-}$) is not stopped.
The variable {\tt IBREM} controls this process.
\begin{DLtt}{MMMMMMMM}
\item[IBREM =0] No bremsstrahlung.
\item[~~~~~~=1] ({\bf D}) bremsstrahlung with generation of $\gamma$.
\item[~~~~~~=2] bremsstrahlung without generation of $\gamma$.
\end{DLtt}
\item[RAYL] Rayleigh effect. The interacting $\gamma$ is not stopped.
The variable {\tt IRAYL} controls this process.
\begin{DLtt}{MMMMMMMM}
\item[IRAYL =0] ({\bf D}) No Rayleigh effect.
\item[~~~~~~=1] Rayleigh effect.
\end{DLtt}
\item[DRAY] $\delta$-ray production. 
The variable {\tt IDRAY} controls this process.
\begin{DLtt}{MMMMMMMM}
\item[IDRAY =0] No $\delta$-rays production.
\item[~~~~~~=1] ({\bf D}) $\delta$-rays production with generation of \Pem.
\item[~~~~~~=2] $\delta$-rays production without generation of \Pem.
\end{DLtt}
\item[ANNI] Positron annihilation. The \Pep is stopped.
The variable {\tt IANNI} controls this process.
\begin{DLtt}{MMMMMMMM}
\item[IANNI =0] No positron annihilation.
\item[~~~~~~=1] ({\bf D}) Positron annihilation with generation of photons.
\item[~~~~~~=2] Positron annihilation without generation of photons.
\end{DLtt}
\item[HADR] Hadronic interactions. The particle is stopped in case of
inelastic interaction, while it is not stopped in case of elastic interaction.
The variable {\tt IHADR} controls this process.
\begin{DLtt}{MMMMMMMM}
\item[IHADR =0] No hadronic interactions.
\item[~~~~~~=1] ({\bf D}) Hadronic interactions with generation of secondaries.
\item[~~~~~~=2] Hadronic interactions without generation of secondaries.
\item[~~~~~~$>$2] Can be used in the user code \Rind{GUPHAD} and \Rind{GUHADR}
to chose a hadronic package. These values have no effect on the hadronic
packages themselves.
\end{DLtt}
\item[LABS] Light ABSorption. This process is the absorption of light photons
(particle type 7) in dielectric materials. It is turned on by default when
the generation of \v{C}erenkov light is requested (data record {\tt CKOV}).
For more information see {\tt [PHYS260]}.
\begin{DLtt}{MMMMMMMM}
\item[ILABS =0] No absorption of photons.
\item[~~~~~~=1] Absorption of photons with possible detection.
\end{DLtt}
\item[STRA]
This flag turns on the collision sampling method to simulate energy
loss in thin materials, particularly gases. For more information see
{\tt [PHYS334]}.
\begin{DLtt}{MMMMMMMM}
\item[ISTRA =0] ({\bf D}) Collision sampling switched off.
\item[~~~~~~=1] Collision sampling activated.
\end{DLtt}
\item[SYNC] Synchrotron radiation in magnetic field.
\begin{DLtt}{MMMMMMMM}
\item[ISYNC =0] ({\bf D}) The synchrotron radiation is not simulated.
\item[~~~~~~=1] Synchrotron photons are generated,
at the end of the tracking step.
\item[~~~~~~=2] Photons are not generated, the energy is deposit locally.
\item[~~~~~~=3] Synchrotron photons are generated,
distributed along the curved path of the particle.
\end{DLtt}
\end{DLtt}
