%%%%%%%%%%%%%%%%%%%%%%%%%%%%%%%%%%%%%%%%%%%%%%%%%%%%%%%%%%%%%%%%%%%
%                                                                 %
%  GEANT manual in LaTeX form                              %
%                                                                 %
%  Michel Goossens (for translation into LaTeX)                   %
%  Version 1.00                                                   %
%  Last Mod. Jan 24 1991  1300   MG + IB                          %
%                                                                 %
%%%%%%%%%%%%%%%%%%%%%%%%%%%%%%%%%%%%%%%%%%%%%%%%%%%%%%%%%%%%%%%%%%%
\Origin{F.Carminati}
\Submitted{14.12.93}       \Revised{14.12.93}
\Version{Geant 3.16}       \Routid{TRAK310}
\Makehead{Altering the order of tracking secondary particles}

When secondary particles are produced during particle transport, if
the user wants to transport them in turn, they must be stored by
a call to \Rind{GSKING} or \Rind{GSKPHO} onto the temporary stack
{\tt JSTAK}. As explained in {\tt [TRAK310]}, this is a LIFO
structure, where the last particle introduced is the first to be extracted
to be transported. It may be sometimes interesting to alter the order
of tracking, for instance to decide whether the current event is
worth completing or whether it should be skipped to go to next one.

It is possible in {\tt GEANT} to alter the behaviour of the {\tt JSTAK}
structure using one of the words of the stack, normally reserved for
the user. Stack ordering is activated via the data record {\tt SORD 1}.
This will set to 1 the variable {\tt ISTORD} in common \FCind{/GCSTAK/},
which is by default initialised to 0.

When stack ordering is activated, the routine \Rind{GLTRAC}, which
fetches the next particle to be tracked from {\tt JSTAK}, instead
of the last one, will select the one with the maximum value of the
user word. This user word must be set by the user in the variable
{\tt UPWGHT} in the common \FCind{/GCTRAK/} prior to the call to
\Rind{GSKING} to store each secondary.

If, for instance, all the protons must be tracked first, in order of
production, and all the other particles after, again according to their
time of production, the following code should appear in \Rind{GUSTEP}:

\begin{verbatim}
      SUBROUTINE GUSTEP
+SEQ,GCTRAK,GCKING
      .
      .
      .
      DO 10 IG=1,NGKINE
*---         Younger particles tracked first
*---         Add to the current time of flight the time delay of
*---         the secondary
         UPWGHT = -TOFG-TOFD(IG)
         IPART = GKIN(5,IG)
         IF(IPART.EQ.14) THEN
*---         This is a proton, add a 100 microseconds to have it
*---         tracked first
            UPWGHT = UPWGHT+1E-4
         ENDIF
*---         Now store the particle in the GEANT stack
         CALL GSKING(IG)
  10  CONTINUE
\end{verbatim}

Users should be aware of the fact that ordering the stack will change the
results, because transport depends on random numbers which will be different
for the same particle in the two cases.
