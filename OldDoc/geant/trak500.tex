%%%%%%%%%%%%%%%%%%%%%%%%%%%%%%%%%%%%%%%%%%%%%%%%%%%%%%%%%%%%%%%%%%%
%                                                                 %
%  GEANT manual in LaTeX form                                     %
%                                                                 %
%  Michel Goossens (for translation into LaTeX)                   %
%  Version 1.00                                                   %
%  Last Mod. Jan 24 1991  1300   MG + IB                          %
%                                                                 %
%%%%%%%%%%%%%%%%%%%%%%%%%%%%%%%%%%%%%%%%%%%%%%%%%%%%%%%%%%%%%%%%%%%
\Origin{R.Brun, M.Hansroul}
\Submitted{01.10.81}   \Revised{16.12.93}
\Version{Geant 3.16}   \Routid{TRAK500}
\Makehead{Tracking routines in magnetic field}

\Shubr{GUSWIM}{(CHARGE,STEP,VECT,VOUT*)}
\begin{DLtt}{MMMMMMMM}
\item[CHARGE] ({\tt REAL}) particle charge in $e$ unit;
\item[STEP] ({\tt REAL}) step size in cm;
\item[VECT] ({\tt REAL}) array of 7 with the initial coordinates, direction 
cosines and momentum ($x$, $y$, $z$, $p_x/p$, $p_y/p$, $p_z/p$, $p$);
\item[VOUT] ({\tt REAL}) array of 7 with the final coordinates, direction 
cosines and momentum after the step.
\end{DLtt}
 
This is the steering routine for calculating the deviation of a charged
particle in magnetic field. It is called by the {\tt GEANT} tracking 
routines: \Rind{GTELEC}, \Rind{GTHADR}, \Rind{GTMUON} and \Rind{GTHION}.
Even if this routine follows the naming conventions of the user routines,
the version provided in the {\tt GEANT} library is quite adequate for most
situation, and users should have a very good reason to modify it.
 
It calls \Rind{GRKUTA}, \Rind{GHELIX}, \Rind{GHELX3} according to the value
1, 2 or 3 of {\tt IFIELD} in common \FCind{/GCTMED/}. {\tt IFIELD} correspond
to the argument to the \Rind{GSTMED} routine (see {\tt [CONS200]}).
{\tt IFIELD = -1} is reserved for user decision in case \Rind{GUSWIM}
is modified by the user.

\Shubr{GHELX3}{(FIELD,STEP,VECT,VOUT*)}

\begin{DLtt}{MMMMMMMM}
\item[FIELD] ({\tt REAL}) field value in kilo-Gauss multiplied by 
the charge of the particle in $e$ units;
\item[STEP] ({\tt REAL}) step size in cm;
\item[VECT] ({\tt REAL}) array of 7, same as for \Rind{GUSWIM};
\item[VOUT] ({\tt REAL}) array of 7, same as for \Rind{GUSWIM};
\end{DLtt}

Transport the particle by a length {\tt STEP} in a magnetic
field. The field is assumed to be uniform and parallel to the
$z$ axis. The value of the field is the {\tt FIELDM} argument to
the \Rind{GSTMED} routine (see {\tt [CONS200]}). The particle is
transported along an helix.  The call to this routine is selected by 
\Rind{GUSWIM} when {\tt IFIELD=3} in the call to \Rind{GSTMED}.

\Shubr{GHELIX}{(CHARGE,STEP,VECT,VOUT*)}

\begin{DLtt}{MMMMMMMM}
\item[CHARGE] ({\tt REAL}) particle charge in $e$ units;
\item[STEP] ({\tt REAL}) step size in cm;
\item[VECT] ({\tt REAL}) array of 7, same as for \Rind{GUSWIM};
\item[VOUT] ({\tt REAL}) array of 7, same as for \Rind{GUSWIM};
\end{DLtt}
 
Transport the particle by a length {\tt STEP} in a magnetic
field. The magnetic field is calculated in the middle of the step
(along a straight line) and it is supposed to be constant along
the step and the particle is transported along an helix. 
The call to this routine is selected by
\Rind{GUSWIM} when {\tt IFIELD=2} in the call to \Rind{GSTMED}.
This routine is intended for magnetic fields with a small gradient.
The value of the field is obtained from the routine \Rind{GUFLD}
which must be coded by the user.

\Shubr{GRKUTA}{(CHARGE,STEP,VECT,VOUT*)}
 
The arguments have the same meaning than those of
\Rind{GHELIX}. Transport a particle in magnetic field using
Runge-Kutta method for solving the kinematic equations (Nystroem 
algorithm~\cite{bib-NATB}, procedure 25.5.20).
The call to this routine is selected by
\Rind{GUSWIM} when {\tt IFIELD=1} in the call to \Rind{GSTMED}.
This method is the slower of the three and it must be used for
magnetic fields with strong gradient. Again the magnetic field is
obtained from the \Rind{GUFLD} routine which the user has to code.

\Shubr{GUFLD}{((VECT,F*)}

\begin{DLtt}{MMMMMMMM}
\item[VECT] ({\tt REAL}) array of 3 with the position in space where the
field has to be returned;
\item[F] ({\tt REAL}) array of 3 with the magnetic field in {\tt VECT},
in kilo-Gauss units.
\end{DLtt}

User routine to return the three components of the magnetic field at
point {\tt VECT} expressed in kilo-Gauss. This routine must be provided
by the user for all tracking media where {\tt IFIELD} is 1 or 2. This
routine will be called very often, at least three times at each step,
so all the care must be put in its optimisation.
