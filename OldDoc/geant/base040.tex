%%%%%%%%%%%%%%%%%%%%%%%%%%%%%%%%%%%%%%%%%%%%%%%%%%%%%%%%%%%%%%%%%%
%                                                                 %
%  GEANT manual in LaTeX form                                     %
%                                                                 %
%  Version 1.00                                                   %
%                                                                 %
%  Last Mod.  9 June 1993 1300   MG                               %
%                                                                 %
%%%%%%%%%%%%%%%%%%%%%%%%%%%%%%%%%%%%%%%%%%%%%%%%%%%%%%%%%%%%%%%%%%%
\Documentation {F.Bruyant, M.Maire}  
\Submitted {01.10.84}               \Revised{16.12.93}
\Version{Geant 3.16}\Routid{BASE040}
\Makehead{Summary of Data Records}
\section{ Introduction }
 
{\tt GEANT} uses the {\tt FFREAD} \cite{bib-FFREAD}
package to read {\it free format} data records in the routine \Rind{GFFGO}.
The keywords accepted by \Rind{GFFGO} can be
classified as:
\begin{enumerate}
\item         general control of the run;
\item         control of the physics processes;
\item         debug and I/O operations;
\item         user applications;
\item         Lund event generation.
\end{enumerate}
 
The data records are listed below by category with the following information:
\begin{DLtt}{MMMM}
\item[KEY] keyword, any number of characters
truncated to the first 4 unless otherwise specified by the user;
\item[N] maximum expected number of variables ({\tt NVAR});
\item[T] type of these variables ({\tt I=INTEGER, R=REAL or M=MIXED})
and for each variable in turn:
\begin{itemize}
\item variable FORTRAN name;
\item short description (more detail in {\tt [ZZZZ010]});
\item labelled common where it is stored;
\item default value, usually from \Rind{GINIT}.
\end{itemize}
\end{DLtt}
When a record is decoded, the values entered by the user
in free format are assigned to the variables in order.
The number of values can be less than {\tt NVAR}. In case of a {\tt MIXED}
type the values entered have agree
with the type of the corresponding variable.
 
For example the data record:
\begin{verbatim}
      RUNG   5   201
\end{verbatim}
presets the run and event number to 5 and 201 respectively.
None of the records mentioned below is mandatory.
\section{ User defined data records }
 
Before calling \Rind{GFFGO} the user may define private data records
through calls to \Rind{FFKEY} as follows:
\begin{verbatim}
     CALL FFKEY('key',VAR(1),NVAR,'type')
\end{verbatim}
They will be interpreted by \Rind{GFFGO} in the same way as the {\tt GEANT}
pre-defined records.

\section{ Summary of {\tt GEANT} data records }

\subsection{General control}
\begin{tabular}{lllllll}
KEY   &N    &I    &VAR  &\parbox[t]{7.5cm}{Short description}
&\tt COMMON  &\Rind{GINIT} \\
\hline
\tt HSTA  &20 &M &\tt LHSTA &
\parbox[t]{7.5cm}{names of required standard histograms, see {\tt [BASE110]}}
&\FCind{/GCLIST/} & {\tt Blank} \\
\tt OPTI & 1 & I & \tt IOPTI &
\parbox[t]{7.5cm}{automatic optimisation of the geometry via \Rind{GSORD}}
&\FCind{/GCOPTI/} & 1 \\
\tt RNDM  &2  &I & \tt NRNDM & initial random number seed (2 words) &
   \FCind{/GCFLAG/} &0    \\
\tt RUNG  &2  &I & \tt IDRUN & user run number                     &
   \FCind{/GCFLAG/} &1      \\
&     &   &  \tt IDEVT  & first user event number                   &
   \FCind{/GCFLAG/} &1      \\
\tt SORD & 1  &I &\tt ISTORD & stack ordering flag                    &
   \FCind{/GCSTAK/}  & 0 \\
\tt TRIG  &1  &I &\tt NEVENT & total number of events to process      &
   \FCind{/GCFLAG/}  & 10000000\\
\tt TIME  &3  &M &\tt TIMINT &time left after initialisation (see \bf Note
below)
       &
   \FCind{/GCTIME/}  \\
&     &   &  \tt TIMEND &time required for termination        &
  \FCind{/GCTIME/} &1      \\
 
&     &   &  \tt ITIME  &test every {\tt ITIME} events &\FCind{/GCTIME/} &1
\end{tabular}

{\bf Note:} the time allowed for the job after initialisation
cannot be set by the user via the data 
record. To set the total time for the job the user should call the
\Rind{TIMEST} routine at the beginning of the program before any
call to {\tt GEANT} routines. This variable in the 
data record has not been removed for backward compatibility.

\subsection{Control of physics processes}
For more information on the use of these flags, see {\tt [PHYS001]}.

\begin{tabular}{lllllll}
KEY   &N    &I    &VAR  &\parbox[t]{7.5cm}{Short description}
 &\tt COMMON  &\Rind{GINIT} \\
\hline
\tt ANNI  &1  &I &\tt IANNI &annihilation   &\FCind{/GCPHYS/}   &1 \\
\tt AUTO  &1  &I &\tt IGAUTO &\parbox[t]{7.5cm}{automatic computation
of the tracking medium parameters} &\FCind{/GCTRAK/}   &1 \\
\tt BREM  &1  &I &\tt IBREM  &bremsstrahlung &\FCind{/GCPHYS/}  &1 \\
\tt CKOV  &1  &I &\tt ICKOV  &
\v{C}erenkov photon generation &\FCind{/GCTMED/}  &0 \\
\tt COMP  &1  &I &\tt ICOMP  &Compton scattering &\FCind{/GCPHYS/} &1 \\
\tt CUTS  &16 &R &\multicolumn{4}{l}{Kinetic energy cuts in GeV:}      \\
 &    &   &  \tt CUTGAM  &cut for  for gammas    &\FCind{/GCCUTS/} &0.001   \\
 &    &   &  \tt CUTELE  &cut for electrons  &\FCind{/GCCUTS/} &0.001   \\
 &    &   &  \tt CUTNEU  &cut for neutral hadrons&\FCind{/GCCUTS/}&0.01 \\
 &    &   &  \tt CUTHAD  &cut for charged hadrons&\FCind{/GCCUTS/}&0.01  \\
 
 &    &   &  \tt CUTMUO  &cut for muons           &\FCind{/GCCUTS/}&0.01\\
 &    &   &  \tt BCUTE   &cut for electron bremsstrahlung&\FCind{/GCCUTS/}&
    {\tt GUTGAM}\\
 &    &   &  \tt BCUTM   &cut for muon and hadron bremsstrahlung
&\FCind{/GCCUTS/}& \tt CUTGAM\\
 &    &   &  \tt DCUTE   & cut for $\delta$-rays by electrons&\FCind{/GCCUTS/}&
     $10^{4}$\\
&  &  & \tt DCUTM   & cut for $\delta$-rays by muons&\FCind{/GCCUTS/}&$10^{4}$\\
 &    &   &  \tt PPCUTM  & \parbox[t]{7.5cm}{total energy cut for 
direct pair production by muons} &
    \FCind{/GCCUTS/}&0.01\\
 &    &   &  \tt TOFMAX  & time of flight cut in seconds
&\FCind{/GCCUTS/}&$10^{10}$\\
 
 &    &    & \tt GCUTS    &5 user words    &\FCind{/GCCUTS/}&0\\
\tt DCAY  &1   &I&\tt IDCAY    &decay &\FCind{/GCPHYS/}&1\\
\tt DRAY  &1   &I&\tt IDRAY    &$\delta$-ray &\FCind{/GCPHYS/}&1\\
\end{tabular}

\begin{tabular}{lllllll}
KEY   &N    &I    &VAR  &\parbox[t]{7.5cm}{Short description}
 &\tt COMMON  &\Rind{GINIT} \\
\hline
\tt ERAN  &3   &M&\multicolumn{4}{l}{cross-section tables structure:}      \\
 &    &R  &    \tt EKMIN  & minimum energy for the cross-section tables
&\FCind{/GCMULO/}&$10^{-5}$ \\
 &    &R  &    \tt EKMAX  & maximum energy for the cross-section tables
&\FCind{/GCMULO/}&$10^{4}$ \\
 &    &I  &    \tt NEKBIN & \parbox[t]{7.5cm}{number of logarithmic bins 
for cross-section tables}
&\FCind{/GCMULO/}&$90$ \\
\tt HADR  &1   &I&\tt IHADR    &hadronic process &\FCind{/GCPHYS/}&1\\
\tt LABS  &1   &I&\tt ILABS    &\v{C}erenkov light absorbtion 
& \FCind{/GCPHYS/} &0\\
\tt LOSS  &1   &I&\tt ILOSS    &energy loss&
   \FCind{/GCPHYS/} &2\\
\tt MULS  &1   &I&\tt IMULS    &multiple scattering &\FCind{/GCPHYS/}&1\\
\tt MUNU  &1   &I&\tt IMUNU    &muon nuclear interaction
&\FCind{/GCPHYS/}&1\\
\tt PAIR  &1   &I&\tt IPAIR    &pair production &\FCind{/GCPHYS/}&1\\
\tt PFIS  &1   &I&\tt IPFIS    &photofission &\FCind{/GCPHYS/}&0\\
\tt PHOT  &1   &I&\tt IPHOT    &photo electric effect &\FCind{/GCPHYS/}&1\\
\tt RAYL  &1   &I&\tt IRAYL    &Rayleigh scattering &\FCind{/GCPHYS/}&0\\
\tt STRA  &1   &I&\tt ISTRA    &energy fluctuation model&\FCind{/GCPHYS/}&0\\
\tt SYNC  &1   &I&\tt ISYNC    &synchrotron radiation
generation &\FCind{/GCPHYS/}&0\\
\end{tabular}
\subsection{Debug and I/O operations}
\begin{tabular}{lllllll}
KEY   &N    &I    &VAR  &\makebox[7.5cm][l]{Short description}
 &\tt COMMON  &\Rind{GINIT} \\
\hline
\tt DEBU &3 &M &\tt IDEMIN&\parbox[t]{7.5cm}{first event to debug.} 
&\FCind{/GCFLAG/}&0\\
&    &  &  \tt IDEMAX&last event to debug    &\FCind{/GCFLAG/}&0\\
 
&    &  &  \tt ITEST&print control frequency&\FCind{/GCFLAG/}&0\\
\tt GET  &20&M &\tt LGET &\parbox[t]{7.5cm}{
{\tt NGET} names of data structures to
fetch (see \bf Note)}&\FCind{/GCLIST/}& {\tt Blank} \\
\tt PRIN &20 &M & \tt LPRIN&\parbox[t]{7.5cm}{
{\tt NPRIN} user keywords to print data
structure (see \bf Note)}&\FCind{/GCLIST/}& {\tt Blank} \\
\tt RGET  &20&M&\tt LRGET&\parbox[t]{7.5cm}{
{\tt NRGET} names of data structures to fetch
from RZ files (see \bf Note)}&\FCind{/GCRZ/} & {\tt Blank} \\
\tt RSAV &20&M&\tt LRSAVE&\parbox[t]{7.5cm}{
{\tt NRSAVE} names of data structures to save
from RZ files (see \bf Note)}&\FCind{/GCRZ/} & {\tt Blank} \\
\tt SAVE &20&M&\tt LSAVE&\parbox[t]{7.5cm}{
{\tt NSAVE} names of data structures to
save (see \bf Note)}&\FCind{/GCLIST/} & {\tt Blank} \\
\tt SWIT &10&I&\tt ISWIT &user flags for debug&\FCind{/GCFLAG/}&0\\
\end{tabular}

{\bf Note:} the user data records for I/O have no effect on the {\tt GEANT}
system, and the user is supposed to analyse them at run time and take
corresponding action. For instance, a use of the {\tt PRIN} data record
could be the following:
\begin{verbatim}
      CALL GLOOK('VOLU',LPRIN,NPRIN,IPRES)
      IF(IPRES.NE.0) THEN
         CALL GPVOLU(0)
      ENDIF
\end{verbatim}

All the names quoted here are given as 4-character strings in input
and their ASCII equivalent is read into the corresponding variable. The same
applies to the user lists of the following section.

\subsection{User applications}
\begin{tabular}{lllllll}
KEY   &N    &I    &VAR  &\parbox[t]{7.5cm}{Short description}
 &\tt COMMON  &\Rind{GINIT} \\
\hline
\tt KINE  &11   &M    &\tt IKINE   &user flag           &\FCind{/GCKINE/}   &0\\
 
&     &     &     \tt PKINE    &10 user words
    &\FCind{/GCKINE/}    & $10^{10}$\\
\tt SETS  &20   &M    &\tt LSETS    &user words for detector sets
    &\FCind{/GCLIST/}&{\tt Blank}\\
\tt STAT  &20   &M    &\tt LSTAT    &user words to control statistics
&\FCind{/GCLIST/}&{\tt Blank}\\
\tt PLOT  &20   &M    & \tt LPLOT    &user words to control
    plots&\FCind{/GCLIST/}&{\tt Blank}\\
\tt GEOM  &20   &M    &\tt LGEOM    &user words to control geometry
    setup&\FCind{/GCLIST/}&{\tt Blank}\\
\tt VIEW  &20   &M    &\tt LVIEW    &user words to control view banks&
    \FCind{/GCLIST/}&{\tt Blank}\\
\end{tabular}

See note in the previous section on the use of these data records.

\subsection{Lund event generation}
\begin{tabular}{lllllll}
KEY   &N    &I    &VAR  &\makebox[7.5cm][l]{Short description}
 &\tt COMMON  &\Rind{GINIT} \\
\hline
\tt LUND  &2    &M   &\tt IFLUND  &flavour code  (See \FCind{/GLUNDI/})&
\FCind{/GCLUND/}& 0\\
 
&     &     R   &\tt ECLUND  &total CMS energy      &\FCind{/GCLUND/}     &94\\
\end{tabular}

\subsection{Scan geometry control}

The scan geometry has been introduced in {\tt GEANT} version 3.15 with the
idea of developing a general parametrisation scheme. While the scan
geometry can already be built, the parametrisation scheme has not yet
been developed, so the following data records have to be considered as a
{\it forward compatibility} feature.

\begin{tabular}{lllllll}
KEY   &N    &I    &VAR  &\makebox[7.5cm][l]{Short description}
 &\tt COMMON  &\Rind{GINIT} \\
\hline
\tt PCUT &  6 & M & \multicolumn{4}{l}{parametrisation control} \\
& & & \tt IPARAM & parametrisation flag & \FCind{/GCPARM/} & 0 \\
& & & \tt PCUTGA & threshold for gammas & \FCind{/GCPARM/} & 0 \\
& & & \tt PCUTEL & threshold for \Pem\Pep & \FCind{/GCPARM/} & 0 \\
& & & \tt PCUTNE & threshold for neutral hadrons& \FCind{/GCPARM/} & 0 \\
& & & \tt PCUTHA & threshold for charged hadrons & \FCind{/GCPARM/} & 0 \\
& & & \tt PCUTMU & threshold for muons& \FCind{/GCPARM/} & 0 \\
\tt PNUM &  2 & I & \multicolumn{4}{l}{parametrisation stack size} \\
& & & \tt MPSTAK & size of the parametrisation stack & \FCind{/GCPARM/} & 1000 \\
& & & \tt NPGENE & number of pseudo-particles generated & \FCind{/GCPARM/} & 20  \\
\tt SCAN &  8 & M & \multicolumn{4}{l}{scan geometry control} \\
& & & \tt SCANFL & scan processing flag ({\tt LOGICAL}) & \FCind{/GCSCAN/} 
& \tt .FALSE.\\
& & & \tt NPHI &  number of divisions in $\phi$  & \FCind{/GCSCAN/} & 90\\
& & & \tt PHIMIN &  minimum value of $\phi$      & \FCind{/GCSCAN/} &  0\\
& & & \tt PHIMAX &  maximum value of $\phi$      & \FCind{/GCSCAN/} &  360\\
& & & \tt NTETA  &  number of divisions in $\theta$  & \FCind{/GCSCAN/} &  90\\
& & & \tt TETMIN &  minimum value of $\theta$      & \FCind{/GCSCAN/} & 0\\
& & & \tt TETMAX &  maximum value of $\theta$      & \FCind{/GCSCAN/} & 180\\
& & & \tt MODTET &  type of $\theta$ division      & \FCind{/GCSCAN/} &  1\\
\tt SCAL & 32 & M & \tt ISLIST & list of volumes to be {\it scanned} &
\FCind{/GCSCAN/} & \tt Blank \\
\end{tabular}

\begin{tabular}{lllllll}
KEY   &N    &I    &VAR  &\makebox[7.5cm][l]{Short description}
 &\tt COMMON  &\Rind{GINIT} \\
\hline
\tt SCAP &  6 & R & \multicolumn{4}{l}{scan parameters} \\
& & & \tt VX   & scan vertex X coordinate & \FCind{/GCSCAN/} &  0 \\
& & & \tt VY   & scan vertex Y coordinate & \FCind{/GCSCAN/} &  0 \\
& & & \tt VZ   & scan vertex Z coordinate & \FCind{/GCSCAN/} &  0 \\
& & & \tt FACTX0 & scale factor for SX0 & \FCind{/GCSCAN/} &  100 \\
& & & \tt FACTL & scale factor for SL & \FCind{/GCSCAN/} &  1000 \\
& & & \tt FACTR & scale factor for R & \FCind{/GCSCAN/} &  100
\end{tabular}
\subsection{Landau fluctuations versus $\delta$-rays}
In order to avoid double counting between energy loss fluctuations
({\tt ILOSS=2}) and generation of $\delta$-rays {\tt IDRAY=1},
if {\tt ILOSS = 2}
the default value for $\delta$-ray generation is set to {\tt 0} and
it cannot be changed.
The different cases are summarised in the table below.
\begin{center}
\begin{tabular}{|l|l|l|l|}
\hline
                 & Full fluctuations&Restricted fluctuations&No fluctuations \\
                 & \tt ILOSS = 2 (D)   &\tt ILOSS = 1 or 3  & \tt ILOSS = 4 \\
\hline
\tt IDRAY        &\tt 0            &\tt 1          & \tt 1             \\
\tt DCUTE        &\tt 10 TeV       &\tt CUTELE     & \tt CUTELE      \\
\tt DCUTM \quad  &\tt 10 TeV \quad &\tt CUTELE     & \tt CUTELE      \\
\hline
\end{tabular}\end{center}
