%%%%%%%%%%%%%%%%%%%%%%%%%%%%%%%%%%%%%%%%%%%%%%%%%%%%%%%%%%%%%%%%%%%
%                                                                 %
%  GEANT manual in LaTeX form                                     %
%                                                                 %
%  Michel Goossens (for translation into LaTeX)                   %
%  Version 1.00                                                   %
%  Last Mod. Jan 24 1991  1300   MG + IB                          %
%                                                                 %
%%%%%%%%%%%%%%%%%%%%%%%%%%%%%%%%%%%%%%%%%%%%%%%%%%%%%%%%%%%%%%%%%%%
\Authors {R.Brun, M.Maire}     \Origin{M.Maire}
\Version{Geant 3.15}\Routid{CONS103}
\Submitted{05/05/86}    \Revised{10/12/92}
\Makehead{Print  Material Cross  sections}
\Shubr{GPRMAT}{(IMATE,IPART,MECA,KDIM,TKIN)}
 
User routine to
{\bf print} and {\bf interpolate} the $dE/dx$ and cross-sections
tabulated in {\tt JMATE} banks corresponding to :
material {\tt IMATE}, particle {\tt IPART}, mechanism name {\tt MECA},
kinetic energies {\tt TKIN}.
 
The {\tt MECA}nism name can be :
\begin{center}
{\tt
'LOSS' 'PHOT' 'ANNI' 'COMP' 'MUNU' 'BREM'  \\
'PAIR' 'DRAY' 'PFIS' 'HADG' 'HADF' 'ALL'
}
\end{center}
For hadrons it also computes the total
hadronic cross section from \Rind{GHEISHA}  ({\tt'HADG'}) or
\Rind{FLUKA} ({\tt 'HADF'}) programs.
 
{\bf Input parameters:}
\begin{DL}{MMMM}
\item[IMATE]    ({\tt INTEGER})  GEANT material number
\item[IPART]   ({\tt INTEGER})  GEANT particle number
\item[MECA]    ({\tt CHARACTER*4}) mechanism name of the table(s) to be
printed; if {\tt MECA = 'ALL'}  all the relevant banks for particle
{\tt  IPART} will be printed
\item[KDIM]   ({\tt INTEGER}) dimension of the array {\tt TKIN}
(maximum 100)
\item[TKIN]   ({\tt REAL}) array of kinetic energies of incident particle
(in GeV)
\end{DL}
{\bf Called by:} \Rind{<USER>}
%\end{document}
 
