%%%%%%%%%%%%%%%%%%%%%%%%%%%%%%%%%%%%%%%%%%%%%%%%%%%%%%%%%%%%%%%%%%%
%                                                                 %
%  GEANT manual in LaTeX form                                     %
%                                                                 %
%  Michel Goossens (for translation into LaTeX)                   %
%  Version 1.00                                                   %
%  Last Mod. Jan 24 1991  1300   MG + IB                          %
%                                                                 %
%%%%%%%%%%%%%%%%%%%%%%%%%%%%%%%%%%%%%%%%%%%%%%%%%%%%%%%%%%%%%%%%%%%
\Origin{G.N.Patrick, L.Urb\'{a}n}
\Submitted{12.12.84}\Revised{16.12.93}
\Version{Geant 3.16}\Routid{PHYS211}
\Makehead{Simulation of e-e+ pair production by photons}

\section{Subroutines}
\Shubr{GPAIRG}{}

\Rind{GPAIRG} generates \Pem\Pep-pair production by photons.
It uses a modified version
of the random number techniques of Butcher and Messel \cite{bib-BUTC}
to sample the secondary electron/positron energies from the Coulomb
corrected Bethe-Heitler differential cross-section.
For the angular distribution of the pair, it calls the function
\Rind{GBTETH}.

Input:    via COMMON \FCind{/GCTRAK/}

Output:   via COMMON \FCind{/GCKING/}

\Rind{GPAIRG} is called from \Rind{GTGAMA},
when the photon reaches its pair production point.

\Sfunc{GBTETH}{VALUE = GBTETH(ENER,PARTM,EFRAC)}
\Rind{GBTETH} calculates the angular distribution of the \Pem\Pep-pair
for pair production and of the photon for bremsstrahlung.
In case of \Pem\Pep-pair production by photons, it gives the
scaled angle for an electron (mass {\tt PARTM}) of energy {\tt ENER}
which is {\tt EFRAC} times the initial energy of the photon.
\Rind{GBTETH} is called by \Rind{GPAIRG}.

\section{ Method}

\subsection{MonteCarlo technique}
We give a very short summary of the random number technique used here
\latexhtml{\cite{bib-BUTC,bib-HAMM}}{\cite{bib-BUTC}, \cite{bib-HAMM}}.
The method is a combination of the composition and
rejection Monte Carlo methods. Suppose we wish to sample $x$ from
the distribution $f(x)$ and the
(normalised) probability density function can be written as
\begin{equation}
f(x)  = \sum_{i=1}^{n}\alpha _{i} f _{i} (x) g_{i} (x)
\end{equation}

where $f _i (x)$ are normalised density functions, $\alpha _i > 0 $ and
$ 0 \leq g_i (x) \leq 1 $.

According to this method, $x$ can sampled in the following way:
\begin{enumerate}
\item
select a random integer $i$ such that $(1\leq i \leq n)$
with probability proportional to $\alpha_i $
\item
select a value $x'$ from the distribution $f_i (x)$
\item
calculate $g_i (x')$ and accept $x = x'$ with probability
$g_i (x')$;
\item if $x'$ is rejected restart from step 1.
\end{enumerate}
It can be shown that this scheme is correct and the mean
number of tries to accept a value is $ \sum_{i}\alpha_i $.

In practice we have a good method of sampling from the distribution $f(x)$, if
\begin{itemize}
\item all the subdistributions $ f_i (x)$ can be sampled easily;
\item the rejection functions $ g_i(x)$ can be evaluated easily/quickly;
\item the mean number of tries is not too large.
\end{itemize}
Thus the different possible decompositions of the distribution
$f(x)$ are not equivalent from the practical point of view (e.g. they
can be very different in computational speed) and it can be very useful
to optimise the decomposition.
A remark of practical importance is that if our distribution is not
normalised ($\int f(x)dx=C>0; C\approx 1$), the method can be used in the same
manner, the only difference is that the mean number of tries in this
case is given by  $\sum_ {i}\alpha_i/C$ .

\subsection{Differential cross-section for pair production}
The Bethe-Heitler differential cross-section with the Coulomb correction for
a photon of energy $E$ to produce a \Pem\Pep-pair with one of the
particles having
energy  $ \epsilon E$  ($ \epsilon$ is the fraction of the
photon energy carried by one particle of the pair)
is given as in \cite{bib-EGS3}:
\begin{equation}
\frac{d\sigma (Z,E,\epsilon )}{d \epsilon} =
     \frac{r_0^2\alpha Z [ Z +\xi (Z) ]}{E^2}
     \left\{[\epsilon^ 2 + ( 1 -\epsilon ) ^ 2 ] [ \Phi _1 (\delta ) -
     \frac{F (Z)}{2} ] +
     \frac{2}{3}\epsilon (1-\epsilon ) [\Phi_2 (\delta ) - 
     \frac{F (Z)}{2} ] \right\}
\end{equation}
where $\Phi_i(\delta)$ are the screening functions
depending on the screening variable $\delta$\\
\[ \begin{array}{LL}

\delta  =  \frac { 136m}{ Z^{1/3}E}
            \frac{ 1}{ \epsilon(1-\epsilon)}
              &        m = \mbox{electron mass} \\
& \\
\hspace{-.2cm}
\left.\begin{array}{L}
\Phi_1(\delta)  =  20.867 - 3.242\delta +0.625\delta^2  \\
\Phi_2(\delta)  =  20.209 - 1.930\delta - 0.086\delta^2  \\
\end{array}        \right\} & \; \delta\leq 1 \\
\Phi_1(\delta) = \Phi_2(\delta) =
21.12 - 4.184 \ln(\delta+0.952)
                    & \; \delta > 1  \\
&\\
F(Z)  = \left\{\begin{array}{L}
        8/3 \ln Z          \\
        8/3 \ln Z+8f_c(Z)  \\
        \end{array}\right.
&  \begin{array}{L}
E<0.05 \; GeV      \\
E\geq 0.05 \; GeV  \\
\end{array}  \\
\\
\xi(Z)=\frac{\ln {(1440/Z^{2/3})}}{\ln
{(183/Z^{1/3})}-f_c(Z)}  & \\
\\
f_c(Z)
& \mbox{the Coulomb correction function}
\end{array} \]

\[ \begin{array}{LCL}
f_c(Z) & = & a(1/(1+a)+0.20206-0.0369a+0.0083*a^2-0.002a^3)\\
    a   & = & (\alpha*Z)^2                                     \\
\alpha  & = & 1/137
\end{array} \]

The kinematical range for the variable $\epsilon$ is
\begin{equation}
\frac {m}{E} \leq \epsilon  \leq 1 - \frac{m}{E}
\end{equation}
The cross-section is symmetric with respect to
the interchange of $\epsilon$ with
$1-\epsilon $, so we can restrict $\epsilon $ to lie in the range
 $\epsilon_0  = m/E\leq \epsilon \leq 1/2 $

After some algebraic manipulations we can decompose the cross-section
as (note: the normalisation is not important):
\[ \begin{array}{LCLLCL}
\frac{d\sigma}{d\epsilon} & = &
\sum^{2}_{i=1}\alpha_i f_i (\epsilon) g_i (\epsilon) \\
\end{array} \]

where

\[ \begin{array}{LCLLCL}
\alpha_1 & = & \frac{(0.5 - \epsilon_0)^2}{3} F_{10} &
\alpha_2 & = & \frac{1}{2}F_{20}    \\
\\
f_1 (\epsilon)& =&
\frac{ 3}{ (0.5 - \epsilon_0)^3}(\epsilon-0.5)^2  &
f_2 (\epsilon)& =& \frac{ 1}{ 0.5-\epsilon_0}    \\
g_1 (\epsilon)&     =&  F_1 / F_{10}      &
g_2 (\epsilon)&  = & F_2 / F_{20}        \\
\end{array} \]

and

\[ \begin{array}{LCLLCL}
F_1   &   = & F_1(\delta) = 3\Phi_1(\delta)-\Phi_2(\delta)-F(Z)       &
F_{10}  & = & F_1(\delta_{min})                                          \\
F_2  &  = & F_2(\delta) = \frac{3}{2}\Phi_1(\delta)+
                        \frac{1}{2}\Phi_2(\delta)-F(Z)            &
F_{20}  & = & F_2(\delta_{min})                                          \\
\\
\delta_{min} & = & 4\frac{ 136m}{ Z^{\frac{1}{3}}3E} & \\
\end{array} \]

$\delta_{min}$ is the minimal value of the
screening variable $\delta$. It can be
seen that the functions $f_i(\epsilon)$ are normalised and
that the functions $ g_i(\epsilon)$ are ``valid" rejection functions.

Therefore, if $r_i$ are uniformly distributed random numbers
($0\leq r_i  \leq 1$), we can sample the
$\epsilon $ ($x$ in the program) in the following way:
\begin{enumerate}
\item
select $i$ to be 1 or 2 according to the following ratio:
\[
\mbox{\tt BPAR}  = \frac{ \alpha_1 }{\alpha_1 +\alpha_2}
\]
If $r_0 < \mbox{\tt BPAR}$ then $i=1$, otherwise if $r_0\geq \mbox{\tt BPAR}$ 
$i=2$;
\item
sample $\epsilon$  from $f_1(\epsilon)$. This can be
performed by the following expressions:

\[
\epsilon = \left\{
\begin{array}{Ll}
0.5 - (0.5 - \epsilon_0 ) {r_1}^\frac{1}{3} &
\mbox{when $i=1$} \\
\epsilon_0 + \left(\frac{1}{2} - \epsilon_0\right) r_1 &
\mbox{when $i=2$}
\end{array} \right.
\]
\item calculate the rejection function $g_i(\epsilon)$.
If $r_2\leq g_{i}(\epsilon)$, accept $ \epsilon$, otherwise return
to step 1.
\end{enumerate}

It should be mentioned that we need a step just after sampling $ \epsilon $
in the step 2, because the cross-section formula becomes
negative at large $ \delta$
and this imposes an upper limit
for $ \delta$;
\[
\delta_{max} = \exp \left[\frac{42.24-F(Z)}{8.368}\right] - 0.952
\]

If we get a $\delta$ value using the sampled $\epsilon$ such that
$\delta > \delta_{max} $ , we
have to start again from the step 1.
After the successful sampling of $\epsilon $, \Rind{GPAIRG}
generates the polar angles
of the electron with respect to an axis defined along the direction
of the parent photon.
The electron and the positron are assumed to have
a symmetric angular distribution.
The energy-angle distribution is given by
Tsai~\latexhtml{\cite{bib-TSAI,bib-TSAI-err}}%
               {\cite{bib-TSAI}, \cite{bib-TSAI-err}}
as following:
\begin{eqnarray}
\frac{d \sigma}{dpd \Omega}
& = & \frac{2 \alpha^2e^2}{\pi k m^4}
      \left\{ \left[ \frac{2x(1-x)}{(1+l)^2}-\frac{12lx(1-x)}{(1+l)^4}\right]
      (Z^{2}+Z) \nonumber  \right. \\
&   & \mbox{} + \left. \left[ \frac{2x^2-2x+1}{(1+l)^2} +
      \frac{4lx(1-x)}{(1+l)^4}
      \right]
      \left[ X-2Z^{2}f((\alpha Z)^{2})\right]
      \right\} \nonumber 
\end{eqnarray}

where $k$ is the photon energy, $p$ and $E$  are the momentum and
the energy of the electron of the $e^+e^-$-pair,
$x=E/k$  and $l = E^{2} \theta^{2}/m^{2}$. This distribution
is quite complicated to sample and, furthermore, considered as
function of the variable
$u = E \theta/m$, it shows a very weak dependence on $Z$, $E$, $k$,
$x = E/k$. Thus, the distribution can be approximated by a function
\begin{equation}
f(u) = C \left( u e^{-au} + d u e^{-3au} \right)
\end{equation}
where
\begin{eqnarray*}
C & = & \frac{9a^{2}}{9 + d} \\
a & = & 0.625 \\
d & = & 27.0 
\end{eqnarray*}
The sampling of the function $f(u)$ can be done in the following way
($r_{i}$ are uniformly distributed random numbers
in [0,1]):
\begin{enumerate}
\item choose between $u e^{-au}$ and $d u e^{-3au}$, with relative
probability given by $9/(9+d)$ and $d/(9+d)$ respectively;
if $r_{1} < 9/(9+d)$ then $b=a$, else $b=3a$;
\item sample $u e^{-bu}$, $u=-(\ln r_{2} + \ln r_{3})/b$.
\end{enumerate}

The azimuthal angle, $\phi$, is generated isotropically.

This information together with the momentum conservation is used to calculate
the momentum vectors of both decay products and to transform
them to the {\tt GEANT} coordinate system.
The choice of which particle in the pair is the electron/positron is
made randomly.

\subsection{Restrictions}
\begin{enumerate}
\item
Effects due to the breakdown of Born approximation at low energies
are ignored (but the Coulomb correction is now included):
\item
as suggested by Ford and Nelson~\cite{bib-EGS4}, for very low energy 
photons ($E\leq 2.1 MeV$) the electron energy is approximated by sampling 
from a uniform distribution over the interval $ m\rightarrow 1/(2E)$.
The reason for this suggestion is that the sampling method used in 
EGS and in the earlier {\tt GEANT} versions becomes progressively more 
inefficient as the pair threshold is approached. This is not true for 
the sampling method outlined above (the efficiency of the method 
practically does not depend on the photon energy), but we
have chosen to keep this approximation;
\item
target materials composed of compounds or mixtures are treated identically
to chemical elements (this is not the case when computing the mean free path!)
using the effective atomic number computed in the routine \Rind{GSMIXT}. 
It can be shown that the error of this type of treatment is small and can 
be neglected;
\item
the differential cross-section implicitly accounts for pair production
in both nuclear and atomic electron fields. However, triplet production
is not generated, and the recoil momentum of the target nucleus/electron
is assumed to be zero.
\end{enumerate}
