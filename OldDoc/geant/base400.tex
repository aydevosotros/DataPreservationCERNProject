%%%%%%%%%%%%%%%%%%%%%%%%%%%%%%%%%%%%%%%%%%%%%%%%%%%%%%%%%%%%%%%%%%%
%                                                                 %
%  GEANT manual in LaTeX form                                     %
%                                                                 %
%  Version 1.00                                                   %
%  Last Mod.  8 June 1993 1300   MG                               %
%                                                                 %
%%%%%%%%%%%%%%%%%%%%%%%%%%%%%%%%%%%%%%%%%%%%%%%%%%%%%%%%%%%%%%%%%%%
\Origin{R.Brun, F.Carena}
\Submitted{01.10.84}  \Revised{16.12.93}
\Version{Geant 3.16}\Routid{BASE400}
\Makehead{Debugging facilities}
The flags {\tt IDEBUG, ITEST} and {\tt ISWIT(1-10)} are available to
in the
common \FCind{/GCFLAG/} for debug control {\tt [BASE030]}.
The array {\tt ISWIT} is filled through the data record
{\tt SWIT}.
Some flags are used by \Rind{GHEISHA} {\tt [PHYS510]} and
by the routine \Rind{GDEBUG}.

The flag {\tt IDEBUG} is set to 1 in \Rind{GTRIGI} for the events
with sequence number from {\tt IDEMIN} to {\tt IDEMAX}, as specified by
the user on the data record {\tt DEBU}.
If {\tt IDEMIN} is negative, debug is
activated also in the initialisation phase.

The flag {\tt ITEST}, set by the user via the data
record {\tt DEBU}, is also used by \Rind{GTRIGI}.
The sequence number, the event number and the random numbers seeds are
printed at the beginning of each event every {\tt ITEST} from
{\tt IDEMIN} to {\tt IDEMAX}.

\section{Debug of data structures}
The contents of the data structures can be dumped by the routine
\Shubr{GPRINT}{(CHNAME,NUMB)}
\begin{DLtt}{MMMMMMMM}
\item[CHNAME] ({\tt CHARACTER*4}) name of a top level data structure;
\item[NUMB] ({\tt INTEGER}) number of the substructure to be printed, 0 for all.
\end{DLtt}

Examples
\begin{itemize}
\item{\tt CALL GPRINT('KINE',0)} prints all banks {\tt JKINE};
\item{\tt CALL GPRINT('KINE',8)} prints {\tt JKINE} bank for track 8;
\item{\tt CALL GPRINT('VOLU',0)} prints all existing volumes.
\end{itemize}
The following names are recognised:
\begin{center}\tt
DIGI,HITS,KINE,MATE,VOLU,ROTM,SETS,TMED,PART,VERT,JXYZ
\end{center}
\Rind{GPRINT} calls selectively the routines:

\begin{center}
\begin{tabular}{llll}
\Rind{GPDIGI}{\tt ('*','*')} &
\Rind{GPHITS}{\tt ('*','*')} &
\Rind{GPKINE}{\tt (NUMB)}    & 
\Rind{GPMATE}{\tt (NUMB)}    \\
\Rind{GPVOLU}{\tt (NUMB)}    &
\Rind{GPROTM}{\tt (NUMB)}    &
\Rind{GPSETS}{\tt ('*','*')} &
\Rind{GPTMED}{\tt (NUMB)}    \\
\Rind{GPPART}{\tt (NUMB)}    &
\Rind{GPVERT}{\tt (NUMB)}    &
\Rind{GPJXYZ}{\tt (NUMB)}    &
\end{tabular}
\end{center}

These routines can be called directly by the user. In case of {\tt SETS},
{\tt HITS} and {\tt DIGI} the content of all detectors of all sets will
be printed, so {\tt NUMB} is irrelevant.

\section{Debug of events}
The development of an event can be followed via the routine:
\Shubr{GDEBUG}{}

which operates under the control of the {\tt ISWIT} array. It is the
user responsibility to call this routine from \Rind{GUSTEP}. If the
{\tt DEBUG} flag is active, the routine will perform as follows:
\begin{DLtt}{MMMMM}
\item[ISWIT(1)] ~

\begin{DLtt}{MMM}
\item[2]the content of the temporary stack for secondaries in the
common \FCind{/GCKING/} is printed;
\end{DLtt}
\item[ISWIT(2)] ~

\begin{DLtt}{MMM}
\item[1]the current point of the track is stored in the {\tt JDXYZ}
bank via the routine \Rind{GSXYZ};
\item[2]the current information on the track is printed via the
routine \Rind{GPCXYZ};
\item[3]the current step is drawn via the routine
\Rind{GDCXYZ};
\item[4]the current point of the track is stored in the {\tt JDXYZ}
bank via the routine \Rind{GSXYZ}. When the particle stops the track
is drawn via the routine \Rind{GDTRAK}
and the space occupied by the track in the structure {\tt JDXYZ}
released;
\end{DLtt}
\item[ISWIT(3)] ~

\begin{DLtt}{MMM}
\item[1]the current point of the track is stored in the {\tt JDXYZ}
bank via the routine \Rind{GSXYZ}.
\end{DLtt}
\end{DLtt}
