%%%%%%%%%%%%%%%%%%%%%%%%%%%%%%%%%%%%%%%%%%%%%%%%%%%%%%%%%%%%%%%%%%%
%                                                                 %
%  GEANT manual in LaTeX form                              %
%                                                                 %
%  Michel Goossens (for translation into LaTeX)                   %
%  Version 1.00                                                   %
%  Last Mod. Jan 24 1991  1300   MG + IB                          %
%                                                                 %
%%%%%%%%%%%%%%%%%%%%%%%%%%%%%%%%%%%%%%%%%%%%%%%%%%%%%%%%%%%%%%%%%%%
\Origin{L. Urb\'{a}n}
\Version{Geant 3.16}\Routid{PHYS340}
\Submitted{28.05.86} \Revised{16.12.93}
\Makehead{Total cross-section and energy loss for bremsstrahlung
by e-e+}

\section{Subroutines}
\Shubr{GBRELA}{}
\Rind{GBRELA} fills the tables for the energy loss of electrons,
positrons and muons due to
bremsstrahlung at initialisation time for different materials.
The energy
binning is set within the array \texttt{ELOW} (common \FCind{/CGMULO/}) in the
routine \Rind{GPHYSI}. In the tables, the dE/dx due to bremsstrahlung
is summed with that due to the ionisation. For energy loss
of electrons and positrons, \Rind{GBRELA} calls the function \Rind{GBRELE}. 
Following pointers are used:
\begin{DLtt}{MMMMMMMMMMMMMMMMM}
\item[JMA = LQ(JMATE-I)] pointer to the I$^{th}$ material;
\item[JEL1 = LQ(JMA-1)]  pointer to $dE/dx$ for $\Pem$;
\item[JEL1+NEK1]           pointer to $dE/dx$ for $\Pep$.
\end{DLtt}
\Rind{GBRELA} is called at initialisation time by \Rind{GPHYSI}.
\Sfunc{GBRELE}{VALUE = GBRELE(ZZ,T,BCUT)}
\Rind{GBRELE} calculates the energy loss due to bremsstrahlung
of an electron with kinetic energy {\tt T} in material 
with atomic number {\tt ZZ}.
It is called by \Rind{GBRELA} and for energies below the cut
{\tt BCUT} it adds the contribution of bremsstrahlung to. 
Above this cut, the bremsstrahlung process is simulated
explicitly (see {\tt [PHYS341]}) and tabulation of these continuous losses
is not needed. \Rind{GBRELE} is called by \Rind{GBRELA}.
\Sfunc{GBFLOS}{VALUE = GBFLOS(T,C)}
\Rind{GBFLOS} calculates a weight factor for the positron continuous
bremsstrahlung energy loss. {\tt T} is the kinetic energy in GeV of the 
positron and C is the energy cut for bremsstrahlung ({\tt BCUTE}). The
value is the ratio of the energy loss due to bremsstrahlung of the
positron to that of the electron so that:
$<${\it positron loss}$>$ = \Rind{GBFLOS} $\times$ $<${\it electron loss}$>$.
\Rind{GBFLOS} is called by \Rind{GBRELA}.
\Shubr{GBRSGA}{} 
\Rind{GBRSGA} calculates the total cross-section for bremsstrahlung
in all materials. It tabulates the mean free path, $\lambda = 
\frac{1}{\Sigma}$ (in cm) as a function of medium and energy. The energy
binning is set within the array {\tt ELOW} (common \FCind{/CGMULO/}) in the
routine \Rind{GPHYSI}. The following pointers are used:
\begin{DLtt}{MMMMMMMMMMMMMMMMMMN}
\item[JMA = LQ(JMATE-I)]pointer to the I$^{th}$ material
\item[JBREM = LQ(JMA-9)]pointer to bremsstrahlung cross-sections
\item[JBREM            ]pointer for  $\Pem$
\item[JBREM+NEK1       ]pointer for  $\Pep$
\item[JBREM+2*NEK1     ]pointer for $\mu^+/\mu^-$
\end{DLtt}
\Rind{GBRSGA} is called at initialisation time by \Rind{GPHYSI}.
\Sfunc{GBRSGE}{VALUE = GBRSGE(ZZ,T,BCUT)}
\Rind{GBRSGE} calculates the total cross-section of 
bremsstrahlung of an electron with kinetic energy {\tt T} 
in material with atomic number {\tt ZZ}.
It is called by \Rind{GBRSGA}. For kinetic energies 
which are below the cut {\tt BCUT} or
for which bremsstrahlung process is not simulated
explicitly (see {\tt [PHYS341]}) it returns 0.  
\Sfunc{GBFSIG}{VALUE = GBFSIG(T,C)}
\Rind{GBFSIG} calculates a 
weight factor for the positron discrete (hard) bremsstrahlung cross section.
{\tt T} is the kinetic energy in GeV of the 
positron and C is the energy cut for bremsstrahlung ({\tt BCUTE}). The
value returned is the ratio of the positron bremsstrahlung cross-section
to that of the electron so that:
<{\it positron cross-section}> = \Rind{GBFSIG} $\times$ <{\it electron 
cross-section}>.  \Rind{GBFSIG} is called by \Rind{GBRSGA}.
\section{Method}
Let's call $d\sigma(Z,T,k)/dk$ the differential cross-section for
production of a photon of energy $k$ by an electron of kinetic energy
$T$ in the field of an atom of charge $Z$, and
$k_c$ the energy cut-off below which the soft photons are treated
as continuous energy loss ({\tt BCUTE} in the program).
Then the mean value of the energy lost by the electron due
to soft photons is
\begin{equation}
  E_{Loss}^{brem} (Z,T,k_c ) = 
\int_{0}^{k_ c}k\frac{d \sigma (Z,T,k)}{dk}dk
\end{equation}
whereas the total cross-section for the emission of 
a photon of energy larger than $k_c$ is
\begin{equation}
 \sigma_{brem} (Z,T,k_c ) = \int_{k_c}^{T}\frac{d \sigma (Z,T,k)}{dk} dk
\end{equation}
Many theories of the bremsstrahlung process exist, each with its own
limitations and regions of applicability. Perhaps the best synthesis
of these theories can be found in the paper of S.M. Seltzer and M.J.
Berger \cite{bib-SEL1}. The authors give a tabulation of the
bremsstrahlung cross-section $d\sigma/dk$ differential in the photon
energy $k$, for electrons with kinetic energies $T$ from 1 keV to 10
GeV. For electron energies above 10 GeV the screened Bethe-Heitler
differential cross-section can be used
\latexhtml{\cite{bib-EGS3,bib-NELS}}{\cite{bib-EGS3}, \cite{bib-NELS}}
together with the Midgal 
corrections~\latexhtml{\cite{bib-MES1,bib-MIGD}}%
                      {\cite{bib-MES1}, \cite{bib-MIGD}}.  
The first of the two Migdal corrections is
important for very high electron energies only ($T \geq$ 1 TeV) and
has the effect of reducing the cross-section.  The second Migdal
correction is effective even at ``ordinary'' energies (100 MeV -- 1
GeV) and it decreases the differential cross-section at photon
energies below a certain fraction of the incident electron energy
($d\sigma/dk$ decreases significantly if $ k/T \leq 10^{-4}$.)

\subsection{Parameterisation of energy loss and total cross-section} 
Using the tabulated cross-section values of Seltzer and Berger together
with the Migdal corrected Bethe-Heitler formula we have computed
$\sigma(Z,T,k_c ) $
and we have used these computed values as ``data points''
in the fitting procedure.
Calculating the ``low energy'' ($T\leq 10$ GeV) data we have applied the
second Midgal correction to the results of Seltzer and Berger.
We have chosen the parameterisations:
\begin{equation}
\label{eq:phys340-1}
\sigma (Z,T,k_c ) =\frac{Z(Z+\xi_{\sigma} )(T+m)^2}
     {T(T+2m)}[ \ln (T/k_c )]^\alpha F_{\sigma} (Z,X,Y)
\mbox{\quad(barn)}\\
\end{equation}
and
\begin{equation}
\label{eq:phys340-2}
E_{Loss}^{brem} (Z,T,k_c ) =\frac{Z(Z+ \xi_l)(T+m)^2 }
      {(T+2m)}\left[\frac{k_c C_ M }{T}\right]^\beta
      F_{l} (Z,X,Y) \mbox{\quad(GeV barn)}
\end{equation}
where  $m$ is the mass of the electron,
\[
\begin{array}{LLL}
X = \ln(E/m), & Y = \ln (v_{\sigma} E/k_c) &
\mbox{\quad for the total cross-section }\sigma \\
X = \ln(T/m), & Y = \ln(k_c/v_lE)       &
\mbox{\quad for the energy loss }  E_{Loss}^{brem}  \\
\end{array}
\]
with $E=T+m$.
The constants $\xi_{\sigma},\ \xi_l,\ \alpha,\ \beta,\ v_{\sigma},\ v_l$
are parameters to be fitted.
\begin{eqnarray*}
C_M &=&\frac{\Dstm 1}{1+\frac{\Tstm n r_0 \lambdar^2  (T+m)^2}{\Tstm\pi k^2_c}}
\end{eqnarray*}
is the Midgal correction factor, with
\begin{DLtt}{MM}
\item[$r_o$] classical electron radius;
\item[$\lambdar$] reduced electron Compton wavelength;
\item[$n$] electron density in the medium.
\end{DLtt}
The factors $(T+m)^2/T(T+2m)$ and $(T+m)^2/(T+2m)$ come from 
the scaled cross-section computed by Seltzer and Berger:
\[
f(k/T) = \frac{\beta^2}{Z^2 }k\frac{d \sigma}{dk} 
       = \frac{T(T+2m)} {(T+m)^2 Z^2}  k \frac{d \sigma}{dk}
\]
 
The functions $F_i(Z,X,Y)$ ($ i=\sigma,l $) have the form
\begin{equation}
 F_i(Z,X,Y) = F_{i0}(X,Y)+ZF_{i1}(X,Y)
\end{equation}
where $\Dstm F_{ij}(X,Y)$ are polynomials of the variables $\Dstm X,Y$
\begin{eqnarray}
F_{i0}(X,Y) & = &  (C_1+C_2X+\ldots+C_6X^5)+(C_7+C_8X+\ldots+C_{12}X^5)Y
                                                             \nonumber \\
          &   & +(C_{13}+C_{14}X+\ldots + C_{18}X^5)Y^2+\ldots
                +(C_{31}+C_{32}X+\ldots+C_{36}X^5)Y^5                  \\
          &   & \mbox{\hspace{11cm}} \Bbfm Y\leq 0  \Ebfm    \nonumber \\
          & = &  (C_1+C_2X+\ldots+C_6X^5)+(C_7+C_8X+\ldots+C_{12}X^5)
                                                             \nonumber \\
          &   & +(C_{37}+C_{38}X+\ldots+C_{42}X^5)Y^2+\ldots+
                 (C_{55}+C_{56}X+\ldots+C_{60}X^5)Y^5        \nonumber \\
          &   & \mbox{\hspace{11cm}} \Bbfm Y > 0\Ebfm        \nonumber \\
F_{i1}(X,Y) & = &  (C_{61}+C_{62}X+\ldots+C_{65}X^4)
                +(C_{66}+C_{67}X+\ldots+C_{70}X^4)Y          \nonumber \\
          &   & +(C_{71}+C_{72}X+\ldots+C_{75}X^4)Y^2+\ldots
                +(C_{81}+C_{82}X+\ldots+C_{85}X^4) Y^4                 \\
          &   &   \mbox{\hspace{11cm}} \Bbfm Y\leq 0\Ebfm    \nonumber \\
          & = &  (C_{61}+C_{62}X+\ldots+C_{65}X^4)
                +(C_{66}+C_{67}X+\ldots+C_{70}X^4)Y          \nonumber \\
          &   & +(C_{86}+C_{87}X+\ldots+C_{90}X^4)Y^2+\ldots
                +(C_{96}+C_{97}X+\ldots+C_{100}X^4)Y^4       \nonumber  \\
          &   &  \mbox{\hspace{11cm}} \Bbfm Y>0 \Ebfm        \nonumber \\
\end{eqnarray}
$F_{ij}(X,Y)$ denotes in fact a function constructed from two
polynomials
\[
F_{ij}(X,Y) = \left \{
\begin{array}{ll}
F_{ij}^{neg}(X,Y) & \mbox{for\hspace{0.5cm}} Y\leq 0 \\
F_{ij}^{pos}(X,Y) & \mbox{for\hspace{0.5cm}} Y>0   
\end{array} \right .
\]
where the polynomials $F_{ij}$ fulfil the conditions
\[
F_{ij}^{neg}(X,Y)_{Y=0} = F_{ij}^{pos}(X,Y)_{Y=0}   \hspace{2cm}
\left ( \frac{\partial F_{ij}^{neg}}{\partial Y} \right )_{Y=0} =
\left ( \frac{\partial P_{ij}^{pos}}{\partial Y} \right )_{Y=0}
\]
We have computed 4000 ``data points'' in the range
\[
Z = 6  ;  13  ;  29  ;  47  ;  74  ;  92  \hspace{1cm}
10 \: \mbox{keV} \leq T \leq 10 \: \mbox{TeV}  \hspace{1cm}
10 \: \mbox{keV} \leq k_c \leq T  
\]
and we have performed a least-squares fit to determine the parameters.
 
The values of the parameters ($\xi_{\sigma}$, $\alpha$, 
$v_{\sigma}$, $C_i $ for $\sigma$ and
$\xi_l$, $\beta$, $V_l$, $C$ for $E_{Loss}^{brem}$) 
can be found in the {\tt DATA} 
statement within the functions
\Rind{GBRSGE} and \Rind{GBRELE}  which compute the formula (\ref{eq:phys340-1})
and (\ref{eq:phys340-2}) respectively.
 
The errors of the parameterisations (\ref{eq:phys340-1}) and (\ref{eq:phys340-2})
can be estimated as

\begin{eqnarray*}
\frac{\Delta\sigma} {\sigma} & = & \left \{
\begin{array}{llr} 
        12-15\%    & \mbox{for    } & T \leq 1 MeV \\
        \leq 5-6\% & \mbox{for    } & 1 MeV < T \leq 10 TeV
\end{array} 
\right . \\[1cm]
\frac{\Delta E_{Loss}^{brem}}
     {E_{Loss}^{brem}} & = & \left \{
\begin{array}{llr}
        10-15\%    & \mbox{for    } & T \leq1 MeV  \\
        5-6\%      & \mbox{for    } & 1 MeV < T \leq 100 GeV\\
        10\%       & \mbox{for    } & 100 GeV < T \leq10 TeV
\end{array} 
\right . 
\end{eqnarray*}

We have performed a fit to the ``data'' without the Midgal corrections, too.
In this case we used the data of Seltzer and Berger without any
correction for $T \leq 10$ GeV and we used the Bethe-Heitler cross-section
for $T \geq 10$ GeV. The parameterised forms of the cross-section and
energy loss are the same as they were in the first fit (i.e. (\ref{eq:phys340-1})
and (\ref{eq:phys340-2})), only the numerical values of the parameters 
have changed.
These values are in {\tt DATA} statements in the functions \Rind{GBRSGE}
and \Rind{GBRELE} and this second kind of parameterisation can be activated
using the \Rind{PATCHY} switch {\tt +USE,BETHE}. (The two
parameterisations give different results for high electron energy.)

The energy loss due to soft photon bremsstrahlung is tabulated at
initialisation time as a function of the medium and of the energy by
routine \Rind{GBRELA} (see {\tt JMATE} data structure).

The mean free path for discrete bremsstrahlung is tabuled at initialisation 
time as a function
of the medium and of the energy by routine \Rind{GBRSGA} (see
{\tt JMATE} data structure).

\subsection{Corrections for $\Pem/\Pep$ differences}

The radiative energy loss for electrons or positrons is:

\begin{eqnarray*}
-\frac{1}{\rho} \left ( \frac{dE}{dx} \right )_{rad}^{\pm} & = & 
\frac{N_{Av} \alpha r_e^2}{A} (T+m) Z^2 \Phi_{rad}^{\pm}(Z,T) \\
\Phi^{\pm}_{rad}(Z,T) & = & \frac{1}{\alpha r_{e}^2 Z^2 (T+m)}
\int^{T}_{0}{k\frac{d\sigma^{\pm}}{dk}dk}
\end{eqnarray*}

Reference \cite{bib-KIM1} says that: \\
{\it ``The differences between the radiative loss of positrons
and electrons are considerable and cannot be disregarded.

[...] The ratio of the radiative energy loss for positrons
to that for electrons obeys a simple scaling law, [...] is a
function only of the quantity $T/Z^2$''}

In other words:

\begin{eqnarray*}
\eta & = & \frac{\Phi_{rad}^{+}(Z,T)}{\Phi_{rad}^{-}(Z,T)} = 
\eta \left (\frac{T}{Z^2}\right )
\end{eqnarray*}

The authors have calculated this function in the range $10^{-7}
\leq \frac{T}{Z^2} \leq 0.5$ (here the kinetic energy T is 
expressed in MeV). Their {\it data} can be fairly accurately
reproduced using a parametrisation:

\begin{eqnarray*}
\eta & = & \left \{ 
\begin{array}{llr}
0 & \mbox{if   } & x \leq -8 \\
\frac{1}{2} + \frac{1}{\pi} \arctan \left( a_1 x + a_3 x^3
+ a_5 x^5 \right ) & \mbox{if  } & -8 < x < 9 \\
1 & \mbox{if   } & x \geq 9
\end{array}
\right .
\end{eqnarray*}

where:

\begin{eqnarray*}
x & = & \log \left ( C \frac{T}{Z^2} \right ) \mbox{(T in GeV)} \\
C & = & 7.5221 \times 10^{6} \\
a_1 & = & 0.415 \\
a_3 & = & 0.0021 \\
a_5 & = & 0.00054
\end{eqnarray*}

This $\Pem/\Pep$ energy loss difference is not a pure low-energy
phenomenon (at least for high Z), as it can be seen from 
Tables~\ref{tb:phys340-1}, \ref{tb:phys340-2} and \ref{tb:phys340-3}.

\begin{table}[hbt]
\begin{centering}
\begin{tabular}{rr|r|r} \hline
\multicolumn{1}{c}{$\frac{T}{Z^2} (GeV)$}
& \multicolumn{1}{c|}{T} 
& \multicolumn{1}{c|}{$\eta$}
& \multicolumn{1}{c}{$\left ( \frac{rad. \ loss}{total \ loss}
\right )_{e^-}$} \\[3mm] \hline
$10^{-9}$ & $\sim 7 keV$ & $\sim 0.1$ & $\sim 0\%$ \\
$10^{-8}$ & $67 keV $ & $\sim 0.2$ & $\sim 1\%$ \\
$2 \times 10^{-7}$ & $1.35 MeV$ & $\sim 0.5$ & $\sim 15\%$ \\
$2 \times 10^{-6}$ & $13.5 MeV$ & $\sim 0.8$ & $\sim 60\%$ \\
$2 \times 10^{-5}$ & $135. MeV$ & $\sim 0.95$ & $> 90\%$ \\ \hline
\end{tabular}
\caption{ratio of the $\Pem/\Pep$ radiative energy loss in lead 
(Z=82).}
\label{tb:phys340-1}
\end{centering}
\end{table}

The scaling holds for the ratio of the total radiative energy
losses, but it is significantly broken for the photon spectrum
in the screened case. 
In case of a point Coulomb charge the
scaling would hold also for the spectrum. 
The scaling can be expressed by:

\begin{eqnarray*}
\frac{\Phi^+}{\Phi^-} = \eta \left ( \frac{T}{Z^2} \right ) 
& \hspace{3cm} &
\frac{\frac{d\sigma^+}{dk}}{\frac{d\sigma^-}{dk}} = 
\mbox{does not scale}
\end{eqnarray*}

If we consider the photon spectrum from bremsstrahlung reported
in \cite{bib-KIM1} we see that:

\begin{eqnarray*}
\frac{d\sigma^{\pm}}{dk} = S^{\pm} \left( \frac{k}{T} \right )
\hspace{2cm}
\frac{S^{+}(k)}{S^{-}(k)} \leq 1 & \hspace{1cm} & S^{+}(1) = 0  
\hspace{2cm} S^{-}(1)  >  0 
\end{eqnarray*}

\begin{table}[hbt]
\begin{centering}
\begin{tabular}{l|rrr|rrr}
\hline
\multicolumn{1}{c}{T(MeV)}
& \multicolumn{3}{|c|}{C}
& \multicolumn{3}{c}{Pb} \\[2mm]
\hspace{1cm}
& $\Delta E^{0}_{l}$ & $\Delta E_{l}$ & $\Delta \sigma_{l}$ 
& $\Delta E^{0}_{l}$ & $\Delta E_{l}$ & $\Delta \sigma_{l}$ \\
\hline
0.02   & -2.86 & -2.86 & +52.00 & -4.89 & -4.69 & +99.80 \\
0.1    & -0.33 & -0.33 & +21.10 & -0.52 & -0.47 & +81.08 \\
1      & +0.07 & +0.07 &  +6.49 & +0.11 & +0.11 & +48.99 \\
10     &  0.00 &  0.00 &  +1.75 &  0.00 & +0.01 & +23.89 \\
$10^2$ &  0.00 &  0.00 &   0.00 &  0.00 &  0.00 &  +9.00 \\
$10^3$ &  0.00 &  0.00 &   0.00 &  0.00 &  0.00 &  +2.51 \\
$10^4$ &  0.00 &  0.00 &   0.00 &  0.00 &  0.00 &  +0.00 \\
\hline
\multicolumn{7}{c}{\hspace{1cm}} \\
\multicolumn{7}{c}{
$\displaystyle{\Delta E_{l} =
100 \frac{E^{-}_{l}-E^{+}_{l}}{E^{-}_{l}}\%}$ \hspace{5mm} and
\hspace{5mm}
$\displaystyle{\Delta \sigma_{l} =
100 \frac{\sigma^{-}_{l}-\sigma^{+}_{l}}{\sigma^{-}_{l}}\%}$
}
\end{tabular}
\caption{Difference in the energy loss and bremsstrahlung
cross-section for $\Pem/\Pep$ in Carbon and Lead with a 
cut for $\gamma$ and $e^{\pm}$ of 10keV. $\Delta E^{0}_{l}$ is
the value without the correction for the difference
$\Pem/\Pep$.}
\label{tb:phys340-2}
\end{centering}
\end{table}

We further assume that:

\begin{eqnarray}
\frac{d\sigma^+}{dk} = f(\epsilon) \frac{d\sigma^-}{dk} 
& \hspace{2cm} & 
\epsilon = \frac{k}{T}
\label{eq:phys340-3}
\end{eqnarray}

In order to satisfy approximately the scaling law for the ratio
of the total radiative energy loss, we require for $f(\epsilon)$:

\begin{eqnarray}
\int^{1}_{0}{f(\epsilon)d\epsilon} & = & \eta
\label{eq:phys340-4}
\end{eqnarray}

From the photon spectra we require:

\begin{eqnarray}
\left . 
\begin{array}{l}
f(0) = 1 \\
f(1) = 0  
\end{array}
\right \} \hspace{2cm} \mbox{for all $Z,T$}
\label{eq:phys340-5}
\end{eqnarray}

We have chosen a simple function $f$:

\begin{eqnarray}
f(\epsilon) & = & C (1-\epsilon)^{\alpha} \hspace{3cm} C,\alpha > 0
\label{eq:phys340-6}
\end{eqnarray}

\begin{table}[hbt]
\begin{centering}
\begin{tabular}{l|rrr|rrr}
\hline
\multicolumn{1}{c}{T(MeV)}
& \multicolumn{3}{|c|}{C}
& \multicolumn{3}{c}{Pb} \\[2mm]
\hspace{1cm}
& $\Delta E^{0}_{l}$ & $\Delta E_{l}$ & $\Delta \sigma_{l}$ 
& $\Delta E^{0}_{l}$ & $\Delta E_{l}$ & $\Delta \sigma_{l}$ \\
\hline
2      & +4.19 & +4.21 &  +7.29 & +4.47 & +6.88 & +61.78 \\
10     & +0.87 & +0.87 &  +1.93 & +0.87 & +1.14 & +26.29 \\
$10^2$ & +0.08 & +0.08 &   0.00 & +0.06 & +0.06 &  +9.10 \\
$10^3$ &  0.00 &  0.00 &   0.00 &  0.00 &  0.00 &  +2.42 \\
$10^4$ &  0.00 &  0.00 &   0.00 &  0.00 &  0.00 &  +0.00 \\
\hline
\multicolumn{7}{c}{\hspace{1cm}} \\
\multicolumn{7}{c}{
$\displaystyle{\Delta E_{l} =
100 \frac{E^{-}_{l}-E^{+}_{l}}{E^{-}_{l}}\%}$ \hspace{5mm} and
\hspace{5mm}
$\displaystyle{\Delta \sigma_{l} =
100 \frac{\sigma^{-}_{l}-\sigma^{+}_{l}}{\sigma^{-}_{l}}\%}$
}
\end{tabular}
\caption{Difference in the energy loss and bremsstrahlung
cross-section for $\Pem/\Pep$ in Carbon and Lead with a 
cut for $\gamma$ and $e^{\pm}$ of 1MeV. $\Delta E^{0}_{l}$ is
the value without the correction for the difference
$\Pem/\Pep$.}
\label{tb:phys340-3}
\end{centering}
\end{table}

\begin{table}[hbt]
\begin{centering}
\begin{tabular}{r|rr|rr}
\multicolumn{5}{c}{$\displaystyle{100 \frac{E^{+}_{dep}-E^{-}_{dep}}
{\sqrt{\sigma_{+}^{2}+\sigma_{-}^{2}}} (\%)}$} \\[5mm]
\hline
\multicolumn{1}{c}{Depth}
& \multicolumn{2}{c}{C}
& \multicolumn{2}{|c}{Pb} \\
\multicolumn{1}{c|}{($X_{0}$ units)} &
No $e^{\pm}$ diff & $e^{\pm}$ diff &
No $e^{\pm}$ diff & $e^{\pm}$ diff \\
\hline
0.5 &   -11.7 & -13.0 &  -0.8 &  -3.9 \\
1.0 &    -5.3 &  -4.9 &  -1.0 &  -4.1 \\
1.5 &    +7.3 &  +8.0 &  -1.4 &  -3.5 \\
2.0 &    +7.1 &  +5.3 &  -0.7 &  -0.0 \\
2.5 &    +4.9 &  +4.3 &  +1.7 &  +3.6 \\
3.0 &    +4.8 &  +4.1 &  +1.1 &  +4.3 \\
3.5 &    +3.3 &  +2.7 &  +2.7 &  +3.1 \\
4.0 &    +3.6 &  +5.3 &  +2.9 &  +3.0 \\
4.5 &    +1.7 &  +2.8 &  +0.5 &  +2.3 \\
5.0 &    +3.4 &  +3.5 &  -1.9 &  +1.8 \\
\hline
\end{tabular}
\caption{Difference in the shower development 
for $\Pem/\Pep$ in Carbon and Lead.
{\it No diff} refers to the
value without the correction for the difference
$\Pem/\Pep$.}
\label{tb:phys340-4}
\end{centering}
\end{table}

from the conditions (\ref{eq:phys340-4}), (\ref{eq:phys340-5}) we get:

\begin{eqnarray*}
C & = & 1 \\
\alpha & = & \frac{1}{\eta} - 1 \hspace{2cm}
\mbox{($\alpha > 0$ because $\eta < 1$)} \\
f(\epsilon) & = & (1-\epsilon)^{\frac{1}{\eta}-1}
\end{eqnarray*}

We have defined weight factors $F_{l}$ and $F_{\sigma}$ for the
positron continuous energy loss and discrete bremsstrahlung cross
section:

\begin{eqnarray}
F_{l} = \frac{1}{\epsilon_{0}} \int^{\epsilon_{0}}_{0}
{f(\epsilon)d\epsilon} & \hspace{3cm} &
F_{\sigma} = \frac{1}{1-\epsilon_{0}} \int^{1}_{\epsilon_{0}}
{f(\epsilon)d\epsilon}
\label{eq:phys340-7}
\end{eqnarray}

where $\epsilon_{0} = \frac{k_c}{T}$ and $k_c$ is the photon
cut {\tt BCUTE}. In this scheme the positron energy loss and
discrete bremsstrahlung can be calculated as:

\begin{eqnarray*}
\left ( - \frac{dE}{dx} \right )^{+} = F_{l} 
\left ( - \frac{dE}{dx} \right )^{-} & \hspace{2cm} &
\sigma^{+}_{brems} = F_{\sigma} \sigma^{-}_{brems}
\end{eqnarray*}

As in this approximation the photon spectra are identical, the
same SUBROUTINE is used for generating $\Pem/\Pep$ bremsstrahlung.
The following relations hold:

\begin{eqnarray*}
F_{\sigma} & = & \eta (1-\epsilon_{0})^{\frac{1}{\eta}-1}
< \eta \\
\epsilon_{0} F_{l} + (1-\epsilon_{0}) F_{\sigma} & = & \eta
\hspace{6cm} \mbox{from the def (\ref{eq:phys340-7})} \\
\Rightarrow F_{l} & = & \eta \frac{1-(1-\epsilon_{0})^{\frac{1}
{\eta}})}{\epsilon_{0}} > \eta \frac{1-(1-\epsilon_{0})}
{\epsilon_{0}} = \eta  \hspace{1cm}
\Rightarrow   \left \{
\begin{array}{l}
F_{l} > \eta \\
F_{\sigma} < \eta
\end{array} \right .
\end{eqnarray*}

which is consistent with the spectra.

The effect of this $\Pem/\Pep$ bremsstrahlung difference can be also
seen in e.m. shower development, when the primary energy is not too
high. An example can be found in table~\ref{tb:phys340-4}.

