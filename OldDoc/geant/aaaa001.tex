%%%%%%%%%%%%%%%%%%%%%%%%%%%%%%%%%%%%%%%%%%%%%%%%%%%%%%%%%%%%%%%%%%
%                                                                 %
%       GEANTR manual in LaTeX form                               %
%                                                                 %
%  Version 1.00                                                   %
%                                                                 %
%  Last Mod. 12 June 1993 1900   MG                               %
%                                                                 %
%%%%%%%%%%%%%%%%%%%%%%%%%%%%%%%%%%%%%%%%%%%%%%%%%%%%%%%%%%%%%%%%%%%
%\Authors{GEANT team}      \Origin{all contributors}              %
%%%%%%%%%%%%%%%%%%%%%%%%%%%%%%%%%%%%%%%%%%%%%%%%%%%%%%%%%%%%%%%%%%%
\Version{Geant 3.21}\Routid{AAAA001}
\Submitted{01.10.84}        \Revised{20.04.94}
\Makehead{Foreword}
As the scale and complexity of High Energy Physics experiments
increase, simulation studies require more and more care
and become essential to

\begin{itemize}
\item design and optimise the detectors,
\item develop and test the reconstruction and analysis programs, and 
\item interpret the experimental data.
\end{itemize}

{\tt GEANT} is a system of detector description and simulation tools that
help physicists in such studies.
The {\tt GEANT} system can be obtained from {\tt CERN} as six 
Patchy~\cite{bib-PATCHY}/CMZ~\cite{bib-CMZ} files:
{\tt GEANT}, {\tt GEANG}, {\tt GEANH}, {\tt GEANF}, {\tt GEANE} and {\tt GEANX}.
The program runs everywhere the CERN Program Library has been installed
\footnote{
At the moment of writing these are the systems on which the CERN Program
Library is maintained: VM/CMS-HPO-XA-ESA, SUN, Silicon Graphics,
CRAY Y-XMP, Apollo 3000 series, Apollo 10000, HP 400 series, HP 700 series,
IBM RS/6000, IBM AIX/370, VAX/VMS, Alliant, Dec Ultrix, NeXT,
IBM MVS-HPO-XA-ESA, Convex, MAC/MPW (partial implementation), 
PC-DOS, PC-Linux, PC-Windows/NT.
}.
\par
The {\tt GEANT} and
{\tt GEANG} files contain most of the basic code. The {\tt GEANH} file
contains the code for the
hadronic showers simulation from the program {\tt GHEISHA}~\cite{bib-GHEI}.
The {\tt GEANF} file contains the source of the routines for hadronic
showers development from the 
{\tt FLUKA}~\cite{bib-FLUK,bib-FLU1,bib-FLU2,bib-FLU3,bib-FLU4,bib-FLU5,bib-FLU6}
program which is interfaced with {\tt GEANT} as an alternative to 
{\tt GHEISHA} to simulate hadronic cascades.
The {\tt GEANE}~\cite{bib-GEANE} file
contains a tracking package to be used, in the context of
event reconstruction, for trajectory estimation and error propagation.
The {\tt GEANX} file
contains
the main program for the interactive version of {\tt GEANT} (GXINT) and
a few examples of application
programs which may help users to get started with {\tt GEANT}.
\par
General information concerning {\tt GEANT}, for example
access to the source code, the list of problems and their
proposed corrections, the context of utilisation on the
CERN machines, the status of some application programs, the acquisition of
documentation, etc., are kept up to date through CERN news and
InterNet news group ({\tt cern.lgeant}) and an
electronic mailing list which is installed on the CERN IBM mainframe
(BITnet/EARN node CERNVM). The name of the list is LGEANT (see below how
to subscribe).
\par
The first version of {\tt GEANT} was written in 1974 as a bare
framework which initially emphasised tracking
of a few particles per event through relatively
simple detectors. The system has been developed
with some continuity over the years~\cite{bib-GEANT2}
\par
New versions may
differ from the previous ones.
Some of the modifications may lead to backward incompatibilities.
The user is therefore invited to read carefully the Patch {\tt HISTORY}
of the current {\tt GEANT} file where all changes are described in detail.
\par
The development and the maintenance of {\tt GEANT} are possible
only thanks to the devoted and continuous collaboration of
physicists around the world who use the program and contribute their
feedback to the authors and maintainers at CERN. It is of course
impossible
to mention all of them, and new names are added frequently to the
list of the contributors.
The {\tt GEANT} team wish to thank them all and expresses
its hope that they will continue to help us.
\par
{\tt GEANT} version 3 originated from an idea of Ren\'{e} Brun and Andy McPherson 
in 1982 during the development of the OPAL simulation program.
GEANT3 was based on the skeleton of {\tt GEANT} version 2 
code~\cite{bib-GEANT2}.
\par
In close collaboration with Ren\'{e} and Andy, Pietro Zanarini developed
the first versions of the graphics system as well as the early versions
of the interactive package initially based on {\tt ZCEDEX}, 
then upgraded to {\tt KUIP}.
\par
Glenn Patrick (RAL) implemented a first version of the electromagnetic 
processes. Tony Baroncelli (Roma) helped in interfacing {\tt GEANT} to his hadronic
shower package {\tt TATINA}. Federico Carminati contributed to the interface
with {\tt GHEISHA}, an hadronic shower package developed by Harn Fesefeldt (Aachen).
\par
Francis Bruyant and Michel Maire (LAPP) made substantial contributions 
to the geometry,
tracking and physics parts of {\tt GEANT} while adapting the system
to the L3 environment. Francis has been, for many years, an essential
collaborator, testing new ideas for the geometry and hits packages.
Michel, together with Elemer Nagy and Vincenzo Innocente, developed the 
{\tt GEANE} system.

\par
A very important contribution to {\tt GEANT} has been made by L\'{a}szl\'{o} 
Urb\'{a}n 
(KFKI Budapest) who has continuously improved the electromagnetic
physics package. Lazlo has spent a considerable amount of time in
reading the relevant papers in the literature and in making comparisons
with experimental results.

\par
Ren\'{e} Brun has coordinated the development and the maintenance of {\tt GEANT}
from 1982 until 1991 (versions 3.00 up to 3.14). Federico Carminati
coordinated the development of the versions 3.15 and 3.16 between 1991 and 1993.
Since January 1994, the responsability for {\tt GEANT} is in the hands of Simone
Giani.
Before assuming this responsability, Simone made substantial improvements
in the graphics and interactive packages. After he has enhanced the power
of the geometry package and the performance of the tracking for a new
version of {\tt GEANT}: 
in March 1994 the version 3.21 has been released and is the current version
of {\tt GEANT}.
\par
Many people contributed their
work or their experience. We have tried to acknowledge their names 
in the manual pages and we apologise for any omissions.

Special mention should be made here of
the following contributions:

S.Banerjee
(contribution to the tracking package),
R.Jones
(contribution to the 
simulation of electromagnetic processes),
K.Lassila-Perini
(interface with FLUKA and MICAP),
G.Lynch
(contribution to the multiple scattering algorithms),
E.Tchernyaev
(original code for hidden-line removal graphics),
J.Salt
(original interface to the GC package).

\par
S. Ravndal did a complete revision and update of the full documentation
for the release
of {\tt GEANT} version 3.21.


Special thanks should go to the authors of the packages interfaced
with {\tt GEANT}, and in particular to Harn Fesefeldt ({\tt GHEISHA}) 
and Alfredo Ferrari ({\tt FLUKA} see later). Their
patience in explaining the internals of their code, their
experience and their 
collaborative and open attitude have been instrumental.

Another special thanks goes to Mike Metcalf, who helped to 
improve the English and the structure of the manual.

\par
Any reader who is not familiar with {\tt GEANT}
should first have a glance at the notes numbered 001 to 009 in each
section of this manual.

Despite our efforts, the documentation is still
incomplete and far from perfect. We accept full
responsibility for its present status.
 
Finally, we express our thanks to Michel Goossens for translating the 
SCRIPT/SGML source of the original {\tt GEANT} manual into \LaTeX.

\section{The GEANT-FLUKA interface}

Since version 3.15, {\tt GEANT} includes an interface with some
{\tt FLUKA}~\cite{bib-FLUK,bib-FLU1,bib-FLU2,bib-FLU3,bib-FLU4,bib-FLU5,bib-FLU6}
routines. This part has been updated and extended in subsequent releases.

{\tt FLUKA} is a standalone code with its own life. Only a few parts have been
included into {\tt GEANT}, namely the ones dealing with hadronic elastic and
inelastic interactions.

The implementation of {\tt FLUKA} routines in {\tt GEANT} does not include
any change, apart from interface ones and those agreed by the {\tt FLUKA}
authors. Whenever different options are available in {\tt FLUKA}, the one
suggested by the authors have been retained. Nevertheless the results
obtained with {\tt FLUKA} routines inside {\tt GEANT} could not be 
representative of the full {\tt FLUKA} performances, since they
generally depend on other parts which are {\tt GEANT} specific.

The routines made available for {\tt GEANT} have been extensively tested and are
reasonably robust. They usually do not represent the latest {\tt FLUKA}
developments, since the policy is to supply for {\tt GEANT} well tested
and reliable code rather than very recent developments with possibly better
physics but also still undetected errors.

It is important that {\tt GEANT} users are aware of the conditions at which
this code has been kindly made available:
\begin{itemize}
\item relevant authorship and references about 
{\tt FLUKA}~\cite{bib-FLUK,bib-FLU1,bib-FLU2,bib-FLU3,bib-FLU4,bib-FLU5,bib-FLU6}
should be clearly
indicated in any publication reporting results obtained with this code;
\item the {\tt FLUKA} authors reserve the right of publishing about 
the physical models
they developed and implemented inside {\tt FLUKA}, {\tt GEANT} users are not 
supposed to extract from the {\tt GEANT}-{\tt FLUKA} code the relevant
routines running them standalone for benchmarks;
\item more generally,
{\tt FLUKA} routines contained in the {\tt GEANF} file are supposed to be
included and used with {\tt GEANT} only: any other use must be authorised
by the {\tt FLUKA} authors.
\end{itemize}

 
\section{Documentation}
 
The main source of documentation on {\tt GEANT} is this manual. Users are
invited to notify any correction or suggestion to the authors.
 
A detailed description of the history of modifications to the {\tt GEANT} code
is contained in the \Lit{$VERSION} Patch in the {\tt GEANT} file.
 
{\tt GEANT} is part of the CERN Program Library, and problems or questions
about {\tt GEANT} should be directed to the Program Library Office (see next
section).
 
{\tt GEANT} problems can be submitted to
an InterNet discussion group {\tt cern.lgeant}.

A mailing list is maintained on CERN's central IBM machine (BITnet node
CERNVM) via the LISTSERV mechanism. LISTSERV acts as a rudimentary
conferencing system, which forwards the mail received to all subscribed
users and to the InterNet group {\tt cern.lgeant}. 
The list is accessible to all users who have an e-mail connection
to CERNVM. To subscribe to the list from a BITnet node a user has to
send the following message to LISTSERV at CERNVM using the local
BITnet message facility:

\begin{center}
\Lit{SUBSCRIBE LGEANT}
\end{center}

From a non-BITnet node, the user can send an ordinary mail message to
the user LISTSERV at CERNVM containing that single line.
 
Together with the {\tt GEANT} library comes a 
{\it correction cradle} (see section on
maintenance policy) which contains the history of the
modifications to the current
version of the {\tt GEANT} program. This file is accessible to all
users in the PRO area and can be found in:

\begin{center}
\Lit{GCORR{\it xxx}.CAR}
\end{center}

where {\it xxx} is the version number.
 
Documentation on the elements of the CERN Program Library used by the {\tt GEANT}
program is available from the CERN Program Library Office.

\section{Update policy}
The {\tt GEANT} program is constantly updated to reflect corrections, most
of the time originating from users' feed-back, and improvements to
the code. This constant evolution, which is one of the reasons for the 
success of {\tt GEANT},
poses the serious problem of managing change without disrupting stability,
which is very important for physicists doing long production runs.
 
In the CERN Program Library maintenance scheme, three versions
of any product are present at the same time on the central systems,
in the
OLD, PRO and NEW areas with those same names. This scheme does not apply to
{\tt GEANT} because every new release usually contains modifications
in the physics which can produce, we hope,
better but often different results with
respect to the previous version. It is therefore appropriate to offer to
the users
an extra level of protection against running inadvertedly the wrong version
by appending the version number to all the files of {\tt GEANT}. In this way the
users will have to change their procedure to change the
version of {\tt GEANT}.
 
On the other hand, the new user should not bother about version numbers and
correction files, and so an alias is installed on all systems without any
version number, always pointing to the latest released version.
 
As said before, users' feed-back is of paramount importance in detecting
problems or areas for improvement in the system, so the new version is made
available in the NEW area well before the official
release. If, on the one hand, those who use this version do so
at their own risk, on the other hand users are encouraged to
perform
as much testing as possible, in order to detect the maximum
number of problems before the final release. Modifications in the pre-release
version are made directly in the source code.

When problems are discovered,
which may seriously affect the validity of the results of the simulation,
they are corrected and the library recompiled in the /new area.
To minimise network
transfer for remote users and in the interest of the stability of the system,
the source code of the released version in PRO is not touched, but rather
the correction is applied via a so-called {\it correction cradle} which is
a file containing the differences between the original and the corrected
version in a format required by PATCHY/CMZ. Both these programs can read the
original source and the correction cradle and produce the corrected source.
The corrected car/cmz source and the corrected binary library are made then
available in /pro at the following CERNLIB release (when /new becomes /pro).

The cmz source files contain the full history of the corrections applied
with proper versioning, so that every intermediate version can be rebuild.
Users at CERN should not need to use the correction                
cradle other than
for documentation purposes. Remote users may want to obtain
the cradle and apply the corrections.    
At every CERNLIB release the correction cradle 
is obviously reset to be empty, as all the
corrections have been applied in the code directly. The correction cradle for
the OLD version is available but has to be considered frozen.
No correction is ever applied to an old version.

New versions of {\tt GEANT} are moved into the PRO area synchronously with
releases of the CERN Program Library. If no new version of {\tt GEANT}
is available
at the time of the release of the Program Library, the {\tt GEANT} files do not
change their location and the production version remains the same.  
 
\section{Availability of the documentation}
 
This document has been produced using \LaTeX\footnote{%
Leslie Lamport, {\it \LaTeX\ -- A Document Preparation
System}. Addison--Wesley, 1985}
with the \Lit{cernman} and \Lit{cerngeant} style options, developed at CERN.
A printable version of each of the sections described in this manual
can be obtained as a compressed PostScript file from CERN 
by anonymous ftp. You can look in the directory described in the
procedure below for more details.
For instance, if you want to transfer the description
of the physics routines, then you can type the following
(commands that you have to type are underlined):\footnote{You can of course 
issue multiple {\tt get} commands in one run.}
 

\vspace*{3mm}
\begin{tabular}{@{\hspace{12mm}}>{\tt}l}
\underline{ftp asis01.cern.ch}\\
Trying 128.141.201.136...\\
Connected to asis01.cern.ch.\\
220 asis01 FTP server (SunOS 4.1) ready.\\
Name (asis01:username): \underline{anonymous}\\
Password: \underline{your\_{}mailaddress}\\
ftp> \underline{binary}        \\
ftp> \underline{cd cernlib/doc/ps.dir/geant}\\
ftp> \underline{get phys.ps.Z} \\
ftp> \underline{quit}          \\
\end{tabular}
