%%%%%%%%%%%%%%%%%%%%%%%%%%%%%%%%%%%%%%%%%%%%%%%%%%%%%%%%%%%%%%%%%%%
%                                                                 %
%  GEANT manual in LaTeX form                                     %
%                                                                 %
%  Michel Goossens (for translation into LaTeX)                   %
%  Version 1.00                                                   %
%  Last Mod. Jan 24 1991  1300   MG + IB                          %
%                                                                 %
%%%%%%%%%%%%%%%%%%%%%%%%%%%%%%%%%%%%%%%%%%%%%%%%%%%%%%%%%%%%%%%%%%%
\Origin{R.Brun, A.McPherson}
\Submitted{15.08.83}              \Revised{16.11.93}
\Version{Geant 3.16}\Routid{GEOM110}

\Makehead{Positioning a volume inside its mother}

\Shubr{GSPOS}{(CHNAME,NR,CHMOTH,X,Y,Z,IROT,CHONLY)}
Places a copy of a volume previously defined by a call to \Rind{GSVOLU} 
inside its mother volume {\tt CHMOTH}.

\begin{DLtt}{MMMMMMMMMM}
\item[CHNAME] ({\tt CHARACTER*4}) name of the volume being positioned;
\item[NR] ({\tt INTEGER}) copy number of the volume {\tt CHNAME} being 
positioned;
\item[CHMOTH]({\tt CHARACTER*4}) name of the volume in which copy
{\tt NR} of {\tt CHNAME} is positioned;
\item[X] ({\tt REAL}) x position of the volume in the mother reference system;
\item[Y] ({\tt REAL}) y position of the volume in the mother reference system;
\item[Z] ({\tt REAL}) z position of the volume in the mother reference system;
\item[IROT] ({\tt INTEGER}) rotation matrix number
describing the orientation of the volume relative to
the coordinate system of the mother;
\item[CHONLY] ({\tt CHARACTER*4}) flag to indicate whether
a point found to be in this volume may also be in other volumes which
are not direct descendants of it -- possible values are {\tt ONLY} and
{\tt MANY}.
\end{DLtt}

The following points are important for a correct use of \Rind{GSPOS}:
\begin{itemize}
\item the position and rotation with which a volume is positioned are
relative to the mother reference system, that is to the reference system
of the volume which contains the copy being positioned. To see how the
reference system is defined for the {\tt GEANT} shapes, see {\tt [GEOM050]};
\item 
\Rind{GSPOS} can be called several times with the same name, to place 
multiple copies of the same volume either in the
same mother volume or in different ones.
The copy number parameter {\tt NR} is user defined, and it allows
the different copies of the same volume to be distinguished. The user is
free to assign any valid integer to this parameter. If two volumes
are positioned with the same copy number, it will be impossible to decide
in which one a particle is during tracking. If the same volume is positioned
twice inside the same mother with the same copy number, the parameters
of the second call to \Rind{GSPOS} will override the first ones for that
copy;
\item all the copies of a volume are identical: in particular they contain
all the same daughters recursively -- conversely, positioning a volume 
inside a mother will insert it in all the copies of that mother;
\item the volumes mentioned in the call must be already defined by a call
to \Rind{GSVOLU};
\item no volume can be positioned in a divided mother;
\item in case of positioning without rotation, {\tt IROT} should be set
to 0 -- note that this is faster than defining a unit matrix;
\item the first volume, containing all the apparatus, cannot be positioned;
\item if the {\tt CHONLY} flag is equal to {\tt ONLY}, the volume being
positioned should not overlap with any other volume except his mother
and his daughters and should not extend beyond the limits of its mother. 
In this case the search routines assume that, when a point
is found in this volume, the only further searching required is in its contents.
Other volumes at the same or higher levels
or in separate branches will not be looked at. The first volume must be 
{\tt ONLY};
\item if the {\tt CHONLY} flag is set to {\tt MANY}, a point found in
this volume can also be in another volume at the same or higher level or in
a different branch of the tree. To decide where a point is the following
algorithm is applied:
\begin{enumerate}
\item the tree is scanned {\it upward} until an {\tt ONLY} mother
with {\it positioned} contents (i.e. non-divided) is found;
\item all the branches of the tree descending from this volume are
scanned to find other volumes where the point may be;
\item a winner is selected among the candidates according to the
following rules:
\begin{itemize}
\item if one of the candidates is {\tt ONLY}, the point is declared in this
volume. If more than one candidate is {\tt ONLY} there is an error in the
geometry definition;
\item amongst several {\tt MANY} candidates, the one at the deepest level
is chosen;
\item amongst several {\tt MANY} candidates at the same level, the first one
found is chosen;
\end{itemize}
\end{enumerate}
Although a considerable effort has been put in making sure that the {\tt MANY}
option works properly and caters for all the possible cases, still it introduces
a considerable complication in the tracking and performance can be
rather poor.

Users should resort to this only when it is not possible to describe the
experimental setup with {\tt ONLY} volumes.
\end{itemize}

The data structure is so designed that, in any case, the parameters
of a volume and the information about the number and position of
its daughters are stored only once. This allows the creation of
very complex geometries without correspondingly large storage
requirements.
