%%%%%%%%%%%%%%%%%%%%%%%%%%%%%%%%%%%%%%%%%%%%%%%%%%%%%%%%%%%%%%%%%%%
%                                                                 %
%  GEANT manual in LaTeX form                                     %
%                                                                 %
%  Michel Goossens (for translation into LaTeX)                   %
%  Version 1.00                                                   %
%  Last Mod. Jan 24 1991  1300   MG + IB                          %
%                                                                 %
%%%%%%%%%%%%%%%%%%%%%%%%%%%%%%%%%%%%%%%%%%%%%%%%%%%%%%%%%%%%%%%%%%%
\Origin{H.C.Fesefeldt, F.Carminati}
\Submitted{20.12.85}\Revised{16.12.93}
\Version{Geant 3.16}\Routid{PHYS240}
\Makehead{Photon-induced fission on heavy materials}
\section{Subroutines}
\Shubr{GPFISI}{}
\Rind{GPFISI} computes and stores in the appropriate bank (see {\tt JMATE} 
data structure) the value of the photo-fission cross sections as a 
function of the energy for all the material for which this process
is activated and with $230 \leq A \leq 240$. 
It is called during initialisation by \Rind{GPHYSI}.

\Shubr{GPFIS}{}
\Rind{GPFIS} is called by the tracking routine \Rind{GTGAMA} to generate
the final state of a photon-induced fission.

\section{Method}
 
This set of routines generates the photon-induced fission in the nuclei of
heavy materials. The code is a straight translation into the {\tt GEANT} style 
of the corresponding code from the {\tt GHEISHA} Monte Carlo program.
 
The information contained in this chapter is mainly taken from the
{\tt GHEISHA} manual to which the user is referred \cite{bib-GHEI}.
In fissionable elements the inelastic reaction of photons may be
accompanied by the fission of the nucleus. This channel is taken into account
only when the nuclear mass is between 230 and 240. The fission cross-section for
uranium is tabulated, and for other elements the parameterisation of the
A-dependence at 3 MeV neutron energy is used:
 
\[
 \sigma_f ( 3 MeV ) = - 67.0+38.7 \:  Z^{4/3 } A
\]

which implies that the fission may occur only for $ Z \geq 90$. The fission 
products are generated in an approximate way. The giant resonance effect is
included here as nuclear photoabsorbtion.
The photo-fission is activated 
by the data record \Rind{PFIS}. 
Secondaries coming from the interaction of the virtual
photon with the nucleus will be in the {\tt GEANT} temporary stack. 
If {\tt IPFIS=2} 
the secondary particles will not be generated and the
energy of the photon will be summed to {\tt DESTEP}.
A table with the photo-fission cross-sections as a function of the
energy is put in a bank at initialization time for all the materials with A
bigger than 230. See material bank structure for details.
