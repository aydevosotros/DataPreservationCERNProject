======================================================================== 229
Return-Path: <@CERNVM.CERN.CH:jvuoskos@DSPAW.CERN.CH>
Received: from CERNVM (NJE origin SMTPIBM@CERNVM) by CERNVM.CERN.CH (LMail
          V1.1d/1.7f) with BSMTP id 5572; Tue, 22 Mar 1994 15:05:31 +0100
Received: from dspaw.cern.ch by CERNVM.CERN.CH (IBM VM SMTP V2R2) with TCP;
   Tue, 22 Mar 94 15:05:30 WET
Received: by dspaw.cern.ch (5.65/DEC-Ultrix/4.3)
        id AA00820; Tue, 22 Mar 1994 15:06:23 +0100
Date: Tue, 22 Mar 1994 15:06:23 +0100
From: jvuoskos@dspaw.cern.ch (Jouko Vuoskoski)
Message-Id: <9403221406.AA00820@dspaw.cern.ch>
To: RAVNDAL@crnvma.cern.ch
 
%%\documentstyle[epsf,amssym,fullpage]{article}
%%\setlength{\oddsidemargin}{10mm}      %% relatively to 1inch(25.4mm)
%%\setlength{\evensidemargin}{10mm}     %% from left side of paper
%%\setlength{\textwidth}{155mm}
%%\setlength{\topmargin}{0mm}   %% relatively to 1inch(25.4mm) from top
%%\setlength{\headheight}{0mm}
%%\setlength{\headsep}{0mm}
%%%%\setlength{\textheight}{297mm}
%%\parskip=5mm
%%%%\parsep=5mm
%%\parindent=0pt
%%\begin{document}
 
\documentstyle[fancyheadings,epsf,times]{article}
\textheight 20 cm \textwidth 14 cm
\oddsidemargin 0.0cm
\evensidemargin 0.0cm
\topmargin 0.0cm
\parskip=0.3cm
\parindent=0cm
\pagestyle{fancy}
\renewcommand{\sectionmark}[1]{\markboth{CADINT 1.20}{CADINT 1.20}}
%%\renewcommand{\sectionmark}[1]{\markboth{#1}{}}
\headwidth=14cm
\rhead[\fancyplain{}{\sl\leftmark}]{\fancyplain{}{\rm\thepage}}
\cfoot{}
 
\begin{document}
\normalsize{
 
\quad
\vspace{2cm}
 
\section*{GEANT-SET interface -- CADINT 1.20}
 
CADINT converts GEANT detector geometries into the SET\footnote{{\it Standard
d'Echange et de Transfert}, a French standard for exchange of CAD/CAM data}
file format.
The detector geometries can be read into
EUCLID-IS (or other CAD-systems with a proper
SET interface) using SET file format.
 
A GEANT detector geometry exported through CADINT
into a CAD-system can be used to
check the coherence of the model used in simulation.
The CADDFAS service at CERN can be used to convert SET files into IGES
\footnote{{\it Initial Graphics Exchange Specification}, an American
standard for exchange of CAD-data} format, which is readable
by most CAD-systems.
For more information about the CADDFAS service, send a mail to {\it
caddfas@cadd.cern.ch} with subject: {\it info}
 
 
\subsection*{The CADINT Command:}
 
The command ``CADINT'' invokes subroutine GTXSET:
 
 
\begin{verbatim}
SUBROUTINE GTXSET(FNAME,ANAME,NBINS,LUNIT,LUNIT2,INST,SITE,DEPT,RESP)
 
FNAME : Name of the SET file
ANAME : Name of the volume to start the conversion
NBINS : Number of instances of each division to be exported
LUNIT : Logical unit for the SET file
LUNI2 : Logical unit for the material file
INST  : Name of the institute
SITE  : Name of the site
DEPT  : Name of the departement
RESP  : Name of the sender
\end{verbatim}
 
The parameters which are related to the geometry are the volume name and the
number of instances of each division. The
other parameters are related to the SET file header and output.
 
 
Selection of volumes can be done using the visibility
attribute, the volume name and the number of
division instances.
All visible volumes and the given number of division instances
will be written into the SET file starting from the given volume.
Volume names will
be written into SET with an index to distinguish different instances
of a volume. In a case of divisions, the indexing of slices will
be reset in each division.
 
Material and tree information is written to a separate file,
with the same name as the SET file, but with extension ``.mat''.
 
An example of a material file:
 
\begin{verbatim}
 
 GEANT-SET MATERIAL LISTING FILE
 --------------------------------
 
 Materials in the geometry described in
 the .SET file:  example.set
 
Volume name  Tracking media   Material         Density
 
CALO         1 AIR            1 AIR            0.12000001E-02
CAL1         1 AIR            1 AIR            0.12000001E-02
MOD1         1 AIR            1 AIR            0.12000001E-02
CAL2         1 AIR            1 AIR            0.12000001E-02
MOD2         1 AIR            1 AIR            0.12000001E-02
CAL3         1 AIR            1 AIR            0.12000001E-02
MOD3         1 AIR            1 AIR            0.12000001E-02
EPO1         4 CARBON         4 CARBON         0.22650001E+01
CHA1         6 BRASS          6 BRASS          0.85600004E+01
TUB1         6 BRASS          6 BRASS          0.85600004E+01
GAS1         5 GAS            5 ARG/ISOBU      0.21360000E-02
 
 
 GEANT TREE
 ----------
 
 The GEANT tree starting from the given volume
 
CALO    6  CAL1  CAL2  CAL3  EPO1  CHA1  EPO1
CAL1  -64  -0.48000000E+02   0.15000000E+01  MOD1
CAL2  -35  -0.43750000E+02   0.25000000E+01  MOD2
CAL3  -13  -0.24050001E+02   0.37000003E+01  MOD3
CHA1  -40  -0.25000000E+02   0.12500000E+01  TUB1
MOD1    6  SHIL  URPL  SHIL  EPO1  CHA1  EPO1
MOD2    4  SHIL  URPL  SHIL  CHA2
MOD3    2  COPL  CHA2
TUB1    1  GAS1
CHA2  -72  -0.23500000E+02   0.65277779E+00  TUB2
TUB2    1  GAS2
 
   ------ end of file -------
 
\end{verbatim}
 
The tree information is in computer interpretable format. After the
volume name there is the number of daughters and daughter names.
In a case of divided volumes the (negative) number of divisions is given after
the name of divided volume followed by the starting coordinate, step
and the name of the division instance.
 
 
 
\subsection*{Some Hints}
 
In order to carry out a succesful transfer of the geometry, you need to be
aware of certain limitations of the interface.  It is wise to communicate
a bit with the engineers to establish a sound transfer methodology.
 
 
\subsubsection*{Size Limitations}
 
CADINT is useful to transfer parts of a detector from GEANT
to a CAD-system.  The full detector description may be too huge
to store in your
CAD-system, so be aware of some limitations here.
A good strategy is to make several SET files, one for each sub-part of the
detector, and then read each file into the CAD-system at a time.
When you export GEANT geometry to a CAD system, be careful not to have
too many volumes, because this will fill up the CAD system.
 
It is impossible to give an exact size-limit, because the limitation
highly depends on the structure of the detector.  Both the GEANT-tree structure
and the type of volumes are important factors. In some cases the
transfer works with  more than 1000 volumes, in other cases you
will reach the limit around 50.
An example: The CDET part of ALEPH (450 volumes) can be read into
EUCLID as one file, the SET file has a size of 140Kb.  But the COIL
part (66 volumes) of ALEPH cannot be read into EUCLID in one piece,
is has to be read in as several files, despite of the fact that the
SET file COIL.SET is only 22Kb.
 
Normally it is a good idea to just export 1 instance of each division
to limit the number of exported volumes.  If you can limit yourself to
sub-detectors of 100 volumes and less you should avoid problems.
 
\subsubsection*{Use Graphics!}
 
The GEANT graphics package  provides excellent capabilities for
viewing parts of a detector which is to be exported.  By using the
DTREE and DSPEC functions, you can simply move
around in the detector tree and look at parts of is on the screen.  By
playing with the visibility attribute and the number of division instances, you
 will be able to export exactly what you need!
 
\subsubsection*{Supply Extra Information if Required}
A plot of the GEANT tree can be useful for the engineer to understand the
structure of the GEANT geometry.  The GEANT volume names are useful references
regarding material and tree information. The tree information is also
provided in the material file, but a graphical presentation is often useful.
 
\subsubsection*{Suppress Non-Relevant Volumes}
Dummy volumes, such as mothers made of air or vacuum should be suppressed (use
the visibility attribute), since the interface is currently not able to discard
 such information.
 
\subsubsection*{Reset GDRAW Parameters After Use of CADINT}
Remember that you must specify the drawing parameters again if you
want use GDRAW to draw your detector after you have used the interface command.
This is because the CAD-interface uses some GDRAW routines.
 
Contact Nils H{\o}imyr/CN/CE or Jouko Vuoskoski/CN/AS if you have any
questions regarding CADINT.
 
\vspace{1cm}
 
Email:  nils@cadd.cern.ch or jvuoskos@dxcern.cern.ch
 
 
 
 
\end{document}
 
 
 
 
