%%%%%%%%%%%%%%%%%%%%%%%%%%%%%%%%%%%%%%%%%%%%%%%%%%%%%%%%%%%%%%%%%%%
%                                                                 %
%  GEANT manual in LaTeX form                              %
%                                                                 %
%  Michel Goossens (for translation into LaTeX)                   %
%  Version 1.00                                                   %
%  Last Mod. Jan 24 1991  1300   MG + IB                          %
%                                                                 %
%%%%%%%%%%%%%%%%%%%%%%%%%%%%%%%%%%%%%%%%%%%%%%%%%%%%%%%%%%%%%%%%%%%
\Origin{R.Brun}
\Submitted{20.08.87}      \Revised{17.12.93}
\Version{Geant 3.16}      \Routid{IOPA500}
\Makehead{Data structure I/O with direct access files}

\Shubr{GRIN}{(CHOBJ,IDVERS*,CHOPT)}
\begin{DLtt}{MMMMMMMM}
\item[CHOBJ] ({\tt CHARACTER*(4)}) name of the data structure to
read in, see {\tt [IOPA300]} for more information on this;
\item[IDVERS] ({\tt INTEGER}) version of the data structure to be read in,
if 0 it will read any version, on output it contains the version of the
data structure read in;
\item[CHOPT] ({\tt CHARACTER*(*)})  option:
\begin{DLtt}{MMMM}
\item[I] perform read only if
{\tt CHOBJ} is an initialisation data structure;
\item[K] perform read only if {\tt CHOBJ} is {\tt KINE} or {\tt TRIG};
\item[T] perform read only if {\tt CHOBJ} is 
{\tt DIGI}, {\tt HEAD}, {\tt HITS}, {\tt KINE},
{\tt VERT} or {\tt JXYZ};
\item[Q] quiet option, no message is printed;
\end{DLtt}
\end{DLtt}

This routine reads {\tt GEANT} data structures from the current working
directory of an {\tt RZ} file (see the {\tt ZEBRA} documentation for
a clarification of this concept). Example:
\begin{verbatim}

      CALL GRFILE(1,'Geometry.dat',' ')
      CALL GRIN ('VOLU',1,' ') 
      CALL GRIN ('MATE',1,' ')
      CALL GRIN ('TMED',1,' ')
      CALL GRIN ('ROTM',1,' ')
      CALL GRIN ('PART',1,' ')
      CALL GRIN ('SCAN',1,' ')
      CALL GRIN ('SETS',1,' ')
\end{verbatim}

The same result can be achieved by:
\begin{verbatim}

      CALL GRFILE(1,'Geometry.dat','I')
\end{verbatim}

\Shubr{GROUT}{(CHOBJ,IDVERS,CHOPT)}
The meaning of the arguments is the same than in the previous routine
\Rind{GRIN}, but for writing instead than for reading.

This routine writes {\tt GEANT} data structures into the current working
directory of an {\tt RZ} file (see the {\tt ZEBRA} documentation for
a clarification of this concept).
Note that if the cross-sections and energy loss tables
are available in the data structure {\tt JMATE}, then they are
saved on the data base.
The data structures saved by this routine can be retrieved
with the routine \Rind{GRIN}.
Before calling this routine a {\\tt RZ} data base must have been
created using \Rind{GRFILE}.
The data base must be closed with \Rind{RZEND}. Example:
\begin{verbatim}

      CALL GRFILE(1,'Geometry.dat','N')
      CALL GROUT ('VOLU',1,' ') 
      CALL GROUT ('MATE',1,' ')
      CALL GROUT ('TMED',1,' ')
      CALL GROUT ('ROTM',1,' ')
      CALL GROUT ('PART',1,' ')
      CALL GROUT ('SCAN',1,' ')
      CALL GROUT ('SETS',1,' ')
\end{verbatim}

The same result can be achieved by:
\begin{verbatim}

      CALL GRFILE(1,'Geometry.dat','NO')
\end{verbatim}

The interactive version of {tt GEANT} provides facilities
to interactively update, create and display objects.


The routines \Rind{GRGET} and \Rind{GRSAVE} are obsolete and should not 
be used.
