%%%%%%%%%%%%%%%%%%%%%%%%%%%%%%%%%%%%%%%%%%%%%%%%%%%%%%%%%%%%%%%%%%%
%                                                                 %
%  GEANT manual in LaTeX form                              %
%                                                                 %
%  Michel Goossens (for translation into LaTeX)                   %
%  Version 1.00                                                   %
%  Last Mod. Jan 24 1991  1300   MG + IB                          %
%                                                                 %
%%%%%%%%%%%%%%%%%%%%%%%%%%%%%%%%%%%%%%%%%%%%%%%%%%%%%%%%%%%%%%%%%%%
\Origin{R.Brun, A.C.McPherson, F.Bruyant}
\Revision{S.Giani}
\Submitted{18.12.83}       \Revised{14.12.93}
\Version{Geant 3.16}       \Routid{GEOM300}

\Makehead{Finding in which volume a point is}

\Shubr{GMEDIA}{(X,NUMED*)}
\begin{DLtt}{MMMMMMMM}
\item[X] ({\tt REAL}) array of dimension 3 with the coordinates in the 
{\tt MRS};
\item[NUMED] ({\tt INTEGER}) medium number, if this is zero the point is 
outside the detector.
\end{DLtt}

Searches the geometrical tree structure to find in which volume the point 
{\tt X} is. The tracking medium is returned in {\tt NUMED}, and the common 
\FCind{/GCVOLU/} is updated.
 
\Rind{GMEDIA} uses the geometry data structure to conduct its search
starting its search from the last volume where a point was found.
If no previous search has been conducted, the first volume is used
as a starting point.

If the point is not inside the current volume, \Rind{GMEDIA} looks in its 
mother and so on, until it finds a volume which contains the point.
It then looks at the contents of this object and so on until the point 
is in a volume but in none of its contents (if any).

If this {\it downward} search terminates in a
{\tt 'MANY'} object, \Rind{GMEDIA} looks for another candidate. See
{\tt [GEOM110]} for a description of the {\tt `MANY'} volumes tratment.

The {\it physical} geometrical tree from the first volume
to the current one is stored in the common block \FCind{GCVOLU}
(see {\tt [BASE030]}) and in the structure {\tt JGPAR} (see {\tt [GEOM199]}).

\Shubr{GTMEDI}{(X,NUMED*)}
\begin{DLtt}{MMMMMMMM}
\item[X] ({\tt REAL}) array of dimension 3 with the coordinates in the 
{\tt MRS};
\item[NUMED] ({\tt INTEGER}) medium number, if this is zero the point is 
outside the detector.
\end{DLtt}

This routine performs the same function than \Rind{GMEDIA}, but it uses
the dynamical information of the particle history to speed-up the 
search:
\begin{itemize}
\item when {\tt INWVOL=2} (common \FCind{/GCTRAK/})
the particle just came out of a volume. In this
case, if {\tt INFROM} (common \FCind{/GCVOLU/}) is positive, it is
interpreted by \Rind{GTMEDI} as the number {\tt IN} of the content
just left, inside the mother volume
where the point {\tt X} is assumed to be. This content will not be
searched again.
\item the daughter of the current volume which limits the
geometrical step of the particle (i.e. where the particle would be heading
moving along a straight line) is searched first.
{\tt INGOTO} (common \FCind{/GCVOLU/}) is set by \Rind{GTNEXT}, to transmit 
the information
on this one volume which has limited the geometrical step {\tt SNEXT} as follows:
\begin{DLtt}{MMMM}
\item[$>$0] {\tt IN}$^{th}$ content;
\item[$=$0] current volume;
\item[$<$0] -{\tt NLONLY}, with {\tt NLONLY} defined as the lowest {\tt 'ONLY'}
level up in the tree which is an ancestor of the {\tt 'MANY'} volume 
where the point {\tt X} is.
\end{DLtt}
\end{itemize}

\Shubr{GINVOL}{(X,ISAME*)}
\begin{DLtt}{MMMMMMMM}
\item[X] ({\tt REAL}) array of dimension 3 with the coordinates in the 
{\tt MRS};
\item[ISAME] ({\tt INTEGER}) return flag.
\end{DLtt}
Checks if particle at point {\tt X} has left current volume.
If so, returns {\tt ISAME = 0} and prepares information useful to
identify the new volume entered, otherwise, returns {\tt ISAME = 1}.
