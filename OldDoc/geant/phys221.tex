%%%%%%%%%%%%%%%%%%%%%%%%%%%%%%%%%%%%%%%%%%%%%%%%%%%%%%%%%%%%%%%%%%%
%                                                                 %
%  GEANT manual in LaTeX form                              %
%                                                                 %
%  Michel Goossens (for translation into LaTeX)                   %
%  Version 1.00                                                   %
%  Last Mod. Jan 24 1991  1300   MG + IB                          %
%                                                                 %
%%%%%%%%%%%%%%%%%%%%%%%%%%%%%%%%%%%%%%%%%%%%%%%%%%%%%%%%%%%%%%%%%%%
\Origin{G.N.Patrick, L.Urb\'{a}n}
\Submitted{26.09.83}\Revised{16.12.93}
\Version{Geant 3.16}\Routid{PHYS221}
\Makehead{Simulation of Compton scattering}
\section{Subroutines}
\Shubr{GCOMP}{}
\Rind{\tt GCOMP} generates the Compton scattering of a photon on an 
atomic electron. It uses the random number techniques of Butcher and 
Messell \cite{bib-BUTC} to sample the scattered photon energy according 
to the Klein-Nishina formula \cite{bib-KLEI}.

The interaction produces one electron, which is put in the \FCind{/GCKING/} 
common block for further tracking. Tracking of the scattered photon will
continue, with direction and energy changed by the interaction.
All input/output information is through {\tt GEANT} common blocks.
 
\begin{tabular}{ll}
Input: & via COMMON \FCind{/GCTRACK/} \\
Output: & via COMMONs \FCind{/GCTRAK/} and \FCind{/GCKING/} 
\end{tabular}
 
Compton scattering is selected in {\tt GEANT} by the input data 
record {\tt COMP}. When Compton scattering is selected, \Rind{GCOMP} 
is called automatically from the {\tt GEANT} photon tracking
routine \Rind{GTGAMA}.

\section {Method}
 
For a complete account of the Monte Carlo methods used the
interested
user is referred to the publications of Butcher and Messel 
\cite{bib-BUTC}, Messel
and Crawford \cite{bib-MESS} and Ford and Nelson \cite{bib-EGS3}.
Only the basic formalism is outlined here.
 
The quantum  mechanical Klein-Nishina differential cross-section
is:
\[
\Phi(E,E') =\frac{X_0 n \pi r_0^2 m_{\rm e}}{E^2}
     \left[\frac{1}{\epsilon}+\epsilon\right]
     \left[1 - \frac{\epsilon \sin^2 \theta}{1+\epsilon^2}\right]
\]
where,\quad
\begin{tabular}[t]{l@{\ = \ }l}
$E$         & energy of the incident photon   \\
$E'$        & energy of the scattered photon  \\
$\epsilon$  & $E'/E$                          \\
$m_{e}$     & electron mass                   \\
$n$         & electron density                \\
$r_0$       & classical electron radius       \\
$X_0$       & radiation length
\end{tabular}
 
Assuming an elastic collision, the scattering angle $\theta$ is
defined by the Compton formula:
 
\[
E'   = E \frac{m_{\rm e}}{ m_{\rm e} + E(1-\cos\theta )}
\]
 
Using the combined ``composition and rejection'' Monte Carlo methods
described in chapter {\tt PHYS211}, we may set:
 
\[
\begin{array}{LcLLcL}
f(\epsilon)   & = & \left[\frac{1}{\epsilon}+\epsilon\right] =
                    \sum^{2}_{i=1} \alpha_i f_i(E)
                    &  \multicolumn{2}{l}{\mbox{for}}
                    & \epsilon_0 > \epsilon > 1     \\
g(\epsilon)   & = & \left[ 1 - \frac{\epsilon\sin^2\theta}{1+\epsilon^2}
                    \right] & \multicolumn{3}{l}{\mbox{rejection function}} \\
\alpha_1      & = & \frac{1}{\ln(1/\epsilon_0)}   &
\alpha_2      & = & \frac{1}{2} (1-\epsilon_0^2)                             \\
f_1(\epsilon) & = & \frac{1}{\epsilon\ln(1/\epsilon_0)} &
f_2(\epsilon) & = & \frac{2\epsilon}{1-\epsilon^2}
\end{array}
\]
 
The value of $\epsilon$ corresponding to the minimum
photon energy (backward scattering) is given by:
\begin{eqnarray*}
\epsilon_0 & = & \frac{1}{1+2E/m_{\rm e}}
\end{eqnarray*}

Given a set of random numbers $r_i$ uniformly distributed in [0,1],
the sampling procedure for $\epsilon$ is the following:
\begin{enumerate}
\item
decide which element of the $f(\epsilon)$ distribution to sample from.
Let $\alpha_T = (\alpha_1+\alpha_2)r_0$. If $\alpha_1\geq\alpha_T$
select $f_1(\epsilon)$, otherwise select $f_2(\epsilon)$;
 
\item  sample $\epsilon$ from the distributions
corresponding to $f_1$ or $f_2$. For $f_1$ this is simply achieved by:
\[
\epsilon = \epsilon_0 e^\alpha_1 r_1
\]
For $f_2$, we change variables and use:
\[
\epsilon' = \left\{ \begin{array}{lll}
\max(r_2,r_3) & \mbox{for } & E/m \geq (E/m+1)r_4 \\
r_5           & \multicolumn{2}{l}{\mbox{for all other cases}}
 \end{array} \right.
\]
Then, $\epsilon = \epsilon_0+(1-\epsilon_0)\epsilon'$;
 
\item calculate $\sin^2\theta = \max(0,t(2-t))$ 
where $t=m_{\rm e}(1-\epsilon)/E'$ 

\item test the rejection function, if $r_6 \leq g(\epsilon)$ accept
$\epsilon$, otherwise return to step 1.
\end{enumerate}

After the successful sampling of $\epsilon$, \Rind{GCOMP} generates the
polar angles of the scattered photon with respect to the direction of
the parent photon. The azimuthal angle, $\phi$, is generated isotropically and
$\theta$ is as defined above. The momentum vector of the scattered
photon is then calculated according to kinematic considerations. Both
vectors are then transformed into the {\tt GEANT} coordinate system.

\section{Restriction}
 
The differential cross-section is only valid for those
collisions in which the energy of the recoil electron is large compared
with its binding energy (which is ignored). However, as pointed out by
Rossi \cite{bib-ROSS}, this has a negligible effect 
because of the small number of
recoil electrons produced at very low energies.
