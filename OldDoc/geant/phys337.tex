%%%%%%%%%%%%%%%%%%%%%%%%%%%%%%%%%%%%%%%%%%%%%%%%%%%%%%%%%%%%%%%%%%%
%                                                                 %
%  GEANT manual in LaTeX form                                     %
%                                                                 %
%  Michel Goossens (for translation into LaTeX)                   %
%  Version 1.00                                                   %
%  Last Mod. Jan 24 1991  1300   MG + IB                          %
%                                                                 %
%%%%%%%%%%%%%%%%%%%%%%%%%%%%%%%%%%%%%%%%%%%%%%%%%%%%%%%%%%%%%%%%%%%
\Authors {F.Carminati} \Origin{N.Van Eijndhoven}
\Version{Geant 3.16}\Routid{PHYS337}
\Submitted {26.07.93}\Revised{16.12.93}
\Makehead {Birks' saturation law}
\section{Subroutines}
\Shubr{GBIRK}{(EDEP)}

\begin{DLtt}{MMMMMMMM}
\item[EDEP] ({\tt REAL}) and on output the energy
equivalent to the calorimeter response in the current step.
\end{DLtt}

Organic scintillators are usually calibrated with particles whose
energy is near the minimum ionisation ($\gamma \beta \approx 4$).
However, the energy response of such scintillators 
for large local energy deposit is
attenuated. \Rind{GBIRK} returns the energy {\it detected} by
the scintillator in the current step.

\section{Method}

The phenomenological description of the response attenuation of
organic scintillators~\cite{bib-BIRK} is known as Birks' law:
\begin{equation}
R = \frac{\Delta E}{1 + C_1 \delta + C_2 \delta^2} 
\hspace{4cm}
\delta = \frac{1}{\rho} \frac{dE}{dx} \hspace{0.5cm} 
\mbox{MeV g$^{-1}$ cm$^{2}$}
\end{equation}
The values quoted in~\cite{bib-BIR1} for the parameters $C_1, C_2$
are:
\[
C_1 = 0.013 \: \mbox{g MeV$^{-1}$ cm$^{-2}$ \hspace{1.5cm} and \hspace{1.5cm}}
C_2 = 9.6 \times 10^{-6} \: \mbox{g$^{2}$ MeV$^{-2}$ cm$^{-4}$}
\]
These values have been measured for various scintillators.
If the charge of the particle is greater than one, a better description
can be obtained by correcting $C_1$:
\begin{equation}
\label{eq:phys337-1}
C_1' = \frac{7.2}{12.6} C_1 \approx 0.5714 C_1
\end{equation}

The values of the parameters of Birks' formula (if defined) are in the
{\tt ZEBRA} bank next to the material bank:
\begin{DLtt}{MMMMMMMMMMMMMMMMMMMM}
\item[JTM = LQ(JMATE-NTMED)] pointer to the current tracking medium bank;
\item[JTMN = LQ(JTM)] pointer to the {\it next} bank to the current
material one, where the specific tracking medium parameters are stored;
\item[MODEL = Q(JTMN+27)] flag controlling the correction (\ref{eq:phys337-1})
for multiply charged particles:
\begin{DLtt}{MMMM}
\item[$\neq$1] correction is not applied;
\item[$=$1] correction is applied;
\end{DLtt}
\item[C1 = Q(JTMN+28)] first parameter of the Birks' formula;
\item[C2 = Q(JTMN+29)] second parameter of the Birks' formula.
\end{DLtt}
These parameters are set via the \Rind{GSTPAR} routine, with
names {\tt BIRK1}, {\tt BIRK2} and {\tt BIRK3}, respectively. For instance,
to define the standard Birks' parameters for tracking medium {\tt ITM}
with correction for multiply charged particles, one would have to insert
the following piece of code, after the definition of tracking media
but before the call to \Rind{GPHYSI}
\begin{verbatim}
      CALL GSTPAR(ITM,'BIRK1',1.)
      CALL GSTPAR(ITM,'BIRK2',0.013)
      CALL GSTPAR(ITM,'BIRK3',9.6E-6)
\end{verbatim}
