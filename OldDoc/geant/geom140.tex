%%%%%%%%%%%%%%%%%%%%%%%%%%%%%%%%%%%%%%%%%%%%%%%%%%%%%%%%%%%%%%%%%%%
%                                                                 %
%  GEANT manual in LaTeX form                              %
%                                                                 %
%  Michel Goossens (for translation into LaTeX)                   %
%  Version 1.00                                                   %
%  Last Mod. Jan 24 1991  1300   MG + IB                          %
%                                                                 %
%%%%%%%%%%%%%%%%%%%%%%%%%%%%%%%%%%%%%%%%%%%%%%%%%%%%%%%%%%%%%%%%%%%
\Origin{F.Bruyant, A.McPherson, M.Maire}
\Submitted{17.12.83}       \Revised{18.11.93}
\Version{Geant 3.16}\Routid{GEOM140}
\Makehead{Division of a Volume into cells of a given size}

This routine creates new volumes from a mother 
by divisions of a given step.

\Shubr{GSDVT}{(CHNAME,CHMOTH,STEP,IAXIS,NUMED,NDVMX)}

\begin{DLtt}{MMMMMMMM}
\item[CHNAME] ({\tt CHARACTER*4}) a unique name for the volume to be generated
by subdivision of the mother volume;
\item[CHMOTH] ({\tt CHARACTER*4}) volume that has to be subdivided;
\item[STEP] ({\tt REAL}) size of the divisions -- this value can be in 
centimeters or degrees according to the value of {\tt IAXIS};
\item[IAXIS] ({\tt INTEGER}) {\it axis} of the division.
\item[NUMED] ({\tt INTEGER}) medium number of the divisions -- this can be 
different from the one of the mother, as the division cells may leave a
portion of the mother undivided (see below) --
if {\tt NUMED} $\leq 0$  the medium of the mother;
\item[NDVMX] ({\tt INTEGER}) expected (maximum) number of divisions -- if
$ \leq 0 $ or $ > 255 $, 255 is assumed.
\end{DLtt}

The full range of the mother will be divided
in sections of the user supplied step. If the step is such that the mother
cannot be divided exactly, the largest possible number
of divisions will be generated, the excess space will be equally
divided between each end of the range of the mother. These extra
spaces will be assumed to belong to the mother.

For more information on the division in general, see {\tt [GEOM130]}.
