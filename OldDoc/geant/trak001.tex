%%%%%%%%%%%%%%%%%%%%%%%%%%%%%%%%%%%%%%%%%%%%%%%%%%%%%%%%%%%%%%%%%%%
%                                                                 %
%  GEANT manual in LaTeX form                              %
%                                                                 %
%  Michel Goossens (for translation into LaTeX)                   %
%  Version 1.00                                                   %
%  Last Mod. Jan 24 1991  1300   MG + IB                          %
%                                                                 %
%%%%%%%%%%%%%%%%%%%%%%%%%%%%%%%%%%%%%%%%%%%%%%%%%%%%%%%%%%%%%%%%%%%
\Documentation{F.Bruyant}
\Submitted{15.08.84}      \Revised{16.12.93}
\Version{Geant 3.16}       \Routid{TRAK001}
\Makehead{The tracking package}
\section{Introduction}
 
In {\tt GEANT} {\it tracking} a particle through a geometry of
objects consists in calculating a set of points in a seven-dimensional
space ($x$, $y$, $z$, $t$, $p_x$, $p_y$, $p_z$) which is called the
trajectory of the particle. This is achieved by integrating the equations 
of motion over successive {\it steps} from one trajectory point to the
next and applying corrections to account for the presence of matter. 

To have a detailed description of the kinematic of the particle it would
be necessary to calculate a trajectory point every time the component
of the momentum change. This is not possible because it would mean 
calculating an enormously large number of points. Processes like the
deviation of a charged particle in a magnetic field, the loss of energy
due to bremsstrahlung and ionisation or the deviation due to elastic
electromagnetic scatterings are essentially continuous. An arbitrary
distinction is thus made between discrete and continuous processes,
controlled by a set of thresholds which can be set by the user.
A particle trajectory is thus a set of points at which a discrete process
has happened. 

The tracking package contains a subprogram
which performs the tracking for
all particles in the current event and for the secondary
products which they generate, plus some tools
for storing the space point coordinates
computed along the corresponding trajectories.

\section{The step size}
 

When tracking particles through a complex structure one of the
critical tasks is the estimation {\it a priori} of the step size. 
This is performed automatically by the program.
For a particle with a given energy the step size depends
primarily on the intrinsic properties of the particle (mass,
charge, lifetime, etc.) and on the characteristics of the current
medium. The dependence may be due either to the (quasi)continuous
processes which usually impose a limit to the interval
of integration (energy loss, multiple scattering ...)
or to the occurrence of a discrete process which introduces
a discontinuity in the trajectory (decay, electromagnetic or hadronic
interaction). 
The modification of the kinematic parameters (position
and energy) due to continuous processes between two discrete ones is
taken into account and the necessary modifications are applied at the end
of the step. 

In addition to the physical effects
there are constraints of a geometrical nature, the step being limited by
the path length to the medium boundary.
In practice, the step size depends ultimately on a set of tolerances and cuts,
which the program will set automatically, but which may be also
optimised by the user, such as:

\begin{itemize}
\item the maximum turning angle due to magnetic field permitted in one step;
\item the maximum fractional energy loss in one step;
\item the maximum geometrical step allowed;
\item the accuracy for crossing medium boundaries;
\item the minimum step size due to either energy loss or multiple scattering.
\end{itemize}

These quantities are part of the so-called tracking medium parameters. They
can either be calculated by the program or 
provided by the user and are stored in the
data structure {\tt JTMED}, through the routine \Rind{GSTMED} (see {\tt [CONS]}). 
Usually, this is done together with the initialisation of the geometrical setup.
The optimisation is by no means trivial as the economy of computing time
should not lead to an unacceptable loss of accuracy.

Other information required for the computation
of the step size is found in the data
structures {\tt JPART} and {\tt JMATE}, for the
properties of the particles and of the materials, and in the
data structure {\tt JVOLUM}, for the current medium and its geometrical
boundaries.
The communication between the tracking package and the
structure {\tt JVOLUM} is achieved through the basic subroutines of the
geometry package \Rind{GMEDIA}, \Rind{GTMEDI},
\Rind{GTNEXT} and \Rind{GINVOL} (see {\tt [GEOM]}).
Some information is computed at tracking time
such as the probability of occurrence of an interaction. 

For convenience
every particle is assigned a tracking type:
\begin{enumerate}
\item $\gamma$ (\Rind{GTGAMA});
\item e$^{\pm}$ (\Rind{GTELEC});
\item neutral hadrons and neutrinos (\Rind{GTNEUT});
\item charged hadrons (\Rind{GTHADR});
\item muons (\Rind{GTMUON});
\item geantinos, particles only sensitive to geometry
used for debugging (\Rind{GTNINO});
\item \v{C}erenkov photons (\Rind{GTCKOV});
\item heavy ions (\Rind{GTHION}).
\end{enumerate}

Which processes have
to be considered for a given particle depends on its tracking type.
For the hadrons it depends also, through the subroutine \Rind{GUPHAD}, on
which hadronic processes from {\tt GHEISHA, FLUKA} have been selected
(see {\tt [PHYS001]}).

\section{Transport subroutines}
 
At the event level the tracking is  controlled by the subroutine \Rind{GTREVE}
called by the subroutine \Rind{GUTREV} where the user can perform additional
actions.
\Rind{GTREVE} loops over all vertices and for every vertex in turn
stores all tracks
in the stack {\tt JSTAK}. Then tracking begins: particles are fetched
from {\tt JSTAK} by \Rind{GLTRAC} which prepares the commons for tracking.
\Rind{GUTRAK} is then called, which calls \Rind{GTRACK}.
The subroutine  \Rind{GTRACK}  transports  the  particle up to the its end:
stop, decay, interaction or escape. During this phase it may
happen that  secondary products have  been generated and  stored by
the user, as explained below, in the {\tt JSTAK} stack, and possibly in the
permanent structure {\tt JKINE}.

The subroutine \Rind{GTRACK} transports the track through the geometrical 
volumes
identifying, through the subroutine
\Rind{GTMEDI}, every new volume which the particle has reached, and storing
the corresponding  material and  tracking medium  constants in  the
common blocks \FCind{/GCMATE/} and \FCind{/GCTMED/}. The actual transport
is performed by a different routine for each tracking type, as indicated in
the previous section.
These compute the physical step size according to the activated
physics processes, and compute the geometrical limit for the step,
only when necessary, through \Rind{GTNEXT}, and propagate the particle
over the computed step.

\section{Magnetic field transport routines}
Once the step size has been decided, transport proceeds along a straight
lines for all neutral particles and for charged particle in absence of
magnetic field. When magnetic field is present, the direction of charged
particles will change along the step. The routine \Rind{GUSWIM} calculates
the deviation in magnetic field over a given step. 
Depending on 
the tracking medium parameter {\tt IFIELD} the \Rind{GUSWIM} routine
calls:
\begin{itemize}
\item \Rind{GRKUTA} for inhomogeneous fields, {\tt IFIELD=1};
\item \Rind{GHELIX} for quasi-homogeneous fields, {\tt IFIELD=2};
\item \Rind{GHELX3} for uniform fields along the $z$ axis, {\tt IFIELD=3}).
\end{itemize}
\Rind{GRKUTA} and \Rind{GHELIX} call the user subroutine
\Rind{GUFLD} to obtain the components of the field in a given point.
\Rind{GHELX3} takes the value of the field from the tracking medium parameter
{\tt FIELDM}.

\section{The subroutine {\tt GUSTEP}}
 
The current track parameters are available in the common
\FCind{/GCTRAK/} together with all variables necessary to 
the tracking routines for the control of the step size. In addition, a few
flags and variables are
stored in the common block \FCind{/GCTRAK/} to record the history of the
current step:

\begin{itemize}
\item The flag {\tt INWVOL} informs on the boundary crossing:
\begin{DLtt}{MMMM}
\item[0] transport inside a volume;
\item[1] entering a new volume or beginning of new track;
\item[2] exiting a volume;
\item[3] exiting the first mother;
\end{DLtt}
\item  The flag {\tt ISTOP} informs on the particle status:
\begin{DLtt}{MMMM}
\item[0] normal transport;
\item[1] particle has disappeared (decay, interaction...);
\item[2] particle has crossed a threshold (time, energy...).
\end{DLtt}
\item The array {\tt LMEC} informs on the {\it mechanisms} active in the
current step. The mechanism names are stored in ASCII equivalent in 
{\tt LMEC(1...NMEC)}.
\item The total energy loss in the current step is
stored in the variable {\tt DESTEP}.
\end{itemize}

This information is necessary for the user to take
the proper actions in the subroutine
\Rind{GUSTEP} which is called by \Rind{GTRACK} at the end of every step
and when entering a new volume.

The variable {\tt NGKINE} in common \FCind{/GCKING/} contains the number of
secondary particles generated at every step, and which are stored in the
same common.
Depending on the application and on the particle type the user may then
take the appropriate action in \Rind{GUSTEP}:
\begin{itemize}
\item ignore the particle;
\item store the secondary produced in the {\tt JSTAK} stack for further
tracking;
\item store the secondary also in the {\tt JKINE/JVERTX} structure where
it will be kept till the end of the event.
\end{itemize}

\section{Connection with the detector response package}
The detector response package ({\tt [HITS]}) allows to establish a 
correspondence between the volumes seen by the particle and
the active components of the detectors. When entering a new volume in
\Rind{GTRACK} the subroutine
\Rind{GFINDS} is called. If the volume has been declared by the user as a
sensitive detector through appropriate calls to \Rind{GSDET} and if the
corresponding tracking medium constant {\tt ISVOL} is non zero,
\Rind{GFINDS} fills the common block \FCind{/GCSETS/} with the information to
identify uniquely the detector component. This enables the user, in
\Rind{GUSTEP}, to record the
hits in the proper {\tt JHITS} substructure {\tt[HITS]}.

\section{Connection with the drawing package}
The coordinates of the space points generated during the tracking are
available at each step in the common block\FCind{/GCTRAK/}. In
\Rind{GUSTEP} the user can store them in the structure {\tt JXYZ} with the
help of the subroutine \Rind{GSXYZ}. This information can be used later
for debugging (subroutine \Rind{GPJXYZ}) or for the graphical representation
of the particle trajectories {\tt[DRAW]}.
