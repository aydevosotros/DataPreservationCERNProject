%%%%%%%%%%%%%%%%%%%%%%%%%%%%%%%%%%%%%%%%%%%%%%%%%%%%%%%%%%%%%%%%%%%
%                                                                 %
%  GEANT manual in LaTeX form                                     %
%                                                                 %
%  Version 1.00                                                   %
%                                                                 %
%  Last Mod. 12 June 1993 1700   MG                               %
%                                                                 %
%%%%%%%%%%%%%%%%%%%%%%%%%%%%%%%%%%%%%%%%%%%%%%%%%%%%%%%%%%%%%%%%%%%
\Authors{R.Brun}               \Origin{GEANT 2}
\Version{Geant 3.15}\Routid{CONS200}
\Submitted{01.06.83}             \Revised{13.04.92}
\Makehead{Tracking Medium Parameters and related routines}
\section{Description of the routines}
\Shubr{GSTMED}{(ITMED,NATMED,NMAT,ISVOL,IFIELD,FIELDM,TMAXFD,STEMAX,
                DEEMAX,EPSIL,STMIN,UBUF,NWBUF)}
Stores the parameters of the tracking
medium {\tt ITMED} in the data structure {\tt JTMED}.
\begin{DL}{MMMMM}
\item[ITMED]      tracking medium number
\item[NATMED]     {\tt CHARACTER*20}, tracking medium name (optionally 
                  terminated by \$)
\item[NMAT]       material number corresponding to {\tt ITMED}
\item[ISVOL]      sensitivity flag
\begin{DL}{MMM}
\item[$\leq$0] not a sensitive volume
\item[$>$0] sensitive volume
\end{DL}
\item[IFIELD]     magnetic field flag
\begin{DL}{MMM}
\item[=0]         no magnetic field
\item[=1]         tracking performed with \Rind{GRKUTA}
\item[=2]         tracking performed with \Rind{GHELIX}
\item[=3]         tracking performed with \Rind{GHELX3}
\end{DL}
\item[FIELDM]     maximum field value (in Kilogauss)
\item[TMAXFD]     maximum angle due to field permitted in
                  one step (in degrees)
\item[STEMAX]     maximum step permitted (cm)
\item[DEEMAX]     maximum fractional energy loss in one step ($0<$
                  {\tt DEEMAX} $\leq 1$)
\item[EPSIL]      boundary crossing precision (cm)
\item[STMIN]      minimum step due to energy loss or multiple scattering 
                  or magnetic field (cm)
\item[UBUF]       array of {\tt NWBUF} additional user parameters
\item[NWBUF]      number of additional user parameters
\end{DL}
 
\Shubr{GFTMED}{(ITMED,NATMED*,NMAT*,ISVOL*,IFIELD*,FIELDM*,TMAXFD*,
                STEMAX*,DEEMAX*,EPSIL*,STMIN*,UBUF*,NWBUF*)}
 
Extracts the parameters describing the tracking medium {\tt ITMED}
from the data structure {\tt JTMED}.
 
\Shubr{GPTMED}{(ITMED)}
 
Prints the tracking medium parameters for the given medium.
\begin{DL}{MMMMM}
\item[ITMED]      Tracking medium {\tt ITMED}.
                  (for all tracking media if {\tt ITMED}=0)
\end{DL}

\section{Automatic calculation of the tracking parameters}

The results of the simulation depend critically on the choice of the tracking
medium parameters. To make of GEANT a reliable and consisten predictive
tool, the calculation of these parameters has been automated.
In a normal GEANT run, when the {\tt IGAUTO}$=1$
(common \FCind{/GCTRAK/}), the following parameters are calculated by the program:
\begin{DL}{MMMMM}
\item[TMAXFD]    is set to:
\begin{DL}{MMMMMM}
\item[$20^\circ$]      if ${\tt TMAXFD}\leq 0$ or ${\tt TMAXFD} > 20$
\item[{\it input value}]     if $0 < {\tt TMAXFD}\leq 20$
\end{DL}
\item[STEMAX] is set to {\tt BIG} ($=10^{10}$)
\item[DEEMAX] is set to:
\begin{DL}{MMMMMM}
\item[$0.25$]    if ${\tt ISVOL} = 0$ and $X_{0} \leq 2$, where $X_{0}$ is the
               radiation length of the material \vspace{.3cm}
\item[$0.25-\displaystyle{\frac{0.2}{\sqrt{X_{0}}}}$] otherwise
\end{DL}
\item[STMIN] is set to: \vspace{.3cm}
\begin{DL}{MMMMMM}
\item[$\displaystyle{\frac{2 R}{\sqrt{X_{0}}}}$] if ${\tt ISVOL} = 0$
                                \vspace{.3cm}
\item[$\displaystyle{\frac{5 R}{\sqrt{X_{0}}}}$] if ${\tt ISVOL} \neq 0$
                                \vspace{.3cm}
\end{DL}
where $R$ is the range in cm of an electron of energy {\tt CUTELE}+200KeV
\end{DL}

If the {\tt IGAUTO } variable is set to 0 via the {\tt AUTO} data record, 
then value given by the user for the above parameters is accepted as the
true parameter value if $>0$, while automatic calculation still takes place
in case the input value is negative.

Users are strongly recommended to begin their simulation with the parameters
as calculated by GEANT. Users who want to modify any of the above
parameters must be sure they understand their function in the program and
the implications of a change.

The {\tt EPSIL} parameter represents the absolute precision with which the
tracking is performed. It has a double meaning. When tracking in magnetic
field, {\tt EPSIL} is the minimum distance for which boundaries are
checked. A particle nearer than {\tt EPSIL} to the boundary is considered
as exiting the current volume. If the end point of the step of a particle in
magnetic field is distant less than {\tt EPSIL} along each axis 
from what would be the end point in absence of magnetic field, then no boundary 
checking is performed. 

In all cases, when a particle is going to reach the
boundary of the current volume with the current step, the step length is 
increased by a quantity ({\tt PREC}, common \FCind{/GCTMED/}) which is set to the 
minimum between $0.1 \times {\tt EPSIL}$
and 10 micron at the beginning of the tracking. This quantity is recalculated
at every step according to the formula:
\begin{equation}
{\tt PREC} = \max \left ( \min \left (\frac{\tt EPSIL}{10}, 10 \mu \right ) ,
\max \left ( | x |, | y |, | z |, S \right ) \times \epsilon \right )
\end{equation}
Where $x,y,z$ are the current coordinates of the particle, $S$ is the length
of the track, and $\epsilon$ is the assumed machine precision.
$\epsilon$ (called {\tt EPSMAC} in the program) is initially set to
$10^{-6}$ for 32 bits machines and $10^{-11}$ for 64 bits machines.
During the tracking, every fifth time that a particle tries unsuccessfully
to exit from the same volume, the machine precision is multiplied by the
number of trials. This accounts for additional losses of precision due to
transformation of coordinates and region of the floating point range where
the real machine precision is different from the above (this happens 
in particular on IBM mainframes with 370 floating point number representation).
%\end{document}
 
 
