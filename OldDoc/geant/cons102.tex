%%%%%%%%%%%%%%%%%%%%%%%%%%%%%%%%%%%%%%%%%%%%%%%%%%%%%%%%%%%%%%%%%%%
%                                                                 %
%  GEANT manual in LaTeX form                                     %
%                                                                 %
%  Michel Goossens (for translation into LaTeX)                   %
%  Version 1.00                                                   %
%  Last Mod. Jan 24 1991  1300   MG + IB                          %
%                                                                 %
%%%%%%%%%%%%%%%%%%%%%%%%%%%%%%%%%%%%%%%%%%%%%%%%%%%%%%%%%%%%%%%%%%%
\Authors {R.Brun, M.Maire}     \Origin{M.Maire}
\Version{Geant 3.15}\Routid{CONS102}
\Submitted{30/05/86}           \Revised{10/12/92}
\Makehead{Plot Material Cross Sections}
\Shubr{GPLMAT}{(IMATE,IPART,MECA,KDIM,TKIN,IDM)}
 
User routine to
{\bf plot} and {\bf interpolate} the $dE/dx$ and cross-sections
tabulated in {\tt JMATE} banks corresponding to:
material {\tt IMATE}, particle {\tt IPART}, mechanism name {\tt MECA},
kinetic energies {\tt TKIN}
 
The {\tt MECA}nism name can be :
\begin{center}
{\tt
'LOSS' 'PHOT' 'ANNI' 'COMP' 'MUNU' 'BREM' \\
'PAIR' 'DRAY' 'PFIS' 'HADG' 'HADF' 'ALL'
}
\end{center}
For Hadronic particles it also computes the total
hadronic cross section from \Rind{GHEISHA}
({\tt'HADG'}) or \Rind{FLUKA} ({\tt 'HADF'}) programs.
 
{\bf Input parameters:}
\begin{DL}{MMMM}
\item[IMATE] ({\tt INTEGER}) {\tt GEANT} material number
\item[IPART] ({\tt INTEGER}) {\tt GEANT} particle number
\item[MECA]  ({\tt CHARACTER*4})
mechanism name of the data to be plotted;  if {\tt MECA =
'ALL'} all the tables for particle  {\tt IPART} will be plotted
\item [KDIM]  ({\tt INTEGER}) dimension of the array {\tt TKIN} (maximum
100)
\item[TKIN] ({\tt REAL}) array of kinetic energies of incident
particle (in GeV)
\end{DL}
 
{\bf Output parameters:}
\begin{DL}{MMMM}
\item[IDM]({\tt INTEGER}) treatment of the created histogram(s):
\begin{DL}{MMMM}
\item[${\tt IDM > 0}$]fill, print, keep histogram(s)
\item[${\tt IDM = 0}$]fill, print, delete histogram(s)
\item[${\tt IDM < 0}$]fill, noprint, keep histogram(s)
\end{DL}
The histogram IDentifier will be:{\tt 10000*IMATE+100*IPART+IMECA}
where, {\tt IMECA} is the link number in structure {\tt JMATE}
(see {\tt [CONS199]}) {\tt  IMECA = 13} for {\tt'HADF'}, {\tt IMECA = 14}
for {\tt'HADG'}
\end{DL}
 
{\bf Called by:} \Rind{<USER>}
%\end{document}
