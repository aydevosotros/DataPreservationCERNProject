\chapter[Photoabsorption ionization model]{Photoabsorption ionization model}

\section{Method}

In the framework of the PhotoAbsorption Ionization (PAI) model we describe the 
ionization energy loss produced by a relativistic charged particle in matter. 
The differential 
cross section $d\sigma_i/d\omega$ of ionizing collisions with the energy transfer
$\omega$ produced by a relativistic
particle in a matter in the most general form can be expressed by the following
equations \cite{asosk}:

\begin{eqnarray}
\frac{d\sigma_i}{d\omega} 
& = & \frac{2\pi Ze^4}{mv^2}    
\left\{
\frac{f(\omega)}{\omega\left|\varepsilon(\omega)\right|^2}
\left[
\ln\frac{2mv^2}{\omega\left|1-\beta^2\varepsilon\right|} -  
\right. \right. \nonumber \\
&& \left. \left.
- \frac{\varepsilon_1 - \beta^2\left|\varepsilon\right|^2}{\varepsilon_2}
\arg(1-\beta^2\varepsilon^*)
\right] + 
 \frac{\tilde{F}(\omega)}{\omega^2}
\right\} ,
\end{eqnarray}

\[
\tilde{F}(\omega) = \int_{0}^{\omega}\frac{f(\omega')}
{\left|\varepsilon(\omega')\right|^2}d\omega' ,
\]

\[
f(\omega) = \frac{m\omega\varepsilon_2(\omega)}{2\pi^2ZN\hbar^2} ,
\]

where $m$ and $e$ are the electron mass and charge respectively, $\hbar$ is
the Planck constant, $\beta = v/c$ is ratio of the particle velocity $v$ to
the speed of light $c$, $Z$ is the effective atomic number, $N$ is the number of
atoms (molecule) in unit volume, and $\varepsilon = \varepsilon_1 + i\varepsilon_2$
 is the complex dielectric constant of the medium. In an isotropic nonmagnetic 
medium
the dielectric constant can be expressed in terms of complex index of refraction, 
$n(\omega) = n_1 + in_2$, $\varepsilon(\omega) = n^2(\omega)$. In the energy range
above the first ionization potential $I_1$ for all matters of practical interest,
particular in all gases, $n_1 \sim 1$ . Therefore the imaginary part of the
dielectric constant can be expressed in terms of the photoabsorption cross
section $\sigma_{\gamma}(\omega)$:

\[
\varepsilon_2(\omega) = 2n_1n_2 \sim 2n_2 = \frac{N\hbar c}{\omega}
\sigma_{\gamma}(\omega) .
\]

The real part of the dielectric constant is calculated in turn from the dispersion
relation:

\[
\varepsilon_1(\omega) - 1 = \frac{2N\hbar c}{\pi}V.p.\int_{0}^{\infty}
\frac{\sigma_{\gamma}(\omega')}{\omega'^2 - \omega^2}d\omega'  ,
\]

where the integral of the pole expression is considered in terms of the
principal value. In practice it is convenient to calculate the contribution from 
the continuos part of the spectrum only. Then we have to use the normalised
photoabsorption cross section $\tilde{\sigma}_{\gamma}(\omega)$ :

\[
\tilde{\sigma}_{\gamma}(\omega) = \frac{2\pi^2\hbar e^2Z}{mc}
\sigma_{\gamma}(\omega)
\left[
\int_{I_1}^{\omega_{max}}\sigma_{\gamma}(\omega')d\omega'
\right]^{-1}, \  \omega_{max} \sim 100 \ keV ,
\]

which satisfies the quantum mechanical sum rule \cite{fano}:

\[
\int_{I_1}^{\omega_{max}}\tilde{\sigma}_{\gamma}(\omega')d\omega' = 
\frac{2\pi^2\hbar e^2Z}{mc} .
\]

The differential cross section of ionizing collisions is expressed by the 
photoabsorption cross section in the continuous spectrum region:

\begin{eqnarray}
\frac{d\sigma_i}{d\omega}
& = & \frac{\alpha}{\pi\beta^2}
\left\{
\frac{\tilde{\sigma}_{\gamma}(\omega)}
{\omega\left|\varepsilon(\omega)\right|^2}
\left[
\ln\frac{2mv^2}{\omega\left|1-\beta^2\varepsilon\right|} - 
\right. \right.   \nonumber \\
&   & \left. \left.
- \frac{\varepsilon_1-\beta^2\left|\varepsilon\right|^2}{\varepsilon_2}
\arg(1-\beta^2\varepsilon^*)
\right] 
 + \frac{1}{\omega^2}\int_{I_1}^{\omega}\frac{\tilde{\sigma}_{\gamma}(\omega')}
{\left|\varepsilon(\omega')\right|^2}d\omega'
\right\} ,
\end{eqnarray}

\[
\varepsilon_2(\omega) = \frac{N\hbar c}{\omega}
\tilde{\sigma}_{\gamma}(\omega) ,
\]

\[
\varepsilon_1(\omega) - 1 = \frac{2N\hbar c}{\pi}V.p.\int_{I_1}^{\omega_{max}}
\frac{\tilde{\sigma}_{\gamma}(\omega')}{\omega'^2 - \omega^2}d\omega'  .
\]

For practical calculations according (9.2) it is convinient to use the 
representation of the photoabsorption cross section as polinom of $\omega^{-1}$ as
was proposed in \cite{sandia}:

\[
\sigma_{\gamma}(\omega) = \sum_{k=1}^{4}a_{k}^{(i)}\omega^{-k} ,
\]

where the coefficients, $a_{k}^{(i)}$ are fitted with experimental data by the
least square method separately in each i-th energy interval. The interval borders
are equal as a rule to the corresponding photoabsorption edges. Then the dielectric
constant can be calculated analytically in elementary functions for all $\omega$
except of photoabsorption edges where the photoabsorption cross section 
experiences breaks and the integral for the real part is not defined in the 
sence of the principal value.

The third term in (9.2), which can be calculated numerically only, results in
complex procedure of calculation of $d\sigma_i/d\omega$. However, one can see
that this term dominates for the energy transfers $\omega > 10\ keV$, where the
function $\left|\varepsilon(\omega)\right|^2 \sim 1$. It is clear from physical
reasons, since the third term represents the Rutheford cross section on atomic
electrons which can be considered as quasifree for given energy transfer \cite
{allis}. In addition, for high energy transfers , 
$\varepsilon(\omega) = 1 - \omega_{p}^{2}/\omega^2 \sim 1$, where $\omega_{p}$ is
the plasma energy of the matter. Therefore the factor
$\left|\varepsilon(\omega)\right|^{-2}$ can be moved from the integral and the 
differential cross section of ionizing collisions can be expressed as:

\begin{eqnarray}
\frac{d\sigma_i}{d\omega}
& = &\frac{\alpha}
{\pi\beta^2\left|\varepsilon(\omega)\right|^2}
\left\{
\frac{\tilde{\sigma}_{\gamma}(\omega)}{\omega}
\left[
\ln\frac{2mv^2}{\omega\left|1-\beta^2\varepsilon\right|} - 
\right. \right. \nonumber \\ 
&   & \left. \left.
- \frac{\varepsilon_1-\beta^2\left|\varepsilon\right|^2}{\varepsilon_2}
\arg(1-\beta^2\varepsilon^*)
\right]
 + \frac{1}{\omega^2}\int_{I_1}^{\omega}\tilde{\sigma}_{\gamma}(\omega')d\omega'
\right\} ,
\end{eqnarray}

which is especially simple in gases when 
$\left|\varepsilon(\omega)\right|^{-2} \sim 1$  for all $\omega > I_1$
\cite{allis}.

\section{Simulation of energy losses}

For given length of track the number of ionizing collisions is simulated by
the Poisson distribution with the mean number proportional to the total cross
section of ionizing collisions:

\[
\sigma_i = \int_{I_1}^{\omega_{max}}\frac{d\sigma(\omega')}{d\omega'}d\omega' ,
\]

while the energy transfer in each collision is simulated according the
distribution proportional to:

\[
\sigma_i(>\omega) = \int_{\omega}^{\omega_{max}}
\frac{d\sigma(\omega')}{d\omega'}d\omega' .
\]

The sum of the energy transfers is equal to energy loss.

\section{Status of  this document}

16.11.98 created by V. Grichine .

\begin{thebibliography}{99}
\bibitem[Asosk82]{asosk} Asoskov V.S., Chechin V.A., Grichine V.M. at el,
{Lebedev Institute annual report, v. 140, p. 3} (1982)
\bibitem[Allis80]{allis} Allison W.W.M., and Cobb J.
{Ann.Rev.Nucl.Part.Sci., v.30,p.253} (1980)
\bibitem[Fano68]{fano} Fano U., and Cooper J.W.
{Rev.Mod.Phys., v. 40, p. 441} (1968)
\bibitem[Biggs90]{sandia} Biggs F., and Lighthill R.,
{Preprint Sandia Laboratory, SAND 87-0070} (1990)
\end{thebibliography}