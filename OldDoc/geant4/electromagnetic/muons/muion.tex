\chapter[Ionisation]{Ionisation}
 

 The class G4MuIonisation calculates the continuous energy loss
  due to ionisation and
 simulates the 'discrete' part of the ionisation , 
the delta ray production or knock-on electron production by muons .
 The formulae used here
 are essentially the same than those in GEANT3 (\cite{hion.geant3}).

\section{Method}

In {\tt GEANT4}, $\delta$-rays are generated generally 
only above a threshold energy
$T_{cut}$ which value is calculated from the parameter {\em cut in range}
for all  charged particles. The total cross-section
for the production of a $\delta$-ray electron of kinetic
energy $T > T_{cut}=$ by a particle of kinetic energy $E$
is:
\begin{equation}
\label{muion.a}
\sigma (E,T_{cut}) = \int_{T_{cut}}^{T_{max}} \frac{d \sigma (E,T)}{dT} dT
\end{equation}
where $T_{max}$ is the maximum energy transferable to the free electron:
\begin{equation}
T_{max} =\frac{2m(\gamma^2 -1)}
      {1+2\gamma\frac{m}{M}+
      \left(\frac{m}{M} \right)^2 }
\end{equation}

The energy lost in
ionising collisions producing $\delta$-rays below ${T_{cut}}$
are included in the continuous energy loss. The mean value of the
energy lost due to these {\it soft} collisions is:
\begin{equation}
\label{muion.b}
E_{loss}(E,T_{cut}) = \int_{0}^{T_{cut}}
T \frac{d \sigma (E,T)} {dT} dT
\end{equation}
where $m$ is the electron mass and $M$ is the mass of the muon.

\section{Continuous energy loss}
The integration of (\ref{muion.c}) leads to the Bethe-Bloch
stopping power or to the restricted energy
loss formula \cite{hion.pdgb}:
\begin{equation}
\label{muion.c}
 \left(\frac{dE}{dx} \right) =
\left \{
\begin{array}{lcl}
 N_{el} \frac{ Z_{inc}^{2}}{\beta^{2}} \left [
\ln \left (
\frac{2 m_{e} \beta^{2} \gamma^{2} T_{max} }{I^{2}}
\right ) - 2 \beta^{2} -
\delta -
\frac{ 2 C_{e}}{Z} \right ]
\hfill & \mbox{if} & T_{cut} \geq T_{max} \\ [0.5cm]
 N_{el} \frac{ Z_{inc}^{2}}{ \beta^{2}} \left [
\ln \left (
\frac{ 2 m_{e} \beta^{2} \gamma^{2} T_{c} }{I^{2}}
\right ) -
\beta^{2} \left ( 1 + \frac{T_{c}}{T_{max}} \right )
 - \delta
-\frac{2 C_{e}}{Z} \right ] \hfill & \mbox{if} & T_{cut} < T_{max}
\end{array}
\right .
\end{equation}

where,
$ N_{el}$ \mbox{is the electron density in the medium} ,
$T _c = \min(T_{cut},T_{max})$ , $m_e$ \mbox{the electron mass}
and $I$ is the average ionisation potential .
There exists a variety of phenomenological approximations for this.
 At present the simple expression  $I=16\:Z^{0.9}$ eV is used for all the 
elements.

$\delta$ is a correction term which takes account of the reduction
in energy loss due to the so-called density effect. This becomes
important at high energy because media have a tendency to become
polarised as the incident particle velocity increases. As a consequence,
the atoms in a medium can no longer be considered as isolated. To correct
for this effect the formulation of Sternheimer~\cite{hion.sternheimer}
is used:

\begin{equation}
\label{muion.d}
\delta = \left \{ \begin{array}{lll}
            0                   &    \mbox{if} & X  < X_0      \\
            4.606X + C + a(X_1-X)^m & \mbox{if} & X_0 \leq X < X_1 \\
            4.606X + C  & \mbox{if}  & X  \geq X_1
            \end{array} \right .
\end{equation}

where the medium-dependent constants are calculated as follows:
\[
\begin{array}{lcllcl}
X & = & \log_{10}(\gamma \beta) = \ln(\gamma^{2} \beta^{2})/4.606 &
\nu_{p} & = & \sqrt{ \frac{N_{el} e^2 }{\pi m} }
\mbox{\hspace{0.25cm}s$^{-1}$ plasma frequency} \\
N_{el} & = & \mbox{electron density in the material}
& C & = & -2 \ln \left ( \frac{I}{h\nu_{p}} \right ) - 1 \\
a & = & \frac{4.606(X_a - X_0)}{(X_1 - X_0)^m} & 4.606 \: X_{a} & = & -C
\end{array}
\]

For condensed media we have:

\[
\begin{array}{ll}
I < 100 \: \mbox{eV} & \left \{
\begin{array}{lcl}
X_{0} & = & \left \{
\begin{array}{lll}
0.2 & \mbox{if} & -C \leq 3.681 \\
-0.326C-1.0 & \mbox{if} & -C > 3.681
\end{array} \right . \\
X_{1} & = & 2.0 \\
m & = & 3.0
\end{array} \right . \\ [1.0cm]
I \geq 100 \: \mbox{eV} & \left \{
\begin{array}{lcl}
X_{0} & = & \left \{
\begin{array}{lll}
0.2 & \mbox{if} & -C \leq 5.215 \\
-0.326C-1.5 & \mbox{if} & -C > 5.215
\end{array} \right . \\
X_{1} & = & 3.0 \\
m & = & 3.0
\end{array} \right .
\end{array}
\]

and in the case of gaseous media $m=3$ and:

\[
\begin{array}{llcrcl}
X_0 = 1.6 & X_1 = 4 & \mbox{for} &       & C & \leq 9.5 \\
X_0 = 1.7 & X_1 = 4 & \mbox{for} & 9.5 < & C & \leq 10 \\
X_0 = 1.8 & X_1 = 4 & \mbox{for} & 10  < & C & \leq 10.5 \\
X_0 = 1.9 & X_1 = 4 & \mbox{for} & 10.5 < & C & \leq 11 \\
X_0 = 2.  & X_1 = 4 & \mbox{for} & 11  < & C & \leq 12.25 \\
X_0 = 2.  & X_1 = 5 & \mbox{for} & 12.25 < & C & \leq 13.804 \\
X_0 = 0.326C-2.5 & X_1 = 5 & \mbox{for} & 13.804 < & C &  \\
\end{array}
\]

$C_e/Z$ is a so-called {\it shell correction term}
which accounts for the fact that, at
low energies for light elements and at all energies for heavy ones, the
probability of collision with the electrons of the inner atomic shells
(K, L, etc.) is negligible. The semi-empirical formula used in
{\tt GEANT4}, applicable to all materials, is due to Barkas \cite{hion.barkas}:
\begin{eqnarray*}
\label{muion.e}
 C_e(I,\eta) & = & (0.42237\eta^{-2}+0.0304\eta^{-4}-0.00038\eta^{-6})
10^{-6} I^2 \\
& +& (3.858\eta^{-2}-0.1668\eta^{-4}+0.00158\eta^{-6})10^{-9}I^3 \\
\eta & = &  \gamma \beta
\end{eqnarray*}
This formula breaks down at low energies, and it only applies for values
of $\eta > 0.13$ (i.e. $T > 0.9$ MeV for a muon). For $\eta \leq
0.13$ the shell correction term is calculated as:

\[
\left . C_{e}(I,\eta) \rule{0mm}{5mm} \right |_{\eta \leq
0.13} = C_{e}(I,\eta=0.13)\frac{\ln \left ( \frac{T}{T_{2l}} \right )}
{\ln \left ( \frac{7.9 \: 10^{-3} \: \rm GeV}{T_{2l}} \right )}
\]

i.e. the correction is {\it switched off} logarithmically from $T=0.9$
MeV to $T=T_{2l}=0.225$ MeV, the latter value is the limit between the use 
 of the Bethe-Bloch formula and the low energy parametrisation .

The mean energy loss can be described by the Bethe-Bloch formula
(\ref{muion.c}) only if the projectile velocity is larger than that
of orbital electrons. In the low-energy region where this is not
verified, a different kind of parameterisation should be used. 
Presently a simple phenomenological parametrisation has been chosen which 
 gives the energy loss as a function of the variable
 $\tau = \frac{T}{M}$.
The parametrisation is given by the expression
\begin{equation}
\label{muion.f}
\frac{dE}{dx}  =  N_{el} \cdot (A\sqrt{\tau}+B \tau )  
       \mbox{\hspace{0.5cm}for\hspace{0.5cm}}
        \tau \leq \tau _{0} 
\end{equation}
\begin{equation}
\label{muion.g}
\frac{dE}{dx}  =  N_{el} \cdot \frac{C}{\sqrt{\tau }} 
              \mbox{\hspace{0.5cm}for\hspace{0.5cm}} 
        \tau _{0} < \tau \leq \tau _{1} 
\end{equation}
 ,the values of the $A,B,C$ and $\tau _{0} ,\tau _{1}$ parameters are
the same than in the case of charged hadrons .This parametrisation is
used for $T < \tau_1 \cdot M$ and                     
  the Bethe-Bloch formula is used above this kinetic energy .       

The energy lost due to the soft $\delta$-rays is tabulated during
initialisation as a function of the medium and of the energy .

\subsection{Total cross-section}
The integration of formula (\ref{muion.a}) gives the total cross-section :
\begin{equation}
\label{muion.h}
\sigma (Z,E,T_{cut}) = 
2 \pi r^2_0 m  \frac{Z}{\beta^2} \left (\frac{1-Y+\beta^2 Y\ln Y}{T_{cut}}
+ \frac{T_{max}-T_{cut}} {2E^2} \right) 
\end{equation}
 , where  $Y = \frac{T_{cut}}{T_{max}} $ .

The mean free path is tabulated during initialisation
as a function of the medium and of the energy . 

\section{Simulation of the knock-on electrons}
\subsection{Differential cross-section}
 The differential cross-section
can be written as:

\begin{equation}
\label{muion.i}
\frac{d \sigma} {dT}= 2 \pi Z r^2_0 m \frac{1}{\beta^2 } \frac{1}{T^2}
\left[1- \beta^ 2 \frac{T}{T_{max} }+ \frac{T^2} {2E^2} \right]
\end{equation}

where $T_{max}$  is the maximum energy transferable to the free electron 
(\ref{muion.b})
and $T_{cut}$  the energy threshold for the $\delta$-ray emission:
$ T_{cut} \leq T \leq  T_{max} $

\subsection{Sampling}
Apart from the normalisation, the cross-section can be written as

\begin{eqnarray*}
f(T) &=& \left(\frac{1}{T_{cut}} -\frac{1}{T_{max}}\right)
\frac{1}{T} \\
g(T) &=& 1 - \beta^2\frac{T}{T_{max}} +\frac{T^2}{2E^2}
\end{eqnarray*}
The variable $\epsilon$ is sampled by:
\begin{enumerate}
\item sample $\epsilon$ from $f(\epsilon)$
\item calculate the rejection function $g(\epsilon)$ and accept the
sampled $\epsilon$ with a probability of $g(\epsilon)$.
\end{enumerate}

After the successful sampling of $\epsilon$,  the polar
angles of the scattered electron are generated 
with respect to the direction of the
incident particle. The azimuthal angle $\phi$ is generated isotropically;
the polar angle
$\theta$ is calculated from the energy momentum conservation.
This information
is used to calculate the energy and momentum of both scattered
particles and to transform them into the global coordinate system.

\section{Status of this document}
  9.10.98  created by L. Urb\'an.
\begin{thebibliography}{99}

\bibitem[GEANT3]{hion.geant3}
  GEANT3 manual ,CERN Program Library Long Writeup W5013 (October 1994).
\bibitem[Mess70]{hion.messel}
  H.Messel and D.F.Crawford. Pergamon Press,Oxford,1970.
\bibitem[Ster71]{hion.sternheimer}
  R.M.Sternheimer. Phys.Rev. B3 (1971) 3681.
\bibitem[PDGB88]{hion.pdgb}
  Particle Data Group . Rev. of Particle Properties.
   Eur. Phys. J. C 3. (1988) 1.
\bibitem[Bark62]{hion.barkas}
  W. H. Barkas. Technical Report 10292,UCRL, August 1962.
\bibitem[Serr67]{hion.serre1}
  C.R.Serre. Techical Report 67 5 , CERN, March 1967.
\bibitem[Serr71]{hion.serre2}
  C.R.Serre. Techical Report 71 18, CERN, September 1971.

\end{thebibliography}

