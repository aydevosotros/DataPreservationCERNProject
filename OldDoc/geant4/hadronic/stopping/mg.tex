\section{Complementary parameterised and theoretical treatment}

Absorption of negative pions and kaons at rest from a nucleus 
is described in literature
(references...) 
as consisting of two main components:
\begin{itemize}
\item a primary absorption process, involving the interaction of the
incident stopped hadron with one or more nucleons of the target nucleus;
\item the deexcitation of the remnant nucleus, left in an excitated 
state as a result of the occurrence of the primary absorption process.
\end{itemize}

This interpretation is supported by several experiments
(references...),
that have measured various features characterizing these processes.
In many cases 
(references...)
the experimental measurements are capable to distinguish the final
products originating from the primary absorption process and those 
resulting from the nuclear deexcitation component.

A set of stopped particle absorption processes is implemented in GEANT4,
based on this two-component model (PiMinusAbsorptionAtRest and 
KaonMinusAbsorptionAtRest classes, for $\pi^{-}$ and $K^{-}$ respectively.
Both implementations adopt the same approach:
the primary absorption component of the process is parameterised,
based on available experimental data; 
the nuclear deexcitation component is handled through theoretical models
(reference to other chapter...).

