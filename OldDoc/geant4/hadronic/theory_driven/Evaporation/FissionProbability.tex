\section{Fission probability calculation.}

\hspace{1.0em}The fission decay channel (only for nuclei with $A > 65$)
is taken into account as a competitor for fragment and photon evaporation
channels.

\subsection{The fission total probability.}
hspace{1.0em}The fission probability (per uni time) $W_{fis}$ 
in the Bohr
and Wheeler theory of fission \cite{BW39} is proportional to the level
density $\rho_{fis}(T)$ ( approximation Eq. ($\ref{SFE13}$) is used) at
the saddle point, i.e.
\begin{equation}
\begin{array}{c}
\label{FP1}W_{fis}=\frac{1}{2\pi \hbar \rho_c(U_c)}\int_{0}^{U_f-B_{fis}}
\rho_{fis}(U_f-B_{fis}-T)dT =\\
=\frac{1 + (C_f - 1)\exp{(C_f)}}{4\pi a_{fis} \exp{(2\sqrt{a_cU_c})}},
\end{array}
\end{equation}
where $U_f= E^{*} - \Delta_f$ and pairing energy
\begin{equation}
\label{FP2} \Delta_{f} = \kappa \frac{14}{\sqrt{A}} \ [MeV]
\end{equation} 
In Eq. ($\ref{FP1}$)  $B_{fis}$ is the fission barrier height.
The value of $C_f = 2\sqrt{a_{fis}(U_f - B_{fis})}$ and $a_c$, $a_{fis}$ are 
the level density parameters of the compound and of the fission saddle point
nuclei, respectively.

The value of the level density parameter is large at the saddle point,
when excitation energy is given by initial excitation energy minus the
fission barrier height, than in the ground state, i. e. $a_{fis} > a$.
$a_{fis} = 1.08 a$ for $Z < 85$, $a_{fis} = 1.04 a$ for $Z \geq 89$ and
$a_f=a[1.04+0.01(89.-Z)]$ for $85 \leq Z < 89$ is used.
 
\subsection{The fission barrier.}

\hspace{1.0em}
The fission barriers are determined as differences 
between the saddle-point and
 ground state masses. In the general case fission barriers are functions 
 of the charge $Z$, atomic mass number $A$, excitation energy of fissioning 
 nuclei $E^{*}$ and their angular momenta $L$ and their deformations 
 $\alpha$. 
 
 Shell structure effects play a role at the fission
barrier. The height of fission barrier can be aprroximated as
\begin{equation}
\label{FP3}B_{fis} = B^{0}_{fis} + \Delta_{Shell} + \Delta_{SP},
\end{equation}
where $B^{0}_{fis}$ is the so-called liquid drop component of the
fission barrier, $\Delta_{Shell}$ is the shell correction to the mass of a
nucleus in the ground state and $\Delta_{SP}$ is the shell correction to
the mass of nucleus in the saddle point.  The last correction is very
important for actinide nuclei. It leads to a double-humped shape of the
fission barrier.
 
There are many models for fission barriers: the phenomenological approach 
of Barashenkov et al. \cite{Barash73}, the semiphenomenological 
approach of Barashenkov and Gereghi \cite{Barash77}, the liquid-drop model (LDM) 
with Myers and Swiatecki's parameters \cite{MS67}, the LDM with Pauli and Ledergerber's 
 parameters \cite{PG71}, the single-Yukawa modified LDM of Krappe and Nix 
 \cite{KN73}, the Yukawa-plus-exponential modified LDM \cite{KNS79}, the subroutine 
 BARFIT of Sierk \cite{Sierk86} which provides macroscopic fission barriers of 
 rotating nuclei in the Yukawa-plus-exponential modified LDM \cite{KNS79}, 
 double-humped fission barriers for transuranium nuclides as proposed in 
 \cite{KIF80}, giving a fixed 
 input value for single-humped fission barriers $B_{fis}$ 
 and giving fixed input values $B^A_{fis}$ and $B^B_{fis}$ for 
 double-humped fission barriers.
  
We use simple semiphenomenological 
approach was suggested by Barashenkov and Gereghi \cite{Barash77}. In their 
approach fission barriers $B_{fis}(A,Z)$ are approximated by 
\begin{equation}
\label{FP4} B_{fis} = B^{0}_{fis} + \Delta^{C}_{Shell}  + \Delta^{C}_{Pair} + 
\delta(A,Z),
\end{equation}
where shell and pairing corrections  for Cameron's
liquid drop mass formula \cite{CAM57}. $\delta(A,Z) = 0$ for even $Z$ 
and even $N = A - Z$, $\delta(A,Z) = 1.248$ \ MeV for odd $A$ and 
$\delta(A,Z) = 2.496$ \ MeV for odd $Z$ and odd $N$ were suggested.

According to this prescription fission barrier heights $B^{0}_{fis}(x)$
vary with the fissility parameter $x$. $B^{0}_{fis}(x)$ is
given by
\begin{equation}
\label{FP5} B^{0}_{fis}(x) = a_{S}A^{2/3} 0.83(1 - x)^3
\end{equation}
for $2/3 \leq x \leq 1$ and
\begin{equation}
\label{FP6} B^{0}_{fis}(x) =a_{S}A^{2/3} 0.38(3/4 -x)
\end{equation}
for $1/3 \leq x \leq 2/3$. 
The fissility parameter $x$ is given by
\begin{equation}
\label{FP7} x = \frac{E^0_C}{2E^0_S}= \frac{(a_C/2a_S)Z^2/A}
{\{1 - k[(N-Z)/A]^2\}},
\end{equation}
where $E^{0}_C$ and $E^{0}_S$ are the Coulomb and surface energies of 
a spherical nucleus, respectively.

The liquid drop model parameters $a_S = 17.9439$\ MeV, $a_C=0.7053$ \ MeV 
and $k = 1.7826$ taken from \cite{CS63} paper.

The fission barrier heights are functions of the excitation energy. We use the 
empirical relation proposed by \cite{Barash73} to estimate the dependence of 
$B_{fis}$ on $E^{*}$:
\begin{equation}
\label{FP8} B_{fis}(E^{*}) =\frac{B_{fis}}{1+\sqrt{(\frac{E^{*}}{2A})}}.
\end{equation} 
