\section{Phenomenological potentials.}
We have taken into account mean field nuclear potentials for nucleons, 
antinucleons, pions, kaons, lambdas and sigmas particles. Thus we have 
a possibility to consider nuclear absorption of these particles and 
therefore to calculate nucleus excitation energy as well as 
the number of excitons. 

\subsection{The nucleon potential.}
\hspace{1.0em}
For nucleons this potential is defined by sum of the Fermi-potential, the binding
energy $B(A,Z)$ and the Coulomb potential $V_{C}(Z,r)$ (for protons):
\begin{equation}
\label{PP1}
V_{i}(A,Z,r)=\frac{[p_{i}^F(r)]^{2}}{2m_{i}}+B(A,Z)+V_{C}(Z,r).
\end{equation}
Nucleons are transported along the straight line paths, i. e. the
curvatures of nucleon trajectories in the nuclear and Coulomb potentials
are neglected (Usualy the curvature effects are simplified in a cascade 
model by taken into account
the refraction and reflection of nucleons due to a jump of potential
\cite{BIST73}). The influence of intranuclear nucleons on the incoming
nucleon is taken into account by adding to its kinetic energy this
effective nuclear potential, i.e. the kinetic energy of a nucleon is
calculated from the bottom of potential well.  For a proton  leaving the
nucleus the kinetic energy of this proton has to be sufficiently high to
overcome the Coulomb barrier (in MeV) given by
\begin{equation}
\label{PP2}
V_{C}(A,Z)=C\frac{Z}{ r_{0}(1+A^{1/3})},
\end{equation}

where $C = 1.44$\ MeVfm. 
As it was already mentioned the influence of intranuclear nucleons on 
the pions, kaons, antinuclons, sigmas and lambdas 
 is  taken into account.
To determine the potential energy $V_{A}(p,r)$, which  in general case is a 
function  of  
momentum $p$ and  position $r$ of these particles
 in the field of a 
nucleus as a whole we use
the real parts of the corresponding optical potential. 

\subsection{The optical potentials.}
\hspace{1.0em}
The optical potential represents all particle-nucleon interactions between 
 the incident particle and the target nucleus. The real part describes the 
 scattering and imaginary part describes the absorption. The real part is 
 used to describe bound states. 
 The simplest form of optical potential is 
\begin{equation}
\label{PP3} V_{opt}=(U + iW)f(r),
\end{equation}
where $U$ and $W$ are the real and imaginary potential depths and $f(r)$ is the
radial form factor. This form factor is usualy taken in the Woods-Saxon form
\begin{equation}
\label{PP4}f(r)=\frac{1}{1+\exp{(\frac{r-R}{a})}}
\end{equation}
where $R$ and $a$ a the radius and surface diffuseness parameters, which 
are different for different nuclei.
Such form of optical potential is used mainly for description of nucleon, 
antiproton, kaon as well as lambda, sigma nucleus interactions.
The optical potential potentials is used for pions has more
 complicated form \cite{SMC79}:
\begin{equation}
\label{PP5} V_{opt} = \beta(k,r)-\nabla \alpha(k,r) - \nabla^2 q(k,r),
\end{equation}
where the functions $\beta$, $\alpha$, and $q$ depend on the coordinate $r$ 
and momentum $p$ of the incident particle.

\subsection{Pion--nucleus interaction potential.}
\hspace{1.0em}
The optical $\pi A$-potential 
of the second order \cite{SMC79} defined by Eq. ($\ref{PP5}$), 
which is used for description of of multiple-
scattering, is  
a rather complicated function 
on the pion momentum and on the pion radius $r$. 
This potential can be either attractive or repulsive. This potential is not 
implemented yet. To simplify situation 
for pions we use  
\begin{equation}
\label{PP6} V_{\pi}(r)=-V^{\pi}_{0}f(r),
\end{equation}
where $V^{\pi}_{0} = 25$ \ MeV and $f(r)$ is choosen in the Woods-Saxon 
(as nuclear density for heavy nuclei) or the Gaussian (as the nuclear 
density for light nuclei) forms.

\subsection{Kaon--nucleus optical potential.}
\hspace{1.0em}
The optical potential is used for an analysis of kaonic atoms can be written in
the form
\begin{equation}
\label{PP17} V_{opt}(r)=-\frac{2\pi}{\mu}(1+\frac{m_{K}}{m})[
a^{eff}_{K^{-}n}\rho_{n}(r)+a^{eff}_{K^{-}p}\rho_{p}(r)],
\end{equation}
where $\mu\approx$ is $K$-nucleus reduced mass, $m_{K}$ is kaon mass and 
 $m$ is mass of the nucleon. $\rho_{n}(r)$ and $\rho_{p}(r)$ are the neutron
 and proton density distributions 
  normalized to the 
number of nucleons. $a^{eff}_{K^{-}n}$ and $a^{eff}_{K^{-}p}$ are complex 
effective scattering lengths for kaon-neutron and kaon-proton interactions, 
respectively.

The simplest form of this optical potential is often used together with Coulomb
 potential:
\begin{equation}
\label{PP18} V_{opt}(r)=-\frac{2\pi}{\mu}(1+\frac{m_{K}}{m})a\rho(r),
\end{equation} 
where $\rho(r)$ is the nuclear density distribution normalized to the 
number of nucleons. The complex coefficient $a=0.35 \pm 0.03 + 
i(0.82 \pm 0.03$ fm can be found in  \cite{Batty82} and $a=0.63 \pm 0.06 + 
i(0.89 \pm 0.05$ fm can be found in \cite{Batty95} are determined 
by fit to the $K$-atom data.

\subsection{ Antiproton-- and sigma--nucleus optical potential.}
\hspace{1.0em}
 For antiprotonic and sigma atoms an optical potential with a form similar to that for 
 kaonic atom is used:
\begin{equation}
\label{PP19} V_{opt}(r)=-\frac{2\pi}{\mu}(1+\frac{m_{{\bar{p}}}}{m})a\rho(r),
\end{equation} 
where $\mu$ is $\bar{p}$-nucleus reduced mass and 
$m_{\bar{p}}$ is the antiproton mass. The complex coefficient $a=1.53 \pm 0.27
 + 
i(2.50 \pm 0.25$ fm  is determined 
by fit to the $\bar{p}$-atom data \cite{Batty82}. To fit the sigma-atom data 
the value of $a=0.36 \pm 0.05
 + 
i(0.19 \pm 0.03$ fm was obtained \cite{Batty82}.

