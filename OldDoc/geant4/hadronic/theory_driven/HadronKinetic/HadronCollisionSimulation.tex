\section{Hadron collision simulation.}
\hspace{1.0em}According to considered hadron interaction 
cross sections we should be 
able to simulate two-body hadron elastic scattering as well as two-body 
inelastic scatterings, meson-barion and meson-meson resonance interaction as 
well as antibarion-barion annihilation.

\subsection{Two-body hadron scatterings.}

\hspace{1.0em}
We distinguish three different mechanisms of the two-body hadron scatterings:
(1) elastic scatterings, (2) two-body scatterings with resonance excitation and  
deexcitation and (3)two-body scattering with string excitation.
To simulate elastic scattering we need to know only the scattered angle 
distribution. The choice of angular distribution is described below.
To simulate the two-body scatterings with resonance excitation and  
deexcitation resonances we have to select a proper channel (according 
to calculated cross sections) and sample resonance mass (according to 
Breit-Wigner distribution with corresponding pole mass and pole width) and 
sample scattered angle (this is also described below).  
The two-body scattering with string excitation
 is simulated using the parton string model, when  
two and more strings (through the diffraction and parton exchange 
mechanisms) can be produced.

\subsection{Angular distribution of two-body hadron scatterings.}

\hspace{1.0em}It is assumed that angular distributions for all relevant
two-body processes are similar and can be approximated by the
differential cross-section of $NN$ elastic scattering. 
This cross 
 section can be written in the form \cite{URQMD97}:
\begin{equation}
\label{HCS1}
  \sigma_{NN \rightarrow NN}(s,t) = \frac{1}{(2 \pi)^{2} s} \lbrack D(s,t)
  + E(s,t) + (s, t \longleftrightarrow u) \rbrack, 
  \end{equation}
  where
 \begin{eqnarray}
  D(s,t)&=& \frac{({\rm g}_{NN}^{\sigma})^{4}}{2(t-m_{\sigma}^{2})^{2}}
 (t- 4 m_N^{2})^{2} + \frac{({\rm g}_{NN}^{\omega})^{4}}{(t-m_{\omega}^{2})^{2}}
 (2 s^{2} + 2st +t^{2} -8m_N^{2}s +8m_N^{4}) \nonumber \\
  &&  + \frac{24({\rm g}_{NN}^{\pi})^{4}}{(t- m_{\pi}^{2})^{2}} m_N^{4}t^{2}
 - \frac{4({\rm g}_{NN}^{\sigma}{\rm g}_{NN}^{\omega})^{2}}
{(t - m_{\sigma}^{2})(t-m_{\omega}^{2})} (2s + t -4m_N^{2})m_N^{2}, \\
  E(s,t)&=& -\frac{({\rm g}_{NN}^{\sigma})^{4}}{8(t-m_{\sigma}^{2})
 (u-m_{\sigma}^{2})} \lbrack t (t+s) + 4m_N^{2}(s-t) \rbrack
  + \frac{({\rm g}_{NN}^{\omega})^{4}}{2(t-m_{\omega}^{2})(u-m_{\omega}^{2})}
 (s- 2m_N^{2}) \nonumber \\ && \times (s-6m_N^{2})  
    - \frac{6({\rm g}_{NN}^{\pi})^{4}}{(t- m_{\pi}^{2})(u-m_{\pi}^{2})}
  (4m_N^{2}-s -t ) m_N^{4}t \nonumber \\
&& + ({\rm g}_{NN}^{\sigma}{\rm g}_{NN}^{\omega})^{2}
 \lbrack \frac{t^{2} - 4m_N^{2}s -10m_N^{2}t +24m_N^{4}}{4(t-m_{\sigma}^{2})
(u-m_{\omega}^{2})} + \frac{(t+s)^{2} - 2m_N^{2}s + 2m_N^{2}t }{4 (t-m_{\omega}
^{2})(u-m_{\sigma}^{2})}  \rbrack \nonumber \\
  && + ({\rm g}_{NN}^{\sigma}{\rm g}_{NN}^{\pi})^{2}
 \lbrack \frac{3m_N^{2}(4m_N^{2}-s-t)(4m_N^{2}-t)}{2(t-m_{\sigma}^{2})
(u-m_{\pi}^{2})} + \frac{3t(t+s)m_N^{2}}{2 (t-m_{\pi}
^{2})(u-m_{\sigma}^{2})}  \rbrack \nonumber \\
  && + ({\rm g}_{NN}^{\omega}{\rm g}_{NN}^{\pi})^{2}
 \lbrack \frac{3m_N^{2}(t+s-4m_N^{2})(t+s-2m_N^{2})}{(t-m_{\omega}^{2})
(u-m_{\pi}^{2})} + \frac{3m_N^{2}(t^{2}-2m_N^{2}t)}{ (t-m_{\pi}
^{2})(u-m_{\omega}^{2})}  \rbrack, 
  \end{eqnarray}
where the function $D$ represents the contribution of the 
direct term and $E$ is the
exchange term. The coupling strengths are ${\rm g}_{NN}^{\sigma}=6.9$,
${\rm g}_{NN}^{\omega}=7.54$ and ${\rm g}_{NN}^{\pi}=1.434$.
$s=(p_1+p_2)^2$, $t=(p_1-p_3)^2 =1/2(s-4m^2_N)(\cos{\theta} - 1)$
and $u= (p_1 - p_4)^2=4m^2_N - s - t$ are Mandelstam's variables
determined by nucleon four momenta $p_{1,2,3,4}$ and by nucleon mass
$m_N$. The $\theta$ is c.m. scattering angle.
  This formula for the differential cross
section of $NN$ elastic scattering is extended to all two-body 
collisions by the replacement $s\rightarrow s - (m_1+m_2)^2 + 4m^2_N$,
where $m_1$ and $m_2$ denote the masses of incoming hadrons.

The finite size hadron effects  
are taken into account by introducing 
a phenomenological form factor at
\begin{equation}
\label{HCS2}
F_{NNA} = \frac{\Lambda_A^2}{\Lambda_A^2-t} \; .
\end{equation}
Here $\Lambda_A$ is the cut-off mass of the meson $A$. These cut-off masses are
$\Lambda_{\sigma}=$1200 MeV, $\Lambda_{\omega}=$808 MeV and 
$\Lambda_{\pi}=$500 MeV. \\

\subsection{Meson-baryon and meson-meson resonance interaction 
simulation.}

\hspace{1.0em}From knowledge of resonance interaction meson-baryon and 
meson-meson cross section we are able to select a resonance, which 
should be created at given initial energy. Outside of resonance high 
energy range inelastic meson-barion and meson-meson interaction is simulated 
using the parton string model, when one (through the quark annihilation 
mechanism) and more strings (through the diffraction and parton exchange 
mechanisms) can be produced.

\subsection{Baryon-antibaryon annihilation simulation.}

\hspace{1.0em} The final state of a baryon-antibarion annihilation can 
be generated via rearrangement of quarks or the formation of one, two 
or three mesonic (with quark and antiquark on the string ends) strings. 
All these processes have strong initial energy dependencies. 
Particularly, initial quark redestribution plays crucial role at stopped 
baryon annihilation and three string production process is main 
process at high energies. 
To simplify a situation we keep only single string production process, 
i. e. we consider process when diquark-antidiquark annihilation 
take place. This process is simulated using the parton string 
model as it is described above.

\subsection{$\pi$-absorption simulation.}

\hspace{1.0em}
Once a pion has been absorbed by a nucleon pair, the pion mass is converted
into kinetic energy of nucleon. Each nucleon has the energy $E_N = m_{\pi}/2$ 
in the center of mass pair. In the center of mass nucleons flay away in
opposite direction isotropically.
The inital momentum of pair is taken as a sum of nucleon Fermi momenta.
 At increasing energy 
 the  
two-steps $p$-wave (resonant) absorption, which is
 going through the excitation, e. g. $\pi + N
\rightarrow \Delta$, and subsequent rescattering, e. g. $\Delta \rightarrow N
N$, resonances plays important role.
