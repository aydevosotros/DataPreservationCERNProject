\section{Simulation of pre-compound reaction.}

\hspace{1.0em}The precompound stage of 
nuclear reaction is considered until nuclear 
system is not an equilibrium state. 
Further emission of nuclear fragments or photons from excited 
nucleus is simulated using an equilibrium 
 model. 


\subsection{Statistical equilibrium condition.}

\hspace{1.0em}In the state of statistical equilibrium, which is
characterized by an eqilibrium number of excitons $n_{eq}$, all three
type of transitions are equiprobable. Thus $n_{eq}$ is fixed by
$\omega_{+2}(n_{eq},U) = \omega_{-2}(n_{eq},U)$. From this condition we
can get
\begin{equation}
\label{PCS1}n_{eq} = \sqrt{0.5+ 2gU}.
\end{equation}

\subsection{Level density of excited ($n$-exciton) states.}

\hspace{1.0em}To obtain Eq. ($\ref{PCS1}$) it was assumed an equidistant 
scheme of single-particle levels with the density $g \approx 0.595 aA$,
where $a$ is the level density parameter, when we have the level density
of the $n$-exciton state as
\begin{equation}
\label{PCS2} \rho_{n}(U) = \frac{g(gU)^{n-1}}{p!h!(n-1)!}.
\end{equation}

\subsection{Transition probabilities.} 

\hspace{1.0em}The partial transition probabilities changing the exciton
number by $\Delta n$ is determined by the squared matrix element
averaged over allowed transitions $<|M|^{2}>$ and the density of final
states $\rho_{\Delta n}(n,U)$, which are really accessible in this
transition. It can be defined as following:
\begin{equation}
\label{PCS3}\omega_{\Delta n}(n,U)=\frac{2\pi}{h}<|M|^{2}>\rho_{\Delta n}(n,U).
\end{equation}
The density of final states $\rho_{\Delta n}(n,U)$ were derived in paper
\cite{Williams70} using the Eq. ($\ref{PCS2}$) for the level density of
the $n$-exciton state and later corrected for the Pauli principle and
indistinguishability of identical excitons in paper \cite{ROB73}:
\begin{equation}
\label{PCS4}\rho_{\Delta n = +2}(n,U)=\frac{1}{2}g\frac{[gU - F(p+1,h+1)]^2}
{n+1} [\frac{gU - F(p+1,h+1)}{gU - F(p,h)}]^{n-1},
\end{equation}
\begin{equation}
\label{PCS5}\rho_{\Delta n = 0}(n,U)=\frac{1}{2}g\frac{[gU - F(p,h)]}{n}
[p(p-1) + 4ph + h(h-1)]
\end{equation}
and
\begin{equation}
\label{PCS6}\rho_{\Delta n = -2}(n,U)=\frac{1}{2}gph(n-2),
\end{equation}
where $F(p,h)=(p^2 + h^2 + p -h)/4 - h/2$ and it was taken to be equal
zero.  To avoid calculation of the averaged squared matrix element
$<|M|^2>$ it was assumed \cite{GMT83} that transition probability
$\omega_{\Delta n = +2}(n,U)$ is the same as the probability for
quasi-free scattering of a nucleon above the Fermi level on a nucleon of
the target nucleus, i. e.
\begin{equation}
\label{PCS7}\omega_{\Delta n =+2}(n,U)=\frac{<\sigma(v_{rel})v_{rel}>}{V_{int}}.
\end{equation}
In Eq. ($\ref{PCS7}$) the interaction volume is estimated as 
$V_{int}=\frac{4}{3}\pi(2r_c + \lambda/2\pi)^3$, 
with the De Broglie wave length
$\lambda/2\pi$ corresponding to the relative velocity 
$<v_{rel}>=\sqrt{2T_{rel}/
m}$, where $m$ is nucleon mass and $r_c = 0.6$ fm. 

The averaging in $<\sigma(v_{rel})v_{rel}>$ is further simplified by
\begin{equation}
\label{PCS8}<\sigma(v_{rel})v_{rel}> =<\sigma(v_{rel})><v_{rel}>.
\end{equation}
For $\sigma (v_{rel})$ we take approximation:
\begin{equation}
\label{PCS9}\sigma(v_{rel})=0.5[\sigma_{pp}(v_{rel})+\sigma_{pn}(v_{rel}]P(T_F/T_{rel}),
\end{equation}
where factor $P(T_F/T_{rel})$ was introduced to take into account the
Pauli principle. It is given by
\begin{equation}
\label{PCS10} P(T_F/T_{rel})=1 - \frac{7}{5}\frac{T_F}{T_{rel}} 
\end{equation}
for $\frac{T_F}{T_{rel}} \leq 0.5$ and 
\begin{equation}
\label{PCS11} P(T_F/T_{rel})=1 - \frac{7}{5}\frac{T_F}{T_{rel}}+ \frac{2}{5}(2 -
\frac{T_{rel}}{T_F})^{5/2} 
\end{equation}
for $\frac{T_F}{T_{rel}} > 0.5$. 

The free-particle proton-proton $\sigma_{pp}(v_{rel})$ and
proton-neutron $\sigma_{pn}(v_{rel})$ interaction cross sections are
estimated using the equations \cite{Metro58}:
\begin{equation}
\label{PCS12}\sigma_{pp}(v_{rel} =
 \frac{10.63}{v^2_{rel}}-\frac{29.92}{v_{rel}}+42.9
\end{equation}
and
\begin{equation}
\label{PCS13}\sigma_{pn}(v_{rel} =
 \frac{34.10}{v^2_{rel}}-\frac{82.2}{v_{rel}}+82.2,
\end{equation}
where cross sections are given in mbarn.
 
The mean relative kinetic energy $T_{rel}$ is needed to calculate
$<v_{rel}>$ and the factor $P(T_F/T_{rel})$ was computed as
$T_{rel}=T_{p}+T_{n}$, where mean kinetic energies of projectile
nucleons $T_p = T_F +U/n$ and target nucleons $T_N = 3T_F/5$,
respecively.
 

Combining Eqs. ($\ref{PCS3}$) - ($\ref{PCS7}$) and assuming that
$<|M|^{2}>$ are the same for transitions with $\Delta n = 0$ and $\Delta
n = \pm 2$ we obtain for another transition probabilities:
\begin{equation}
\begin{array}{c}
\label{PCS14}\omega_{\Delta n =0}(n,U)= \\
=\frac{<\sigma(v_{rel})v_{rel}>}{V_{int}}
\frac{n+1}{n}[\frac{gU - F(p,h)}{gU - F(p+1,h+1)}]^{n+1}
\frac{p(p-1) + 4ph +h(h-1)}{gU - F(p,h)}
\end{array} 
\end{equation}
and 
\begin{equation}
\begin{array}{c}
\label{PCS15}\omega_{\Delta n 
= -2}(n,U)= \\
=\frac{<\sigma(v_{rel})v_{rel}>}{V_{int}}
 [\frac{gU - F(p,h)}{gU - F(p+1,h+1)}]^{n+1}
 \frac{ph(n+1)(n-2)}{[gU - F(p,h)]^2}.
\end{array} 
\end{equation}

\subsection{Emission probabilities for nucleons.} 

\hspace{1.0em}Emission process probability has been choosen similar as 
in the classical equilibrium Weisskopf-Ewing model \cite{WE40}.
Probability to emit nucleon $b$ in the energy interval $(T_b, T_b+dT_b)$
is given
\begin{equation}
\label{PCS16}W_{b}(n,U,T_b) = \sigma_{b}(T_b)\frac{(2s_b+1)\mu_b}{\pi^2 h^3}
R_b(p,h)
\frac{\rho_{n-b}(E^{*})}{\rho_n(U)}T_b,
\end{equation}
where $\sigma_{b}(T_b)$ is the inverse (absorption of nucleon $b$)
reaction cross section, $s_b$ and $m_b$ are nucleon spin and reduced
mass, the factor $R_b(p,h)$ takes into account the condition for the
exciton to be a proton or neutron, $\rho_{n-b}(E^{*})$ and $\rho_n(U)$
are level densities of nucleus after and before nucleon emission are
defined in the evaporation model, respectively and $E^{*}=U-Q_b-T_b$ is the
excitation energy of nucleus after fragment emission.
 
\subsection{Emission probabilities for complex fragments.}

\hspace{1.0em}It was assumed \cite{GMT83} that nucleons inside excited
nucleus are able to "condense" forming complex fragment.  The
"condensation" probability to create fragment consisting from $N_b$
nucleons inside nucleus with $A$ nucleons is given by
\begin{equation}
\label{PCS17} \gamma_{N_b}=N^3_b(V_b/V)^{N_b -1}=N^3_b(N_b/A)^{N_b -1},
\end{equation}
where $V_b$ and $V$ are fragment $b$ and nucleus volumes, respectively.
The last equation was estimated \cite{GMT83} as the overlap integral of
(constant inside a volume) wave function of independent nucleons with
that of the fragment.

During the prequilibrium stage a "condense" fragment can be emitted.
The probability to emit a fragment can be written as \cite{GMT83}
\begin{equation}
\label{PCS18}W_{b}(n,U,T_b) =\gamma_{N_b}R_b(p,h)
 \frac{\rho(N_b, 0, T_b + Q_b)}{g_b(T_b)}
 \sigma_{b}(T_b)\frac{(2s_b+1)\mu_b}{\pi^2 h^3}
\frac{\rho_{n-b}(E^{*})}{\rho_n(U)}T_b,
\end{equation}
where 
\begin{equation}
\label{PCS19}g_b(T_b)=\frac{V_b(2s_b+1)(2\mu_b)^{3/2}}{4\pi^2 h^3}(T_b+Q_b)^{1/2}  
\end{equation}
is the single-particle density for complex fragment $b$, which is
obtained by assuming that complex fragment moves inside volume $V_b$ in
the uniform potential well whose depth is equal to be $Q_b$, and the
factor $R_b(p,h)$ garantees correct isotopic composition of a fragment
$b$.

\subsection{The total  probability.}

\hspace{1.0em}This probability is defined as 
\begin{equation}
\label{PCS20} W_{tot}(n,U) =\sum_{\Delta n =+2,0,-2}\omega_{\Delta n }(n,U) +
\sum_{b=1}^{6}W_b(n,U),
\end{equation} 
where total emission $W_b(n,U)$ probabilities to emit fragment $b$ can
be obtained from Eqs. ($\ref{PCS16}$) and ($\ref{PCS18}$) by
integration over $T_b$:
\begin{equation}
\label{PCS21} 
W_{b}(n,U)=\int_{V_b}^{U-Q_b} W_b(n,U,T_b)dT_b.
\end{equation}

 
\subsection{Calculation of kinetic energies for emitted particle.}

\hspace{1.0em}The equations ($\ref{PCS16}$) and ($\ref{PCS18}$)
 are
 used to sample kinetic energies of emitted fragment.

\subsection{Angular distribution of emitted fragments.}

\hspace{1.0em}The fragment  emission is considered to be isotropic  in
the nucleus rest frame. However, such assumption contradicts
experimental angular distribution of emitted fragments.  To improve the
isotropic approximation the next simple prescription \cite{GMT83} can be
used.  It is considered that the particle emission will be isotropic in
the proper $n$-exciton system and the incoming nuclear momentum is
shared only by the $n$-exciton system rather than whole nucleus. Thus in
the nucleus rest frame the angular distribution of emitted particles
will be anisotropic. 
There is another approach \cite{Mant75}, \cite{Akker80} 
related on the fast particle approximation. In this model the angular
orientation of the nucleus at each exciton collision step is defined by the 
direction of the fast particle, which changes gradually in a series of
two--body collisions. The transition rate between different exciton states and 
the emission probability is assumed to be factorizable in  angle dependent
and energy dependent factors.

\subsection{The angular momenta of 
evaporated fragments.}
 
\hspace{1.0em} The angular momenta of and evaporated particles
are considered as classical vectors ${\bf l_i}$ and estimated in the sharp
cut-off approximation \cite{IT69}, \cite{IT73} according to 
\begin{equation}
\label{PCS22}P(l_b)dl_b \sim l_bdm_b, 0 \leq l_b \leq l_b^{max}
\end{equation}
where
\begin{equation}
\label{PCS23} l_b^{max}=\sqrt{2\mu_b(E_b - V_b)}R_b/\hbar.
\end{equation}
Here $R_b$ is the radius of the interaction of the $b$th 
emmitted particle with the 
residual nucleus, $E_b$,$V_b$ and $\mu_b$ are the energy 
in the center-of-mass system,
Coulomb barrier and reduced mass of particle, respectively. 
The spins of the emmitted 
particles are not taken into account when estimating the 
angular momentum of the
residual nuclei. 
Angular momenta of residual nuclei are calculated 
without taking into
account the spin of initial target nucleus and of 
intermediate nuclei during emission
of particles.

\subsection{Parameters of residual nucleus.}

\hspace{1.0em}After fragment emission we  update parameter
of decaying nucleus:
\begin{equation}
\label{PCS24} 
\begin{array}{c}
A_f=A-A_b; Z_f=Z-Z_b; P_f = P_0 - p_b; \\ 
E_f^{*}=\sqrt{E_f^2-\vec{P}^2_f} - M(A_f,Z_f); \vec{L}_f = \vec{L}_0-\vec{l_b}. 
\end{array}
\end{equation}
Here $p_b$ is the evaporated fragment four momentum.
Angular momenta of residual nuclei are calculated without taking into
account the spin of initial target nucleus and
 of intermediate nuclei during emission
of fragments.

\subsection{MC procedure.}

\hspace{1.0em}Thus the Monte Carlo simulation of pre-equlibrium process
will be outlined as following:
\begin{enumerate}
\item For given excitation energy $U$, atomic number $A$ and number of excitons
 $n$ calculate the equilibrium number $n_{eq}$ of excitons according to
 Eq. ($\ref{PCS1}$).  If exciton number $n \geq n_{eq}$ then further
 emission of fragment should be simulated using an equilibrium model. If
 $n < n_{eq}$ then perform next step.
 
\item Taking into account reaction threshold condition  and 
the factor $R_b(p,h)$ calculate transition (according to Eqs.
($\ref{PCS7}$),($\ref{PCS13}$) and ($\ref{PCS14}$)) and emission
probabilities (according to Eqs. ($\ref{PCS16}$) and ($\ref{PCS18}$)), 
which should be
integrated over $T_b$. Then
use the Eq. ($\ref{PCS20}$) to normalize calculated probabilities to
obtain statistical weights for subprocesses.  Select a subprocess
according to calculated statistical weights.

\item In case of system
transition update the number of excitons, if it is needed and proceed
step (1). In case of fragment emission one should perform the next step.

\item Sample a fragment kinetic energy (according to Eq. ($\ref{PCS16}$) or 
 ($\ref{PCS18}$).
 
\item Sample fragment angles using the isotropical
angular distribution in the exciton system rest frame. Calculate
fragment momentum and perform boost to the nucleus rest frame.

\item Update characteristics of the residual nucleus according to Eqs.
($\ref{PCS24}$) and proceed step (1).
\end{enumerate}
      
