\section{Reaction initial state.}

\hspace{1.0em} The GEANT4 precompound model is considered as an extension 
of the hadron kinetic model. It gives a possibility to extend the low 
energy range of the hadron kinetic model for
nucleon-nucleus inelastic collision and it provides a "smooth" transition 
from kinetic stage of reaction described by the hadron kinetic model to 
the equilibrium stage of reaction described by the equilibrium deexcitation 
models. 

The initial information for calculation of pre-compound
nuclear stage consists from the atomic mass number $A$, charge $Z$ of residual
 nucleus, its four momentum $P_0$, angular 
momentum $\vec{L}_0$,
excitation energy $U$ and number of
excitons $n$ equals the sum of number of particles $p$ (from them $p_Z$
are charged) and number of holes $h$.

There are some peculiarities to get the initial information about pre-compound 
stage 
 for different kind of reactions. Particularly,
the excitation energy of nucleus in the case of nucleon--nucleus interaction 
can be calculated as
\begin{equation}
\label{RISPM1} U = T_{init} + B(A,Z),
\end{equation}
where $T_{init}$ is initial nucleon kinetic energy and $B(A,Z)$ is 
binding energy of nucleon. The initial number of exciton is a model 
parameter and usualy defined from comparison with experimental data. The 
recommended initial configuration is $2p1h$, i. e. $n=3$.

Another way to obtain this initial information is to invoke the hadron 
kinetic model. This model predicts the excitation energy and other parameters 
of residual nucleus. Particularly, the exciton numbers are calculated within 
the hadron kinetic model.

The excitons (particles and holes) energies are calculated in the kinetic 
 model from
the Fermi energy $T_F$.
All cascade nucleons with kinetic energies above $T_{F}$ and absorbed 
by a nucleus are called particles.
The holes are results of cascade interactions, when particles stroke
nucleons from nucleus and they occupy states below $T_{F}$.

At the preequilibrium stage of reaction we take into account all
possible nuclear transition the number of excitons $n$ with $\Delta n =
+2, -2, 0$ \cite{GMT83}, which defined by transition probabilities. 
 Only emmision of neutrons, protons, deutrons, thritium and helium nuclei
  are taken
into account.
